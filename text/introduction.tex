Code deficiencies and bugs constitute an unavoidable part of software systems.
%
In safety-critical systems, like aircrafts or medical equipment, 
even a single bug can lead to catastrophic impacts
such as injuries or death.
%
Even in less critical code, 
programs use various runtime assertions to establish correctness properties. 
%
Formal verification can be used to statically 
discharge assertions and 
track code deficiencies by proving or disproving correctness properties 
of a system. 
%
However, at its current state formal verification is a cumbersome process
that is rarely used by mainstream developers. 

We present \toolname, a \textit{usable} program verifier that aims to 
establish formal verification as an 
integral part of the development process.
%
A usable verifier \textit{naturally integrates}
the specification of correctness properties
in the development process. 
%
Moreover, verification should be \textit{automatic},
%with minimal user effort,
requiring no explicit proofs or complicated annotations. 
%
At the same time, the specification language should be \textit{expressive},
allowing the user to write arbitrary correctness properties. 
%
Finally, a usable verifier should be tested in \textit{real-world programs}. 

\href{http://goto.ucsd.edu/~rjhala/liquid/haskell/blog/about/}{\toolname} is a 
 verifier for Haskell programs
that takes as input Haskell source code, annotated with 
correctness specifications in the form of refinement types and checks whether the code satisfies the specifications. 
%
We designed \toolname in such a way as to satisfy most of the 
aforementioned criteria of a usable verifier.
%%

% integrated to the language (\cite{ICFP14})
\mypara{Natural Integration} of correctness specifications 
comes by our choice of the functional programming language Haskell as a target language.
%
% Functios are first class objects
% No mutation or side effects
% Rich type system natural to express specifications
Haskell's first class functions lead to modular specifications. 
The lack of mutations and side-effects allows a direct correspondence between 
source code and logic. 
%
Correctness specifications are naturally added to Haskell's expressive type system
as \textit{refinement types}, \ie types annotated with logical predicates.
As an example, the type 
\begin{code}
  type NonZero = {v:Int | 0 != v}
\end{code}
describes a value @v@ which is an integer 
and the refinement specifies that this value is not zero.
%
The specification language is simple
as most programmers are familiar with both
its ingredients, \ie Haskell types and logical formulas.

\mypara{Real World Applications} have been verified using 
\toolname. 
%
We proved critical safety and functional correctness of 
more that 10K lines of popular Haskell libraries (Chapter~\ref{chapter:tool})
with minimal amount of annotations.
%
We verified correctness of \textit{array-based sorting algorithm} ($\texttt{Vector-Algorithms}$),
preservation of \textit{binary search tree} properties ($\texttt{Data.Map}$, $\texttt{Data.Set}$),
preservation of \textit{uniqueness invariants} ($\texttt{XMonad}$), 
\textit{low-level memory safety} ($\texttt{Bytestring}$, $\texttt{Text}$), 
and even found and fixed a subtle correctness bug related to unicode handling in $\texttt{Text}$.
%
In the above libraries we automatically 
proved \textit{totality} and \textit{termination} of 
all interface functions. 
%
Even though most of Haskell's features facilitate verification, 
lazy semantics rendered standard refinement typing unsound. 

% Laziness
% The state: refinement types have been used for eager
% Eager: binders to terminating values
% Partial correctness: 
\textbf{Soundness Under Lazy Evaluation} (Chapter~\ref{chapter:refinedhaskell}) describes
how we adjusted refinement typing to soundly 
verify Haskell's lazy programs. 
% 
Refinement types were introduced in 1991 % ~\cite{freeman}
and since then have been successfully applied to many eager languages.
%
When checking an expression, such type systems
implicitly assume that all the free variables in the expression are
bound to values. 
%
This property is trivially guaranteed by eager evaluation, but 
does not hold in a lazy setting.
% 
Thus, to be sound and precise,
a refinement type system for Haskell 
must take into account which subset of binders
actually reduces to values. 
%
To track potentially diverging binders, 
we built a termination checker 
whose correctness is recursively checked by refinement types.


%% 
\textbf{Automatic Verification} comes by constraining refinements in specifications 
to decidable logics. 
Program verification checks that the source code satisfies 
a set of specifications. 
%
A trivial example is to \textit{specify} that the second argument of a division operator 
is different than zero, 
by writing the following specification:
@div :: Int -> NonZero -> Int@.
To \textit{check} whether an expression with type 
@{v:Int | 0 < v}@
is a safe argument to the division operator, the system checks 
whether $ 0 < v $ implies $0 \neq v$. 
%
By constraining all predicates to be drawn from decidable logics, 
such implications can be \textit{automatically} checked via an % SMT
Satisfiability Modulo Theories (SMT) 
solver.
%
\textit{Liquid Types}~\cite{LiquidPLDI09} are a subset of refinement types 
that achieve automation and type inference by 
constraining the language of the logical predicates to quantifier-free \textit{decidable} logics, including 
logical formulas, linear arithmetic and uninterpreted functions.
%

\textbf{Expressiveness} of the specifications is critically hindered 
by our choice to constrain the language of predicates to decidable logics.
%
Liquid types specifications are naturally used to describe first order properties
but prevent modular, higher order specifications. 
%
Consider a function that sorts lists of integers, with type 
@sort :: [Int] -> [Int]@.
%
Using \toolname we can specify that 
sorting positive numbers returns a list of positive numbers,
but we cannot give a modular specification 
accounting for \textit{all} different kinds of numbers
@sort@ will be invoked. 
%
We developed ``Abstract'' and ``Bounded'' refinement types to allow for 
modular specifications while preserving SMT decidability. 

In \textbf{Abstract Refinement Types}~(Chapter~\ref{chapter:abstractrefinements})
we parameterize a type over its refinements
allowing modular specifications while preserving SMT-based decidable type checking. 
%
As an example, 
since @sort@ preserves the elements of the input list,
we can use abstract refinements to specify
that \textit{for every} refinement $p$ on integers, 
@sort@ takes a list of integers that satisfy $p$ and returns a list of integers that satisfy
the same refinement $p$.
\begin{code}
  sort :: forall <p :: Int -> Bool>. [{v:Int | p v}] -> [{v:Int | p v}]
\end{code}
%
With this modular specification, 
we can prove that @sort@ preserves the property that all the input numbers satisfy, 
for any property, ranging from being positive numbers to 
being numbers that are safe keys for a security protocol.
%
We used abstract refinements to describe modular properties of recursive data structures. 
With such abstractions we simultaneously 
reasoned about challenging invariants such as 
\textit{sortedness and uniqueness of lists} or 
preservation of \textit{red-black invariants or heap properties} on trees. 
%
Without abstract refinements reasoning about each of these invariants would 
require a special purpose analysis. 
%
Crucially, abstractions over refinements 
preserve SMT-based decidability, simply by encoding 
refinement parameters as uninterpreted propositions within the ground
refinement logic.

\textbf{Bounded Refinement Types}~(Chapter~\ref{boundedrefinements})
constrain and relate abstract refinement and let us express even more interesting invariants
while preserving SMT-decidability.
%
As an example, we used bounds on refinement types to 
reason about \textit{stateful computations}.
We expressed the pre- and post-conditions of the computations with 
two abstract refinements, $p$ and $q$ respectively and used bounds to impose constraints upon them. 
For instance, when sequencing two computation we bound the first post-condition $q_1$ to imply 
the second pre-condition $p_2$.
%
%%As a trivial example, we can bound the abstract refinement $p$ of \maximum 
%%to only refer to refinements satisfying \textit{some} constraint, 
%%say that $p$ is \textit{upward closed} satisfying the property $\forall x.\ p\ x \Rightarrow p\ (x+1)$.
%%%and use it to constraint the type of \maximum only to refinements that are upward closed, like positive %%%As an example, we can bound the abstract refinement $p$ of \maximum 
%%%to only refer to refinements satisfying \textit{some} constraint, 
%%%say that $p$ is \textit{upward closed}.
%%%% 
%%%We define a \textit{bound} \bupclosed 
%%%$$ \texttt{bound}\ \bupclosed\ p = \forall x.\ p\ x \Rightarrow p\ (x+1)$$
%%%and use it to constraint the type of \maximum only to refinements that are upward closed, like positive numbers
%%%$$\maximum :: \forall p :: \tint \rightarrow \texttt{Bool}. \
%%%              (\bupclosed\ p) \Rightarrow
%%%              \texttt{List}\ \{ v:\tint\ |\ p\ v\} \rightarrow \{ v:\tint\ |\ p\ v\} $$
%%%%
%%%Bounded quantification expands the expressiveness
%%%of refinement typing by allowing us to 
%%%both constrain but also 
%%%define relations or transformations over the abstracted refinements.
%
We implemented the above idea in a refined Haskell $\texttt{IO}$ state monad that 
encodes \textit{Floyd-Hoare logic state transformations}
and used this encoding to track capabilities and resource usage. 
%%
Moreover, we encoded \textit{safe database access} using 
abstract refinements to encode key-value properties 
and bounds to express the constraints imposed by 
relational algebra operators, like disjointedness, union \etc. 
%
Bounds are internally translated to ``ghost'' functions, 
thus the increased expressiveness comes while preserving the automated
and decidable SMT-based type checking 
that makes liquid typing effective in practice.
%
Abstract and Bounded refinement types 
do allow modular higher order specifications, 
but the expressiveness of the specifications is crucially 
restricted by the fact that, 
for automatic verification, arbitrary, Haskell functions are not allowed to 
appear in the refinements.

\textbf{Refinement Reflection}~(Chapter~\ref{refinementrflection})
allows arbitrary, terminating, Haskell functions to 
appear into the specifications 
as uninterpreted functions thus preserving automatic and decidable 
type checking. 
%
The key idea is to reflect the ​code implementing a​
user-defined function into the function's (output)
refinement type.
%
As a consequence, at \emph{uses} of the function,
the function definition is unfolded into the refinement logic
in a precise and predictable manner.
%
With Refinement Reflection, 
the user can write arbitrarily expressive (fully dependent type)
specifications expressing theorems about the code, 
but to prove such theorems the user needs to manually provide 
appropriate proof terms. 
%
We used reflection to verify
that many widely used instances of the Monoid,
Applicative, Functor and Monad typeclasses 
satisfy key algebraic laws needed to
making the code using the typeclasses safe.
%
Finally, transforming a mature language---with
highly tuned parallel runtime---into a theorem
prover  enables us to build the parallel applications, 
like an efficient String Matcher (Chapter~\ref{stringmatcher}), 
and prove it equivalent with its na\"ive, sequential version.

In short, \toolname is a usable verifier for real world Haskell applications 
as it allows for natural integration of expressive, type based specifications 
that can be automatically verified using SMT solvers. 
