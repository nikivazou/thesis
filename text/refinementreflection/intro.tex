%%\begin{comment}
%%
%%Dependent types: Coq and Adga fully dependent types,
%%
%%Utopia: extends liquidHaskell to fully dependent types while preserving decidable and predictable type cheking.
%%
%%\begin{itemize}
%%\item \textit{Curry–Howard correspondence} into action!
%%Theorems are refinement types, proofs are plain Haskell programs.
%%\item Type directed proofs: if it (Haskell) type checks, your proof is (most probably) correct!
%%\item Type directed proofs: if it (liquid) type checks, your proof is provably correct!
%%\end{itemize}
%%\end{comment}

\section{Intoduction}\label{sec:intro}

Is decidable and predictable dependent
type programming in a general purpose
programming language utopic?
%
Traditional dependent type
languages, like Coq~\citep{coq-book}
and Adga~\citep{agda},
have been extensively used for
program verification, but they
are not typically used for
general-purpose programming
as they disallow key features
like non-termination, exceptions
and IO.
%
Such features are embedded in verification oriented
general purpose languages, such as
Dafny~\citep{dafny} and \fstar~\citep{fstar}
that allow for semi-automated, SMT based verification.
%

Verification in such languages is
unpredictable~\citep{Leino16}.
Programs are represented as a
logical model in the SMT logic,
and verification relies on SMT
heuristics that an instable.
%
In between, \liquidHaskell restricts
specification languages to
decidable logics ensuring
predictable verification in
the general-purpose language
Haskell, in the cost of low
expressiveness in specifications.

We present \libname,
an extension to Liquid Types that allows
interpretation of terminating program functions
while preserving predictable and decidable type checking.
%
The common route~\citep{dafny, fstar, HALO} for program function
interpretation is to explicitly model function's behavior as an SMT axiom.
Axiom instantiation relies on SMT heuristics that are impredictable,
leading to the ``butterfly effect'' where
modification in a part of the program changes the outcome of verification in other part.
%
Instead of creating SMT axioms, \libname captures the behavior of functions
in function's return type.
%
That way, implicit axioms instantiations occur at each function invocation
and are fully controlled by the programmer.

\mypara{Function Axiomatization} into the SMT logic has been
commonly used to model program functions in SMT-based program
verification. As an example, consider the fibonacci function
%
\begin{code}
  -- I am a comment fib :: Int -> Int
  fib n
    | n == 0    = 0
    | n == 1    = 1
    | otherwise = fib (n-1) + fib (n-2)
\end{code}

An interpretation of @fib@ function can be a logical axiom that exactly captures its behavior:
$$
\begin{array}{rll}
\forall n. \fib{n} = &\texttt{if}\ n = 0 & \texttt{then}\ 0\\
& \texttt{else if}\ n = 1 & \texttt{then}\ 1\\
& \texttt{else}\ & \fib{(n-1)} + \fib{(n-2)}\\
\end{array}
$$

\mypara{Instantiation} of such axioms is problematic as can lead to
matching loops and unpredictable bahaviors.
%
Consider that the existence of the term \fib{n}
triggers the above axiom during verification.
Then, the instantiation $x := \fib{n}$ will produce the terms
\fib{(n-1)} and \fib{(n-2)}, which give rise to other possible instantiations.
%
To prevent matching loops a `fuel'' parameter~\citep{Amin2014ComputingWA}
can be used to restrict the numbers of the SMT
unrollings of each function.


\liquidHaskell followed a simpler approach to prevent matching loops
and keep verification decidable.
%
The implementation of recursive functions was not encoded at all in the logic.
%
Worse, program functions were (syntactically) not allowed
to appear in the specifications.

\libname extends \liquidHaskell allowing interpretation of terminating
program functions while preserving predictability in three steps.
%
The user marks with @axiomatize@
the program functions she wants to refer to in the specifications, sat @f@.

Then, \libname will
\begin{enumerate}
\item Create a logical uninterpreted function with the same name @f@.
\item Automatically strengthen the result type of @f@'s to exactly capture @f@'s behavior.
\item Unfold the function's definition once into the logic, at each program function invocation.
\end{enumerate}

As an example, say that the user wants to reason about the fibonacci function.
Since @fib@ is provably terminating, the user can ``axiomatize'' it.
\begin{code}
  axiomatize fib
\end{code}

\mypara{Step 1: Definition of} @fib@ \textbf{\emph{in the logic.}}
Initially, the axiomatization annotation declares an uninterpreted function @fib@ in the logic
\begin{code}
  fib :: Int -> Int
\end{code}
%
The logical @fib@ is not directly connected to the program function @fib@.
As far as the logic is concerned, @fib@ only satisfies the congruence axiom
%
$$\forall n, m. n = m \Rightarrow \fib{n} = \fib{m}$$
%

\mypara{Step 2: Connecting logical and program} @fib@ \textbf{\emph{via liquid types.}}
Liquid types are Haskell types refined with logical predicates.
As an example, we define the Liquid Type that describes natural numbers
as values @v@ that are integers and moreover satisfy the predicate @0<=v@:
\begin{code}
  type Nat = {v:Int | 0 <= v}
\end{code}
%
The user can use the @Nat@ Liquid Type alias to specify that @fib@ maps
natural numbers to natural numbers
\begin{code}
  fib :: Nat -> Nat
\end{code}
%
Axiomatization of @fib@  \textit{automatically} strengthens the user specified type for @fib@ to
exactly capture @fib@'s behavior via the homonymous uninterpreted function:
\begin{code}
  fib :: n:Nat
      -> {v:Nat |
          v == if n == 0 then 0
               else if n == 1 then 1
               else fib (n-1) + fib (n-2)
          }
\end{code}
%
With this type each invocation of the program @fib@ function automatically
unfolds the @fib@ definition \textit{once} in the logic.

\mypara{Step 3: Function invocation as logical unfolding.}
When the user invokes @fib@ with, say @0@, in @v == fib 0@,
it can be verified that both @v == fib 0@ and @v == 0@.
Thus, the @fib@ function is unfolded once in the logic concluding that
$\fib{0} = 0$.

As an example, by invoking @fib@ on both @2@, @1@ and @0@, we can verify
that @fib 2 == 1@
\begin{code}
  fib2_safe :: {fib 2 == 1}
  fib2_safe = fib 2 == fib 1 + fib 0
\end{code}
Where we use @{p}@ $\defeq$ @{v:t | p}@ when @t@ is obvious from the context
and @v@ does not appear on @p@.

Note that verification on @safe@ relies on the invocations of @fib@
but not on the way these invocations are combined.
%
SMT-based verifies that directly axiomatize program functions do not require such invocations.
For example, \fstar and Dafny can verify the following @fib2@ specification that fails in
\liquidHaskell
%
\begin{code}
  fib2_unsafe :: {fib 2 == 1}
  fib2_unsafe = ()
\end{code}
%
Yet, verification of @fib2_safe@ requires invoking @fib@ on @2@, @1@, and @0@
so that the @fib@ function is unfolded exactly once on these arguments.
%
But, the combination of these function calls in not important.
For example the following term suffices to prove the equality @fib 2 == 2@
\begin{code}
  fib2_safe_tuple :: {fib 2 == 1}
  fib2_safe_tuple = (fib 2, fib 1, fib 0)
\end{code}
%
Even though @fib2_safe@ provides the same information as
@fib2_ safe_tuple@,
it has a better structure regarding the proof it provides.
%
We implemented the library \libname, that smoothly complements the
current extensions in \liquidHaskell, by providing operators that
intuitively structure proofs.

In this paper we present how we extended \liquidHaskell
to allow for arbitrary terminating expressions appear in the specifications.
%
This extension allows the user to specify theorems about program functions
using Liquid Types and prove such theorems using Haskell Code,
instantiating Curry–Howard correspondence.
\begin{itemize}
\item First, we describe the library \libname,
a Haskell library that structures ``axiomatized'' Haskell's functions invocations
into readable, type directed proofs that are checked using \liquidHaskell
(\S~\ref{sec:library}).
%
\item Then, we use \libname to state and prove theorems theorems
on integers (\eg fibonacci function is increasing)
and inductive lists (\eg associativity of append, map fusion, and monad laws)
%
(\S~\ref{sec:examples}).
\item Next, we formalize our system by showing that
the presented extension is sound with respect to denotational semantics
%
(\S~\ref{sec:formalism})
and we present a transformation from a core calculus to
the SMT logic that via lambda lifting and defunctionalization
preserves decidable type checking
of higher order specifications (\S~\ref{sec:algorithmic}).
%
In present of lambda expressions in the specifications,
our translation can (optionally) insert extensionality axioms
that express $\alpha$- and $\beta$-equivalence in the logic,
but do not preserve undecidability and thus unpredictability of type checking.

\item Finally, we evaluated our extension by verification of
arithmetic properties of the Ackermann functions,
class laws on inductive data types,
folding properties,
and by combining proof terms on (potentially diverging)
Haskell programs (\S~\ref{sec:evaluation}).
\end{itemize}
