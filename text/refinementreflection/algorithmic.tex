\section{Algorithmic Verification}\label{sec:algorithmic}

Next, we describe \smtlan, a conservative approximation
of \corelan where the undecidable type subsumption rule
is replaced with a decidable one, yielding an SMT-based
algorithmic type system that enjoys the same soundness
guarantees.

\subsection{The SMT logic \smtlan}

\begin{figure}[t!]
\vspace{-5mm}
\centering
$$
\begin{array}{rrcl}
\emphbf{Predicates} 
  & \pred & ::= &
    \pred \binop \pred \spmid
    \unop \pred \\
  && \spmid & n \spmid b \spmid x \spmid \dc \spmid  x\ \overline{\pred}\\
%%  && \spmid & \forall \overline{\tbind{x}{\sort}}. \pred
  && \spmid & \eif{\pred}{\pred}{\pred}
\\[0.03in]

\emphbf{Integers} 
  & n
  & ::= & 0, -1, 1, \dots
\\[0.03in]

\emphbf{Booleans} 
  & b
  & ::= & \etrue \spmid \efalse
\\[0.03in]

\emphbf{Bin Operators} 
  & \binop
  & ::= & = \spmid < \spmid \land \spmid + \spmid - \spmid \dots
\\[0.03in]

\emphbf{Un Operators} 
  & \unop
  & ::= & \lnot \spmid \dots 
\\[0.03in]

\emphbf{Sort Args} 
  & \sort_a
  & ::= & \tint \spmid \tbool \spmid \tuniv 
         \spmid \tsmtfun{\sort_a}{\sort_a}
\\[0.03in]
\emphbf{Sorts} 
  & \sort
  & ::=  & \sort_a \rightarrow \sort
\end{array}
$$
\caption{\textbf{Syntax of \smtlan}.}
\label{fig:smtsyntax}
\vspace{-2mm}
\end{figure}


\mypara{Syntax: Terms \& Sorts}
%
Figure~\ref{fig:smtsyntax} summarizes the syntax
of \smtlan, the \emph{sorted} (SMT-)
decidable logic of quantifier-free equality,
uninterpreted functions and linear
arithmetic (QF-EUFLIA) ~\citep{Nelson81,SMTLIB2}.
%
The \emph{terms} of \smtlan include
integers $n$,
booleans $b$,
variables $x$,
data constructors $\dc$ (encoded as constants),
fully applied unary \unop and binary \binop operators,
and application $x\ \overline{\pred}$ of an uninterpreted function $x$.
%
The \emph{sorts} of \smtlan include built-in
integer \tint and \tbool for representing
integers and booleans.
%
%% NV reflected functions and measures are first order
%% NV because
%% NV 1. they can be partially applied
%% NV 2. they can be passed as arguments
The interpreted functions of \smtlan, \eg
the logical constants $=$ and $<$,
%% NV and the uninterpreted functions app and lam
%% NV but we have not introduced these yet
have the function sort $\sort \rightarrow \sort$.
%
Other functional values in \corelan, \eg
reflected \corelan functions and
$\lambda$-expressions, are represented as
first-order values with
uninterpreted sort \tsmtfun{\sort}{\sort}.
%
%%The uninterpreted functions of \smtlan, which
%%correspond to reflected \corelan functions,
%%have the function sort $\sort \rightarrow \sort$.
%%%
%%Other functional values in \corelan, \eg
%%$\lambda$-expressions, are represented as
%%first-order values in \smtlan with
%%uninterpreted sort \tsmtfun{\sort}{\sort}.
%%%
The universal sort \tuniv represents all other values.

\mypara{Semantics: Satisfaction \& Validity}
%
An assignment $\sigma$ is a mapping from
variables to terms
%
${\sigma \defeq \{ \assignto{x_1}{\pred_1}, \ldots, \assignto{x_n}{\pred_n} \}}$.
%
We write
%
${\sigma \models \pred}$
%
if the assignment $\sigma$ is a
\emph{model of} $\pred$, intuitively
if $\sigma\ \pred$ ``is true''~\cite{Nelson81}.
%
A predicate $\pred$ \emph{is satisfiable} if
there exists ${\sigma\models\pred}$.
%
A predicate $\pred$ \emph{is valid} if
for all assignments ${\sigma\models\pred}$.


\subsection{Transforming \corelan into \smtlan}
%
\label{subsec:embedding}

\newcommand\emptyaxioms{\ensuremath{\emptyset}\xspace}
\newcommand\andaxioms[2]{\ensuremath{{#1}\cup {#2}}\xspace}

\begin{figure}
\emphbf{Transformation}\hfill{\fbox{\tologicshort{\Gamma}{e}{\typ}{\pred}{\sort}{\smtenv}{\axioms}}}
$$
\inference{
}{
	\tologicshort{\env}{b}{\tbool}{b}{\tbool}{\emptyset}{\emptyaxioms}
}[\lgbool]
\qquad
\inference{
}{
	\tologicshort{\env}{n}{\tint}{n}{\tint}{\emptyset}{\emptyaxioms}
}[\lgint]
$$

$$
\inference{
    \tologicshort{\env}{e_1}{\typ}{\pred_1}{\embed{\typ}}{\smtenv}{\axioms_1} &
    \tologicshort{\env}{e_2}{\typ}{\pred_2}{\embed{\typ}}{\smtenv}{\axioms_2}
}{
	\tologicshort{\env}{e_1\binop e_2}{\tbool}{\pred_1 \binop\pred_2}{\tbool}{\smtenv}{\andaxioms{\axioms_1}{\axioms_2}}
}[\lgbinGEN]
$$

$$
\inference{
	\tologicshort{\env}{e}{\tbool}{\pred}{\tbool}{\smtenv}{\axioms}
}{
	\tologicshort{\env}{\unop e}{\tbool}{\unop\pred}{\tbool}{\smtenv}{\axioms}
}[\lgun]
\qquad
\inference{
}{
	\tologicshort{\env}{x}{\env(x)}{x}{\embed{\env(x)}}{\emptyset}{\emptyaxioms}
}[\lgvar]
$$

$$
\inference{
}{
	\tologicshort{\env}{c}{\constty{\odot}}{\smtvar{c}}{\embed{\constty{\odot}}}{\emptyset}{\emptyaxioms}
}[\lgpop]
\qquad
\inference{
}{
	\tologicshort{\env}{\dc}{\constty{\dc}}{\smtvar{\dc}}{\embed{\constty{\dc}}}{\emptyset}{\emptyaxioms}
}[\lgdc]
$$


%%$$
%%\inference{
%%  	\axioms_{f_1} = \forall \tbind{x}{\sort_x}.\smtappname{\sort_x}{\sort}\ f\ x = \pred \\
%%  	\axioms_{f_2} = \forall \tbind{g}{\sort'},\tbind{x}{\sort_x}.
%%  	\smtappname{\sort_x}{\sort}\ f\ x = \smtappname{\sort_x}{\sort}\ g\ x \Rightarrow f = g \\
%% 	f\ \text{fresh} &
%% 	\sort' = \embed{\tfun{x}{\typ_x}{\typ}} &
%% 	\sort  = \embed{\typ} &
%% 	\sort_x = \embed{\typ_x} \\
%% 	\tologicshort{\env,\tbind{x}{\typ_x}}{e}{\typ}{\pred}{\sort}{\smtenv, \tbind{x}{\sort_x}}{\axioms} &
%% 	\hastype{\env}{(\efun{x}{}{e})}{(\tfun{x}{\typ_x}{\typ})}\\
%%}{
%%	\tologicshort{\env}{\efun{x}{}{e}}{(\tfun{x}{\typ_x}{\typ})}
%%	        {f}{\sort'}{\smtenv, \tbind{f}{\sort'}}{\andaxioms{\{\axioms_{f_1}, \axioms_{f_2}\}}{\axioms}}
%%}[\lgfun]
%%$$

$$
\inference{
    \tologicshort{\env, \tbind{x}{\typ_x}}{e}{}{\pred}{}{}{} &
  	\hastype{\env}{(\efun{x}{}{e})}{(\tfun{x}{\typ_x}{\typ})}\\
}{
	\tologicshort{\env}{\efun{x}{}{e}}{(\tfun{x}{\typ_x}{\typ})}
	        {\smtlamname{\embed{\typ_x}}{\embed{\typ}}\ {x}\ {\pred}}
	        {\sort'}{\smtenv, \tbind{f}{\sort'}}{\andaxioms{\{\axioms_{f_1}, \axioms_{f_2}\}}{\axioms}}
}[\lgfun]
$$

$$
\inference{
	\tologicshort{\env}{e'}{\typ_x}{\pred'}{\embed{\typ_x}}{\smtenv}{\axioms'}
	&
	\tologicshort{\env}{e}{\tfun{x}{\typ_x}{\typ}}{\pred}{\tsmtfun{\embed{\typ_x}}{\embed{\typ}}}{\smtenv}{\axioms}
	& 
	\hastype{\env}{e}{{\typ_x}\rightarrow{\typ}}
}{
	\tologicshort{\env}{e\ e'}{\typ}{\smtappname{\embed{\typ_x}}{\embed{\typ}}\ {\pred}\ {\pred'}}{\embed{\typ}}{\smtenv}{\andaxioms{\axioms}{\axioms'}}
}[\lgapp]
$$


$$
\inference{
	\tologicshort{\env}{e}{\tbool}{\pred}{\tbool}{\smtenv}{\axioms} & 
	\tologicshort{\env}{e_i\subst{x}{e}}{\typ}{\pred_i}{\embed{\typ}}{\smtenv}{\axioms_i}
}{
	\tologicshorttwolines{\env}{\ecaseexp{x}{e}{\etrue \rightarrow e_1; \efalse \rightarrow e_2}}{\typ}
	 {\eif{\pred}{\pred_1}{\pred_2}}{\embed{\typ}}{\smtenv}{\andaxioms{\axioms}{\axioms_i}}
}[\lgcaseBool]
$$

$$
\inference{
	\tologicshort{\env}{e}{\typ_e}{\pred}{\embed{\typ_e}}{\smtenv}{\axioms}\\
	\tologicshort{\env}{e_i\subst{\overline{y_i}}{\overline{\selector{\dc_i}{}\ x}}\subst{x}{e}}{\typ}{\pred_i}{\embed{\typ}}{\smtenv}{\axioms_i}
}{
	\tologicshorttwolines{\env}{\ecase{x}{e}{\dc_i}{\overline{y_i}}{e_i}}{\typ}
	 {\eif{\smtappname{}{}\ \checkdc{\dc_1}\ \pred}{\pred_1}{\ldots} \ \mathtt{else}\ \pred_n}{\embed{\typ}}{\smtenv}
	 {\andaxioms{\axioms}{\axioms_i}}
}[\lgcase]
$$
\caption{\textbf{Transforming \corelan terms into \smtlan.}}
\label{fig:defunc}
\end{figure}

%
The judgment
\tologicshort{\env}{e}{\typ}{\pred}{\sort}{\smtenv}{\axioms}
states that a $\corelan$ term $e$ is transformed,
under an environment $\env$, into a
$\smtlan$ term $\pred$.
%
The transformation rules are summarized in Figure~\ref{fig:defunc}.

\mypara{Embedding Types}
%
We embed \corelan types into \smtlan sorts as:
%
$$
\begin{array}{rclcrcl}
\embed{\tint}                       & \defeq &  \tint &  &
\embed{T}                           & \defeq &  \tuniv \\
\embed{\tbool}                      & \defeq &  \tbool & &
\embed{\tfun{x}{\typ_x}{\typ}} & \defeq & \tsmtfun{\embed{\typ_x}}{\embed{\typ}}
\end{array}
$$
%%%The embedding extends to typing environments:
%%%% by embedding the types of the environment
%%%$$
%%%\embedsort{\{\tbind{x_1}{\typ_1}, \dots, \tbind{x_n}{\typ_n}\}}
%%%  \defeq
%%%  \{\tbind{x_1}{\embed{\typ_1}}, \dots, \tbind{x_n}{\embed{\typ_n}}
%%%  \}
%%%$$

\mypara{Embedding Constants}
%
Elements shared on both \corelan and \smtlan
translate to themselves.
%
These elements include
booleans (\lgbool),
integers (\lgint),
variables (\lgvar),
binary (\lgbinGEN)
and unary (\lgun)
operators.
%
SMT solvers do not support currying,
and so in \smtlan, all function symbols
must be fully applied.
%
Thus, we assume that all applications
to primitive constants and data
constructors are \emph{saturated},
%% NV eta converted
\ie fully applied, \eg by converting
source level terms like @(+ 1)@ to
@(\z -> z + 1)@.
%

%%% Thus, to translate \corelan's partially applied operators,
%%% we define an uninterpreted function
%%% $$
%%% \tbind{\smtvar{c}}{\embed{\constty{c}}}
%%% $$
%%% for every functional constant $c$ in \corelan.
%%% %
%%% For example, $+ 1$ will be translated to application of $\smtvar{+}$ to $1$, while
%%% $1+2$ will be translated to the identical $1+2$.

%%\spara{Lambda Lifting}
%%%
%%Since \smtlan does not support $\lambda$-functions.
%%the translation lifts function to axiomatized variables.
%%%
%%Rule~\lgfun
%%translates the term $\efun{x}{\typ}{e}$ to
%%a fresh variable $f$ that satisfies two axioms:
%%(1). $\beta$-reduction,
%%that is $f$ applied to $x$ is equal to $e$, and
%%(2). extentionality,
%%that is for every other function $g$ and argument $x$,
%%if $f$ applied to $x$ is equal to $g$ applied to $x$,
%%then $f = g$.

\mypara{Embedding Functions}
%
As \smtlan is a first-order logic, we
embed $\lambda$-abstraction and application
using the uninterpreted functions
\smtlamname{}{} and \smtappname{}{}.
%
We embed $\lambda$-abstractions
using $\smtlamname{}{}$ as shown in rule~\lgfun.
%
The term $\efun{x}{}{e}$ of type
${\typ_x \rightarrow \typ}$ is transformed
to
${\smtlamname{\sort_x}{\sort}\ x\ \pred}$
of sort
${\tsmtfun{\sort_x}{\sort}}$, where
%
$\sort_x$ and $\sort$ are respectively
$\embed{\typ_x}$ and $\embed{\typ}$,
%
${\smtlamname{\sort_x}{\sort}}$
is a special uninterpreted function
of sort
${\sort_x \rightarrow \sort\rightarrow\tsmtfun{\sort_x}{\sort}}$,
and
$x$ of sort $\sort_x$ and $r$ of sort $\sort$ are
the embedding of the binder and body, respectively.
%
As $\smtlamname{}{}$ is just an SMT-function,
it \emph{does not} create a binding for $x$.
%
Instead, the binder $x$ is renamed to
a \emph{fresh} name pre-declared in
the SMT environment.


\mypara{Embedding Applications}
%
Dually, we embed applications via
defunctionalization~\citep{Reynolds72}
using an uninterpreted \emph{apply}
function
$\smtappname{}{}$ as shown in rule~\lgapp.
%
The term ${e\ e'}$, where $e$ and $e'$ have
types ${\typ_x \rightarrow \typ}$ and $\typ_x$,
is transformed to
${\tbind{\smtappname{\sort_x}{\sort}\ \pred\ \pred'}{\sort}}$
where
%
$\sort$ and $\sort_x$ are respectively $\embed{\typ}$ and $\embed{\typ_x}$,
the
${\smtappname{\sort_x}{\sort}}$
is a special uninterpreted function of sort
${\tsmtfun{\sort_x}{\sort} \rightarrow \sort_x \rightarrow \sort}$,
and
$\pred$ and $\pred'$ are the respective translations of $e$ and $e'$.


\mypara{Embedding Data Types}
%
Rule~\lgdc translates each data constructor to a
predefined \smtlan constant ${\smtvar{\dc}}$ of
sort ${\embed{\constty{\dc}}}$.
%
Let $\dc_i$ be a non-boolean data constructor such that
$$
\constty{\dc_i} \defeq \typ_{i,1} \rightarrow \dots \rightarrow \typ_{i,n} \rightarrow \typ
$$
Then the \emph{check function}
${\checkdc{{\dc_i}}}$ has the sort
$\tsmtfun{\embed{\typ}}{\tbool}$,
and the \emph{select function}
${\selector{\dc}{i,j}}$ has the sort
$\tsmtfun{\embed{\typ}}{\embed{\typ_{i,j}}}$.
%
Rule~\lgcase translates case-expressions
of \corelan into nested $\mathtt{if}$
terms in \smtlan, by using the check
functions in the guards, and the
select functions for the binders
of each case.
%
%\mypara{Reflecting DataTypes}
%
% The above approach  makes it straightforward
% to reflect functions over datatypes into \smtlan.
%
For example, following the above, the body of the list append function
%
%%% reflect (++) :: xs:[Int] -> ys:[Int] -> [Int]
\begin{code}
  []     ++ ys = ys
  (x:xs) ++ ys = x : (xs ++ ys)
\end{code}
%
is reflected into the \smtlan refinement:
%
$$
\ite{\mathtt{isNil}\ \mathit{xs}}
    {\mathit{ys}}
    {\mathtt{sel1}\ \mathit{xs}\
       \dcons\
       (\mathtt{sel2}\ \mathit{xs} \ \mathtt{++}\  \mathit{ys})}
$$
%
We favor selectors to the axiomatic translation of
HALO~\citep{halo} and \fstar~\cite{fstar} to avoid
universally quantified formulas and the resulting
instantiation unpredictability.

%% $$
%% \tbind{\checkdc{\dc}}{\embed{\typ \rightarrow \tbool}}
%% \ \text{with}\ \constty{\dc} = \typ_1 \rightarrow \dots \rightarrow \typ_n\rightarrow\typ
%% $$
%% and the field selector is used to substitute the data constructor quantified variables $\overline{y_i}$:
%% eg. if \dc is [] then i == 0
%%     if \dc is (:) :: a -> [a] -> [a] then
%%         \dc_1 = head :: [a] -> a
%%         \dc_2 = tail :: [a] -> [a]
%% $$
%% \tbind
      %% {\embed{\typ \rightarrow \typ_i}}
%% \ \text{with}\ \constty{\dc} = \typ_1 \rightarrow \dots \rightarrow \typ_n\rightarrow\typ, i \leq n
%% $$
%% %
%% For example, the body of the @length@ function from~\S~\ref{sec:examples}
%% translates to the condition $\eif{\isN\ xs}{0}{1+\texttt{length} (\etail\ xs)}$,
%% as $\etail \defeq \selector{\dcons}{2}$.

\subsection{Correctness of Translation}

Informally, the translation relation $\tologicshort{\env}{e}{}{\pred}{}{}{}$
is correct in the sense that if $e$ is a terminating boolean expression
then $e$ reduces to \etrue \textit{iff} $\pred$ is SMT-satisfiable
by a model that respects $\beta$-equivalence.

%%\mypara{Type Preservation}
%%%
%%The \emph{initial environment} \smtenvinit
%%maps the uninterpreted symbols used
%%by the translation, namely
%%%
%%$\smtlamname{}{}$,
%%$\smtappname{}{}$,
%%$\smtvar{\dc}$,
%%$\checkdc{{\dc_i}}$,
%%$\selector{\dc}{{i,j}}$
%%and fresh binder names $x$ used  in $\smtlamname{}{}$
%%to their respective sorts.
%%%
%%The judgment $\smthastype{\smtenv}{\pred}{\sort}$ states
%%that the term $\pred$ has sort $\sort$ in environment
%%$\smtenv$. (We omit the standard derivation rules
%%for brevity.)
%%%
%%The translation is type (sort) preserving.
%%
%%\begin{lemma}
%%%  [Type Transformation]
%%If \tologicshort{\env}{e}{\typ}{p}{\sort}{\smtenv}{\axioms},
%%and \hastype{\env}{e}{\typ}, then
%%\smthastype{\smtenvinit, \embedsort{\env}}{p}{\embed{\typ}}.
%%\end{lemma}
%%
%%% are defined in the
%%% %
%%% Thus, \smtenvinit includes
%%% $$
%%% \begin{array}{rcll}
%%% \smtvar{c}  &\colon &\embed{\constty{c}}
  %%% &\forall c\in \corelan\\
%%% \smtlamname{\sort_x}{\sort}&\colon&\sort_x \rightarrow \sort\rightarrow\tsmtfun{\sort_x}{\sort}
  %%% &\forall \sort_x, \sort\in \smtlan\\
%%% \smtappname{\sort_x}{\sort}&\colon&\tsmtfun{\sort_x}{\sort} \rightarrow \sort_x \rightarrow \sort
  %%% &\forall \sort_x, \sort\in \smtlan\\
%%% \smtvar{\dc}&\colon&\embed{\constty{\dc}}
  %%% &\forall\dc\in\corelan\\
%%% \checkdc{\dc}&\colon&\embed{T \rightarrow \tbool}
  %%% &\forall \dc\in \corelan\ \text{of data type}\ T \\
%%% \selector{\dc}{i}&\colon&\embed{T \rightarrow \typ_i}
  %%% &\forall \dc\in \corelan\ \text{of data type}\ T \\
  %%% &&&\text{and}\ i\text{-th argument}\ \typ_i \\
%%% {x} & \colon&{\sort}&\text{for each lambda argument} \\
%%% \end{array}
%%% $$


% \mypara{Lifted Substitutions}


%
%% as defined
%% in~\citep{Vazou15} remove bottoms from expressions
%% in substitutions and translate via
%% \tologic{\emptyset}{\star}{}{\star}{}{}{}
%% to a set of models $\sigma \in \theta^\perp$
%% where each bottom maps to
%% %
%%
%% Such models $\sigma \in \theta^\perp$
%% map variables in $\theta$ to values
%% in the logic, without providing
%% interpretations for the
%% $\smtlamname{}{}$ and $\smtappname{}{}$.

\NV{below we use substitution in lambda s which is not formally defined}
%
\begin{definition}[$\beta$-Model]\label{def:beta-model}
A $\beta-$model $\bmodel$ is an extension of a model $\sigma$
where $\smtlamname{}{}$ and $\smtappname{}{}$
satisfy the axioms of $\beta$-equivalence:
$$
\begin{array}{rcl}
\forall x\ y\ e. \smtlamname{}{}\ x\ e
  & = & \smtlamname{}{}\ y\ (e\subst{x}{y}) \\
\forall x\ e_x\ e. (\smtappname{}{}\ (\smtlamname{}{}\ x\ e)\ e_x
  & = &  e\subst{x}{e_x}
\end{array}
$$
\end{definition}

\mypara{Semantics Preservation}
%
We define the translation of a \corelan term
into \smtlan under the empty environment as
${\embed{e} \defeq \pred}$
if ${\tologicshort{\emptyset}{\refa}{}{\pred}{}{}{}}$.
%
A \emph{lifted substitution}
$\theta^\perp$ is a set of models $\sigma$
where each ``bottom'' in the substitution
$\theta$ is mapped to an arbitrary logical
value of the respective sort~\citep{Vazou14}.
%
We connect the semantics of \corelan and translated
\smtlan via the following theorems:
% terms can connect evaluation of boolean
% \corelan expression to \smtlan predicates.

\begin{theorem}\label{thm:embedding-general}
If ${\tologicshort{\env}{\refa}{}{\pred}{}{}{}}$,
then for every ${\sub\in\interp{\env}}$
and every ${\sigma\in {\sub^\perp}}$,
if $\evalsto{\applysub{\sub^\perp}{\refa}}{v}$
then $\sigma^\beta \models \pred = \embed{v}$.
\end{theorem}

% For Boolean expressions we specialize the above to

\begin{corollary}\label{thm:embedding}
If ${\hastype{\env}{\refa}{\tbool}}$, $e$ reduces to a value and
${\tologicshort{\env}{\refa}{\tbool}{\pred}{\tbool}{\smtenv}{\axioms}}$,
then for every ${\sub\in\interp{\env}}$
and every ${\sigma\in {\sub^\perp}}$,
$\evalsto{\applysub{\sub^\perp}{\refa}}{\etrue}$ iff
$\sigma^\beta \models \pred$.
\end{corollary}



\subsection{Decidable Type Checking}
\begin{figure}[t!]
\centering
$$
\begin{array}{rrcl}
\emphbf{Refined Types} \quad
  & \typ
  & ::=   & \tref{v}{\btyp^{[\tlabel]}}{\reft} \spmid \tfun{x}{\typ}{\typ}
\\[0.10in]
\end{array}
$$
\emphbf{Well Formedness}\hfill{\fbox{\aiswellformed{\env}{\typ}}}\\
$$
\inference{
  \ahastype{\env,\tbind{v}{\btyp}}{\refa}{\tbool^{\tlabel}}
}{
  \aiswellformed{\env}{\tref{v}{\btyp}{\refa}}
}[\rwbase]
$$
\emphbf{Subtyping}\hfill{\fbox{\aissubtype{\env}{\typ}{\typ'}}}\\
$$
\inference{
\env' \defeq \env,\tbind{v}{\{\btyp^\tlabel | \refa\}} &
\tologicshort{\env'}{\refa'}{\tbool}{\pred'}{}{}{} &
\smtvalid{\vcond{\env'}{\pred'}}
%
}{
 \aissubtype{\env}{\tref{v}{\btyp}{\refa}}{\tref{v}{\btyp}{\refa'}}
}[\rsubbase]
$$
%%%% %\NV{REVERT TO OLD DEFINITIONS, what is e'?}
%%%% $$
%%%% \inference{
%%%% \tologicshort{\env'}{\refa_1}{\tbool}{\pred_1}{\tbool}{\smtenv_1}{\axioms_1} &
%%%% \tologicshort{\env'}{\refa_2}{\tbool}{\pred_2}{\tbool}{\smtenv_1}{\axioms_1} \\
%%%% % \isvalid{\env,\tbind{v}{\btyp}}{\refa_1}{\refa_2}
%%%% \env' \defeq \env,\tbind{v}{\btyp^\tlabel} &
%%%% % \tologicshort{\env'}{\refa'}{\tbool}{\pred'}{\tbool}{\smtenv'}{\axioms'} &
%%%% \smtvalid{\vcond{\env'}{\pred_1 \Rightarrow \pred_2}}
%%%% %
%%%% }{
  %%%% \aissubtype{\env}{\tref{v}{\btyp}{\refa_1}}{\tref{v}{\btyp}{\refa_2}}
%%%% }[\rsubbase]
%%%% $$
%%% \emphbf{Implication}\hfill{\isvalid{\env}{\refa_1}{\refa_2}}\\
%%% $$
%%% \inference{
  %%% \tologicshort{\env}{\refa_1}{\tbool}{\pred_1}{\tbool}{\smtenv_1}{\axioms_1} &
  %%% \tologicshort{\env}{\refa_2}{\tbool}{\pred_2}{\tbool}{\smtenv_2}{\axioms_i} \\
  %%% \text{is SMT-valid}\ (\embedexpr{\env} \Rightarrow \pred_1 \Rightarrow \pred_2)
%%% }{
  %%% \isvalid{\env}{\refa_1}{\refa_2}
%%% }
%%% $$
%%% \emphbf{Typing}\hfill{\ahastype{\env}{\prog}{\typ}}\\
\caption{\textbf{Algorithmic Typing (other rules in Figs~\ref{fig:syntax} and \ref{fig:typing}.)}}
\label{fig:modifications}
\end{figure}

Figure~\ref{fig:modifications} summarizes the modifications required
to obtain decidable type checking.
%
Namely, basic types are extended with labels that track termination
and subtyping is checked via an SMT solver.

\mypara{Termination}
%
Under arbitrary beta-reduction semantics
(which includes lazy evaluation), soundness
of refinement type checking requires checking
termination, for two reasons:
%
(1)~to ensure that refinements cannot diverge, and
(2)~to account for the environment during subtyping~\citep{Vazou14}.
%
We use \tlabel to mark provably terminating
computations, and extend the rules to use
refinements to ensure that if
${\ahastype{\env}{e}{\tref{v}{\btyp^\tlabel}{r}}}$,
then $e$ terminates~\citep{Vazou14}.
%
%% Here we assume termination is checked by an oracle,
%% but we can use refinement types themselves to prove
%% correctness of the termination labeling


\mypara{Verification Conditions}
The \emph{verification condition} (VC)
${\vcond{\env}{\pred}}$
is \emph{valid} only if the set of values
described by $\env$, is subsumed by
the set of values described by $\pred$.
%
$\env$ is embedded into logic by conjoining
(the embeddings of) the refinements of
provably terminating binders~\cite{Vazou14}:
%
%% We only trust refinements of terminating
%% expressions, as every diverging expression
%% can be unsoundly refined \efalse.
%% $$
%% \embed{\env} \defeq
  %% \bigwedge\{ p \mid \tbind{x}{\tref{v}{\btyp^{\tlabel}}{e}} \in \env
   %% \land \tologicshort{\env}{e\subst{v}{x}}{\btyp}{p}{\embed{\btyp}}{\smtenv}{\axioms}
   %% \}
%% $$
\begin{align*}
\embed{\env} \defeq & \bigwedge_{x \in \env} \embed{\env, x} \\
\intertext{where we embed each binder as}
\embed{\env, x} \defeq & \begin{cases}
                           \pred  & \text{if } \env(x)=\tref{v}{\btyp^{\tlabel}}{e},\
                                    \tologicshort{\env}{e\subst{v}{x}}{\btyp}{\pred}{\embed{\btyp}}{\smtenv}{\axioms} \\
                           \etrue & \text{otherwise}.
                         \end{cases}
\end{align*}

%We use the embedding of environment to decidably check subtyping.
%As defined in Figure~\ref{fig:modifications},
%\tref{v}{\btyp}{\refa_1} is subtype of \tref{v}{\btyp}{\refa_1}
%under the environment \env, when
%$\refa_i$ transforms to $\pred_i$ with axioms $\axioms_i$
%and assuming $\embedexpr{\env}$ and the axioms $\axioms_i$
%$\pred_i$ implies $\pred_2$.

\mypara{Subtyping via SMT Validity}
%
We make subtyping, and hence, typing decidable,
by replacing the denotational base subtyping
rule $\rsubbase$ with a conservative,
algorithmic version that uses an SMT
solver to check the validity of the subtyping VC.
%
We use Corollary~\ref{thm:embedding} to prove
soundness of subtyping. 
%
\begin{lemma}\label{lem:subtyping} %[Conservative Subtyping]
If {\aissubtype{\env}{\tref{v}{\btyp}{e_1}}{\tref{v}{\btyp}{e_2}}}
then {\issubtype{\env}{\tref{v}{\btyp}{e_1}}{\tref{v}{\btyp}{e_2}}}.
\end{lemma}

%
\mypara{Soundness of \smtlan}
%
Lemma~\ref{lem:subtyping} directly implies the soundness of \smtlan.
%
\begin{theorem}[Soundness of \smtlan]\label{thm:soundness-smt}
If \ahastype{\env}{e}{\typ} then \hastype{\env}{e}{\typ}.
\end{theorem}


\begin{comment}
\begin{proof}
By rule \rsubbase, we need to show that
$\forall \sub\in\interp{\env}.
  \interp{\applysub{\sub}{\tref{v}{\btyp}{\refa_1}}}
  \subseteq
  \interp{\applysub{\sub}{\tref{v}{\btyp}{\refa_2}}}$.
%
We fix a $\sub\in\interp{\env}$.
and get that forall bindings
$(\tbind{x_i}{\tref{v}{\btyp^{\downarrow}}{\refa_i}}) \in \env$,
$\evalsto{\applysub{\sub}{e_i\subst{v}{x_i}}}{\etrue}$.

Then need to show that for each $e$,
if $e \in \interp{\applysub{\sub}{\tref{v}{\btyp}{\refa_1}}}$,
then $e \in \interp{\applysub{\sub}{\tref{v}{\btyp}{\refa_2}}}$.

If $e$ diverges then the statement trivially holds.
Assume $\evalsto{e}{w}$.
We need to show that
if $\evalsto{\applysub{\sub}{e_1\subst{v}{w}}}{\etrue}$
then $\evalsto{\applysub{\sub}{e_2\subst{v}{w}}}{\etrue}$.

Let \vsub the lifted substitution that satisfies the above.
Then  by Lemma~\ref{thm:embedding}
for each model $\bmodel \in \interp{\vsub}$,
$\bmodel\models\pred_i$, and $\bmodel\models q_1$
for
$\tologicshort{\env}{e_i\subst{v}{x_i}}{\btyp}{\pred_i}{\embed{\btyp}}{\smtenv_i}{\axioms_i}$
$\tologicshort{\env}{e_i\subst{v}{w}}{\btyp}{q_i}{\embed{\btyp}}{\smtenv_i}{\beta_i}$.
%
Since \aissubtype{\env}{\tref{v}{\btyp}{e_1}}{\tref{v}{\btyp}{e_2}} we get
$$
\bigwedge_i \pred_i
\Rightarrow q_1 \Rightarrow q_2
$$
thus $\bmodel\models q_2$.
%
By Theorem~\ref{thm:embedding} we get $\evalsto{\applysub{\sub}{\refa_2\subst{v}{w}}}{\etrue}$.
\end{proof}
\end{comment}
