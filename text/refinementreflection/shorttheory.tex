\section{Refinement Reflection}
\label{sec:formalism}
\label{sec:types-reflection}
\label{sec:refinementreflection:theory}
Next, we formalize refinement reflection
via a core calculus \corelan.
%
We define a decidable SMT language \smtlan to approximate the
higher order, potentially diverging
target language \corelan
and present a decidable and sound type system
for \corelan.

\subsection{Syntax}

\begin{figure}[t!]
\centering
$$
\begin{array}{rrcl}
\emphbf{Operators} \quad
  & \odot
  & ::= & = \spmid  <
\\[0.03in]

\emphbf{Constants} \quad
  & c
  & ::=
  & \land \spmid \lnot \spmid \odot \spmid +,-,\dots  \\
  && \spmid & \etrue\spmid \efalse \spmid 0, 1,-1, \dots
\\[0.03in]

\emphbf{Values} \quad
  & w & ::=&  c
             \spmid \efun{x}{\typ}{e} \spmid D\ \overline{w}
\\[0.03in]

\emphbf{Expressions} \quad
  & e & ::=    & w \spmid x \spmid \eapp{e}{e}      \\
  &   & \spmid & \ecase{x}{e}{\dc}{\overline{x}}{e}
  %% &   & \spmid & \eletb{x}{\gtyp}{e}{e}
\\[0.03in]

\emphbf{Binders} \quad
  & \bd & ::= & e \spmid \eletb{x}{\gtyp}{\bd}{\bd}
\\[0.03in]

\emphbf{Program} \quad
  & \prog & ::= & \bd \spmid \erefb{x}{\gtyp}{e}{\prog}
\\[0.03in]

% \emphbf{Logical Labels} \quad
%  & L & ::= & \la \spmid \lm
%\\[0.03in]

\emphbf{Basic Types} \quad
  & \btyp
  & ::=
  & \tint \spmid \tbool \spmid T
\\[0.03in]

\emphbf{Refined Types} \quad
  & \typ
  & ::=      & \tref{v}{\btyp}{\reft} \spmid \tfun{x}{\typ}{\typ}
\\[0.05in]
\end{array}
$$
\caption{\textbf{Syntax of \corelan}}
\label{fig:syntax}
\end{figure}

\begin{figure}[t!]
\centering

\emphbf{Contexts}\hfill{{$\fbox{\textit{C}}$}}
$$
\begin{array}{rcl}
  C & ::=    & \bullet \\
    & \spmid & C\ e \spmid c\ C \spmid D\ \overline{e}\ C\ \overline{e}\\
    & \spmid & \ecase{y}{C}{\dc_i}{\overline{x}}{e_i}
  \\[0.03in]
\end{array}
$$

\emphbf{Reductions}\hfill{{$\fbox{\goesto{\prog}{\prog'}}$}}
$$
\begin{array}{rcl}
C[\prog]
  & \hookrightarrow
  & C[\prog'],\quad \text{if}\ \goesto{\prog}{\prog'}
  \\
{c\ v}
  & \hookrightarrow
  & {\ceval{c}{v}}
  \\
{({\efun{x}{\typ}{e})}\ {e'}}
  & \hookrightarrow
  & {\SUBST{e}{x}{e'}}
  \\
{\ecase{y}{\dc_j\ \overline{e}}{\dc_i}{\overline{x_i}}{e_i}}
  & \hookrightarrow
  & {\SUBST{\SUBST{e_j}{y}{\dc_j\ \overline{e}}}{\overline{x_i}}{\overline{e}}}
  \\
{\erefb{x}{\gtyp}{e}{\prog}}
  & \hookrightarrow
  & {\SUBST{\prog}{x}{\efix{(\efun{x}{\gtyp}{e})}}}
  \\
{\eletb{x}{\gtyp}{\bd_x}{\bd}}
  & \hookrightarrow
  % & {\SUBST{\prog}{x}{e}}
  & {\SUBST{\bd}{x}{\efix{(\efun{x}{\gtyp}{\bd_x})}}}
  \\
%% {\eletrec{x}{e_x}{\gtyp}{[L]}{\prog}}
  %% & \hookrightarrow
  %% & {\SUBST{\prog}{x}{\efix{\efun{x}{\gtyp}{e_x}}}}
  %% \\
{\efix{\prog}}
  & \hookrightarrow
  & {\prog\ (\efix{\prog})}
\end{array}
$$
\caption{\textbf{Operational Semantics of \corelan}}
\label{fig:semantics}
\end{figure}

%
Figure~\ref{fig:syntax} summarizes the syntax of \corelan,
which is essentially the calculus \undeclang (from~\S~\ref{sec:language})
with explicit recursion and a special $\erefname$ binding
to denote terms that are reflected into the refinement logic.
%
The elements of \corelan are layered into
primitive constants, values, expressions, binders
and programs.

\mypara{Constants}
The primitive constants of \corelan
include all the primitive logical
operators $\op$, here, the set $\{ =, <\}$.
%
Moreover, they include the
primitive booleans $\etrue$, $\efalse$,
integers $\mathtt{-1}, \mathtt{0}$, $\mathtt{1}$, \etc,
and logical operators $\mathtt{\land}$, $\mathtt{\lor}$, $\mathtt{\lnot}$, \etc.

\mypara{Data Constructors}
%
Data constructors are special constants.
% Each data type has an equality predicate $\haseq{T}$
% that is true only if values of type $T$ can be finitely compared.
For example, the data type \tintlist, which represents
finite lists of integers, has two data constructors: $\dnull$ (nil)
and $\dcons$ (cons).

\mypara{Values \& Expressions}
%
The values of \corelan include
constants, $\lambda$-abstractions
$\efun{x}{\typ}{e}$, and fully
applied data constructors $D$
that wrap values.
%
The expressions of \corelan
include values, variables $x$,
applications $\eapp{e}{e}$, and
$\mathtt{case}$ expressions.

\mypara{Binders \& Programs}
%
A \emph{binder} $\bd$ is a series of possibly recursive
$\mathtt{let}$ definitions, followed by an expression.
%
A \emph{program} \prog is a series of $\erefname$
definitions, each of which names a function
that is reflected into the refinement
logic, followed by a binder.
%
The stratification of programs via binders
is required so that arbitrary recursive definitions
are allowed in the program but cannot be inserted into the logic
via refinements or reflection.
%
(We \emph{can} allow non-recursive $\mathtt{let}$
binders in expressions $e$, but omit them for simplicity).

\subsection{Operational Semantics}
We define $\hookrightarrow$ to be the small step, call-by-name
$\beta$-reduction semantics for \corelan.
%
We evaluate reflected terms %$\erefb{x}{\gtyp}{e}{\prog}$
as recursive $\mathtt{let}$ bindings, with extra termination-check
constraints imposed by the type system:
%
$$
\erefb{x}{\gtyp}{e}{\prog}
\hookrightarrow
\eletb{x}{\gtyp}{e}{\prog}
$$
%
We define $\evalsto{}{}$ to be the reflexive,
transitive closure of $\evals{}{}$.
%
Moreover, we define $\betaeq{}{}$ to be the reflexive,
symmetric, and transitive closure of $\evals{}{}$.

\mypara{Constants} Application of a constant requires the
argument be reduced to a value; in a single step, the
expression is reduced to the output of the primitive
constant operation, \ie ${c\ v} \hookrightarrow \ceval{c}{v}$.
%
For example, consider $=$, the primitive equality
operator on integers.
%
We have $\ceval{=}{n} \defeq =_n$
where $\ceval{=_n}{m}$ equals \etrue
iff $m$ is the same as $n$.
%

\mypara{Equality}
We assume that the equality operator
is defined \emph{for all} values,
and, for functions, is defined as
extensional equality.
%
That is, for all
$f$ and
$f'$,
$\evals{(f = f')}{\etrue}$
   $\mbox{iff}$
  $\forall v.\ \betaeq{f\ v}{f'\ v}$.
%
We assume source \emph{terms} only contain implementable equalities
over non-function types; while function extensional equality only appears
in \emph{refinements} and is approximated by the underlying logic.


\subsection{Types}

\corelan types include basic types, which are \emph{refined} with predicates,
and dependent function types.
%
\emph{Basic types} \btyp comprise integers, booleans, and a family of data-types
$T$ (representing lists, trees \etc).
%
For example, the data type \tintlist represents lists of integers.
%
We refine basic types with predicates (boolean-valued expressions \refa) to obtain
\emph{basic refinement types} $\tref{v}{\btyp}{\refa}$.
%
%
We use \tlabel to mark provably terminating
computations and use
refinements to ensure that if
${{e}\text{:}{\tref{v}{\btyp^\tlabel}{\refa'}}}$,
then $e$ terminates (as in chapter~\ref{chapter:refinedhaskell}).
%
Finally, we have dependent \emph{function types} $\tfun{x}{\typ_x}{\typ}$
where the input $x$ has the type $\typ_x$ and the output $\typ$ may
refer to the input binder $x$.
%
We write $\btyp$ to abbreviate $\tref{v}{\btyp}{\etrue}$,
and \tfunbasic{\typ_x}{\typ} to abbreviate \tfun{x}{\typ_x}{\typ} if
$x$ does not appear in $\typ$.
%
%We use $r$ to refer to refinements.

\mypara{Constants}
For each constant $c$ we define its type \constty{c}, \eg%
%
$$
\begin{array}{lcl}
\constty{3} &\doteq& \tref{v}{\tint}{v = 3}\\
\constty{+} &\doteq& \tfun{\ttx}{\tint}{\tfun{\tty}{\tint}{\tref{v}{\tint}{v = x + y}}}\\
\constty{\leq} &\doteq& \tfun{\ttx}{\tint}{\tfun{\tty}{\tint}{\tref{v}{\tbool}{v \Leftrightarrow x \leq y}}}\\
\end{array}
$$
%

\subsection{Refinement Reflection}\label{subsec:logicalannotations}
%
The key idea in our work is to
\emph{strengthen} the output type of functions
with a refinement that \emph{reflects} the
definition of the function in the logic.
%
We do this by treating each
%
$\erefname$-binder
%
(${\erefb{f}{\gtyp}{e}{\prog}}$)
%
as a $\eletname$-binder with a reflected singleton type
%
(${\eletb{f}{\exacttype{\gtyp}{e}}{e}{\prog}}$)
%
during type checking (rule $\rtreflect$ in Figure~\ref{fig:typing}).

\mypara{Reflection}
%
We write \exacttype{\typ}{e} for the \emph{reflection}
of the term $e$ into the type $\typ$,  defined by strengthening
\typ as:
%
$$
\begin{array}{lcl}
\exacttype{\tref{v}{\btyp}{r}}{e}
  & \defeq
  & \tref{v}{\btyp}{r \land v = e}\\
\exacttype{\tfun{x}{\typ_x}{\typ}}{\efun{y}{}{e}}
  & \defeq
  & \tfun{x}{\typ_x}{\exacttype{\typ}{e\subst{y}{x}}}
\end{array}
$$
%
As an example, recall from \S~\ref{sec:refinementreflection:overview}
that the @reflect fib@ strengthens the type of
@fib@ with the refinement @fibP@.
%% NV In Overview, we have fibP v n = v = fib n && fibR v n
%% NV Here we get the reflection part (fibR v n)
%% NV which we can verify
%% NV at each fix invocation we also get the v = fib n portion
%% NV via the exact rule
%% NV We can not add the v = fib n as a port condition, because
%% NV we cannot prove it.


\mypara{Consequences for Verification}
%
Reflection has two consequences for verification.
%
First, the reflected refinement is \emph{not trusted};
it is itself verified (as a valid output type)
during type checking.
%
Second, instead of being tethered to quantifier
instantiation heuristics or having to program
``triggers'' as in Dafny~\citep{dafny} or
\fstar~\citep{fstar},
%
the programmer can predictably ``unfold'' the
definition of the function during a proof simply
by ``calling'' the function, which, as discussed
in~\S~\ref{sec:evaluation}, we have found to be
a very natural way of structuring proofs.


\subsection{The SMT logic \smtlan}
\corelan is a higher order, potentially diverging language
that cannot be used for decidable verification.
%
Next, we describe \smtlan, a conservative, first order
approximation of \corelan where higher order features are
approximated with uninterpreted functions,
yielding an SMT-based
algorithmic logic that enjoys soundness and decidability.

\begin{figure}[t!]
\vspace{-5mm}
\centering
$$
\begin{array}{rrcl}
\emphbf{Predicates} 
  & \pred & ::= &
    \pred \binop \pred \spmid
    \unop \pred \\
  && \spmid & n \spmid b \spmid x \spmid \dc \spmid  x\ \overline{\pred}\\
%%  && \spmid & \forall \overline{\tbind{x}{\sort}}. \pred
  && \spmid & \eif{\pred}{\pred}{\pred}
\\[0.03in]

\emphbf{Integers} 
  & n
  & ::= & 0, -1, 1, \dots
\\[0.03in]

\emphbf{Booleans} 
  & b
  & ::= & \etrue \spmid \efalse
\\[0.03in]

\emphbf{Bin Operators} 
  & \binop
  & ::= & = \spmid < \spmid \land \spmid + \spmid - \spmid \dots
\\[0.03in]

\emphbf{Un Operators} 
  & \unop
  & ::= & \lnot \spmid \dots 
\\[0.03in]

\emphbf{Sort Args} 
  & \sort_a
  & ::= & \tint \spmid \tbool \spmid \tuniv 
         \spmid \tsmtfun{\sort_a}{\sort_a}
\\[0.03in]
\emphbf{Sorts} 
  & \sort
  & ::=  & \sort_a \rightarrow \sort
\end{array}
$$
\caption{\textbf{Syntax of \smtlan}.}
\label{fig:smtsyntax}
\vspace{-2mm}
\end{figure}


\mypara{Syntax}
%
Figure~\ref{fig:smtsyntax} summarizes the syntax
of \smtlan, the \emph{sorted} (SMT-)
decidable logic of quantifier-free equality,
uninterpreted functions and linear
arithmetic (QF-EUFLIA) ~\citep{Nelson81,SMTLIB2}.
%
The \emph{terms} of \smtlan include
integers $n$,
booleans $b$,
variables $x$,
data constructors $\dc$ (encoded as constants),
fully applied unary \unop and binary \binop operators,
and application $x\ \overline{\pred}$ of an uninterpreted function $x$.
%
The \emph{sorts} of \smtlan include the built-in
\tint and \tbool.
%
The interpreted functions of \smtlan, \eg
the logical constants $=$ and $<$,
%% NV and the uninterpreted functions app and lam
%% NV but we have not introduced these yet
have the function sort $\sort \rightarrow \sort$.
%
Other functional values in \corelan, \eg
reflected \corelan functions and
$\lambda$-expressions, have the first-order
uninterpreted sort \tsmtfun{\sort}{\sort}.
%
The universal sort \tuniv represents all other values.

\subsection{Transforming \corelan into \smtlan}
%
\label{subsec:embedding}

%\newcommand\emptyaxioms{\ensuremath{\emptyset}\xspace}
\newcommand\andaxioms[2]{\ensuremath{{#1}\cup {#2}}\xspace}

\begin{figure}
\emphbf{Transformation}\hfill{\fbox{\tologicshort{\Gamma}{e}{\typ}{\pred}{\sort}{\smtenv}{\axioms}}}
$$
\inference{
}{
	\tologicshort{\env}{b}{\tbool}{b}{\tbool}{\emptyset}{\emptyaxioms}
}[\lgbool]
\qquad
\inference{
}{
	\tologicshort{\env}{n}{\tint}{n}{\tint}{\emptyset}{\emptyaxioms}
}[\lgint]
$$

$$
\inference{
    \tologicshort{\env}{e_1}{\typ}{\pred_1}{\embed{\typ}}{\smtenv}{\axioms_1} &
    \tologicshort{\env}{e_2}{\typ}{\pred_2}{\embed{\typ}}{\smtenv}{\axioms_2}
}{
	\tologicshort{\env}{e_1\binop e_2}{\tbool}{\pred_1 \binop\pred_2}{\tbool}{\smtenv}{\andaxioms{\axioms_1}{\axioms_2}}
}[\lgbinGEN]
$$

$$
\inference{
	\tologicshort{\env}{e}{\tbool}{\pred}{\tbool}{\smtenv}{\axioms}
}{
	\tologicshort{\env}{\unop e}{\tbool}{\unop\pred}{\tbool}{\smtenv}{\axioms}
}[\lgun]
\qquad
\inference{
}{
	\tologicshort{\env}{x}{\env(x)}{x}{\embed{\env(x)}}{\emptyset}{\emptyaxioms}
}[\lgvar]
$$

$$
\inference{
}{
	\tologicshort{\env}{c}{\constty{\odot}}{\smtvar{c}}{\embed{\constty{\odot}}}{\emptyset}{\emptyaxioms}
}[\lgpop]
\qquad
\inference{
}{
	\tologicshort{\env}{\dc}{\constty{\dc}}{\smtvar{\dc}}{\embed{\constty{\dc}}}{\emptyset}{\emptyaxioms}
}[\lgdc]
$$


%%$$
%%\inference{
%%  	\axioms_{f_1} = \forall \tbind{x}{\sort_x}.\smtappname{\sort_x}{\sort}\ f\ x = \pred \\
%%  	\axioms_{f_2} = \forall \tbind{g}{\sort'},\tbind{x}{\sort_x}.
%%  	\smtappname{\sort_x}{\sort}\ f\ x = \smtappname{\sort_x}{\sort}\ g\ x \Rightarrow f = g \\
%% 	f\ \text{fresh} &
%% 	\sort' = \embed{\tfun{x}{\typ_x}{\typ}} &
%% 	\sort  = \embed{\typ} &
%% 	\sort_x = \embed{\typ_x} \\
%% 	\tologicshort{\env,\tbind{x}{\typ_x}}{e}{\typ}{\pred}{\sort}{\smtenv, \tbind{x}{\sort_x}}{\axioms} &
%% 	\hastype{\env}{(\efun{x}{}{e})}{(\tfun{x}{\typ_x}{\typ})}\\
%%}{
%%	\tologicshort{\env}{\efun{x}{}{e}}{(\tfun{x}{\typ_x}{\typ})}
%%	        {f}{\sort'}{\smtenv, \tbind{f}{\sort'}}{\andaxioms{\{\axioms_{f_1}, \axioms_{f_2}\}}{\axioms}}
%%}[\lgfun]
%%$$

$$
\inference{
    \tologicshort{\env, \tbind{x}{\typ_x}}{e}{}{\pred}{}{}{} &
  	\hastype{\env}{(\efun{x}{}{e})}{(\tfun{x}{\typ_x}{\typ})}\\
}{
	\tologicshort{\env}{\efun{x}{}{e}}{(\tfun{x}{\typ_x}{\typ})}
	        {\smtlamname{\embed{\typ_x}}{\embed{\typ}}\ {x}\ {\pred}}
	        {\sort'}{\smtenv, \tbind{f}{\sort'}}{\andaxioms{\{\axioms_{f_1}, \axioms_{f_2}\}}{\axioms}}
}[\lgfun]
$$

$$
\inference{
	\tologicshort{\env}{e'}{\typ_x}{\pred'}{\embed{\typ_x}}{\smtenv}{\axioms'}
	&
	\tologicshort{\env}{e}{\tfun{x}{\typ_x}{\typ}}{\pred}{\tsmtfun{\embed{\typ_x}}{\embed{\typ}}}{\smtenv}{\axioms}
	& 
	\hastype{\env}{e}{{\typ_x}\rightarrow{\typ}}
}{
	\tologicshort{\env}{e\ e'}{\typ}{\smtappname{\embed{\typ_x}}{\embed{\typ}}\ {\pred}\ {\pred'}}{\embed{\typ}}{\smtenv}{\andaxioms{\axioms}{\axioms'}}
}[\lgapp]
$$


$$
\inference{
	\tologicshort{\env}{e}{\tbool}{\pred}{\tbool}{\smtenv}{\axioms} & 
	\tologicshort{\env}{e_i\subst{x}{e}}{\typ}{\pred_i}{\embed{\typ}}{\smtenv}{\axioms_i}
}{
	\tologicshorttwolines{\env}{\ecaseexp{x}{e}{\etrue \rightarrow e_1; \efalse \rightarrow e_2}}{\typ}
	 {\eif{\pred}{\pred_1}{\pred_2}}{\embed{\typ}}{\smtenv}{\andaxioms{\axioms}{\axioms_i}}
}[\lgcaseBool]
$$

$$
\inference{
	\tologicshort{\env}{e}{\typ_e}{\pred}{\embed{\typ_e}}{\smtenv}{\axioms}\\
	\tologicshort{\env}{e_i\subst{\overline{y_i}}{\overline{\selector{\dc_i}{}\ x}}\subst{x}{e}}{\typ}{\pred_i}{\embed{\typ}}{\smtenv}{\axioms_i}
}{
	\tologicshorttwolines{\env}{\ecase{x}{e}{\dc_i}{\overline{y_i}}{e_i}}{\typ}
	 {\eif{\smtappname{}{}\ \checkdc{\dc_1}\ \pred}{\pred_1}{\ldots} \ \mathtt{else}\ \pred_n}{\embed{\typ}}{\smtenv}
	 {\andaxioms{\axioms}{\axioms_i}}
}[\lgcase]
$$
\caption{\textbf{Transforming \corelan terms into \smtlan.}}
\label{fig:defunc}
\end{figure}

%
A \emph{type environment} $\env$ is a sequence of type bindings
$\tbind{x_1}{\typ_1},$ $\ldots,\tbind{x_n}{\typ_n}$.
We use the type environment to define the judgment
\tologicshort{\env}{e}{\typ}{\pred}{\sort}{\smtenv}{\axioms}
that transforms a $\corelan$ term $e$
% under an environment $\env$,
into a $\smtlan$ term $\pred$.
%
Most of the transformation rules are identity
and can be found in~\cite{vazou16techrep}.
Here we discuss the non-identity ones.

\mypara{Embedding Types}
%
We embed \corelan types into \smtlan sorts as:
%
\[
\begin{array}{rclcrcl}
\embed{\tint}                       & \defeq &  \tint & \quad &
\embed{T}                           & \defeq &  \tuniv \\
\embed{\tbool}                      & \defeq &  \tbool & \quad &
\embed{\tfun{x}{\typ_x}{\typ}} & \defeq & \tsmtfun{\embed{\typ_x}}{\embed{\typ}}
\end{array}
\]


\mypara{Embedding Constants}
%
Elements shared on both \corelan and \smtlan
translate to themselves.
%
These elements include
booleans,
integers,
variables,
binary
and unary
operators.
%
SMT solvers do not support currying,
and so in \smtlan, all function symbols
must be fully applied.
%
Thus, we assume that all applications
to primitive constants and data
constructors are fully applied, \eg by converting
source terms like @(+ 1)@ to
@(\z -> z+1)@.
%


\mypara{Embedding Functions}
%
As \smtlan is a first-order logic, we
embed $\lambda$-abstraction
using the uninterpreted function
\smtlamname{}{}.
%
$$
\inference{
    \tologicshort{\env, \tbind{x}{\typ_x}}{e}{}{\pred}{}{}{} &
        \hastype{\env}{(\efun{x}{}{e})}{(\tfun{x}{\typ_x}{\typ})}\\
}{
	\tologicshort{\env}{\efun{x}{}{e}}{(\tfun{x}{\typ_x}{\typ})}
	        {\smtlamname{\embed{\typ_x}}{\embed{\typ}}\ {x}\ {\pred}}
	        {\sort'}{\smtenv, \tbind{f}{\sort'}}{\andaxioms{\{\axioms_{f_1}, \axioms_{f_2}\}}{\axioms}}
}[\lgfun]
$$
%
The term $\efun{x}{}{e}$ of type
${\typ_x \rightarrow \typ}$ is transformed
to
${\smtlamname{\sort_x}{\sort}\ x\ \pred}$
of sort
${\tsmtfun{\sort_x}{\sort}}$, where
%
$\sort_x$ and $\sort$ are respectively
$\embed{\typ_x}$ and $\embed{\typ}$,
%
${\smtlamname{\sort_x}{\sort}}$
is a special uninterpreted function
of sort
${\sort_x \rightarrow \sort\rightarrow\tsmtfun{\sort_x}{\sort}}$,
and
$x$ of sort $\sort_x$ and $r$ of sort $\sort$ are
the embedding of the binder and body, respectively.
%
As $\smtlamname{}{}$ is an SMT-function,
it \emph{does not} create a binding for $x$.
%
Instead, $x$ is renamed to
a \emph{fresh} name pre-declared in
the SMT logic.


\mypara{Embedding Applications}
%
Dually, we embed applications via
defunctionalization~\citep{Reynolds72}
with the uninterpreted $\smtappname{}{}$ function.
%
$$
\inference{
	\tologicshort{\env}{e'}{\typ_x}{\pred'}{\embed{\typ_x}}{\smtenv}{\axioms'}
	&
	\tologicshort{\env}{e}{\tfun{x}{\typ_x}{\typ}}{\pred}{\tsmtfun{\embed{\typ_x}}{\embed{\typ}}}{\smtenv}{\axioms}
	&
	\hastype{\env}{e}{{\typ_x}\rightarrow{\typ}}
}{
	\tologicshort{\env}{e\ e'}{\typ}{\smtappname{\embed{\typ_x}}{\embed{\typ}}\ {\pred}\ {\pred'}}{\embed{\typ}}{\smtenv}{\andaxioms{\axioms}{\axioms'}}
}[\lgapp]
$$
%
The term ${e\ e'}$, where $e$ and $e'$ have
types ${\typ_x \rightarrow \typ}$ and $\typ_x$,
is transformed to
${\tbind{\smtappname{\sort_x}{\sort}\ \pred\ \pred'}{\sort}}$
where
%
$\sort$ and $\sort_x$ are $\embed{\typ}$ and $\embed{\typ_x}$,
the
${\smtappname{\sort_x}{\sort}}$
is a special uninterpreted function of sort
${\tsmtfun{\sort_x}{\sort} \rightarrow \sort_x \rightarrow \sort}$,
and
$\pred$ and $\pred'$ are the respective translations of $e$ and $e'$.


\mypara{Embedding Data Types}
%
We translate each data constructor to a
predefined \smtlan constant ${\smtvar{\dc}}$ of
sort ${\embed{\constty{\dc}}}$.
$$
	\tologicshort{\env}{\dc}{\constty{\dc}}{\smtvar{\dc}}{\embed{\constty{\dc}}}{\emptyset}{\emptyaxioms}
$$
%
For each datatype, we assume the existence of reflected functions that
\emph{check} the top-level constructor
and \emph{select} their individual fields.
%
For example, for lists, we assume the existence of measures:
%
\begin{mcode}
  isNil []     = True     isCons (x:xs) = True
  isNil (x:xs) = False    isCons []     = False

  sel1 (x:xs)  = x        sel2 (x:xs)   = xs
\end{mcode}
%
Due to the simplicity of their syntax the above checkers and selectors
can be automatically instantiated in the logic
(\ie without actual calls to the reflected functions at source level)
using the measure mechanism (\S~\ref{subsec:measures}).

To generalize, let $\dc_i$ be a data constructor such that
$$
\constty{\dc_i} \defeq \typ_{i,1} \rightarrow \dots \rightarrow \typ_{i,n} \rightarrow \typ
$$
Then the \emph{check function}
${\checkdc{{\dc_i}}}$ has the sort
$\tsmtfun{\embed{\typ}}{\tbool}$,
and the \emph{select function}
${\selector{\dc}{i,j}}$ has the sort
$\tsmtfun{\embed{\typ}}{\embed{\typ_{i,j}}}$.
%


% \mypara{Checking and Projection}
%
We translate case-expressions
of \corelan into nested $\mathtt{if}$
terms in \smtlan, by using the check
functions in the guards, and the
select functions for the binders
of each case.
$$
\inference{
	\tologicshort{\env}{e}{\typ_e}{\pred}{\embed{\typ_e}}{\smtenv}{\axioms} &&
	\tologicshort{\env}{e_i\subst{\overline{y_i}}{\overline{\selector{\dc_i}{}\ x}}\subst{x}{e}}{\typ}{\pred_i}{\embed{\typ}}{\smtenv}{\axioms_i}
}{
	\tologicshorttwolines{\env}{\ecase{x}{e}{\dc_i}{\overline{y_i}}{e_i}}{\typ}
	 {\eif{\smtappname{}{}\ \checkdc{\dc_1}\ \pred}{\pred_1}{\ldots} \ \mathtt{else}\ \pred_n}{\embed{\typ}}{\smtenv}
	 {\andaxioms{\axioms}{\axioms_i}}
}[\lgcase]
$$
%
For example, the body of the list append function
%
\begin{code}
  []     ++ ys = ys
  (x:xs) ++ ys = x : (xs ++ ys)
\end{code}
%
is reflected into the \smtlan refinement:
%
$$
\ite{\mathtt{isNil}\ \mathit{xs}}
    {\mathit{ys}}
    {\mathtt{sel1}\ \mathit{xs}\
       \dcons\
       (\mathtt{sel2}\ \mathit{xs} \ \mathtt{++}\  \mathit{ys})}
$$
%
We favor selectors to the axiomatic translation of
HALO~\citep{halo} to avoid
universally quantified formulas and the resulting
instantiation unpredictability.


\subsection{Typing Rules}
\begin{figure}
\emphbf{Well Formedness}\hfill{\fbox{\iswellformed{\env}{\typ}}}\\

$$
\inference{
	\hastype{\env,\tbind{v}{\btyp}}{\refa}{\tbool}
}{
	\iswellformed{\env}{\tref{v}{\btyp}{\refa}}
}[\rwbase]
\inference{
	\iswellformed{\env}{\typ_x} &
	\iswellformed{\env,\tbind{x}{\typ_x}}{\typ}
}{
	\iswellformed{\env}{\tfun{x}{\typ_x}{\typ}}
}[\rwfun]
$$

\emphbf{Subtyping}\hfill{\fbox{\issubtype{\env}{\typ_1}{\typ_2}}}\\

$$
\inference{
	\forall \sub\in\interp{\env}.
	\interp{\applysub{\sub}{\tref{v}{\btyp}{\refa_1}}}
	\subseteq
	\interp{\applysub{\sub}{\tref{v}{\btyp}{\refa_2}}}
}{
	\issubtype{\env}{\tref{v}{\btyp}{\refa_1}}{\tref{v}{\btyp}{\refa_2}}
}[\rsubbase]
$$

$$
\inference{
	\issubtype{\env}{\typ_x'}{\typ_x} &
	\issubtype{\env,\tbind{x}{\typ_x'}}{\typ}{\typ'}
}{
	\issubtype{\env}{\tfun{x}{\typ_x}{\typ}}{\tfun{x}{\typ_x'}{\typ'}}
}[\rsubfun]
$$

\emphbf{Typing}\hfill{\fbox{\hastype{\env}{\prog}{\typ}}}\\
$$
\inference{
	\tbind{x}{\gtyp}\in\env
}{
	\hastype{\env}{x}{\gtyp}
}[\rtvar]
\qquad
\inference{
}{
	\hastype{\env}{c}{\constty{c}}
}[\rtconst]
$$

$$
\inference{
	\hastype{\env}{\prog}{\typ'} &
	\issubtype{\env}{\typ'}{\typ}
}{
	\hastype{\env}{\prog}{\typ}
}[\rtsub]
$$
$$
\inference{
    % \haseq{\btyp} &
	\hastype{\env}{e}{\tref{v}{\btyp}{\reft_r}}
}{
	\hastype{\env}{e}{\tref{v}{\btyp}{\reft_r\land v = e}}
}[\rtexact]
$$
$$
\inference{
	\hastype{\env, \tbind{x}{\typ_x}}{e}{\typ}
}{
	\hastype{\env}{\efun{x}{\typ}{e}}{\tfun{x}{\typ_x}{\typ}}
}[\rtfun]
$$

$$
\inference{
	\hastype{\env}{e_1}{(\tfun{x}{\typ_x}{\typ})} &&
	\hastype{\env}{e_2}{\typ_x}
}{
	\hastype{\env}{e_1\ e_2}{\typ}
}[\rtapp]
$$

%%% $$
%%% \inference{
	%%% \hastype{\env}{e}{\gtyp_x} &
	%%% \hastype{\env, \tbind{x}{\gtyp_x}}{\prog}{\typ} &
	%%% \iswellformed{\env}{\typ}
%%% }{
	%%% \hastype{\env}{\elet{x}{e}{\gtyp_x}{}{\prog}}{\typ}
%%% }[\rtlet]
%%% $$
%%%
%%% $$
%%% \inference{
	%%% \hastype{\env, \tbind{x}{\gtyp_x}}{e}{\gtyp_x} &
	%%% \hastype{\env, \tbind{x}{\gtyp_x}}{\prog}{\typ} &
	%%% \iswellformed{\env}{\typ}
%%% }{
	%%% \hastype{\env}{\eletrec{x}{e}{\gtyp_x}{}{\prog}}{\typ}
%%% }[\rtletrec]
%%% $$
%% \NV{For soundness, it is important that f cannot appear in $\gtyp_2$}
$$
\inference
	{\hastype{\env, \tbind{x}{\gtyp_x}}{\bd_x}{\gtyp_x} &
	 \iswellformed{\env, \tbind{x}{\gtyp_x}}{\typ_x} \\
	 \hastype{\env, \tbind{x}{\gtyp_x}}{\bd}{\gtyp} &
	 \iswellformed{\env}{\typ}}
	{\hastype{\env}{\eletb{x}{\gtyp_x}{\bd_x}{\bd}}{\typ}}
	[\rtlet]
$$

$$
\inference
	{\hastype{\env}
	 				 {\eletb{f}{\exacttype{\gtyp_f}{e}}{e}{\prog}}
					 {\typ}
	}
	{\hastype{\env}
					 {\erefb{f}{\gtyp_f}{e}{\prog}}
					 {\typ}
	}[\rtreflect]
$$

$$
\inference{
	\hastype{\env}{e}{\tref{v}{T}{e_r}} & \iswellformed{\env}{\typ} \\
	& \forall i. \constty{\dc_i} = \tfunbasic{\overline{\tbind{y_j}{\typ_j}}}{\tref{v}{T}{e_{r_i}}} \\
	& \hastype{\env, \overline{\tbind{y_j}{\typ_j}}, \tbind{x}{\tref{v}{T}{e_r \land e_{r_i}}} }{e_i}{\typ}
}{
	\hastype{\env}{\ecase{x}{e}{\dc_i}{\overline{y_i}}{e_i}}{\typ}
}[\rtcase]
$$
$$
\inference{
	\iswellformed{\env}{\typ}
}{
	\hastype{\env}
	        {\efix{}}
	        {\tfun{x}{\typ}{\tfun{y}{\typ}{\typ}}}
}[\rtfix]
$$
\caption{Typing of \corelan}
\label{fig:typing}
\end{figure}

%
Next, we present the
typing, well-formedness, and subtyping~\citep{Knowles10,Vazou14}
rules of \corelan.

\mypara{Typing}
A judgment \hastype{\env}{\prog}{\typ} states that
the program $\prog$ has the type $\typ$ in
the environment $\env$.
That is, when the free variables in $\prog$ are
bound to expressions described by $\env$, the
program $\prog$ will evaluate to a value
described by $\typ$.

\mypara{Rules}
%
All but two of the rules are standard~\cite{Knowles10,Vazou14}.
%
First, rule \rtreflect is used to strengthen the type of each
reflected binder with its definition, as described previously
in \S~\ref{subsec:logicalannotations}.
%
Second, rule \rtexact strengthens the expression with
a singleton type equating the value and the expression
(\ie reflecting the expression in the type).
%
This is a generalization of the ``selfification'' rules
from \cite{Ou2004,Knowles10}, and is required to
equate the reflected functions with their definitions.
%
For example, the application $(\fib\ 1)$ is typed as
${\tref{v}{\tint}{\fibdef\ v\ 1 \wedge v = \fib\ 1}}$ where
the first conjunct comes from the (reflection-strengthened)
output refinement of \fib~\S~\ref{sec:refinementreflection:overview} and
the second comes from rule~\rtexact.
%
%%Finally, rule \rtfix is used to type the intermediate
%%$\texttt{fix}$ expressions that appear, not in the
%%surface language but as intermediate terms in the
%%operational semantics.


\mypara{Well-formedness}
A judgment \iswellformed{\env}{\typ} states that
the refinement type $\typ$ is well-formed in
the environment $\env$.
%
Following chapter~\ref{chapter:refinedhaskell}, the type $\typ$ is well-formed if all
the refinements in $\typ$ are $\tbool$-typed,
provably terminating expressions in $\env$.
%
%%\mypara{Termination}
%%%
%%Under arbitrary beta-reduction semantics
%%(which includes lazy evaluation), soundness
%%of refinement type checking requires checking
%%termination, for two reasons:
%%%
%%(1)~to ensure that refinements cannot diverge, and
%%(2)~to account for the environment during subtyping~\citep{Vazou14}.
%%%
%%We use \tlabel to mark provably terminating
%%computations, and extend the rules to use
%%refinements to ensure that if
%%${\ahastype{\env}{e}{\tref{v}{\btyp^\tlabel}{r}}}$,
%%then $e$ terminates~\citep{Vazou14}.
%%%

\mypara{Subtyping}
A judgment \issubtype{\env}{\typ_1}{\typ_2} states
that the type $\typ_1$ is a subtype of %the type
$\typ_2$ in the environment $\env$.
%
Informally, $\typ_1$ is a subtype of $\typ_2$ if
the refinement of $\typ_1$ \emph{implies}
the refinement of $\typ_2$
under the assumptions described by $\env$.
%
Subtyping of basic types reduces to implication checking.

\mypara{Verification Conditions}
The implication or \emph{verification condition} (VC)
${\vcond{\env}{\pred}}$
is \emph{valid} only if the set of values
described by $\env$, is subsumed by
the set of values described by $\pred$.
%
$\env$ is embedded into logic by conjoining
(the embeddings of) the refinements of
provably terminating binders (Chapter~\ref{chapter:refinedhaskell}):
%
\begin{align*}
\embed{\env} \defeq & \bigwedge_{x \in \env} \embed{\env, x} \\
\intertext{where we embed each binder as}
\embed{\env, x} \defeq & \begin{cases}
                           \pred  & \text{if } \env(x)=\tref{v}{\btyp^{\tlabel}}{e},\
                                    \tologicshort{\env}{e\subst{v}{x}}{\btyp}{\pred}{\embed{\btyp}}{\smtenv}{\axioms} \\
                           \etrue & \text{otherwise}.
                         \end{cases}
\end{align*}


It is important to note that since \smtlan
is carefully restricted to SMT-decidable theories,
VC checking, and thus type checking of \corelan, is decidable.

\subsection{Soundness}

Following \undeclang from chapter~\ref{chapter:refinedhaskell}, in~\citep{vazou16techrep}, we show that
if validity checking respects the axioms of $\beta$-equivalence,
then \corelan is sound.

We define the $\beta$-equivalence axioms on the uninterpreted function
that represent $\lambda$-abstraction (\smtlamname{}{}) and
and application (\smtappname{}{}).
$$
\begin{array}{rcl}
\forall x\ y\ e. \smtlamname{}{}\ x\ e
  & = & \smtlamname{}{}\ y\ (e\subst{x}{y}) \\
\forall x\ e_x\ e. (\smtappname{}{}\ (\smtlamname{}{}\ x\ e)\ e_x)
  & = &  e\subst{x}{e_x}
\end{array}
$$
%
 We prove that when validity checking assumes the $\beta$-equivalence
 axioms, \corelan is sound.

\begin{theorem}{[Soundness of \corelan]}\label{thm:safety}
Assuming the $\beta$-equivalence axioms,
if \hastype{\emptyset}{\prog}{\typ}
       and $\evalsto{\prog}{w}$ then $\hastype{\emptyset}{w}{\typ}$.
\end{theorem}

Theorem~\ref{thm:safety} lets us interpret well typed terminating programs as proofs of
propositions.
%
For example, in \S~\ref{sec:refinementreflection:overview} we verified that
%
$\fibincrname\text{ :: }{\tfun{n}{\tnat}{\ttref{\fib{n} \leq \fib{(n+1)}}}}$.
%
Via soundness of \corelan, we get runtime monotonicity of @fib@.
$$
\forall n. \evalsto{0 \leq n}{\etrue} \Rightarrow \evalsto{\fib{n} \leq \fib{(n+1)}}{\etrue}
$$
% \section{Algorithmic Verification}\label{sec:algorithmic}

\mypara{Approximation of $\beta$-equivalence}
%
Though sound and precise, directly extending the logic with $\beta$-equivalence axioms
would render SMT validity checking undecidable.
%
Next, we discuss an incomplete,
yet decidable, technique that allows the
user to manually instantiate the $\beta$-equivalence axioms
when required for precise typing.
