%%% \footnote{Anecdotally, with proof-to-code ratios of 6:1
%%% instead of 50:1 or 100:1 with classical proof assistants~\citep{Ironfleet}},

%%% In present of lambda expressions in
%%% the specifications, our translation
%%% can (optionally) insert extensionality
%%% axioms that express $\alpha$- and
%%% $\beta$-equivalence in the logic,
%%% but do not preserve undecidability
%%% and thus unpredictability of type checking.

Structure
* Introduction
* Overview
* Refinement Reflection
* Algorithmic Checking
* Proof Combinator Library
* Evaluation
* Related Work
--------------------------------------------------------------------------------
Dependently typed languages like demonstrate the
immense potential of writing proofs \emph{of}
programs, \emph{by} programs.
%
They enable developers to \emph{specify} correctness
properties and to \emph{verify} those properties, by
writing both the propositions and the proofs as
ordinary terms in the language, thereby uniting
programming and proving into complementary parts
of a whole.
%
Unfortunately, so far, these benefits require one
to work in one of these specially designed languages.
%
Wouldn't it be great to program proofs and write
proofs of programs in \emph{your} favorite language
and libraries, and have them be built and executed
using \emph{your} favorite compilers and run-times?



We used \libname as an assistant
to state and prove theorems
of Haskell functions
on Integers and inductive data types (Lists and Maybe).
%
In this section, we present by examples how \libname
can be used to prove properties on linear arithmetic
(\eg @fib@ is increasing), higher order properties on
integers and inductive data types (\eg generalization
of increasing property and map fusion), and interpretation
of $\lambda$-terms (\eg associativity of monadic bind).

\spara{Proving Strategies.}
%
The proofs reply on four main proving strategies.
We use color notations to make the proofs more readable.
%
\begin{itemize}
\item\textbf{Theorem application.}
We prove
@f op! g ?  thm e1 .. en@
when the theorem @thm x1 ... xn@
is of the form $(\ttf\ \op ! \ \ttg) \subst{\overline{x}}{\overline{e}}$.
\item \textbf{Definition folding.}
Folding the definition of @f e1 ... en == g@ once
proves that
@rf e1 ... e2 ==! g@
(in which case we color @f@ with red).
%
\item \textbf{Definition unfolding.}
Unfolding the definition of @f e1 ... en == g@ once
proves that
@g ==! gf e1 ... e2@
(in which case we color @f@ with green).
\item \textbf{SMT knowledge}
We rely on SMT interpretations to prove
linear arithmetic properties (\eg @1 < 2@)
or $\eta$-reduction (\eg @e = (\x -> e) x@).
\end{itemize}

\subsection{Arithmetic Properties}
%
We start by proving arithmetic properties
of Haskell functions.  @fib_incr@ states
and proves that the @fib@ function
from \S~\ref{sec:intro} is increasing.
%
\begin{code}
  fib_incr :: n:Nat -> {fib n <= fib (n+1)}
  fib_incr n
    | n == 0
    =   rfib 0
    <!  gfib 1
    *** QED
    | n == 1
    =   fib 1
    <=! fib 1 + fib 0
    <=! gfib 2
    *** QED
    | otherwise
    =   rfib n
    ==! fib (n-1) + fib (n-2)
    <=! fib n + fib (n-2) ? fib_incr (n-1)
    <=! fib n + fib (n-1) ? fib_incr (n-2)
    <=! gfib (n+1)
    *** QED
\end{code} %$
Proofs are total and terminating Haskell functions.
@fib_incr@ proves that @fib n@ is increasing using three cases.
1. If @n==0@,
  then by unfolding @fib@ we get @fib 0 == 0@.
  SMT proves that $0 < 1$, which in turn is folded to @1 == fib 1@.
2. If @n==1@, since @fib@ always returns @Nat@, SMT proves that @fib 1 <= fib 1 + fib 0@
 which is folded to @fib 2@.
3. Otherwise,
  @fib n@ is unfolded to @fib n = fib (n-1) + fib (n-2)@.
  We recursively apply the theorem to the (provably) smaller arguments @n-1@ and @n-2@,
  and fold the result to get @fib n + fib (n-1) = fib (n+1)@.

\subsection{Higher Order Theorems}\label{subsec:higherorder}
Next, we state and prove a generalization of increasing functions.
Namely, that for every function @f@,
if there is a proof that @f@ is increasing, that is,
an expressions with type @z:Nat -> {f z <= f (z+1)}@,
then, for every @x:Nat@ and @y@ greater than @x@,
we prove that @f x <= f y@.
%
\begin{code}
  type Greater N = {v:Int | N < v}

  gen_incr :: f:(Nat -> Int)
           -> (z:Nat -> {f z <= f (z+1)})
           -> x:Nat
           -> y:Greater x
           -> {f x <= f y}
           / [y]
  gen_incr f thm x y
    | x + 1 == y
    =   f x
    <=! f (x+1) ? thm x
    <=! f y
    *** QED
    | x + 1 < y
    =   f x
    <=! f (y-1) ? gen_incr f thm x (y-1)
    <=! f y     ? thm (y-1)
    *** QED
\end{code}
%
We prove the theorem by induction on @y@,
which is stated by the termination metric~\citep{Vazou15} @[y]@.
%
If @x+1 == y@, then we apply the @thm@ argument.
Otherwise, @x+1<y@ and the theorem holds by induction on @y@.

We use the above theorem to directly prove
that @fib@ in increasing with the generalized notion:
%
\begin{code}
  fib_incr_gen :: n:Nat
               -> m:Greater n
               -> {fib n <= fib m}
  fib_incr_gen = gen_incr_eq fib fib_incr
\end{code}


\RJ{FIXME, moved from examples}
\subsection{Example: Reflecting Lists} \label{subsec:list}

%
In the rest of this section, we use \libname to prove
properties about a user defined inductive List data type
\begin{code}
  data L [length] a = N | C a (L a)
\end{code}
%
The annotation @[length]@ on the List definition
states that \liquidHaskell will use the @length@ of lists
as a default termination metric on functions inductive on lists.
%
Where @length@ is defined as expected to map lists to natural numbers
\begin{code}
  measure length
  length :: L a -> Nat
  length N        = 0
  length (C _ xs) = 1 + length xs
\end{code}
%
The @measure@ definition (as defined in~\citep{Vazou14})
translates the @length@ function to refinements in the list's
data constructors.
%
Specifically, the above definition strengthens list's data constructors
as
\begin{code}
 N :: {v:L a | length v == 0 }
 C :: x:a -> xs:L a
   -> {v:L a | length v == 1 + length xs }
\end{code}

Moreover, the \liquidHaskell flag @exact-data-constructors@,
automatically generates the
logical record selectors and checkers for lists.
%
Specifically, the following measures will get generated:
\begin{code}
  is_N N          = True
  is_N (C _ _)    = False

  is_C (C _ _)    = True
  is_C N          = False

  sel_C_1 (C x _) = x
  sel_C_2 (C _ x) = x
\end{code}

\subsection{Append Associativity}\label{subsec:append}
Next we recursively define and axiomatize list append
\begin{code}
  axiomatize (++)
  (++) :: L a -> L a -> L a
  N        ++ ys = ys
  (C x xs) ++ ys = C x (xs ++ ys)
\end{code}
%
The above axiomatization, will automatically produce a type for append
that exactly captures its implementation.
%
\begin{code}
  (++) :: xs:L a -> ys:L a ->
    {v:L a |
       v == if (is_N xs) then ys
            else C (sel_C_1 xs)
                   (sel_C_2 xs ++ ys) }
\end{code}
%
The formal translation of arbitrary terminating functions
is formalized in \S~\ref{sec:algorithmic}.
