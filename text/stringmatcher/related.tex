\section{Related Work}\label{sec:related}


\paragraph{SMT-Based Verification}
%
SMT solvers have been extensively used to automate
reasoning on verification languages like
Dafny~\cite{dafny}, Fstar~\cite{fstar} and Why3~\cite{why3}.
%
These languages are designed for verification,
thus have limited support for commonly used language
features like parallelism and optimized libraries
that we use in our verified implementation.
%
Refinement Types~\cite{ConstableS87,FreemanPfenningDONTCITE91,Rushby98}
on the other hand, target existing general purpose languages,
such as
ML~\cite{pfenningxi98,GordonRefinement09,LiquidPLDI08},
C~\cite{deputy,LiquidPOPL10},
Haskell~\cite{Vazou14},
Racket~\cite{RefinedRacket}
and Scala~\cite{refinedscala}.
However, before Refinement Reflection~\cite{reflection} was introduced,
they only allowed ``shallow'' program specifications,
that is, properties that only talk about behaviors of program functions
but not functions themselves.
%

\paragraph{Dependent Types}
Unlike Refinement Types, dependent type systems,
like Coq~\cite{coq-book}, Adga~\cite{agda} and Isabelle/HOL~\cite{isabelle} allow for ``deep'' specifications
which talk about program functions,
such as the program equivalence reasoning we presented.
%
Compared to (Liquid) Haskell,
these systems allow for tactics and heuristics
that automate proof term generation
but lack SMT automations and
general-purpose language features,
like non-termination, exceptions and IO.
%
Zombie~\cite{Zombie,Sjoberg2015} and Fstar~\cite{fstar} allow dependent types to
coexist with divergent and effectful programs,
but still lack the optimized libraries,
like @ByteSting@, which come
with a general purpose language
with long history, like Haskell.



\paragraph{Parallel Code Verification}
Dependent type theorem provers have been used before to
verify parallel code.
%
BSP-Why~\cite{bspwhy} is an extension to Why2 that is using both Coq and SMTs
to discharge user specified verification conditions.
%
Daum~\cite{daum07} used Isabelle to formalize the semantics
of a type-safe subset of C, 
by extending Schirmer's~\cite{schirmer06}
formalization of sequential imperative languages.
%
Finally, Swierstra~\cite{wouter10} formalized mutable arrays in Agda
and used them to reason about distributed maps and sums.

One work  closely related to ours is
SyDPaCC~\cite{SyDPaCC}, a Coq library that
automatically parallelizes list homomorphisms
by extracting parallel Ocaml versions of user provided Coq functions.
%
Unlike SyDPaCC, we are not automatically generating the parallel
function version, because of engineering limitations
(\S~\ref{sec:evaluation}).  Once these are addressed, code extraction
can be naturally implemented by turning
Theorem~\ref{theorem:two-level} into a Haskell type class with a
default parallelization method.
%
SyDPaCC used maximum prefix sum as a case study,
whose morphism verification is
much simpler than our string matching case study.
%
Finally, our implementation consists of
2K lines of Liquid Haskell, which we consider verbose and aim to
use tactics to simplify.
On the contrary, the SyDPaCC implementation
requires three different languages:
2K lines of Coq with tactics, 600 lines of Ocaml and 120 lines of C,
and is considered ``very concise''.
