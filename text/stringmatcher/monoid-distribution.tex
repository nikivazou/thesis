\section{Verification of Distribution over Monoid Concatenation}~\label{sec:monoid:distribution}
In this section we prove distribution over monoid concatenation. 
%
Concretely, if a function @f :: i -> m@
from some input @i@ to a monoid @m@, 
distributes on the input
@f (i1 stringMappend i2) == (f i1) mappend (f i2)@, 
then @f@ distributes over monoid concatenation: 
@f i == mconcat (map f (chunk n i))@. 
%
For the above property to hold, the input @i@
should satisfy certain properties, 
for example, the input should define the methods
@stringMappend@ and @chunk@. 
%
First~\S~\ref{subsec:rstring} we defined Refined Strings 
as an example of proper inputs, and then~\S~\ref{subsec:distribution}
we prove distribution over Monoid Concatenation on Refined Strings.  


\subsection{Refined Strings}\label{subsec:rstring}
We represent the input as a Refined String 
defined at a library @RString.hs@
with implementation irrelevant for the proofs.  
%
\NV{introduce RString here is good, but implementation and constact time indexing seems irrelevant here}
\begin{code}
data RString
\end{code}
%
@RString.hs@
defines the monoid operators @stringMempty@ and @(stringMappend)@
that satisfy the monoid laws. 
%
Moreover, it defines string manipulation operators
that come refined via various theorems. 
%
For example, the library exposes the
operators @takeStr@ and @dropStr@
\begin{code}
takeStr :: i:Nat -> xs:{RString | i <= stringLen xs } 
  -> {v:RString | stringLen v == i }
dropStr :: i:Nat -> xs:{RString | i <= stringLen xs } 
  -> {v:RString | stringLen v == stringLen xs - i}
\end{code}
%
and asserts that appending a list split via @take@ and @drop@
exactly reconstructs the list.
\begin{code}
concatTakeDrop 
  :: i:Nat -> xs:{RString | i <= stringLen xs} 
  -> {xs == takeStr i xs stringMappend dropStr i xs }
\end{code}
%
The operators defined in @RString@ are already reflected in the logic. 
%
Thus we use them to further define operators
that we reflect and use in the theorems. 
%
For example, @takeStr@ and @dropStr@ are used to define 
@chunkStr@ 
\begin{code}
chunkStr i xs 
  | i <= 1 ||stringLen xs <= i 
  = C xs N 
  | otherwise
  = C (takeStr i xs) (chunkString i (dropStr i xs))
\end{code}
%
We reflect @chunkStr@ into logic and use it to define and prove 
distribution over monoid concatenation. 


\subsection{Distribution over Monoid Concatenation on Refined Strings}\label{subsec:distribution}
%
We use the refinement type specification for @monoidDirstibution@
to specify and prove  that for each monoid @m@, 
if the function @f:RString -> m@ distributes over @RString@
then it distributes over monoids. 
%
Note that the assumption of the theorem, 
is expressed in the specifications as a functional argument 
that makes the proper assumptions for @f@. 
%
To prove distribution over monoids for a specific @f@, 
we need to invoke @monoidDirstibution@ 
on @f@ and a proper proof argument that @f@ distributes over @RString@.
%
\begin{theorem}[Monoid Distribution]\label{theorem:monoid:distribution}
If @m@ is a monoid, then each function @f:RString -> m@
that distributes over @RString@, distributes over monoid concatenation.
%
\begin{code}
monoidDirstibution
  :: f:(RString -> m)
  -> (x1:_ -> x2:_ -> {f (x1 stringMappend x2) == (f x1) <> (f x2)} )
  -> is:RString
  -> n:Int 
  -> {f is == mconcat (map f (chunkStr n is))}
\end{code}
\end{theorem}

\begin{proof}
The proof proceeds by induction on the length of the input.
%
\begin{code}
monoidDirstibution f thm is n  
  | stringLen is <= n || n <= 1
  =   mconcat (map f (chunkStr n is))
  ==. mconcat (map f (C is N))
  ==. mconcat (f is `C` map f N)
  ==. f is <> mconcat N
  ==. f is <> mempty 
  ==. f is ? idRight (f is)
  *** QED 
monoidDirstibutionf f thm is n  
  =   mconcat (map f (chunkStr n is))
  ==. mconcat (map f (C takeIs (chunkStr n dropIs))) 
  ==. mconcat (f takeIs `C` map f (chunkStr n dropIs))
  ==. f takeIs <> f dropIs
       ? monoidDirstibution f thm dropIs n  
  ==. f (takeIs <+> dropIs)
       ? thm takeIs dropIs
  ==. f is 
       ? concatTakeDrop n is 
  *** QED  
  where
    dropIs = dropStr n is 
    takeIs = takeStr n is 
\end{code}
%
In the base case we use rewriting and right identity on the monoid @f is@. 
%
In the inductive case, 
we use the inductive hypothesis on the input @dropIs = dropStr n is@, 
that is provably smaller than @is@ as @n > 1@. 
%
Then, by our assumption argument @thm takeIs dropIs@
we get basic distribution of @f@, that is 
@f takeIs <> f dropIs == f (takeIs stringMappend dropIs)@. 
%
Finally, we merge @takeIs stringMappend dropIs@ to @is@
using the property @concatTakeDrop@ take is exported by the @RString@
library.
\qed\end{proof}