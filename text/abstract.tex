Code deficiencies and bugs constitute an unavoidable part of software systems.
%
In safety-critical systems, like aircrafts or medical equipment, 
even a single bug can lead to catastrophic impacts
such as injuries or death.
%
Formal verification can be used to statically 
track code deficiencies by proving or disproving correctness properties 
of a system. 
%
However, at its current state formal verification is a cumbersome process
that is rarely used by mainstream developers, mostly because it targets non general purpose languages (e.g., Coq, Agda, Dafny).

We present \toolname, a \textit{usable} program verifier that aims to
establish formal verification as an
integral part of the development process.
%
\toolname \textit{naturally integrates}
the specification of correctness properties
as logical refinements of Haskell's types. 
%
Moreover, it uses the abstract interpretation framework of liquid types
to \textit{automatically} check correctness of specifications via 
Satisfiability Modulo Theories (SMT) solvers
requiring no explicit proofs or complicated annotations.
%
Finally, the specification language is arbitrary \textit{expressive},
allowing the user to write general correctness properties about their code, 
thus turning Haskell into a theorem prover. 

Transforming a mature language 
--- with optimized libraries and highly tuned parallelism ---
into a theorem prover enables us to verify a wide variety of properties 
on real world applications.
%
We used \toolname to verify shallow invariants of existing Haskell code, 
\eg memory safety of the optimized string manipulation library @ByteString@.
%
Moreover, we checked deep, sophisticated properties of parallel Haskell code, 
\eg program equivalence of a na\"ive string matcher and its parallelized version. 
%
Having verified about 20K of Haskell code, we present how \toolname 
serves as a prototype verifier in a future where formal techniques will 
be used to facilitate, instead of hinder, software development. 