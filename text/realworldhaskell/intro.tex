% \section{Introduction}\label{sec:introduction}
Refinement types enable specification of complex invariants 
by extending the base type system with \emph{refinement predicates} 
drawn from decidable logics. For example,
%
\begin{code}
  type Nat = {v:Int | 0 <= v}
  type Pos = {v:Int | 0 < v}
\end{code}
%
are refinements of the basic type @Int@ with a logical predicate 
that states the \emph{values} @v@ being described must be 
\emph{non-negative} and \emph{postive} respectively. 
%
We can specify \emph{contracts} of functions by refining function types. 
For example, the contract for @div@
%
\begin{code}
  div :: n:Nat -> d:Pos -> {v:Nat | v <= n}
\end{code}
%
states that @div@ \emph{requires} a non-negative dividend @n@ and a positive
divisor @d@ and \emph{ensures} that the result is less than the dividend.
%
If a program (refinement) type checks, we can be sure that @div@ will never 
throw a divide-by-zero exception.

Refinement types \citep{ConstableS87,Rushby98} 
have been implemented for several languages like
ML~\cite{pfenningxi98,GordonTOPLAS2011,LiquidPLDI08},
C~\cite{deputy,LiquidPOPL10},
TypeScript~\cite{Vekris16},
Racket~\cite{RefinedRacket} and Scala~\cite{refinedscala}.
%
Here we present \toolname,
a refinement type checker for Haskell.
%
In this chapter we start with an example driven informal and practical overview
of \toolname.
%
In particular, we try to answer the following questions:
%
\begin{enumerate}
  \item What properties can be specified with refinement types?
  \item What inputs are provided and what feedback is received?
  \item What is the process for modularly verifying a library?
  \item What are the limitations of refinement types? 
\end{enumerate}

We attempt to investigate these questions, by using the
refinement type checker \toolname, to specify and verify a variety of 
properties of over 10,000 lines of Haskell code from popular 
libraries, including @containers@, \hbox{@hscolor@,} @bytestring@, @text@, 
@vector-algorithms@ and @xmonad@. 
%
\begin{itemize}
\item First (\S~\ref{sec:liquidhaskell}), 
we present a high-level overview of \toolname, through a tour 
of its features.
%

\item Second, we present a qualitative discussion of the kinds of properties
that can be checked -- ranging from generic application independent 
criteria like totality (\S~\ref{sec:totality}), 
\ie that a function is defined for all inputs (of a given type)
and termination, 
(\S~\ref{sec:termination}) 
\ie that a recursive function cannot diverge,
to application specific concerns like memory safety (\S~\ref{sec:memory-safety}) 
and functional correctness properties (\S~\ref{sec:structures}).
%
\item Finally (\S~\ref{sec:realworld:evaluation}), we present a quantitative evaluation of the approach, with a view
towards measuring the efficiency and programmer's effort required for
verification, 
and we discuss various limitations of the approach which could
provide avenues for further work.
\end{itemize}
