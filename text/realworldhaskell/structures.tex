\newcommand\lbxmonad{\texttt{xmonad}\xspace}

\section{Functional Correctness Invariants}\label{sec:structures}

So far, we have considered a variety of general, application independent
correctness criteria. Next, let us see how we can use \toolname to specify 
and statically verify critical, application specific correctness properties,
using two illustrative case studies: red-black trees and the stack-set data
structure introduced in the \lbxmonad system.

\subsection{Red-Black Trees}\label{sec:redblack}

Red-Black trees have several non-trivial invariants that are ideal for 
illustrating the effectiveness of refinement types and contrasting with
existing approaches based on GADTs~\cite{Kahrs01}.
%
The structure can be defined via the following Haskell type:
%
\begin{code}
  data Col    = R | B
  data Tree a = Leaf 
              | Node Col a (Tree a) (Tree a)
\end{code}
%
However, a @Tree@ is a valid Red-Black tree only if it 
satisfies three crucial invariants:
%
\begin{itemize}
  \item{\emphbf{Order:}} 
    The keys must be binary-search ordered, \ie the key at each node must
    lie between the keys of the left and right subtrees of the node,
  \item{\emphbf{Color:}}
    The children of every \emph{red} @Node@ must be colored \emph{black}, 
    where each @Leaf@ can be viewed as black,
  \item{\emphbf{Height:}}
    The number of black nodes along any path from each @Node@ to its @Leaf@s 
    must be the same.
\end{itemize}

Red-Black trees are especially tricky as various operations create 
trees that can temporarily violate the invariants. Thus, while 
the above invariants can be specified with singletons and GADTs, 
encoding all the properties (and the temporary violations) results
in a proliferation of data constructors that can somewhat obfuscate 
correctness. In contrast, with refinements, we can specify and verify
the invariants in isolation (if we wish) and can trivially compose
them simply by \emph{conjoining} the refinements.

\mypara{Color Invariant}
To specify the color invariant, we define a \emph{black-rooted tree} as:
%
\begin{code}
  measure isB           :: Tree a -> Prop 
  color (Node c x l r)  = c == B
  color (Leaf)          = True
\end{code}
%
and then we can describe the color invariant simply as:
%
\begin{code}
  measure isRB        :: Tree a -> Prop
  isRB (Leaf)         = True
  isRB (Node c x l r) = isRB l && isRB r &&
                        c = R => (isB l && isB r)
\end{code}
%
The insertion and deletion procedures create intermediate \emph{almost} 
red-black trees where the color invariant may be violated at the root. 
Rather than create new data constructors we define almost red-black 
trees with a measure that just drops the invariant at the root:
%
\begin{code}
  measure almostRB        :: Tree a -> Prop
  almostRB (Leaf)         = True
  almostRB (Node c x l r) = isRB l && isRB r
\end{code}

\mypara{Height Invariant}
To specify the height invariant, we define a black-height measure:
%
\begin{code}
  measure bh        :: Tree a -> Int
  bh (Leaf)         = 0
  bh (Node c x l r) = bh l + if c = R then 0 else 1
\end{code}
%
and we can now specify black-height balance as:
%
\begin{code}
  measure isBal        :: Tree a -> Prop
  isBal (Leaf)         = true
  isBal (Node c x l r) = bh l = bh r 
                       && isBH l && isBH r 
\end{code}
%
Note that @bh@ only considers the left sub-tree, 
but this is legitimate, because @isBal@ will 
ensure the right subtree has the same @bh@.

\mypara{Order Invariant}
We refine the data definition of @Tree@ 
to encode the ordering property:
%
\begin{code}
  data Tree a
    = Leaf
    | Node { c   :: Col
           , key :: a
           , lt  :: Tree {v:a | v < key }
           , rt  :: Tree {v:a | key < v } }
\end{code}
%

\mypara{Composing Invariants}
Finally, we can compose the invariants and define a 
Red-Black tree with the alias:
%
\begin{code}
  type RBT a = {v:Tree a | isRB v && isBal v}
\end{code}
%
An almost Red-Black tree is the above with @isRB@ 
replaced with @almostRB@, \ie does not require any 
new types or constructors.
If desired, we can ignore a particular invariant 
simply by replacing the corresponding refinement 
above with @true@.
Given the above -- and suitable signatures \toolname 
verifies the various insertion, deletion and rebalancing
procedures for a Red-Black Tree library.

\subsection{Stack Sets in XMonad}\label{sec:xmonad}

\lbxmonad is a dynamically tiling \texttt{X11} 
window manager that is written and configured in Haskell. 
The set of windows managed by XMonad is organized into a
hierarchy of types. At the lowest level we have a 
\emph{set} of windows @a@ represented as a @Stack a@
%
\begin{code}
  data Stack a = Stack { focus :: a   
                       , up    :: [a] 
                       , down  :: [a] }
\end{code}
%
The above is a zipper~\cite{zipper} where @focus@ is the 
``current" window and @up@ and @down@ the windows ``before"
and ``after" it.
%
Each \hbox{@Stack@} is wrapped inside a @Workspace@ that has
additional information about layout and naming:
%
\begin{code}
  data Workspace i l a = Workspace 
     { tag    :: i
     , layout :: l
     , stack  :: Maybe (Stack a) }
\end{code}
%
which is in turn, wrapped inside a @Screen@:
%
\begin{code}
  data Screen i l a sid sd = Screen 
    { workspace    :: Workspace i l a
    , screen       :: sid
    , screenDetail :: sd }
\end{code}
%
The set of all screens is represented by the top-level zipper:
%
\begin{code}
  data StackSet i l a sid sd = StackSet 
    { cur :: Screen i l a sid sd  
    , vis :: [Screen i l a sid sd]
    , hid :: [Workspace i l a]   
    , flt :: M.Map a RationalRect } 
\end{code}


\mypara{Key Invariant: Uniqueness of Windows}
The key invariant for the @StackSet@ type is that each window @a@
should appear at most once in a @StackSet i l a sid sd@. That is,
a window should \emph{not be duplicated} across stacks or workspaces.
Informally, we specify this invariant by defining a measure for the 
\emph{set of elements} in a list, @Stack@, @Workspace@ and @Screen@,
and then we use that measure to assert that the relevant sets are 
disjoint.

\mypara{Specification: Unique Lists} To specify that the set of elements
in a list is unique, \ie there are no duplicates in the list we first define
a measure denoting the set using Z3's~\citep{z3} built-in theory of sets:
%
\begin{code}
  measure elts :: [a] -> Set a 
  elts ([])   = emp  
  elts (x:xs) = cup (sng x) (elts xs) 
\end{code}
%
Now, we can use the above to define uniqueness:
%
\begin{code}
  measure isUniq :: [a] -> Prop 
  isUniq ([])   =  true 
  isUniq (x:xs) =  notIn x xs && isUniq xs
\end{code}
%
where @notIn@ is an abbreviation: 
%
\begin{code}
  predicate notIn X S = not (mem X (elts S))
\end{code}

\mypara{Specification: Unique Stacks}
We can use @isUniq@ to define unique, \ie, duplicate free,
@Stack@s as:
%
\begin{code}
  data Stack a = Stack 
   { focus :: a   
   , up    :: {v:[a] | Uniq1 v focus}
   , down  :: {v:[a] | Uniq2 v focus up} }
\end{code}
%
using the aliases
%
\begin{code}
  predicate Uniq1 V X    = isUniq V  && notIn X V
  predicate Uniq2 V X Y  = Uniq1 V X && disjoint Y V
  predicate disjoint X Y = cap (elts X) (elts Y) = emp
\end{code}
%
\ie the field @up@ is a unique list of elements different 
from @focus@, and the field @down@ is additionally disjoint 
from @up@.

\mypara{Specification: Unique StackSets}
It is straightforward to lift the @elts@ measure to 
the @Stack@ and the wrapper types @Workspace@ and 
@Screen@, and then correspondingly lift @isUniq@ to 
@[Screen]@ and \hbox{@[Workspace]@.}
%
Having done so, we can use those measures to refine 
the type of @StackSet@ to stipulate that there are 
no duplicates:
%
\begin{code}
  type UniqStackSet i l a sid sd 
    = {v: StackSet i l a sid sd | NoDups v} 
\end{code}
%
using the predicate aliases
%
\begin{code}
  predicate NoDups V 
    =  disjoint3 (hid V) (cur V) (vis V) 
    && isUniq (vis V) && isUniq (hid V)
  
  predicate disjoint3 X Y Z 
    =  disjoint X Y && disjoint Y Z && disjoint X Z
\end{code}
%
\toolname automatically turns the record selectors of 
refined data types to measures that return the values 
of appropriate fields, hence @hid x@ (resp. @cur x@, @vis x@)
are the values of the \hbox{@hid@,} @cur@ and @vis@ fields of
a @StackSet@ named @x@.

%%%% \begin{code}
%%%% data Stack a = Stack { focus :: a   
%%%%                      , up    :: ULNEq a focus
%%%%                      , down  :: ULNEq a focus }
%%%% 
%%%% data StackSet i l a sid sd = StackSet 
%%%%    { lcurrent  ::  Screen i l a sid sd   
%%%%    , lvisible  :: [Screen i l a sid sd]
%%%%    , lhidden   :: [Workspace i l a]
%%%%    , lfloating :: M.Map a RationalRect     
%%%%    }
%%%% \end{code}
%%%% %
%%%% data Stack a = Stack { focus  :: !a   
%%%%                      , up     :: [a]   
%%%%                      , down   :: [a] } 
%%%%                                   
%%%% data Stack a = Stack { focus :: a   
%%%%                      , up    :: UListDif a focus
%%%%                      , down  :: UListDif a focus }
%%%% 
%%%% data Workspace i l a = Workspace  { tag :: !i, layout :: l, stack :: Maybe (Stack a) }
%%%% 
%%%% data Workspace i l a = Workspace  { tag :: i, layout :: l, stack :: Maybe (UStack a) }
%%%% 
%%%% type UStack a = {v:(Stack a) | (ListDisjoint (getUp v) (getDown v))}
%%%% 
%%%% 
%%%% data Screen i l a sid sd = Screen { workspace     :: !(Workspace i l a)
%%%%                                   , screen        :: !sid
%%%%                                   , screenDetail  :: !sd }
%%%% 
%%%% data StackSet i l a sid sd =
%%%%     StackSet { current  :: !(Screen i l a sid sd)    -- ^ currently focused workspace
%%%%              , visible  :: [Screen i l a sid sd]     -- ^ non-focused workspaces, visible in xinerama
%%%%              , hidden   :: [Workspace i l a]         -- ^ workspaces not visible anywhere
%%%%              , floating :: M.Map a RationalRect      -- ^ floating windows
%%%%              } 
%%%% 
%%%% A workspace is just a @Stack@ of virtual workspaces 
%%%% tagged with a tag @i@ and its layout @l@
%%%% @Workspace i l a@.
%%%% 
%%%% To view a workspace on a physical screen one needs to 
%%%% associate the workspace with physical screen's id @sid@
%%%% and details @sd@, 
%%%% forming the new data structure @Screen i l a sid sd@.
%%%% 
%%%% Xinerama in X11 allows viewing multiple virtual workspaces
%%%% simultaneously. 
%%%% %
%%%% While only one the current one will ever be in focus (i.e. will
%%%% receive keyboard events), other workspaces may be passively
%%%% viewable.  
%%%% %
%%%% We thus need to track which virtual workspaces are
%%%% associated (viewed) on which physical screens.  
%%%% %
%%%% To keep track of
%%%% this, \lbxmonad's main data structure  @StackSet i l a sid sd@ 
%%%% keeps, apart from the current screen,
%%%% separate lists of visible but non-focused
%%%% workspaces (@Screen@) , and non-visible workspaces (@Workspace@).

%%%% \mypara{Unique Stack} 
%%%% Assume the existence of a predicate 
%%%% @ULNeq (X::a) (XS::[a])@ that ensures that 
%%%% (1)~the list @XS@ has no duplicates and
%%%% (2)~@X@ is not an element of @XS@.
%%%% %
%%%% With that magical predicate we define an \emph{almost unique}
%%%% stack: 
%%%% \begin{code}
%%%% data Stack a = Stack { focus :: a   
%%%%                      , up    :: ULNEq a focus
%%%%                      , down  :: ULNEq a focus }
%%%% \end{code}
%%%% The stack has a @focus@ element @a@ and 
%%%% and two unique lists of @a@'s @up@ and @down@ of the focus.
%%%% 
%%%% The above Stack is almost unique, as an element 
%%%% may appear both in the @up@ and the @down@ lists.
%%%% %
%%%% We define a type alias for @U@nique@Stack@
%%%% that rejects with possibility:
%%%% 
%%%% \begin{code}
%%%% type UStack a = {v:(Stack a) |  (LDisjoint (up v) (down v))}
%%%% 
%%%% predicate LDisjoint X Y = 
%%%%   (Set_emp (Set_cap (elts X) (elts Y)))
%%%% \end{code}
%%%% 
%%%% 
%%%% Note that the above definitions crucially depend on set theoretic properties.
%%%% Thus, verification of \lbxmonad is achieved using an SMT back-end 
%%%% that supports set theory (like Z3~\citep{z3}).
%%%% 
%%%% We slightly modify the above measure definition
%%%% to define @dups@, a measure that returns the 
%%%% duplicate elements of a list.
%%%% %
%%%% \begin{code}
%%%%   measure dups :: [a] -> Set a
%%%%   dups ([])  = emp v
%%%%   dups(x:xs) = if mem x (elts xs) 
%%%%                  then cup (sng x) (dups xs)
%%%%                  else (dups xs)
%%%% \end{code}
%%%% 
%%%% Using @dups@ we define the magical list type @ULNEq a N@ 
%%%% as a list @v@ that has no duplicates, \ie the set @dups v@
%%%% is empty, and @N@ does not belong to the set @elts v@.
%%%% 
%%%% \begin{code}
%%%% type ULNEq a N = {v:[a] | ( (UL v) && (not (LElt N v))}
%%%% predicate LElt N LS  = (Set_mem N (elts LS)) 
%%%% predicate UL     LS  = (Set_emp (dups LS)   )
%%%% \end{code}
%%%% 
%%%% Throughout verification
%%%% we need to establish and use the invariant 
%%%% that each Stack is Unique.
%%%% %
%%%% This is achieved by the following annotation
%%%% \begin{code}
%%%% using (Stack a) as (UStack a)
%%%% \end{code}
%%%% that allows \toolname to 
%%%% (1)~ use the invariant:
%%%% each time a Stack value is retrieved from the environment
%%%% \toolname strengthens its type with the disjointness information;
%%%% (2)~ prove the invariant:
%%%% each time @Stack@ data constructor is used,
%%%% an disjoint constraint should be proved.
%%%% Failure to prove this constraint will raise an 
%%%% ``Invariant Check'' error.
%%%% 
%%%% \mypara{Unique StackSet} 
%%%% Establishing Uniqueness on StackSets is a generalization of the above procedure.
%%%% 
%%%% The definition of a @StackSet@ includes the @current@ Screen, the list of @visible@ screens,
%%%% and the list of @hidden@ workspaces.
%%%% \begin{code}
%%%% data StackSet i l a sid sd = StackSet 
%%%%    { lcurrent  ::  Screen i l a sid sd   
%%%%    , lvisible  :: [Screen i l a sid sd]
%%%%    , lhidden   :: [Workspace i l a]
%%%%    , lfloating :: M.Map a RationalRect     
%%%%    }
%%%% \end{code}
%%%% %
%%%% \toolname automatically turns the record selectors of refined data types
%%%% to measures that return the appropriate fields.
%%%% %
%%%% Thus we infix the refined selectors with an @l@
%%%% to distinguish between the haskell (\eg, @current@)
%%%% and the logical (\eg, @lcurrent@) selectors. 
%%%% 
%%%% To prove absence of duplicates we need to @use@
%%%% only @StackSet@s that have no duplicates:  
%%%% \begin{code}
%%%% using (StackSet i l a sid sd) 
%%%%  as  {v:StackSet i l a sid sd|(NoDuplicates v)}
%%%% \end{code}
%%%% 
%%%% @NoDuplicates@ ensures that the elements of
%%%% @hidden@, @current@, and @visible@ 
%%%% workspaces are mutually disjoint
%%%% and that the @visible@ and @hidden@ workspaces
%%%% have no duplicates.
%%%% %
%%%% \begin{code}
%%%% predicate NoDuplicates SS = 
%%%%     (Disjoint3  (workspacesElts (lhidden  SS)) 
%%%%                 (screenElts     (lcurrent SS)) 
%%%%                 (screensElts    (lvisible SS))) 
%%%%   &&
%%%%     (Set_emp (screensDups    (lvisible SS))) 
%%%%   &&
%%%%     (Set_emp (workspacesDups (lhidden  SS)))
%%%% \end{code}
%%%% %
%%%% Again we used recursively defined measures to grap 
%%%% the elements and the duplicates of the structures.
%%%% For example, @screenElts@ returns 
%%%% the elements of the stack of the workspace of the screen, 
%%%% and is used by @screensDup@ to grap the duplicates of 
%%%% a list of Screens.
%%%% 
\mypara{Verification}
\toolname uses the above refined type to verify the key invariant,
namely, that no window is duplicated.
%
% However, the verification process, while eventually successful, 
% revealed several limitations of our approach.
%
Three key actions of the, eventually successful, verification process
can be summarized as follows:
\begin{itemize}
\item\emph{Strengthening library functions.} 
  \lbxmonad repeatedly concatenates the lists of a Stack. %edit to fix overful box
  %
  To prove that for some \hbox{@s:Stack a@,} @(up s ++ down s)@ is a unique list,
  the type of @(++)@ needs to capture that concatenation of two unique and
  disjoint lists is a unique list.
  %
  For verification, we assumed that Prelude's @(++)@ satisfies this property.
  %
  But, not all arguments of @(++)@ are unique disjoint lists:
  @"StackSet" ++ "error"@ is a trivial example that does not satisfy
  the assumed preconditions of @(++)@ thus creating a type error.
  % 
  Currently, \toolname does not support intersection types, 
  thus we used an unrefined @(++.)@ variant of @(++)@ for such cases.
     
\item\emph{Restrict the functions' domain.}
%% \RJ{Seems like you HAVE to do this -- has nothing to do with LH?}
  @modify@ is a @maybe@-like function that given a default value @x@,
  a function @f@, and a StackSet @s@, applies @f@ on the @Maybe@
  values inside @s@. 
  %
\begin{code}
modify :: x:{v:Maybe (Stack a) | isNothing v}
       -> (y:Stack a -> Maybe {v:Stack a | SubElts v y})
       -> UniqStackSet i l a s sd 
       -> UniqStackSet i l a s sd
\end{code}
	%
        Since inside the StackSet @s@ each @y:Stack a@ could be replaced
    with either the default value @x@ or with @f y@, we need to
    ensure that both these alternatives will not insert duplicates.
	%
	This imposes the curious precondition that the default
	value should be @Nothing@.

			
	\item\emph{Code inlining}
    %
    Given a tag @i@ and a StackSet @s@,  @view i s@ will set the current Screen 
    to the screen with tag @i@, if such a screen exists in @s@.
    %
    Below is the original definition for @view@ in case when a screen with tag 
    @i@ exists in visible screens
    %
\begin{code}
view :: (Eq s, Eq i) => i 
     -> StackSet i l a s sd 
     -> StackSet i l a s sd
view i s    
  | Just x <- find ((i==).tag.workspace) (visible s)
  = s { current = x
      , visible = current s 
                : deleteBy (equating screen) x (visible s) } 
\end{code}
    %
    Verification of this code is difficult as we cannot suitably type @find@. 
    %
    Instead we \emph{inline} the call to @find@ and the field update into a 
    single recursive function @raiseIfVisible i s@ that in-place replaces @x@ 
    with the current screen.  
\end{itemize}

Finally, \lbxmonad comes with an extensive suite of QuickCheck properties,
that were formally verified in Coq~\cite{Swierstra2012}. In future work~\ref{chapter:conclusion},
it would be interesting to do a similar verification with \toolname, 
to compare the refinement types to proof-assistants.

%%% TODO \subsection{QuickCheck Properties}
%%% TODO 
%%% TODO \lbxmonad is tested against $113$ \texttt{quickcheck} properties.
%%% TODO %
%%% TODO Of those $15$ check the uniqueness invariant 
%%% TODO and the rest $113$ check various functional properties.
%%% TODO %
%%% TODO We started the endeavour of verifying these properties with \toolname.
%%% TODO %
%%% TODO We looked at a sample of $15$ properties to conclude that
%%% TODO which we categorized as follows:
%%% TODO \begin{itemize}
%%% TODO \item\emph{Easy to be proved.}
%%% TODO Consider the \texttt{quickcheck} property that checks that @view@ing 
%%% TODO a @StackSet a@ is idempotent:
%%% TODO \begin{code}
%%% TODO prop_view_idem (x :: T) (i :: NonNegative Int) 
%%% TODO   = i `tagMember` x ==> view i (view i x) == (view i x)
%%% TODO \end{code}
%%% TODO %
%%% TODO The above property directly translated to a haskell function
%%% TODO \begin{code}
%%% TODO type Valid     = {v:Bool | (Prop v) }
%%% TODO 
%%% TODO prop_view_idem :: StackSet i l a sid sd -> i -> Valid
%%% TODO prop_view_idem x i 
%%% TODO   | i `tagMember` x = view i (view i) == v
%%% TODO   | otherwise       = True
%%% TODO \end{code}
%%% TODO %
%%% TODO By the above type signature,
%%% TODO \ie by the result type @Valid@, 
%%% TODO we specify that the function should always returns True.
%%% TODO %
%%% TODO When typechecking the above function,
%%% TODO \toolname proves that the property holds.
%%% TODO %
%%% TODO \toolname is able to verify this property as the result type of @view@
%%% TODO is strengthens with a refinement 
%%% TODO (in this case @EqTag x i => x = v@) 
%%% TODO that directly implies this property.
%%% TODO 
%%% TODO The above is generalizing to (10/17) properties that we checked:
%%% TODO strengthening the function types by refinements that \toolname can prove
%%% TODO is sufficient to verify these properties.
%%% TODO 
%%% TODO \item\emph{Can be estimated.}
%%% TODO In some other properties (like checking that @view@ing is reversible),
%%% TODO two StackSets (@s1@ and @s2@) were normalized before being compared,
%%% TODO that is their elements were first sorted.
%%% TODO %
%%% TODO In our logic we do not support any operation that can normalize structures in such a way.
%%% TODO %
%%% TODO Thus we cannot prove this exact property.
%%% TODO %
%%% TODO Instead we approximated it, by proving that proving that @s1@ and @s2@
%%% TODO have the same sets of elements.
%%% TODO %
%%% TODO We approximated (3/17) of the properties.
%%% TODO 
%%% TODO \item\emph{Their proof cannot be supported, currently.} [1]
%%% TODO One \texttt{quickcheck} property checks 
%%% TODO that @i@ cannot belong to an empty stackset.
%%% TODO %
%%% TODO We used abstract refinements to encode empty stacksets.
%%% TODO %
%%% TODO Proving the above property would be easy 
%%% TODO if we could mix abstract and concrete refinements in logical formulas
%%% TODO or if \toolname supported sum types.
%%% TODO %
%%% TODO Both the alternatives constitutes features that
%%% TODO we would like to extend \toolname with in the near future.
%%% TODO %
%%% TODO Still, currently we are not able to prove such kind of properties.
%%% TODO \item\emph{Cannot be expressed}
%%% TODO Other properties %, like @prop_focus_left_master@ 
%%% TODO check that the order of the windows is not affected by certain operations.
%%% TODO %
%%% TODO Though not infeasible we acknowledge that \toolname 
%%% TODO is not appropriate for reasoning about order preserving
%%% TODO and verification of such properties 
%%% TODO would require many code modifications.
%%% TODO %
%%% TODO (3/17) are order preserving properties.
%%% TODO \end{itemize}
