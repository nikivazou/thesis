\newcommand\lbxmonad{\texttt{xmonad}\xspace}

\NV{(R3)
 Regarding your XMonad experience, you might want to compare your approach to the work of Wouter Swierstra: "Xmonad in Coq (Experience Report)
}
\section{XMonad: From Testing to Static Verification}\label{sec:xmonad}

In this section we discuss our experience in the on-going endeavour of 
static verification of \lbxmonad.
%
We start by summarizing the data structures used in \lbxmonad, as described in the documentation.
%
We continue by specifying and proving \lbxmonad's invariant, \ie, that
virtual workspaces are unique
%
Finally, we comment on proving \lbxmonad properties to conclude 
that static verification can be not much harder than testing.

\subsection{The structure}

\lbxmonad is a dynamically tiling \texttt{X11} 
window manager that is written and configured in Haskell.

The window (@a@) is a set of virtual workspaces. 
On each workspace is a stack of windows. 
A given workspace is always current, and a given
window on each workspace has focus. 
The focused window on the current
workspace is the one which will take user input. 
%
\lbxmonad uses a Zipper~\citep{zipper} data structure 
to directly track focus on the virtual workspaces, 
and calls this structure @Stack a@.

A workspace is just a @Stack@ of virtual workspaces 
tagged with a tag @i@ and its layout @l@
@Workspace i l a@.

To view a workspace on a physical screen one needs to 
associate the workspace with physical screen's id @sid@
and details @sd@, 
forming the new data structure @Screen i l a sid sd@.

Xinerama in X11 allows viewing multiple virtual workspaces
simultaneously. 
%
While only one the current one will ever be in focus (i.e. will
receive keyboard events), other workspaces may be passively
viewable.  
%
We thus need to track which virtual workspaces are
associated (viewed) on which physical screens.  
%
To keep track of
this, \lbxmonad's main data structure  @StackSet i l a sid sd@ 
keeps, apart from the current screen,
separate lists of visible but non-focused
workspaces (@Screen@) , and non-visible workspaces (@Workspace@).

\subsection{Uniqueness of windows}
We proved that each window @a@ appears once in the @StackSet i l a sid sd@
bottom-up: 
We proved that each @Stack@ has unique elements
and that all the stacks that appear in @StackSet@ are disjoint.

\mypara{Unique Stack} 
Assume the existence of a predicate 
@ULNeq (X::a) (XS::[a])@ that ensures that 
(1)~the list @XS@ has no duplicates and
(2)~@X@ is not an element of @XS@.
%
With that magical predicate we define an \emph{almost unique}
stack: 
\begin{code}
data Stack a = Stack { focus :: a   
                     , up    :: ULNEq a focus
                     , down  :: ULNEq a focus }
\end{code}
The stack has a @focus@ element @a@ and 
and two unique lists of @a@'s @up@ and @down@ of the focus.

The above Stack is almost unique, as an element 
may appear both in the @up@ and the @down@ lists.
%
We define a type alias for @U@nique@Stack@
that rejects with possibility:

\begin{code}
type UStack a = {v:(Stack a) | 
  (LDisjoint (up v) (down v))}

predicate LDisjoint X Y = 
  (Set_emp (Set_cap (listElts X) (listElts Y)))
\end{code}

where @listElts@ is a recursively defined measure that 
returns the Set of elements of the input list, 
that is build in \toolname.
\begin{code}
measure listElts :: [a] -> (Set a) 
 listElts([])   = {v | (Set_emp v) }
 listElts(x:xs) = {v | v = 
   (Set_cup (Set_sng x) (listElts xs))} 
\end{code}

Note that the above definitions crucially depend on set theoretic properties.
Thus, verification of \lbxmonad is achieved using an SMT back-end 
that supports set theory (like Z3~\citep{z3}).

We slightly modify the above measure definition
to define @listDup@, a measure that returns the duplicate
elements of a list.
%
\begin{code}
measure listDup :: [a] -> (Set a)
 listDup([])   = {v | (Set_emp v)}
 listDup(x:xs) = {v | v = 
   if (Set_mem x (listElts xs)) 
    then (Set_cup (Set_sng x) (listDup xs))
    else (listDup xs))                      }
\end{code}

Using @listDup@ we define the magical list type @ULNEq a N@ 
as a list @v@ that has no duplicates, \ie the set @listDup v@
is empty, and @N@ does not belong to the set @listElts v@.

\begin{code}
type ULNEq a N = {v:[a] | ( (UL v) && (not (LElt N v))}

predicate LElt N LS  = (Set_mem N (listElts LS)) 
predicate UL     LS  = (Set_emp (listDup LS)   )
\end{code}

Throughout verification
we need to establish and use the invariant 
that each Stack is Unique.
%
This is achieved by the following annotation
\begin{code}
using (Stack a) as (UStack a)
\end{code}
that allows \toolname to 
(1)~ use the invariant:
each time a Stack value is retrieved from the environment
\toolname strengthens its type with the disjointness information;
(2)~ prove the invariant:
each time @Stack@ data constructor is used,
an disjoint constraint should be proved.
Failure to prove this constraint will raise an 
``Invariant Check'' error.

\mypara{Unique StackSet} 
Establishing Uniqueness on StackSets is a generalization of the above procedure.

The definition of a @StackSet@ includes 
the @current@ Screen, 
the list of @visible@ screens,
and the list of @hidden@ workspaces.
\begin{code}
data StackSet i l a sid sd = StackSet 
   { lcurrent  ::  Screen i l a sid sd   
   , lvisible  :: [Screen i l a sid sd]
   , lhidden   :: [Workspace i l a]
   , lfloating :: M.Map a RationalRect     
   }
\end{code}
%
\toolname automatically turns the record selectors of refined data types
to measures that return the appropriate fields.
%
Thus we infix the refined selectors with an @l@
to distinguish between the haskell (\eg, @current@)
and the logical (\eg, @lcurrent@) selectors. 

To prove absence of duplicates we need to @use@
only @StackSet@s that have no duplicates:  
\begin{code}
using (StackSet i l a sid sd) 
 as  {v:StackSet i l a sid sd|(NoDuplicates v)}
\end{code}

@NoDuplicates@ ensures that the elements of
@hidden@, @current@, and @visible@ 
workspaces are mutually disjoint
and that the @visible@ and @hidden@ workspaces
have no duplicates.
%
\begin{code}
predicate NoDuplicates SS = 
    (Disjoint3  (workspacesElts (lhidden  SS)) 
                (screenElts     (lcurrent SS)) 
                (screensElts    (lvisible SS))) 
  &&
    (Set_emp (screensDups    (lvisible SS))) 
  &&
    (Set_emp (workspacesDups (lhidden  SS)))
\end{code}
%
Again we used recursively defined measures to grap 
the elements and the duplicates of the structures.
For example, @screenElts@ returns 
the elements of the stack of the workspace of the screen, 
and is used by @screensDup@ to grap the duplicates of 
a list of Screens.

\mypara{Verification Procedure}
Using the above unique types make \toolname verify the absence of duplicates.
%
It was not surprising that when we first run \toolname 
against these stronger types many type errors where created.
%
The 
can be summarized as follows:
\begin{itemize}
	\item\emph{Strengthening library functions.}
      \lbxmonad repeatedly concatenates the list fields of a Stack.
      %
      To prove that for some `s::UStack a`, `(up s ++ down s)` is a unique list, 
      the type of `(++)`
      needs to capture that concatenation of two unique and disjoint lists is a unique list.
      %
      For verification, we assumed that Prelude's `(++)` satisfies this property.
      %
      But, not all arguments of `(++)` are unique disjoint lists:
      @"StackSet" ++ "error"@ is a trivial example that does not satisfy
      the assumed preconditions of `(++)` thus creating a type error.
      % 
      Sadly, \toolname does not currently support sum types, 
      thus we used an unrefined `(++.)` variant of `(++)` for such cases.
%%
%%	Thus function @mapLayout@ seemed en auto-provable one, 
%%	as it only modify the layout elements @l@ 
%%	which cannot affect the uniqueness of @a@ elements.
%%	%
%%\begin{code}
%%mapLayout :: (l -> l') 
%%          -> StackSet i l a s sd 
%%          -> StackSet i l' a s sd
%%\end{code}
     
	\item\emph{Restrict the functions' domain.}
	@modify@ is a @maybe@ like function
	that given a default value @x@,
	a function @f@ and a StackSet @s@,
	applies @f@ on the @Maybe (Stack a)@ values inside @s@. 
	%
\begin{code}
modify :: 
    x:{v:Maybe (UStack a) | (isNothing v)}
 -> f:(y:(UStack a) 
     -> Maybe {v:UStack a) | (SubElts v y)})
 -> s:StackSet i l a s sd 
 -> StackSet i l a s sd
\end{code}
	%
	Since inside the StackSet each @y:Maybe (Stack a)@
	could be replaced with either the default value @x@
	or @f y@ we need to ensure that both there alternatives
	will not insert duplicates.
	%
	This imposes the interesting precondition that the default
	value should be @Nothing@, which dramatically restricts 
	@modify@'s domain.
	%
	Given this restriction, \toolname should verify 
	that all user functions satisfy it.
			
	\item\emph{Code inlining}
    %
    Given a tag @i@ and a StackSet @s@, 
    @view i s@ will set the current Screen 
    to the screen with tag @i@, 
    if such screen exists in @s@.
    %
    Below is the original definition for @view@
    in case when a screen with tag @i@ exists in 
    visible screens
    %
\begin{code}
view :: (Eq s, Eq i) => i 
    -> StackSet i l a s sd -> StackSet i l a s sd
view i s    
  | Just x <- L.find ((i==).tag.workspace) 
                     (visible s)
  = s { current = x
      , visible = current s : 
          L.deleteBy (equating screen) x 
                     (visible s)} 
\end{code}
    %
    Verification of this code is difficult
    as the properties of all intermediate values 
    are too complicated to be expressed.
    %
    Instead we replaces this code with a call to a 
    recursive function @raiseIfVisible i s@
    that in-place replaces @x@ with the current screen.  
  
\end{itemize}

\subsection{QuickCheck Properties}

\lbxmonad is tested against $113$ \texttt{quickcheck} properties.
%
Of those $15$ check the uniqueness invariant 
and the rest $113$ check various functional properties.
%
We started the endeavour of verifying these properties with \toolname.
%
We looked at a sample of $15$ properties to conclude that
which we categorized as follows:
\begin{itemize}
\item\emph{Easy to be proved.}
Consider the \texttt{quickcheck} property that checks that @view@ing 
a @StackSet a@ is idempotent:
\begin{code}
prop_view_idem (x :: T) (i :: NonNegative Int) 
  = i `tagMember` x ==> view i (view i x) == (view i x)
\end{code}
%
The above property directly translated to a haskell function
\begin{code}
type Valid     = {v:Bool | (Prop v) }

prop_view_idem :: StackSet i l a sid sd -> i -> Valid
prop_view_idem x i 
  | i `tagMember` x = view i (view i) == v
  | otherwise       = True
\end{code}
%
By the above type signature,
\ie by the result type @Valid@, 
we specify that the function should always returns True.
%
When typechecking the above function,
\toolname proves that the property holds.
%
\toolname is able to verify this property as the result type of @view@
is strengthens with a refinement 
(in this case @EqTag x i => x = v@) 
that directly implies this property.

The above is generalizing to (10/17) properties that we checked:
strengthening the function types by refinements that \toolname can prove
is sufficient to verify these properties.

\item\emph{Can be estimated.}
In some other properties (like checking that @view@ing is reversible),
two StackSets (@s1@ and @s2@) were normalized before being compared,
that is their elements were first sorted.
%
In our logic we do not support any operation that can normalize structures in such a way.
%
Thus we cannot prove this exact property.
%
Instead we approximated it, by proving that proving that @s1@ and @s2@
have the same sets of elements.
%
We approximated (3/17) of the properties.

\item\emph{Their proof cannot be supported, currently.} [1]
One \texttt{quickcheck} property checks 
that @i@ cannot belong to an empty stackset.
%
We used abstract refinements to encode empty stacksets.
%
Proving the above property would be easy 
if we could mix abstract and concrete refinements in logical formulas
or if \toolname supported sum types.
%
Both the alternatives constitutes features that
we would like to extend \toolname with in the near future.
%
Still, currently we are not able to prove such kind of properties.
\item\emph{Cannot be expressed}
Other properties %, like @prop_focus_left_master@ 
check that the order of the windows is not affected by certain operations.
%
Though not infeasible we acknowledge that \toolname 
is not appropriate for reasoning about order preserving
and verification of such properties 
would require many code modifications.
%
(3/17) are order preserving properties.
\end{itemize}
