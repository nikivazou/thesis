\section{Totality}\label{sec:totality}
%% OK \NV{DONE intro (R1)
%% OK Sections 3 and 4 cover "totality" and "termination" respectively. It would be helpful to explain these terms at the start of section 3, so that the two are distinguished appropriately.
%% OK }
Well typed Haskell code can go very wrong:
%
\begin{code}
  *** Exception: Prelude.head: empty list
\end{code}
%
As our first application, let us see how to use 
\toolname to statically guarantee the absence
of such exceptions, \ie, to prove various 
functions \emph{total}.

\subsection{Specifying Totality}

First, let us see how to specify the notion of
totality inside \toolname. Consider the source of 
the above exception:
%
\begin{code}
  head :: [a] -> a
  head (x:_) = x
\end{code}
%
Most of the work towards totality checking is done by 
the translation to GHC's Core, in which every function 
\emph{is} total, but may explicitly call an \emph{error} 
function that takes as input a string that describes the 
source of the pattern-match failure and throws an exception.
%
For example @head@ is translated into
%
\begin{code}
  head d = case d of 
             x:xs -> x
             []   -> patError "head"
\end{code}

Since every core function is total, but may explicitly 
call error functions, to prove that the source function is 
total, it suffices to prove that @patError@ 
will \emph{never} be called.
%
We can specify this requirement by giving the error 
functions a @false@ pre-condition:
%
\begin{code}
  patError :: {v:String | false } -> a
\end{code}
%
The pre-condition states that the input type is \emph{uninhabited}
and so an expression containing a call to @patError@ will only type 
check if the call is \emph{dead code}.


\subsection{Verifying Totality}

The (core) definition of @head@ does not typecheck
as is; but requires a pre-condition that states that the function
is only called with non-empty lists. Formally, we do so by 
defining the alias
%
\begin{code}
  predicate NonEmp X = 0 < len X 
\end{code}
%
and then stipulating that 
%
\begin{code}
  head :: {v : [a] | NonEmp v} -> a
\end{code}
%
To verify the (core) definition of @head@, \toolname uses the above signature
to check the body in an environment
%
\begin{code}
  d :: {0 < len d}
\end{code}
%
When @d@ is matched with @[]@, the environment is 
strengthened with the corresponding refinement from 
the definition of @len@, \ie,
%
\begin{code}
  d :: {0 < (len d) && (len d) = 0}
\end{code}
%
Since the formula above is a contradiction, \toolname concludes that the
call to @patError@ is dead code, and thereby verifies the totality 
of @head@. Of course, now we have pushed the burden of proof onto clients
of @head@ -- at each such site, \toolname will check that the argument 
passed in is indeed a @NonEmp@ list, and if it successfully does so, then
we, at any uses of @head@, can rest assured that @head@ will never throw an 
exception. 

\mypara{Refinements and Totality} 
While the @head@ example is quite simple, in general, refinements make
it very easy to prove totality in complex situations, where we must track
dependencies between inputs and outputs. For example, consider the @risers@
function from \cite{catch}:
%
\begin{code}
  risers []       = []
  risers [x]      = [[x]]
  risers (x:y:zs) 
    | x <= y      = (x:s) : ss 
    | otherwise   = [x] : (s:ss) 
    where 
      s:ss    = risers (y:etc)
\end{code}
%
The pattern match on the last line is partial; its core translation is
%
\begin{code}
  let (s, ss) = case risers (y:etc) of
                  s:ss -> (s, ss)
                  []   -> patError "..."
\end{code}
%
What if @risers@ returns an empty list? 
Indeed, @risers@ \emph{does}, on occasion, return an empty list per its
first equation. However, on close inspection, it turns out that 
\emph{if} the input is non-empty, \emph{then} the output is also
non-empty. Happily, we can specify this as:
%
\begin{code}
  risers :: l:_ -> {v:_ | NonEmp l => NonEmp v} 
\end{code}

\toolname verifies that @risers@ meets the above specification, 
and hence that the @patError@ is dead code as at that 
site, the scrutinee is obtained from calling @risers@ with a
@NonEmp@ list.

\mypara{Non-Emptiness via Measures}
Instead of describing non-emptiness indirectly using @len@, a 
user could a special measure:
%
\begin{code}
  measure nonEmp  :: [a] -> Prop
  nonEmp (x:xs)   = true
  nonEmp []       = false

  predicate NonEmp X = nonEmp X
\end{code}
%
After which, verification would proceed analagous to the above.

\mypara{Total Totality Checking} 
@patError@ is one of many possible errors thrown by non-total functions.  
@Control.Exception.Base@ has several others (@recSelError@, @irrefutPatError@, \etc) which serve the purpose of making 
core translations total.
%
Rather than hunt down and specify @false@ preconditions one
by one, the user may automatically turn on totality checking 
by invoking \toolname with the \cmdtotality command line option, 
at which point the tool systematically checks that all the above 
functions are indeed dead code, and hence, that all definitions are total.

\subsection{Case Studies}

We verified totality of two libraries: \lbhscolour and \lbmap, earlier versions
of which had previously been proven total by \texttt{catch}~\citep{catch}.

\mypara{\lbmap} 
is a widely used library for (immutable) key-value maps, implemented
as balanced binary search trees.
Totality verification of \lbmap was quite straightforward.
We had previously verified termination and the crucial 
binary search invariant~\citep{vazou13}. To verify 
totality it sufficed to simply re-run verification with
the \cmdtotality argument.
%
All the important specifications were already captured by the types, 
and no additional changes were needed to prove totality.
%
%% \RJ{was it trivially total? i.e. is it total if you strip out all refinements
%% from specs?}
%% \NV{No, it fails in 6 functions all of which can trivially be reasoned to be total}
%% \NV{hedgeUnion, hedgeDiff, hedgeMerge, submap', join, merge}
%% \NV{The interesting story is that during verification \emph{we accidentally modified}
%% turn a function to partial, see my commit 041f1f0fea4d34ee41f50dbf7ce43e3c084c2743}
%

This case study illustrates an advantage of \toolname over specialized provers 
(\eg, \texttt{catch}~\citep{catch}), namely it can be used to prove totality, termination and
functional correctness at the same time, facilitating a nice reuse of
specifications for multiple tasks.

%% DONE \NV{(R3)
%% DONE Before discussing HsColour, I'd give a brief explanation of what it is.
%% DONE }
\mypara{\lbhscolour} is a library for generating syntax-highlighted LATEX and HTML from
Haskell source files.
Checking \lbhscolour was not so easy, as in some cases assumptions are used about the 
structure of the input data:
%
For example, @ACSS.splitSrcAndAnnos@ handles an
input list of @String@s and assumes that whenever
a specific @String@ (say @breakS@) appears then 
at least two @String@s (call them @mname@ and @annots@)
follow it in the list.
Thus, for a list @ls@ that starts with @breakS@ 
the irrefutable pattern  @(_:mname:annots) = ls@
should be total.
%
Currently it is somewhat cumbersome to specify such 
properties, and these are interesting avenues for future work.
%
Thus to prove totality, we added a dynamic check that 
validates that the length of the input @ls@ exceeds @2@.

%% measure follows a b c = \case 
%%   []   -> true
%%   x:xs -> if x == a then first2 b c xs else follows a b c xs
%% 
%% measure first2 b c = \case
%%   []   -> false
%%   x:xs -> x == b && first1 c xs
%% 
%% measure first1 c = \case
%%   []   -> false
%%   x:xs -> x == c
%% 
%% Worse, \toolname has no way to express such an invariant: 
%% \toolname naturally describes invariants that recursively 
%% hold for every list element and 
%% reaches its limitations when reasoning about non-recursive
%% properties.

In other cases assertions were imposed via monadic checks, 
for example @HsColour.hs@ reads the input arguments and 
checks their well-formedness using 
%
\begin{code}
  when (length f > 1) $ errorOut "..."
\end{code}
%
Currently \toolname does not support monadic reasoning that 
allows assuming that @(length f <= 1)@
holds when executing the action \emph{following} the @when@ check. 
%
Finally, code modifications were required to capture properties 
that currently we do not know how to express with \toolname.
%
For example, @trimContext@ checks if there is an element that 
satisfies @p@ in the list @xs@; if so it defines 
%
@ys = dropWhile (not . p) xs@
%
and computes @tail ys@.
%
By the check we know that @ys@ has at least one element, the 
one that satisfies @p@, but this is a property that we could 
not express in \toolname.

%%% \mynote{Bug}
%%% %
%%% \RJ{WHY? Seems like a simple GHC CHECK?}
%%% %
%%% On the positive side, totality verification revealed a subtle bug:
%%% %
%%% The instance @Enum@ of @Highlight@ does not define the @toEnum@ 
%%% method. In core, this reduces to a call to the error function 
%%% @noMethodBinding@.
%%% %
%%% Even though this totality bug can be tracked by GHC compilation,
%%% it exposes the strengths of our totality checker.

On the whole, while proving totality can be cumbersome 
(as in \lbhscolour) it is a nice side benefit of refinement
type checking, and can sometimes be a fully automatic corollary
of establishing more interesting safety properties (as in \lbmap).
