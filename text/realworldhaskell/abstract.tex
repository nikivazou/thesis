\begin{abstract}
Haskell has many delightful features. 
%
Perhaps the one most beloved by its users is its type 
system that allows developers to specify and verify a
variety of program properties at compile time. 
%
However, many properties, typically those that depend 
on relationships \emph{between} program values are 
impossible, or at the very least, cumbersome to encode 
within the existing type system. 
%
Many such properties can be verified using a combination 
of Refinement Types and external SMT solvers.
%
We describe the refinement type checker \toolname, which
we have used to specify and verify a variety of properties 
of over 10,000 lines of Haskell code from various popular 
libraries, including 
\verb+containers+, 
\verb+hscolour+, 
\verb+bytestring+, 
\verb+text+, 
\verb+vector-algorithms+ and 
\verb+xmonad+. 
%
First, we present a high-level overview of \toolname, 
through a tour of its features. 
%
Second, we present a qualitative discussion of the kinds 
of properties that can be checked -- ranging from generic 
application independent criteria like totality and termination,
to application specific concerns like memory safety and 
data structure correctness invariants.
%
Finally, we present a quantitative evaluation of the approach, 
with a view towards measuring the efficiency and programmer effort
required for verification, and discuss the limitations of the approach. 
\end{abstract}
