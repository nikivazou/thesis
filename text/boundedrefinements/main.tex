\documentclass{sigplanconf}

% \pagestyle{plain}
\usepackage{times}
\usepackage{amsmath,amssymb, latexsym}

\usepackage[latin1]{inputenc}
\usepackage{amsmath}
\usepackage{amsfonts}
\usepackage{amssymb}
\usepackage{amsthm}

\usepackage[inference]{semantic}
\usepackage{enumerate}
\def\url{}
\usepackage{xspace}
\usepackage{epsfig}
\usepackage{mathpartir}
\usepackage{booktabs}
\usepackage{paralist}
\usepackage{hyperref}

\def\qed{\hfill$\Box$}

% space at the beginning of an environment:
\def\@envspa{\hspace{0.3em}}
\def\@sa{\hspace{-0.2em}}
\def\@sb{\hspace{0.5em}}
\def\@sc{\hspace{-0.1em}}

\def\sk{\smallskip}		% space before and after theorems

\newtheorem{notation}{Notation}{\itshape}{}
\newtheorem{invariant}{Invariant}
\newtheorem{lemma}{Lemma}
\newtheorem*{lemma*}{Lemma}
\newtheorem{definition}{Definition}
\newtheorem{theorem}{Theorem}
\newtheorem*{theorem*}{Theorem}

\usepackage{color}
\usepackage{textcomp}
\usepackage{soul}
% \newcommand\highlight[2]{{\setlength\fboxsep{1pt}\colorbox{#1}{#2}}}
\newcommand\highlight[2]{{\sethlcolor{#1}\hl{#2}}}

\definecolor{colorNV}{rgb}{1,0.8,1}
\def\NV{\highlight{colorNV}}

\definecolor{colorRJ}{rgb}{0.2,1.0,0.3}
\def\RJ{\highlight{colorRJ}}

\definecolor{colorAB}{rgb}{0.8,0.8,0}
\def\AB{\highlight{colorAB}}


\def\nv#1{\NV{\sf NV:$\clubsuit$ #1$\clubsuit$}}
\def\rj#1{\RJ{\sf RJ:$\clubsuit$ #1$\clubsuit$}}
\def\ab#1{\AB{\sf AB:$\clubsuit$ #1$\clubsuit$}}

\usepackage{commands}
\usepackage{liquidHaskell}
\usepackage{listings}

% uncomment next line to restore colors
% \def\withcolor{}

\ifdefined\withcolor
	\definecolor{haskellblue}{rgb}{0.0, 0.0, 1.0}
	\definecolor{haskellblue}{rgb}{1.0, 0.0, 0.0}
	\definecolor{gray_ulisses}{gray}{0.55}
	\definecolor{castanho_ulisses}{rgb}{0.71,0.33,0.14}
	\definecolor{preto_ulisses}{rgb}{0.41,0.20,0.04}
	\definecolor{green_ulisses}{rgb}{0.0,0.4,0.0}
\else
	\definecolor{haskellblue}{gray}{0.1}
	\definecolor{haskellred}{gray}{0.1}
	\definecolor{gray_ulisses}{gray}{0.1}
	\definecolor{castanho_ulisses}{gray}{0.1}
	\definecolor{preto_ulisses}{gray}{0.1}
	\definecolor{green_ulisses}{gray}{0.1}
\fi


\def\codesize{\normalsize}

\lstdefinelanguage{HaskellUlisses} {
	basicstyle=\ttfamily\footnotesize,
	sensitive=true,
	morecomment=[l][\color{gray_ulisses}\ttfamily\codesize]{--},
	%% morecomment=[s][\color{gray_ulisses}\ttfamily\codesize]{\{-}{-\}},
	morestring=[b]",
	stringstyle=\color{haskellred},
	showstringspaces=false,
	numberstyle=\codesize,
	numberblanklines=true,
	showspaces=false,
	breaklines=true,
	showtabs=false,
    literate={
           {`}{{{$^{\backprime}{}$}}}1
           {'}{{{$^{\prime}{}$}}}1
           % {QED}{{{\color{lcolor}QED}}}3
           % {***}{{{\color{lcolor}***}}}3
           {?}{{{$\therefore$}}}1
           {<=}{{$\leq$}}1
           {theta}{{$\theta$}}1
           {env}{{$\Gamma$}}1
           {|-}{{$\vdash$}}1
           {<=!}{{{\color{lcolor}<=!}}}3
           {!=}{{$\neq$}}2
           {forall}{{$\forall$}}1
           {->}{{$\rightarrow$}}2
           {=*}{{$\eqfun$}}2
           {<=>}{{$\Leftrightarrow$}}3
           {=>}{{$\Rightarrow$}}2
           {<:}{{$\preceq$}}1
           {mempty}{{$\mempty$}}1
           {mappend}{{$\mappend$}}1
           {<>}{{$\mappend$}}1
           {stringMempty}{{$\stringMempty$}}1
           {<+>}{{$\stringMappend$}}1
           {stringMappend}{{$\stringMappend$}}1
           {listMempty}{{[]}}1
           {listMappend}{{++}}2
           {epsilon}{{$\epsilon$}}1
           {eta}{{$\eta$}}1
           {&&&}{&&&}3
           {&&}{{$\land$}}1
           {_m}{{${}_m$}}1
           {_n}{{${}_n$}}1
           {m^+}{{m${}^{+}$}}2
           },
	emph=
	{[1]
		FilePath,IOError,abs,acos,acosh,all,and,any,appendFile,approxRational,asTypeOf,asin,
		asinh,atan,atan2,atanh,basicIORun,break,catch,ceiling,chr,compare,concat,concatMap,
		const,cos,cosh,curry,cycle,decodeFloat,denominator,digitToInt,div,divMod,drop,
		dropWhile,either,elem,encodeFloat,enumFrom,enumFromThen,enumFromThenTo,enumFromTo,
		error,even,exp,exponent,fail,filter,flip,floatDigits,floatRadix,floatRange,floor,
		fmap,foldl,foldl1,foldr,foldr1,fromDouble,fromEnum,fromInt,fromInteger,
		fromRational,fst,gcd,getChar,getContents,getLine,head,id,inRange,index,init,intToDigit,
		interact,ioError,isAlpha,isAlphaNum,isAscii,isControl,isDenormalized,isDigit,isHexDigit,
		isIEEE,isInfinite,isLower,isNaN,isNegativeZero,isOctDigit,isPrint,isSpace,isUpper,iterate,
		last,lcm,length,lex,lexDigits,lexLitChar,lines,log,logBase,lookup,map,mapM,mapM_,max,
		maxBound,maximum,maybe,min,minBound,minimum,mod,negate,not,notElem,numerator,odd,
		or,pi,pred,primExitWith,print,product,properFraction,putChar,putStr,putStrLn,quot,
		quotRem,range,rangeSize,read,readDec,readFile,readFloat,readHex,readIO,readInt,readList,readLitChar,
		readLn,readOct,readParen,readSigned,reads,readsPrec,realToFrac,recip,rem,repeat,replicate,
		reverse,round,scaleFloat,scanl,scanl1,scanr,scanr1,seq,sequence,sequence_,show,showChar,showInt,
		showList,showLitChar,showParen,showSigned,showString,shows,showsPrec,significand,signum,sin,
		sinh,snd,span,splitAt,sqrt,subtract,succ,sum,tail,take,takeWhile,tan,tanh,threadToIOResult,toEnum,
		toInt,toInteger,toLower,toRational,toUpper,truncate,uncurry,undefined,unlines,until,unwords,unzip,
		unzip3,userError,words,writeFile,zip,zip3,zipWith,zipWith3,listArray,doParse,for,initTo,
        maxEvens,create,get,set,initialize,idVec,fastFib,fibMemo,
        insert,union,split,size,fromList,initUpto,trim,quickSort,insertSort,append,upperCase,
        copy, group, doDownLoop, mapAccumR, peekByteOff,
        pokeByteOff,spanByte, 
        good, bad, foo, explode, 
        fib, ack, 
        tLen,
        memcpy,writeChar,unsafeWrite,unsafeFreeze,
        singleton
	},
	emphstyle={[1]\color{haskellblue}},
	emph=
	{[2]
		Bool,Char,Double,Either,Float,IO,Integer,Int,Maybe,Ordering,Rational,Ratio,ReadS,ShowS,String,
		Word8,Nat,NonZero,Nat64,Text,ByteString,ByteStringSZ,ByteStringN,
        Ptr,ForeignPtr,CSize
        InPacket,Tree,Prop,TreeEq,TreeLt,Vec,
        NullTerm,IncrList,DecrList,UniqList,BST,MinHeap,MaxHeap,
        PtrN,ByteStringN,ByteStringEq,VO,ByteStringsEq,ByteStringNE
	},
	emphstyle={[2]\color{castanho_ulisses}},
	emph=
	{[3]
		case,class,data,deriving,do,else,if,return,def,import,in,infixl,infixr,instance,let,
		module,measure,predicate,of,primitive,then,refinement,type,where,lazy
	},
	emphstyle={[3]\color{preto_ulisses}\textbf},
	emph=
	{[4]
		quot,rem,div,mod,elem,notElem,seq
	},
	emphstyle={[4]\color{castanho_ulisses}\textbf},
	emph=
	{[5]
		PS,Tip,Node,EQ,False,GT,Just,LT,Left,Nothing,Right,True,Show,Eq,Ord,Num
	},
	emphstyle={[5]\color{green_ulisses}}
}

%%%ORIG
%%%\lstnewenvironment{code}
%%%{\textbf{Haskell Code} \hspace{1cm} \hrulefill \lstset{language=HaskellUlisses}}
%%%{\hrule\smallskip}

%V1
%\lstnewenvironment{code}
%{\smallskip \lstset{language=HaskellUlisses}}
%{\smallskip}

\lstnewenvironment{code}
{\lstset{language=HaskellUlisses}}
{}

\lstnewenvironment{mcode}
{\lstset{language=HaskellUlisses,mathescape}}
{}

\lstMakeShortInline[language=HaskellUlisses]@




%\lstMakeShortInline[language=HaskellUlisses,basicstyle=\ttfamily\normalsize,breakatwhitespace]@

\usepackage{flushend}


% FONT SETTINGS
%% \usepackage{fontspec}
%% \defaultfontfeatures{Mapping=tex-text,Scale=MatchLowercase}
%% \renewcommand\allcapsspacing[1]{{\addfontfeature{LetterSpace=15}#1}}
%% \renewcommand\smallcapsspacing[1]{{\addfontfeature{LetterSpace=10}#1}}

\usepackage{inconsolata}


\usepackage{thmtools}
\declaretheoremstyle[%
  spaceabove=-6pt,%
  spacebelow=6pt,%
  headfont=\normalfont\itshape,%
  postheadspace=1em,%
  qed=\qedsymbol,%
  headpunct={}
]{mystyle}
\declaretheorem[name={Proof Sketch:},style=mystyle,unnumbered,
]{proofsketch}


\newcommand\ifextended[2]{#2}
\sloppy

\special{papersize=8.5in,11in}
\setlength{\pdfpageheight}{\paperheight}
\setlength{\pdfpagewidth}{\paperwidth}

\conferenceinfo{CONF 'yy}{Month d--d, 20yy, City, ST, Country} 
\copyrightyear{20yy} 
\copyrightdata{978-1-nnnn-nnnn-n/yy/mm} 
\doi{nnnnnnn.nnnnnnn}

% Uncomment one of the following two, if you are not going for the 
% traditional copyright transfer agreement.

%\exclusivelicense                % ACM gets exclusive license to publish, 
                                  % you retain copyright

%\permissiontopublish             % ACM gets nonexclusive license to publish
                                  % (paid open-access papers, 
                                  % short abstracts)

\titlebanner{banner above paper title}        % These are ignored unless
\preprintfooter{short description of paper}   % 'preprint' option specified.



\title{Bounded Refinement Types
\thanks{This work was supported by NSF grants CCF-1422471, C1223850, CCF-1218344,
        a Microsoft Research Ph.D Fellowship and a generous gift from Microsoft
        Research.}\ifextended{\\ Supplementary Material}{}}

\authorinfo{Niki Vazou \and Alexander Bakst \and Ranjit Jhala}{UC San Diego, USA}{}

\renewcommand{\keywords}{\paragraph*{Keywords}}

\begin{document}
\toappear{}
\maketitle

\begin{abstract}
\begin{abstract}
Refinement Reflection turns your favorite programming 
language into a proof assistant by reflecting
the ​code implementing a​ user-defined function 
into the function's (output) refinement type. 
%
As a consequence, at \emph{uses} of the function, 
the function definition is unfolded into the refinement logic 
in a precise, predictable and most
importantly, programmer controllable way.
%
In the logic, we encode functions and lambdas using uninterpreted symbols
preserving SMT-based decidable verification. 
In the language, we provide a library of combinators that lets programmers 
compose proofs from basic refinements and function definitions.
%
We have implemented our approach in the Liquid Haskell system, 
thereby converting Haskell into an interactive proof assistant, 
that we used to verify a variety of properties ranging 
from arithmetic properties of higher order, recursive functions
to the Monoid, Applicative, Functor and Monad type class laws 
for a variety of instances.
\end{abstract}
\end{abstract}


\category{D.2.4}{Software/Program Verification}{}
\category{D.3.3}{Language Constructs and Features}{Polymorphism}
\category{F.3.1}{Logics and Meanings of Programs}{Specifying and Verifying and Reasoning about Programs}

\keywords
haskell, refinement types, abstract interpretation


\begin{comment}
Must program verifiers always choose between expressiveness
and automation?
%
On the one hand, tools based on higher order logics
and full dependent types impose no limits on expressiveness,
but require user-provided (perhaps, tactic-based) proofs.
%
On the other hand, tools based on Refinement Types~\cite{Rushby98,pfenningxi98}
trade expressiveness for automation. For example, the refinement types
%
\begin{code}
  type Pos     = {v:Int | 0 < v}
  type IntGE x = {v:Int | x <= v}
\end{code}
%
specify subsets of @Int@ corresponding to values
that are positive or larger than some other value @x@
respectively. By limiting the refinement predicates to
SMT-decidable logics~\cite{NelsonOppen}, refinement type
based verifiers eliminate the need for explicit proof terms,
and thus automate verification.

% We can specify contracts like pre- and post-conditions by
% suitably refining the input and output types of functions.

This high degree of automation has enabled the
use of refinement types for a variety of verification
tasks, ranging from array bounds checking~\cite{LiquidPLDI08},
termination and totality checking~\cite{LiquidICFP14},
protocol validation~\cite{GordonTOPLAS2011,FournetCCS11},
and securing web applications~\cite{SwamyOAKLAND11}.
%
Unfortunately, this automation comes at a price.
To ensure predictable and decidable type checking, we must
limit the logical formulas appearing in specification types
to decidable (typically quantifier free) first order theories,
thereby precluding \emph{higher-order} specifications that
are essential for \emph{modular} verification.
\end{comment}

In this chapter we introduce \emph{Bounded Refinement Types} which enable 
\emph{bounded quantification} over refinements. 
%
Previously (Chapter~\ref{chapter:abstractrefinements}),
we developed Abstract Refinement Types, a mechanism
for quantifying type signatures over abstract refinement parameters.
%
We preserved decidability of checking and inference
by encoding abstractly refined types with uninterpreted functions
obeying the decidable axioms of congruence~\cite{NelsonOppen}. 
%
While useful,
refinement quantification was not enough to enable higher order abstractions
requiring fine grained \emph{dependencies between} abstract refinements.
%
In this chapter, we solve this problem by enriching signatures
with bounded quantification. 
%
The \emph{bounds} correspond to Horn
implications between abstract refinements, which, as in the classical
setting, correspond to subtyping constraints that must be satisfied 
by the concrete refinements used at any call-site. This
addition proves to be remarkably effective.

\begin{itemize}
\item
First, we demonstrate via a series of short examples how bounded refinements
enable the specification and verification of diverse textbook higher order
abstractions that were hitherto beyond the scope of decidable refinement
typing~(\S~\ref{sec:boundedrefinementtypes:overview}).

\item
Second, we formalize bounded types and show how bounds are translated
into ``ghost'' functions, reducing type checking and inference to the
``unbounded'' setting of chapter~\ref{chapter:abstractrefinements}, 
thereby ensuring that checking
remains decidable. Furthermore, as the bounds are Horn constraints, we
can directly reuse the abstract interpretation of Liquid Typing~\citep{LiquidPLDI08}
to automatically infer concrete refinements at instantiation
sites~(\S~\ref{sec:check}).

\item
Third, to demonstrate the expressiveness of bounded refinements, we
use them to build a typed library for extensible dictionaries, to
then implement a relational algebra library on top of those
dictionaries, and to finally build a library for type-safe
database access~(\S~\ref{sec:database}).

\item
Finally, we use bounded refinements to develop a \emph{Refined State Transformer}
monad for stateful functional programming, based upon Filli\^atre's method
for indexing the monad with pre- and post-conditions~\citep{Filliatre98}.
%
We use bounds to develop branching and looping combinators whose types
signatures capture the derivation rules of Floyd-Hoare logic, thereby
obtaining a library for writing verified stateful computations~(\S~\ref{sec:state}).
%
We use this library to develop a refined IO monad that tracks capabilities
at a fine-granularity, ensuring that functions only access specified
resources~(\S~\ref{sec:files}).
\end{itemize}

We have implemented Bounded Refinement Types in \toolname.
The source code of the examples (with slightly more verbose concrete syntax)
is at \cite{liquidhaskellgithub}.
%
While the construction of these verified abstractions is possible with full
dependent types, bounded refinements
%
keep checking automatic and decidable,
%
use abstract interpretation to automatically synthesize
refinements (\ie pre- and post-conditions and loop invariants),
and most importantly
%
enable retroactive or \emph{gradual} verification as when
erase the refinements, we get valid programs in the
host language.
%
Thus, bounded refinements point a way towards 
both automated and expressive verification. 
%
%%% Local Variables:
%%% mode: latex
%%% TeX-master: "main"
%%% End:

\section{Overview}
\label{sec:refinementreflection:overview}
\label{sec:examples}

We begin with an overview of refinement reflection and
how it allows us to write proofs \emph{of} and \emph{by}
Haskell functions.

\subsection{Refinement Types}

First, we recall some preliminaries about refinement types
and how they enable shallow specification and verification.

\mypara{Refinement types} are the source program's (here
Haskell's) types decorated with logical predicates drawn
from a(n SMT decidable) logic~\citep{ConstableS87,Rushby98}.
%
For example, we can define the @Nat@ type by refining
Haskell's @Int@ type with a predicate @0 <= v@:
%
\begin{code}
  type Nat = { v:Int | 0 <= v }
\end{code}
%
Here, @v@ names the value described by the type:
the above can be read as the
``set of @Int@ values @v@ that are not less than 0".
The refinement is drawn from the logic of quantifier
free linear arithmetic and uninterpreted functions
(QF-UFLIA~\cite{SMTLIB2}).

\mypara{Specification \& Verification}
%
We can use refinements to define and type the
textbook Fibonacci function as:
%
\begin{code}
  fib :: Nat -> Nat
  fib 0 = 0
  fib 1 = 1
  fib n = fib (n-1) + fib (n-2)
\end{code}
%
Here, the input type's refinement specifies a
\emph{pre-condition} that the parameters must
be @Nat@, which is needed to ensure termination,
and the output types's refinement specifies a
\emph{post-condition} that the result is also a @Nat@.
%
Refinement type checking lets us specify
and (automatically) verify the shallow property
that if @fib@ is invoked with a non-negative
@Int@, then it terminates and yields
a non-negative @Int@.

\mypara{Propositions}
%
We can use refinements to define a data type
representing propositions simply as an alias
for unit, a data type that carries no useful
runtime information:
%
\begin{mcode}
  type $\typp$ = ()
\end{mcode}
%
which can be \emph{refined} with
propositions about the code.
%
For example, the following states the proposition
$2 + 2$ equals $4$.
%
\begin{mcode}
  type Plus_2_2_eq_4 = { v: $\typp$ | 2 + 2 = 4 }
\end{mcode}
%
For clarity, we abbreviate the above type by omitting
the irrelevant basic type $\typp$ and variable @v@:
%
\begin{mcode}
  type Plus_2_2_eq_4 = { 2 + 2 = 4 }
\end{mcode}
%
Function types encode universally quantified propositions:
%
\begin{mcode}
  type Plus_com = x:Int -> y:Int -> { x + y = y + x }
\end{mcode}
%
The parameters @x@ and @y@ refer to input
values. Any inhabitant of the above type is a
proof that @Int@ addition is commutative.

\mypara{Proofs}
%
We \emph{prove} the above theorems by providing inhabitants to type specifications
in forms of Haskell programs. To ease this task \toolname
provides primitives to construct proof terms by
``casting'' expressions to \typp.
%
\begin{mcode}
  data QED = QED

  (**) :: a -> QED -> $\typp$
  _ ** _  = ()
\end{mcode}
%
To resemble mathematical proofs, we make this casting post-fix.
Thus, we write @e ** QED@ to cast @e@ to a value of \typp.
%
For example, we can prove the above propositions by writing
%
\begin{code}
  pf_plus_2_2 :: Plus_2_2_eq_4
  pf_plus_2_2 = trivial ** QED

  pf_plus_comm :: Plus_comm
  pf_plus_comm = \x y -> trivial ** QED

  trivial = ()
\end{code}
%
Via standard refinement type checking, the above code yields
the respective verification conditions (VCs),
%
\begin{align*}
                      2 + 2 & = 4 \\
  \forall \ x,\ y\ .\ x + y & = y + x
\end{align*}
%
which are easily proved valid by the SMT solver, allowing us
to prove the respective propositions.

\mypara{A Note on Bottom:} Readers familiar with Haskell's
semantics may be feeling anxious about whether the
dreaded ``bottom", which inhabits all types, makes our
proofs suspect.
%
Fortunately, as described in \cite{Vazou14}, \toolname
ensures that all terms with non-trivial refinements
provably evaluate to (non-bottom) values, thereby making
our proofs sound.

\subsection{Refinement Reflection}

Suppose we wish to prove properties about the @fib@
function, \eg @fib 2@ equals @1@.
%
\begin{code}
  type fib2_eq_1 = { fib 2 = 1 }
\end{code}
%
%% \NV{By Standard refinement type checking, you mean liquid types, not FStar}
Standard refinement type checking runs into two problems.
%
First, for decidability and soundness, arbitrary user-defined
functions do not belong the refinement logic, \ie we cannot
\emph{refer} to @fib@ in a refinement.
%
Second, the only information that a refinement type checker
has about the behavior of @fib@ is its shallow type
specification @Nat -> Nat@ which is far too weak to verify
@fib2_eq_1@.
%
To address both problems, we use the following annotation,
which sets in motion the three steps of refinement reflection:
%
\begin{code}
  reflect fib
\end{code}

\mypara{Step 1: Definition}
%
The annotation tells \toolname to declare an
\emph{uninterpreted function} @fib :: Int -> Int@
in the refinement logic.
%
By uninterpreted, we mean that the logical @fib@
is \emph{not} connected to the program function
@fib@; in the logic, @fib@
only satisfies the \emph{congruence axiom}
%
$$\forall n, m.\ n = m\ \Rightarrow\ \fib{n} = \fib{m}$$
%
On its own, the uninterpreted function is not
terribly useful, as it does not let us prove
% It lets us prove theorems like
% $$\forall m,\ n.\ m = n \Rightarrow \fib{m} = \fib{n}$$
%
%% \begin{code}
  %% fib_cong :: n:Nat -> m:Nat -> {m=n => fib m = fib n}
  %% fib_cong = trivial ** QED
%% \end{code}
%% %
%but not
@fib2_eq_1@ which requires reasoning about the
\emph{definition} of @fib@.

\mypara{Step 2: Reflection}
%
In the next key step, \toolname reflects the
definition into the refinement type of @fib@
by automatically strengthening the user defined
type for @fib@ to:
%
\begin{code}
  fib :: n:Nat -> { v:Nat | fibP v n }
\end{code}
%
where @fibP@ is an alias for a refinement
\emph{automatically derived} from the
function's definition:
%
\begin{mcode}
  fibP v n = v = if n = 0 then 0 else
                 if n = 1 then 1 else
                 fib(n-1) + fib(n-2)
\end{mcode}

\mypara{Step 3: Application}
%
With the reflected refinement type,
each application of @fib@ in the code
automatically unfolds the @fib@ definition
\textit{once} in the logic.
%
We prove @fib2_eq_1@ by:
%
\begin{code}
  pf_fib2 :: { fib 2 = 1 }
  pf_fib2 = let t0 = #fib# 0 
                t1 = #fib# 1
                t2 = #fib# 2 
            in  ()
\end{code}
%
We write @#f#@ to denote places where the
unfolding of @f@'s definition is important.
%
Via refinement typing, the above proof yields the
following verification condition that is
discharged by the SMT solver, even though @fib@
is uninterpreted:
%
\begin{align*}
   (\fibdef\ (\fib\ 0)\ 0) \ \wedge\ (\fibdef\ (\fib\ 1)\ 1) \ \wedge\ 
   (\fibdef\ (\fib\ 2)\ 2) \  \Rightarrow\ (\fib{2} = 1)
\end{align*}
%
Note that the verification of @pf_fib2@ relies
merely on the fact that @fib@ was applied
to (\ie unfolded at) @0@, @1@ and @2@.
%
The SMT solver automatically \emph{combines}
the facts, once they are in the antecedent.
The following is also verified:
%
\begin{code}
  pf_fib2' :: { fib 2 = 1 }
  pf_fib2' = [ #fib# 0, #fib# 1, #fib# 2 ] ** QED
\end{code}
%
%
Thus, unlike classical dependent typing, refinement
reflection \emph{does not} perform any type-level
computation.

\mypara{Reflection vs. Axiomatization}
%
An alternative \emph{axiomatic} approach,
used by Dafny~\citep{dafny} and
\fstar~\citep{fstar},
is to encode @fib@ using a universally
quantified SMT formula (or axiom):
$$\forall n.\ \fibdef\ (\fib\ n)\ n$$
%
Axiomatization offers greater automation than
reflection. Unlike \toolname, Dafny
%and \fstar
will verify the following by
\emph{automatically instantiating} the above
axiom at @2@, @1@ and @0@:
%
\begin{code}
  axPf_fib2 :: { fib 2 = 1 }
  axPf_fib2 = trivial ** QED
\end{code}

The automation offered by axioms is a bit of a
devil's bargain, as axioms render checking of
the VCs \emph{undecidable}.
%
In practice, automatic axiom instantation can
easily lead to infinite ``matching loops''.
%
For example, the existence of a term \fib{n} in a VC
can trigger the above axiom, which may then produce
the terms \fib{(n-1)} and \fib{(n-2)}, which may then
recursively give rise to further instantiations
\emph{ad infinitum}.
%
To prevent matching loops an expert must carefully
craft ``triggers'' and provide a ``fuel''
parameter~\citep{Amin2014ComputingWA} that can be
used to restrict the numbers of the SMT unfoldings,
which ensure termination, but can cause the axiom
to not be instantiated at the right places.
%
In short, per the authors of Dafny, the
undecidability of the VC checking and its
attendant heuristics makes verification
unpredictable~\citep{Leino16}.

\subsection{Structuring Proofs}

In contrast to the axiomatic approach,
with refinement reflection, the VCs are
deliberately designed to always fall in
an SMT-decidable logic, as function symbols
are uninterpreted.
%
It is up to the programmer to unfold the
definitions at the appropriate places,
which we have found, with careful design
of proof combinators, to be quite
a natural and pleasant experience.
%
To this end, we have developed a library
of proof combinators that permits reasoning
about equalities and linear arithmetic,
inspired by Agda~\citep{agdaequational}.

\mypara{``Equation'' Combinators}
%
We equip \toolname with a family of
equation combinators @op.@ for each
logical operator @op@ in
$\{=, \not =, \leq, <, \geq, > \}$,
the operators in the theory QF-UFLIA.
%
The refinement type of @op.@  \emph{requires}
that $x \odot y$ holds and then \emph{ensures}
that the returned value is equal to @x@.
%
For example, we define @=.@ as:
%
\begin{code}
  (=.) :: x:a -> y:{a| x=y} -> {v:a| v=x}
  x =. _ = x
\end{code}
%
and use it to write the following ``equational" proof:
%
\begin{code}
  eqPf_fib2 :: { fib 2 = 1 }
  eqPf_fib2 =  #fib# 2
            =. #fib# 1 + #fib# 0
            =. 1
            ** QED
\end{code} %$

\mypara{``Because'' Combinators}
%
Often, we need to compose ``lemmata'' into larger
theorems. For example, to prove @fib 3 = 2@ we
may wish to reuse @eqPf_fib2@ as a lemma.
%
To this end, \toolname has a ``because'' combinator:
%
\begin{mcode}
  ($\because$) :: ($\typp$ -> a) -> $\typp$ -> a
  f $\because$ y = f y
\end{mcode}
%
The operator is simply an alias for function
application that lets us write
%
@ x op. y $\because$ p@ (instead of @(op.) x y p@)
where @(op.)@ is extended to accept an \textit{optional} third proof
argument via Haskell's typeclass mechanisms.
%
We use the because combinator to
prove that @fib 3 = 2@ with a Haskell function:
%
\begin{mcode}
  eqPf_fib3 :: { fib 3 = 2 }
  eqPf_fib3 =  #fib# 3
            =. fib 2 + #fib# 1
            =. 2              $\because$ eqPf_fib2
            ** QED
\end{mcode}

\mypara{Arithmetic and Ordering}
%
SMT based refinements let us go well beyond just equational
reasoning. Next, lets see how we can use arithmetic and
ordering to prove that @fib@ is (locally) increasing,
%
\ie for all $n$, $\fib{n} \leq \fib{(n+1)}$
%
\begin{mcode}
  fibUp :: n:Nat -> { fib n <= fib (n+1) }
  fibUp n
    | n == 0
    =  #fib# 0 <. #fib# 1
    ** QED

    | n == 1
    =  fib 1 <=. fib 1 + fib 0 <=. #fib# 2
    ** QED

    | otherwise
    =  #fib# n
    =. fib (n-1) + fib (n-2)
    <=. fib n     + fib (n-2) $\because$ fibUp (n-1)
    <=. fib n     + fib (n-1) $\because$ fibUp (n-2)
    <=. #fib# (n+1)
    ** QED
\end{mcode} %$

\mypara{Case Splitting and Induction}
%
The proof @fibUp@ works by induction on @n@.
%
In the \emph{base} cases @0@ and @1@, we simply assert
the relevant inequalities. These are verified as the
reflected refinement unfolds the definition of
@fib@ at those inputs.
%
The derived VCs are (automatically) proved
as the SMT solver concludes $0 < 1$ and $1 + 0 \leq 1$
respectively.
%
In the \emph{inductive} case, @fib n@ is unfolded
to  @fib (n-1) + fib (n-2)@, which, because of the
induction hypothesis (applied by invoking @fibUp@
at @n-1@ and @n-2@) and the SMT solver's arithmetic
reasoning, completes the proof.

\mypara{Higher Order Theorems}
%
Refinements smoothly accomodate higher-order reasoning.
%
For example, lets prove that every locally increasing
function is monotonic, \ie
if @f z <= f (z+1)@ for all @z@,
then @f x <= f y@ for all @x < y@.
%
\begin{mcode}
  fMono :: f:(Nat -> Int)
        -> fUp:(z:Nat -> {f z <= f (z+1)})
        -> x:Nat
        -> y:{x < y}
        -> {f x <= f y} / [y]
  fMono f inc x y
    | x + 1 == y
    =  f x <=. f (x+1) $\because$ fUp x
           <=. f y
           ** QED

    | x + 1 < y
    =  f x <=. f (y-1) $\because$ fMono f fUp x (y-1)
           <=. f y     $\because$ fUp (y-1)
           ** QED
\end{mcode}
%
We prove the theorem by induction
on @y@, which is specified by the
annotation @/ [y]@ which states
that @y@ is a well-founded
termination metric that decreases
at each recursive call~\citep{Vazou14}.
%
% All reflected functions are proved terminating.
% When the annotation metric is not explicit Liquid Haskell
% successfully uses heuristics to automatically prove termination. 
%
If @x+1 == y@, then we use @fUp x@.
%
Otherwise, @x+1 < y@, and we use
the induction hypothesis \ie apply
@fMono@ at @y-1@, after which
transitivity of the less-than
ordering finishes the proof.
%
We can use the general @fMono@
theorem to prove that @fib@
increases monotonically:
%
\begin{code}
  fibMono :: n:Nat -> m:{n<m} -> {fib n <= fib m}
  fibMono = fMono fib fibUp
\end{code}


\subsection{Case Study: Deterministic Parallelism}
\label{sec:detpar}

%% The natural integration of deep verification with a language like Haskell makes
%% it possible to engage in lightweight, incremental verification of program
%% properties.

One benefit of an in-language prover is that it lowers the barrier to {\em
  small} verification efforts that touch only a fraction of the program, and yet
ensure critical invariants that Haskell's type system cannot.  Here we consider
parallel programming, which is commonly considered error prone and entails
proof obligations on the user that typically go unchecked.

The situation is especially precarious with parallel programming frameworks that
claim to be {\em determinstic} and thus usable within purely functional
programs.  These include Deterministic Parallel Java (DPJ \cite{DPJ}), Concurrent
Revisions for .NET~\cite{concurrent-revisions-oopsla}, and Haskell's
LVish~\cite{kuper2014freeze}, Accelerate~\cite{accelerate-icfp13}, and
REPA~\cite{repa-icfp10}.
%
Accelerate's parallel fold function, for instance, claims to be
deterministic---and its purely functional type means the Haskell optimizer will
{\em assume} its referential transparency---but its determinism depends on an
associativity guarantee which must be assured {\em by the programmer} rather than the
type system.
%
Thus simply folding the minus function, @fold (-) 0 arr@, is sufficient to
violate determinism and Haskell's pure semantics.


Likewise, DPJ goes to pains to develop a new type system for parallel
programming, but then provides a ``commutes'' annotation for methods updating
shared state, compromising the {\em guarantee} and going back to trusting the
user. LVish has the same Achilles heel. Consider set insertion:

\begin{code}
  insert :: Ord a => a -> Set s a -> Par s ()
\end{code}

Here @insert@ returns an (effectful) @Par@ computation, which can be run within a
pure function to produce a pure result.  At first glance it would seem that
trusting the implementation of the concurrent set is sufficient to assure a
deterministic outcome.  Yet the interface has an @Ord@ constraint. This
 polymorphic function works with user-defined data types, and thus
user-defined orderings.  What if the user fails to implement a total order?
Then, even a correct implementation of, e.g. a concurrent
skiplist~\cite{concurrent-skiplist}, can reveal
different insertion orders due to concurrency.

%% verifiedInsert :: HasPut e => VerifiedOrd a
%%                => a -> ISet s a -> Par e s ()

% \mypara{LVish}
%% We demonstrate the use of \toolname{} to ensure guarantees of
%% deterministic parallel programming. We choose this case study, because, to the
%% best of our knowledge, there exists no practical deterministic parallel
%% programming system, including user-defined parallel folds, which does not have
%% {\em soundness holes}---due to trusted assumptions of user code.

%% {\em LVish}\cite{kuper2014freeze} is a programming library for Haskell, which
%% exposes effectful parallel programming against lattice-variables (LVars) whose
%% states change monotonically during parallel regions of program execution. LVish
%% programs operate on Haskell data types, and LVish requires the operations on
%% these datatypes to satisfy some first order laws, which cannot be expressed in
%% Haskell. However, we can leverage \toolname to verify these properties for
%% arbitrary user-defined datatypes.

%% LVish provides two implementations of concurrent sets, @PureSet@ and @SLSet@,
%% where the underlying data structure is a size-balanced binary tree and
%% concurrent skiplist respectively. The @insert@ operation on a set requires a
%% total ordering on the elements, we can express that in the type signature by
%% \new{The implementation doesn't change, in fact, the}
%% @VerifiedOrd@ \new{methods do not even need to exist at runtime. A sufficiently
%%   smart compiler could optimize away these proof obligations during code
%%   generation.}
% \RN{Let's save the issue of runtime impact for the eval.}

In summary, parallel programs naturally need to communicate, but the mechanisms
of that communication---such as folds or inserts into a shared
structure---typically carry additional proof obligations.  This in turn makes
parallelism a liability.  But we can remove the risk with verification.

% But what if we could use verification to remove the risk?

% through contributions to shared structures (otherwise they are really separate
% programs)


\mypara{Verified typeclasses}
%
Our solution is simply to change the @Ord@ constraint above to
@VerifiedOrd@.
\begin{mcode}
  insert :: VerifiedOrd a => a -> Set s a -> Par s ()
\end{mcode}
%
This constraint changes the interface but not the implementation of @insert@.
%
% \NV{Why does insert now requires Verified Ord? Is it using the extra methods
% in the implementation?}
%% \note{VerifiedSemigroup story + lifting + isomorphism ("bootstrapping
%%   instances") + detpar propaganda}
%
%% It is an informal requirement when using
%% typeclasses in GHC that some typeclass laws be satisfied. For example, the @Ord@
%% typeclass in GHC requires that the $\leq$ operation be a total order. Using
%% \toolname, we can extend it to include the required properties of a total order,
%% which we call a @VerifiedOrd@.
The additional methods of the verified type class don't add operational
capabilities, but rather impose additional proof obligations:

\begin{code}
  class Ord a => VerifiedOrd a where
   antisym :: x:a -> y:a -> { x <= y && y <= x => x = y }
   trans   :: x:a -> y:a -> z:a -> { x <= y && y <= z => x <= z }
   total   :: x:a -> y:a -> { x <= y || y <= x }
\end{code}

% ---------------------------------------------------------------
% \mypara{Verified Monoids}

Similarly, we can extend
the @Monoid@ typeclass to a @VerifiedMonoid@, with refinements
expressing @Monoid@ laws.
%
\begin{code}
  class Monoid a => VerifiedMonoid a where
   lident :: x:a -> { mempty <> x = x }
   rident :: x:a -> { x <> mempty = x }
   assoc  :: x:a -> y:a -> z:a -> { x <> (y <> z) = (x <> y) <> z }
\end{code}
The @VerifiedMonoid@ typeclass constraint requires the binary operation
to be associative, thus can be safely used to fold on
an unknown number of processors.
%% A parallel fold requires the underlying binary operation to be associative and
%% have a well-behaved identity element, or a @Monoid@.


%% We can then extend the @ParFoldable@ typeclass to a @VerifiedParFoldable@ which
%% enforces a @VerifiedMonoid@ constraint.

%% \NV{Not sure if the below code adds any information: too difficult to follow,
%%   especially for non Haskell people} \NV{I suggest to say similarly to Verified
%%   Ord and add a link to appendix}
%%\RN{I concur with Niki -- we often whitewash away details of the library for
%%  presentation purposes.  E.g. we are not going to explain effect signatures in
%%  this paper.}

%% \begin{code}
%% class ParFoldable c
%%    => VerifiedParFoldable c where
%%   verifiedPmapFold :: forall m e s a .
%%   ( ParFuture m, HasGet e
%%   , HasPut e, FutContents m a,
%%   , VerifiedMonoid a )
%%   => (ElemOf c -> m e s a) -- compute one
%%                            -- result
%%   -> c                     -- element generator
%%                            -- to consume
%%   -> m e s a
%% \end{code}


%%  -------------------------------------------------------------------------

\mypara{Verified instances for primitive types}
@VerifiedOrd@ instances for primitive types like @Int@, @Double@ are trivial to
write; they just appeal to the SMT solver's built-in theories.
%
For example, the following is a valid totality proof on @Int@.
\begin{code}
  totInt :: x:Int -> y:Int -> {x <= y || y <= x}
  totInt _ _ = trivial ** QED
\end{code}

\mypara{Verified instances for algebraic datatypes}
%
To prove the class laws for user defined algebraic datatypes,
refinement reflection allows for structurally inductive proof terms.
%
For example, we can inductively define Peano numerals
%
\begin{code}
  data Peano = Z | S Peano
\end{code}
%
We can compare two @Peano@ numbers via
\begin{code}
  reflect leq :: Peano -> Peano -> Bool
  leq Z _         = True
  leq (S n) Z     = False
  leq (S n) (S m) = leq n m
\end{code}
%
In \S~\ref{sec:refinementreflection:theory} we will describe
exactly how the reflection mechanism (illustrated
via @fibP@) is extended to account for ADTs like @Peano@.
%
\toolname automatically checks
that @leq@ is total~\citep{Vazou14}, which
lets us safely @reflect@ it into the logic.

Next, we prove that @leq@ is total on @Peano@ numbers
%
\begin{mcode}
  totalPeano :: n:Peano -> m:Peano -> {leq n m || leq m n} / [toInt n + toInt m]
  totalPeano Z m = leq Z m ** QED
  totalPeano n Z = leq Z n ** QED
  totalPeano (S n) (S m)
   =  leq (S n) (S m) || leq (S m) (S n)
   =. leq n m || leq m n
   =. True $\because$ totalPeano m n
   ** QED
\end{mcode}
The proof goes by induction, splitting cases on
whether the number is zero or non-zero. Consequently,
we pattern match on the parameters @n@ and @m@, and furnish
separate proofs for each case.
%
In the ``zero" cases, we simply unfold the definition
of @leq@.
%
In the ``successor" case, after unfolding we (literally)
apply the induction hypothesis by using the because operator.
%
The termination hint @[toInt n + toInt m]@,
where @toInt@ maps @Peano@ numbers to integers,
is used to verify well-formedness of the @totalPeano@
proof term.
%
\toolname's termination and totality checker
use the hint to
verify that we are in fact doing induction
properly~(\S~\ref{sec:types-reflection}).

Similarly to @totalPeano@, we can define the rest of the @VerifiedOrd@
proof methods and use them to create the verified instance.
%
\begin{code}
  instance Ord Peano where
    (<=) = leq

  instance VerifiedOrd Peano where
    total = totalPeano
\end{code}
%
Proving all the four @VerifiedOrd@ laws
is a burden on the programmer.
%becomes a burden as the datatype grows more complicated.
%
Since @Peano@ is isomorphic to @Nat@s,
next we present how
to reduce the @Peano@ proofs into the
SMT automated integer proofs.

\mypara{Isomorphisms}
%
In order to reuse proofs for a custom datatype,
we provide a way to translate verified instances between isomorphic types~\cite{barthe2001type}.
%% If our datatype is isomorphic to a nesting of binary sums and products, we
%% should be able to reusing existing proofs.
%% To verify operations on custom data types efficiently, we
%% need to be able lift verified instances on one type to another.
%
We design a typeclass @Iso@ which witnesses the fact that
two types are isomorphic.
%, with respect to the built-in equality in \toolname{}
%which is a congruence.

\begin{mcode}
  class Iso a b where
    to      :: a -> b
    from    :: b -> a
    to$\circ$from :: x:a -> {to (from x) = x}
    from$\circ$to :: x:a -> {from (to x) = x}
\end{mcode}
%
For two isomorphic types @a@ and @b@
we compare instances of @b@ using @a@'s
comparison method.
%
\begin{mcode}
  instance (Ord a, Iso a b) => Ord b where
    x <= y = from x <= from y
\end{mcode}
%
Then, we prove that @VerifiedOrd@ laws are closed under isomorphisms.
%
For example, we prove totality of comparison on @b@s
using the @VerifiedOrd@ totality on @a@s

\begin{mcode}
  isoTotal :: (VerifiedOrd a, Iso a b) => x:b -> y:b -> {x <= y || y <= x}
  isoTotal x y
   =  x <= y || y <= x
   =. (from x) <= (from y) || (from y) <= (from x)
      $\because$ total (from x) (from y)
   ** QED
\end{mcode}
%
We use @isoTotal@ to create a verified instance on @b@s.
\begin{mcode}
  instance (VerifiedOrd a, Iso a b) => VerifiedOrd b where
    total   = isoTotal
\end{mcode}
%
With the above technique,
and using Haskell's instances,
getting a @VerifiedOrd@ instance for @Peano@
reduces to definition of an @Iso Nat Peano@.
%\VC{Iso (Either () Peano) Peano}

\mypara{Proof Composition via Products}
Finally, we present a mechanism to automatically
reduce proofs on product types to proofs of the product components.
%
For example, lexicographic ordering preserves the ordering laws.
%
First, we use class instances to define lexicographic ordering.
%
\begin{mcode}
  instance (VerifiedOrd a, VerifiedOrd b) => Ord (a, b) where
    (x1, y1) <= (x2, y2) = if x1 == x2 then y1 <= y2 else x1 <= x2
\end{mcode}
%
Then, we prove that lexicographic ordering
preserves the ordering laws.
%
For example, it preserves totality.
%
\begin{mcode}
  prodTotal :: (VerifiedOrd a, VerifiedOrd b)
            => p:(a, b) -> q:(a, b) -> {p <= q || q <= p}
  prodTotal p@(x1, y1) q@(x2, y2)
   =  p <= q || q <= p
   =. if x1 == x2 then (y1 <= y2 || y2 <= y1) else True 
      $\because$ total x1 x2
   =. if x1 == x2 then True                   else True 
      $\because$ total y1 y2
   ** QED
\end{mcode}
%
Finally, using the @prodTotal@ proof method,
we conclude that each instance defined via the lexicographic
ordering is indeed a verified instance.
%
\begin{mcode}
  instance (VerifiedOrd a, VerifiedOrd b) => VerifiedOrd (a, b) where
    total   = prodTotal
\end{mcode}
%
For example the type @(Peano, Peano)@ is derived to be a @VerifiedOrd@ instance.

In short, we can decompose an algebraic datatype into an isomorphic type using sums and
products to generate verified instances for arbitrary Haskell
datatypes. This could be combined with the Glasgow Haskell Compiler's (GHC) support
for generics~\cite{ghc-generics} to automate the derivation of verified instances
for user datatypes.
In \S\ref{sec:eval-parallelism}, we use these ideas to develop fully safe
interfaces to LVish modules, as well as verifying programming patterns from DPJ.

\section{Formalism}\label{sec:check}

Next we formalize Bounded Refinement Types by defining
a core calculus \boundedcorelan and showing how it can
be reduced to \corelan, the core language of Abstract
Refinement Types~\ref{chapter:abstractrefinements}.
%
We start by defining the syntax~(\S~\ref{sec:syntax-corelan})
and semantics~(\S~\ref{sec:semantics-corelan}) of \corelan
and the syntax of \boundedcorelan~(\S~\ref{sec:syntax-boundedcorelan}).
%
Next, we provide a translation from \boundedcorelan to
\corelan ~(\S~\ref{sec:translation}).
%
Then, we prove soundness by showing that our translation
is semantics preserving~(\S~\ref{sec:soundness}).
%
Finally, we describe how type inference remains
decidable in the presence of bounded refinements~(\S~\ref{sec:infer}).


\subsection{Syntax of \corelan}\label{sec:syntax-corelan}

\newcommand{\ra}[1]{\renewcommand{\arraystretch}{#1}}
\ra{0.9}
\begin{figure}[t!]
\centering
\captionsetup{justification=centering}
$$
\begin{array}{rrcl}
\emphbf{Expressions} \quad 
  & e & ::=     & x 
                  \spmid c 
                  \spmid \efunt{x}{\rtyp}{e} 
                  \spmid \eapp{e}{x}      \spmid
                  \elet{x}{e}{e}{\rtyp}   \\
  &   &  \spmid & \etabs{\tvar}{e}  
                  \spmid \etapp{e}{\rtyp} 
                  \epabs{\rvar}{\rtyp}{e}
                  \spmid \epapp{e}{\constraint}     \\[0.03in] 
  
\emphbf{Constants} \quad
  & c 
  & ::= 
  & \etrue \spmid \efalse \spmid \ecrash \spmid
   0 \spmid 1 \spmid -1 \spmid \dots
  \\[0.05in] 
  
\emphbf{Parametric Refinements} \quad 
& \constraint & ::= & \reft 
                        \spmid \efunt{x}{b}{\constraint}  \\[0.03in]


\emphbf{Predicates} \quad 
  & \creft & ::= & c \spmid \lnot \creft 
                   \spmid \creft = \creft 
                   \spmid % \creft < \creft 
                   \dots  \\[0.05in] 

\emphbf{Atomic Refinements} \quad 
  & \areft & ::= & \creft 
                   \spmid \rvapp{\rvar}{x} \\[0.03in] 

\emphbf{Refinements} \quad 
  & \reft & ::= & \areft 
                  \spmid \areft \wedge \reft 
                  \spmid \areft \Rightarrow \reft \\[0.03in] 

\emphbf{Basic Types} \quad 
  & b 
  & ::= & \texttt{Int}
          \spmid \texttt{Bool}
          \spmid \texttt{a}    \\[0.03in]

%% \emphbf{Refined Basic Types} \quad 
%%   & r 
%%   & ::= 
%%   % &      \tpref{b}{\areft}{\reft} 
%%   \\[0.05in]

\emphbf{Types} \quad 
  & \rtyp
  & ::=      & \tref{b}{\reft} \spmid
  \trfun{x}{\rtyp}{\rtyp}{\reft} \\[0.03in]

\emphbf{Bounded Types} \quad 
  & \bt
  & ::= & \rtyp \\[0.05in]

\emphbf{Schemata} \quad 
  & \sigma
  & ::= & \bt
          \spmid \ttabs{\tvar}{\sigma}
          \spmid \tpabs{\rvar}{\rtyp}{\sigma} \\[0.03in]
\end{array}
$$
\caption{Stratified Syntax of \texorpdfstring{\corelan}{LamB}.}
\label{fig:boundedrefinements:syntax}
\end{figure}


%%%% \begin{figure}[t!]
%%%% $$
%%%% \begin{array}{rrcl}
%%%% \centering
%%%% \emphbf{Expressions} \quad 
%%%%   & e & ::=
%%%%   &  \dots \spmid  \elet{x}{e}{e}{\rtyp}
%%%%   \\[0.05in]
%%%% 
%%%% \end{array}
%%%% $$
%%%% \caption{\textbf{Syntactic extension from \corelan to \letcorelan}}
%%%% \label{fig:letsyntax}
%%%% \end{figure}




We build our core language on top of \corelan, the language
of Abstract Refinement Types\ref{chapter:abstractrefinements}.
%
Figure~\ref{fig:syntax} summarizes the syntax of \corelan,
a polymorphic $\lambda$-calculus extended with abstract
refinements.
%
For an easier transition to the syntax of Bounded Refinement Types, 
we rewrite the syntax of \corelan as initially presented in Figure~\ref{fig:abstractrefinements:syntax}
by stratifying type schemata to include bounded types, that for now, 
are plain types.


\mypara{The Expressions} of \corelan include the usual variables $x$,
primitive constants $c$, $\lambda$-abstraction $\efunt{x}{\rtyp}{e}$,
application $\eapp{e}{e}$,
let bindings $\elet{x}{e}{e}{\rtyp}$,
type abstraction $\etabs{\alpha}{e}$,
and type application $\etapp{e}{\rtyp}$.
(We add let-binders to \corelan
from Figure~\ref{fig:abstractrefinements:syntax} as they can be reduced to $\lambda$-abstractions
in the usual way).
%
The parameter $\rtyp$ in the type application is a \emph{refinement
type}, as described shortly.  Finally, \corelan includes refinement
abstraction $\epabs{\rvar}{\rtyp}{e}$, which introduces a refinement
variable $\rvar$ (with its type $\rtyp$), which
can appear in refinements inside $e$, and the corresponding refinement
application $\epapp{e}{\constraint}$ that substitutes an abstract refinement
with the parametric refinement $\constraint$, \ie
refinements $\reft$ closed under lambda abstractions.
%%%\nv{Suggestion: replace $\epapp{e}{e}$ with $\epapp{e}{\phi}$ in the syntax}
%%%\nv{If abstract refinements can be instantiated with arbitrary expressions,}
%%%\nv{then all the careful syntax that ensures that refinements r come from}
%%%\nv{decidable theory is useless.}
%%%\nv{In Abstract refinement paper we had}
%%%\nv{arbitrary expressions as refinements thus we did not have this problem}
%%%\nv{It seems that bounds $\phi$, ie x-parametric refinements, is a good}
%%%\nv{candidate to substitute abstract refinements, ie replace $\epapp{e}{e}$}
%%%\nv{with $\epapp{e}{\phi}$}
%%%\nv{Question: is this a coincidence?}

\mypara{The Primitive Constants} of \corelan include
\etrue, \efalse, @0@, @1@, @-1@, \etc. In addition, we include a
special untypable constant \ecrash that models ``going wrong''.
Primitive operations return a crash when invoked with inputs
outside their domain, \eg when @/@ is invoked with @0@ as the
divisor or when an @assert@ is applied to \efalse.

\mypara{Atomic Refinements} $\areft$ are either concrete or abstract refinements.
%
A \emph{concrete refinement} \creft is a boolean valued expression
(such as a constant, negation, equality, \etc)
drawn from a \emph{strict subset} of the language of expressions
which includes only terms that
%
(a)~neither diverge nor crash and
%
(b)~can be embedded into an SMT decidable logic including
%
the quantifier free theory of linear arithmetic and uninterpreted
functions~\cite{LiquidICFP14}.
%
An \emph{abstract refinement} $\rvapp{\pi}{x}$ is an application of
a refinement variable $\pi$ to a sequence of program variables.
%
A \emph{refinement} \reft is either a conjunction or
implication of atomic refinements.
%
To enable inference, we only allow implications to appear within
bounds $\constraint$ (\S~\ref{sec:infer}).


\mypara{The Types of} \corelan, written $\rtyp$, include basic types,
dependent functions and schemata quantified over type and refinement
variables $\tvar$ and $\rvar$ respectively.
%
A basic type is one of $\tbint$, $\tbbool$, or a type
variable $\alpha$.
%
A refined type $\rtyp$ is either a refined basic type $\tref{b}{\reft}$,
or a dependent function type $\trfun{x}{\rtyp}{\rtyp}{\reft}$ where
the parameter $x$ can appear in the refinements of the output type.
%
(We include refinements for functions, as refined type variables can be
replaced by function types. However, typechecking ensures these refinements
are trivially true).
%
In \corelan bounded types $\bt$ are just a synonym for types $\rtyp$.
%
Finally, schemata quantify bounded types over type
and refinement variables.

\subsection{Syntax of \boundedcorelan}\label{sec:syntax-boundedcorelan}

\begin{figure}[t!]
\centering
\captionsetup{justification=centering}
$$
\begin{array}{rrcl}
\centering

\emphbf{Bounded Types} \quad 
  & \bt         & ::= & \rtyp 
                        \spmid \tconstraint{\constraint}{\bt} \\[0.03in]

\emphbf{Expressions} \quad 
  & e           & ::= & \dots 
                        \spmid \econstraint{\constraint}{e} 
                        \spmid \econstantconstraint{e}{\constraint} \\
\end{array}
$$
\caption{Extending Syntax of \corelan to \boundedcorelan.} 
\label{fig:boundedsyntax}
\end{figure}
\ra{1.0}

Figure~\ref{fig:boundedsyntax} shows how we obtain the syntax for
\boundedcorelan by extending the syntax of \corelan with
\emph{bounded} types.

\mypara{The Types} of \boundedcorelan extend those of \corelan with
bounded types $\bt$, which are the types $\rtyp$ guarded by bounds
$\constraint$.

%% {HERE: Colin:How does $\phi$ influence or  interact with $\rho$?  It’s not clear here.}
\mypara{The Expressions} of \boundedcorelan extend those of \corelan
with \emph{abstraction} over bounds $\econstraint{\constraint}{e}$ and
\emph{application} of bounds $\econstantconstraint{e}{\constraint}$.
%
Intuitively, if an expression $e$ has some type $\bt$
then $\econstraint{\constraint}{e}$ has the type
$\tconstraint{\constraint}{\bt}$.
%
We include an explicit bound application form
$\econstantconstraint{e}{\constraint}$ to simplify
the formalization; these applied bounds are automatically
synthesized from the type of $e$, and are of the form
$\overline{\efunt{x}{\bt}{}}{\etrue}$.

\mypara{Notation.}
We write
$b$,
%$\tref{b}{\reft}$ and
$\tpp{b}{\rvapp{\pi}{x}}$,
$\tpref{b}{\rvapp{\pi}{x}}{\reft}$
to abbreviate
$\tref{b}{\etrue}$,
$\tref{b}{\rvapp{\pi}{x}\ \vref}$,
$\tref{b}{\reft \wedge \rvapp{\pi}{x}\ \vref}$
%% $\tpref{b}{\true}{\true}$,
%% $\tpref{b}{\true}{\reft}$, and
%% $\tpref{b}{\areft}{\true}$
respectively.
We say a type or schema is \emph{non-refined} if all the
refinements in it are $\true$.
%
We get the \textit{shape} of a type $\rtyp$ (\ie the System-F type)
by the function $\toshape{\rtyp}$ defined:
%
\begin{align*}
\toshape{\tref{b}{\reft}} \defeq &\ b \\
\toshape{\trfun{x}{\rtyp_1}{\rtyp_2}{\reft}} \defeq &\ \toshape{\rtyp_1} \rightarrow \toshape{\rtyp_2}
\end{align*}

\subsection{Translation from \boundedcorelan to \corelan}
\label{sec:translation}

Next, we show how to translate a term from \boundedcorelan to
one in \corelan. We assume, without loss of generality that the
terms in \boundedcorelan are in Administrative Normal Form
(\ie all applications are to variables).

\mypara{Bounds Correspond To Functions} that explicitly
``witness'' the fact that the bound constraint holds at a
given set of ``input'' values.
%
That is we can think of each bound as a universally quantified
relationship between various (abstract) refinements; by ``calling''
the function on a set of input values $x_1,\ldots,x_n$, we get
to \emph{instantiate} the constraint for that particular set
of values.


\mypara{Bound Environments} \benv are used by our translation
to track the set of
bound-functions (names) that are in scope at each program point.
%
These names are distinct from the regular program variables that
will be stored in Variable Environments \cenv.
%
We give bound functions distinct names so that they cannot appear
in the regular source, only in the places where calls are inserted
by our translation.
%
The translation ignores refinements entirely; both environments
map their names to their non-refined types.

\mypara{The Translation is formalized} in
Figure~\ref{fig:translation} via a
relation $\txexpr{\cenv}{\benv}{e}{e'}$
that translates the expression
$e$ in $\boundedcorelan$ into
$e'$ in $\corelan$.
%
Most of the rules in figure~\ref{fig:translation}
recursively translate the sub-expressions.
%
Types that appear inside expressions are syntactically restricted to
not contain bounds,
thus types inside expressions do not require translation.
%
Here we focus on the three interesting rules:

\begin{enumerate}
%
\item \emphbf{At bound abstractions} $\econstraint{\constraint}{e}$
 we convert the bound $\constraint$ into a bound-function
 parameter of a suitable type,
%
\item \emphbf{At variable binding sites} \ie $\lambda$- or let-bindings,
 we \emph{use} the bound functions to \emph{materialize} the
 bound constraints for all the variables in scope after the binding,
%
\item \emphbf{At bound applications} $\econstantconstraint{e}{\constraint}$
 we \emph{provide} regular functions that witness that the bound constraints hold.
%
\end{enumerate}

\begin{figure}[t!]
\centering
\captionsetup{justification=centering}
$$
\begin{array}{rrcl}
\centering
 \emphbf{Variable Environment} \quad 
   & \cenv & ::=
   & \emptyset \spmid  \EXT{\cenv}{x}{\utyp}
   \\[0.05in]
 
 \emphbf{Bound Environment} \quad 
   & \benv & ::=
   & \emptyset \spmid  \EXT{\benv}{x}{\utyp}
\end{array}
$$

\judgementHead{Translation}{\txexpr{\cenv}{\benv}{e}{e}}
$$
\inference{
}{
	\txexpr{\cenv}{\benv}{x}{x}
}[\txVar]
\qquad
\inference{
}{
	\txexpr{\cenv}{\benv}{c}{c}
}[\txCon]
$$
%
$$
\inference{
	\cenv' = \EXT{\cenv}{x}{\toshape{\rtyp}} && \txexpr{\cenv'}{\benv}{e}{e'} 
}{
	\txexpr{\cenv}{\benv}{\efunt{x}{\rtyp}{e}}{\efunt{x}{\rtyp}{\closure{\cenv'}{\benv}{e'}{x\colon\rtyp}}}
}[\txFun]
$$
% 
$$
\inference{
	\txexpr{\cenv}{\benv}{e_x}{e_x'} && \cenv' = \EXT{\cenv}{x}{\toshape{\rtyp}} &&
	\txexpr{\cenv'}{\benv}{e}{e'}
}{
	\txexpr{\cenv}{\benv}{\elet{x}{e_x}{e}{\rtyp}}
	{\elet{x}{e_x'}{\closure{\cenv'}{\benv}{e'}}{\tau}{x\colon\rtyp}}
}[\txLet]
$$
%
$$
\inference{
	\txexpr{\cenv}{\benv}{e_1}{e_1'} &&
	\txexpr{\cenv}{\benv}{e_2}{e_2'}
}{
	\txexpr{\cenv}{\benv}{\eapp{e_1}{e_2}}{\eapp{e_1'}{e_2'}}
}[\txApp]
$$
%
$$
\inference{
	\txexpr{\cenv}{\benv}{e}{e'}
}{
	\txexpr{\cenv}{\benv}{\etabs{\alpha}{e}}{\etabs{\alpha}{e'}}
}[\txTAbs]
\quad
\inference{
	\txexpr{\cenv}{\benv}{e}{e'}
}{
	\txexpr{\cenv}{\benv}{\etapp{e}{\rtyp}}{\etapp{e'}{\rtyp}}
}[\txTApp]
$$
%
$$
\inference{
	\txexpr{\cenv}{\benv}{e}{e'}
}{
	\txexpr{\cenv}{\benv}{\epabs{\rvar}{\rtyp}{e}}{\epabs{\rvar}{\rtyp}{e'}}
}[\txPAbs]
\quad
\inference{
	\txexpr{\cenv}{\benv}{e_1}{e_2'} &&
	\txexpr{\cenv}{\benv}{e_1}{e_2'}
}{
	\txexpr{\cenv}{\benv}{\epapp{e_1}{e_2}}{\epapp{e_1'}{e_2'}}
}[\txPApp]
$$
% 
$$
\inference{
	\text{fresh}\ f &&
	\txexpr{\cenv}{\benv, f\colon\toshape{\txbound{\constraint}}}{e}{e'}
}{
	\txexpr{\cenv}{\benv}{\econstraint{\constraint}{e} }{\efunt{f}{\txbound{\constraint}}{e'}}
}[\txCAbs]
\quad
\inference{
	\txexpr{\cenv}{\benv}{e}{e'}
}{
	\txexpr{\cenv}{\benv}{\econstantconstraint{e}{\constraint}}{\eapp{e'}{\ctofun{\constraint}}}
}[\txCApp]
$$
%%\nv{Note that basic and function types (\ie $\tau$) cannot have}
%%\nv{constraints,thus they do not require translation!}
%%\rj{I don't understand the above comment}
%%\nv{the above comment says that if type application had}
%%\nv{type schema instead of t (basic or function types only)}
%%\nv{then bounds could appear in the types inside expressions}
%%\nv{so, our translation would have to translate types too}
%%\nv{ie, e [t] translates to e' [t']}
\caption{Translation Rules from $\boundedcorelan$ to  $\corelan$.}
\label{fig:translation}
\end{figure}



\mypara{1.  Rule~\txCAbs} translates bound abstractions
$\econstraint{\phi}{e}$ into a plain $\lambda$-abstraction.
%
In the translated expression $\efunt{f}{\txbound{\constraint}}{e'}$
the bound becomes a function named $f$ with type
$\txbound{\constraint}$ defined:
%
\begin{align*}
\txbound{\efunt{x}{b}{\constraint}} \defeq & \tfun{x}{b}{\txbound{\constraint}}\\
\txbound{r} \defeq & \tref{\tbbool}{r}
\end{align*}
%
That is, $\txbound{\constraint}$ is a function type whose
final output carries the refinement corresponding to
the constraint in $\constraint$.
%
Note that the translation generates a fresh name $f$ for
the bound function (ensuring that it cannot be used in
the regular code) and saves it in the bound environment
$\benv$ to let us materialize the bound constraint when
translating the body $e$ of the abstraction.

\mypara{2. Rules~\txFun and~\txLet} materialize bound
constraints at variable binding sites ($\lambda$-abstractions
and let-bindings respectively).
%
%Even though application  of the axiomatic functions is required at specific places
%
If we view the bounds as universally quantified constraints
over the (abstract) refinements, then our translation exhaustively
and eagerly \emph{instantiates} the constraints at each point that
a new binder is introduced into the variable environment, over all
the possible candidate sets of variables in scope at that point.
%
The instantiation is performed by $\closure{\cenv}{\benv}{e}{x:\rtyp}$
%
%% \begin{figure}[t]
$$\begin{array}{rcl}
\closure{\cenv}{\benv}{e}{\tbind{x}{\utyp}}
  & \defeq & {\wraplet{e}{\cands{\cenv}{\benv}}} \\[0.05in]
\wraplet{e}{\set{e_1,\ldots,e_n}}
  & \defeq & {\elett{t_1}{e_1}{\ldots} \elett{t_n}{e_n}{e}} \\
  &        & {\mbox{where $t_i$ are fresh \tbbool binders}} \\[0.05in]
\cands{\cenv}{\benv}
  & \defeq & \{ \  f\ \overline{x}  \ | \ \tbind{f}{\utyp} \leftarrow \benv
                  , \ \overline{\tbind{x}{\_}} \leftarrow \cenv,\
    \EXT{\cenv}{f}{\utyp} \vdash_B \tbind{ f\ \overline{x}}{\tbbool}\ \} \
\end{array}$$
%% \caption{\textbf{Materializing Bound Functions}}
%% \label{fig:closure}
%% \end{figure}
% described in Figure~\ref{fig:closure} that
The function takes the environments
$\cenv$ and $\benv$, an expression $e$ and a variable $x$ of type
$\rtyp$ and uses let-bindings to materialize all the bound
functions in $\benv$ that accept the variable $x$.
%
Here, $\cenv \vdash_B \tbind{e}{\utyp}$ is the standard typing
derivation judgment for the non-refined System F and so
we elide it for brevity.
%%%\nv{Note that with the above definitions,}
%%%\nv{$\closure{\cenv}{\benv}{e}{\tbind{x}{\utyp}}$}
%%%\nv{will instantiate the bounds to all combinations}
%%%\nv{of variables, not only the ones that contain $x$}
%%%\nv{so you get much much repetition of instantiations}
%%%\nv{which is not unsound, but redundant}

\mypara{3. Rule~\txCApp} translates bound applications
$\econstantconstraint{e}{\constraint}$ into plain $\lambda$
abstractions that witness that the bound constraints
hold.
%
That is, as within $e$, bounds are translated to a bound
function (parameter) of type $\txbound{\constraint}$, we
translate $\constraint$ into a $\lambda$-term that, via
subtyping must have the required type $\txbound{\constraint}$.
%
We construct such a function via $\ctofun{\constraint}$
that depends only on the \emph{shape} of the bound,
\ie the non-refined types of its parameters (and not
the actual constraint itself).
%%
\begin{align*}
\ctofun{\reft} \defeq & \true \\
\ctofun{\efunt{x}{b}{\constraint}} \defeq &  \efunt{x}{b}{\ctofun{\constraint}}
\end{align*}
%
This seems odd: it is simply a constant function, how
can it possibly serve as a bound? The answer is that
subtyping in the translated \corelan term will verify
that in the context in which the above constant function
is created, the singleton $\true$ will indeed carry
the refinement corresponding to the bound constraint,
making this synthesized constant function a valid
realization of the bound function.
%
Recall that in the example @ex2@ of the overview (\S~\ref{sec:overview:implementation})
the subtyping constraint that decides is the constant $\true$
is a valid bound reduces to the equation \ref{vc:ex2}
that is a tautology.
%% \rj{ex2 EXAMPLE HERE?} \ref{vc:ex2}

\subsection{Soundness}\label{sec:soundness}

\begin{comment}
Definition of \erase{\dot}:

\begin{align*}
\erase{x}                      &= x \\
\erase{c}                      &= c\\
\erase{\eapp{e_1}{e_2}}        &= \eapp{\erase{e_1}}{\erase{e_2}}\\
\erase{\etabs{\alpha}{e}}      &= \etabs{\alpha}{\erase{e}}\\
\erase{\etapp{e}{\tau}}        &= \etapp{\erase{e}}{\erase{\tau}}\\
\erase{\epabs{\rvar}{\tau}{e}} &= \epabs{\rvar}{\erase{\tau}}{\erase{e}}\\
\erase{\epapp{e_1}{e_2}}           &= \epapp{\erase{e_1}}{\erase{e_2}}\\
%
\erase{\elet{x}{e_x}{e}{\tau}} &= \elet{x}{\erase{e_x}}{\erase{e}}{\erase{\tau}}\\
%
\erase{\econstraint{\phi}{e}}  &= \erase{e}\\
\erase{\econstantconstraint{e}{\phi}} &= \erase{e}
\end{align*}

\begin{definition}% [Operational]
$ e\boundedgoestostar{c} \Leftrightarrow {\erase{e}}\goestostar{c} $.
\end{definition}

The below does not hold:
\begin{lemma*}[Operational]%[Equivalence of Operational]
\label{theorem:operational}
If
   $\vdash_{B} e : \tau$ and
   $\txexpr{\emptyset}{\emptyset}{e}{e'}$
then
   $\erase{e} \goestostar{c} \Leftrightarrow {e'} \goestostar{c} $.
\end{lemma*}

Counterexample e = \{φ} =>  0

\end{comment}

\mypara{The Small-Step Operational Semantics} of \boundedcorelan
are defined by extending a similar semantics for \corelan
which is a standard call-by-value calculus where abstract
refinements are boolean valued functions.
%
Let $\stepcore$ denote the transition relation defining
the operational semantics of \corelan and \tclos{\stepcore}
denote the reflexive transitive closure of $\stepcore$.
%
We obtain the transition relation $\stepboundedcore$:
%
\begin{align*}
\econstantconstraint{(\econstraint{\constraint}{e})}{\constraint} &\boundedgoesto{e} &
e  & \stepboundedcore e', \text{if}\ {e \stepcore e'}
\end{align*}
%%%$$
%%%\inference{
%%%}{
%%%\econstantconstraint{(\econstraint{\constraint}{e})}{\constraint} \boundedgoesto{e}
%%%}[\rtobound]
%%%\qquad \qquad
%%%\inference{e \stepcore e'}
%%%          {e \stepboundedcore e'}[\rulename{O-Oth}]
%%%$$
%
Let $\boundedgoestostar{}$ denote the reflexive transitive
closure of $\stepboundedcore$.

\mypara{The Translation is Semantics Preserving} in the sense that
if a source term $e$ of $\boundedcorelan$ reduces to a constant
then the translated variant of $e'$ also reduces to the same
constant (as show in~\citep{vazou15techrep}):

\begin{lemma*}[Semantics Preservation]\ifextended{[Semantics Preservation]}{}
\label{theorem:operational}
If $\txexpr{\emptyset}{\emptyset}{e}{e'}$ and
   $e \boundedgoestostar{c}$
then $e' \goestostar{c}$.
\end{lemma*}

%%\rj{what about the extra let-bindings? techrep.}
%%
\ifextended{
\begin{proof}
By assumption, there exists a sequence
$e \equiv e_1 \boundedgoesto{e_2} \boundedgoesto{} \dots
\boundedgoesto{e_n\equiv c} $.
%
Let $i$ be the largest index in which rule \rtobound was applied.
%
Then, for some $\phi$ and $\phi'$,
$e_i$ contains a sub-expression of the form
$\econstantconstraint{(\econstraint{\phi}{e_i^0})}{\phi'}$.
%
Let $e_i^1$ be the expression we get if we replace
$\econstantconstraint{(\econstraint{\phi}{e_i^0})}{\phi'}$
with $e_i^0$ in $e_i$.
%
By the way we choose $i$, there exist a sequence
$e_i^1 \goestostar{c}$.

Let $e_i^2$
be the expression we get if we replace
$\econstantconstraint{(\econstraint{\phi}{e_i^0})}{\phi'}$
with $\eapp{(\efunt{f}{\txbound{\phi}}{e_i^0})}{(\ctofun{\phi'})}$ in $e_i$.
%
Then, since $f$ does not appear in $e_i^0$,
$e_i^2 \goestostar{c}$.
%
Finally,
let $g \defeq \ctofun{\phi'}$, then
by the definition of
$\ctofun{\cdot}$
we have that  $\forall e_1 \dots e_n$
if
there exists a type $\tau$ such that
$\emptyset \vdash g \ e_1 \dots e_n : \tau $,
then $g \ e_1 \dots e_n \goestostar{true}$.
%
Thus, for any expression,
if $e \goestostar{c}$, then $\elett{t}{f \ e_1 \dots e_n}{e}\goestostar{c}$

From the above, by the way we choose $i$ we have that
there exists a sequence
%
$\txex{e_i} \hookrightarrow \dots \hookrightarrow {c}$.

Since $n$ is finite, we iteratively apply the above procedure to
$e \equiv e_1\boundedgoesto{} \dots \boundedgoesto{} \txex{e_i}\hookrightarrow \dots \hookrightarrow {c}$.
until we get the sequence $ {\txex{e}}\goestostar{c}$.
\end{proof}}{}

\mypara{The Soundness of \boundedcorelan} follows by combining
the above Semantics Preservation
Lemma % ~\ref{theorem:operational}
with the soundness of \corelan.

\mypara{To Typecheck a \boundedcorelan program} $e$ we translate it
into a \corelan program $e'$ and then type check $e'$ using the rules of Figure~\ref{fig:rules}; 
if the latter check is safe, then we are guaranteed that the source term $e$ will
not crash:

\begin{theorem*}[Soundness]
\label{theorem:bounded}
If $\txexpr{\emptyset}{\emptyset}{e}{e'}$ and
   $\hastype{\emptyset}{e'}{\sigma}$
then $e \not \boundedgoestostar{\ecrash}$.
\end{theorem*}

\subsection{Inference}\label{sec:infer}

A critical feature of bounded refinements is that we can
automatically synthesize instantiations of the abstract
refinements by:
%
(1)~generating templates corresponding to the unknown types
    where fresh variables $\kvar$ denote the unknown refinements
    that an abstract refinement parameter $\rvar$ is instantiated
    with,
(2)~generating subtyping constraints over the resulting templates,
    and
(3)~solving the constraints via abstract interpretation.

\mypara{Inference Requires Monotonic Constraints.}
Abstract interpretation only works if the constraints
are \emph{monotonic}~\citep{cousotcousot77}, which in this case
means that the $\kvar$ variables, and correspondingly,
the abstract refinements $\rvar$ must only appear in
\emph{positive} positions within refinements (\ie not
under logical negations).
%
The syntax of refinements shown in Figure~\ref{fig:syntax}
violates this requirement via refinements of the
form $\rvapp{\rvar}{x} \Rightarrow \reft$.
%
%% (Note that the syntax otherwise only allows for positive
%% occurrences as $\neg (\rvapp{\rvar}{x})$ is disallowed.)

\mypara{We restrict implications to bounds} \ie prohibit
them from appearing elsewhere in type specifications.
%
Consequently, the implications only appear in the
\emph{output} type of the (first order) ``ghost''
functions that bounds are translated to.
%
The resulting subtyping constraints only have
\emph{implications inside super-types}, \ie as:
$$
\isSubType{\Gamma}{\stref{b}{\areft}}{\stref{b}{\areft_1 \Rightarrow \dots \Rightarrow \areft_{n} \Rightarrow\areft_q}}
$$
%
By taking into account the semantics of subtyping, we can
push the antecedents into the environment, \ie transform
the above into an equivalent constraint in the form:
$$
\isSubType{
\EXTT{
 \EXTT{\Gamma}{x_1}{\sxref{b_1}{\areft_1'}{x_1}},\dots
}{x_n}{\sxref{b_n}{\areft_n'}{x_n}}
}
{\stref{b}{\areft'}}
{\stref{b}{\areft_q'}}
$$
where all the abstract refinements variables $\rvar$
(and hence instance variables $\kvar$) appear positively,
ensuring that the constraints are monotonic, hence permitting
inference via Liquid Typing~\citep{LiquidPLDI08}.

\begin{comment}

\subsection{Semantics of \corelan}\label{sec:semantics-corelan}

\begin{figure*}[ht!]
\judgementHead{Well-Formedness}{\isWellFormed{\Gamma}{\sigma}}

$$\inference
    {\hastype{\Gamma, \vref:b}{\reft}{\tbbool}}
    {\isWellFormed{\Gamma}{\tref{b}{\reft}}}
    [\wtBase]
\qquad
\inference
    {
	%\hastype{\Gamma, v:\tfun{x}{\rtyp_x}{\rtyp}}{e}{\tbbool} &&
	\hastype{\Gamma}{\reft}{\tbbool} &&
    \isWellFormed{\Gamma}{\rtyp_x} &&
	\isWellFormed{\Gamma, x:\rtyp_x}{\rtyp}
    }
    {\isWellFormed{\Gamma}{\trfun{x}{\rtyp_x}{\rtyp}{\reft}}}
    [\wtFun]
$$

$$
\inference
  {\isWellFormed{\Gamma, \rvar:\rtyp}{\sigma}}
  {\isWellFormed{\Gamma}{\tpabs{\rvar}{\rtyp}{\sigma}}}
  [\wtPred]
\quad
\inference
    {\isWellFormed{\Gamma}{\sigma}}
    {\isWellFormed{\Gamma}{\ttabs{\alpha}{\sigma}}}
    [\wtPoly]
$$

\medskip \judgementHead{Subtyping}{\isSubType{\Gamma}{\sigma_1}{\sigma_2}}

$$
\inference
   {(\inter{\Gamma} \Rightarrow \inter{\reft_1} 
                 \Rightarrow  \inter{\reft_2})
                 \ \text{is valid}}
   {\isSubType{\Gamma}{\tref{b}{\reft_1}}{\tref{b}{\reft_2}}}
   [\tsubBase]
\qquad
\inference{
	\isSubType{\Gamma}{\rtyp_2}{\rtyp_1} &
	\isSubType{\Gamma, x_2:{\rtyp_2}}{\SUBST{\rtyp_1'}{x_1}{x_2}}{\rtyp_2'}	
   }
   {\isSubType{\Gamma}
	  {\trfun{x_1}{\rtyp_1}{\rtyp_1'}{\reft_1}}
	  {\trfun{x_2}{\rtyp_2}{\rtyp_2'}{\true}}
}[\tsubFun]
$$


$$
\begin{array}{c}
\inference
   {\isSubType{\Gamma, \rvar:\rtyp}{\sigma_1}{\sigma_2}}
   {\isSubType{\Gamma}{\tpabs{\rvar}{\rtyp}{\sigma_1}}{\tpabs{\rvar}{\rtyp}{\sigma_2}}}
   [\tsubPred]
\qquad
\inference
   {\isSubType{\Gamma}{\sigma_1}{\sigma_2}}
   {\isSubType{\Gamma}{\ttabs{\alpha}{\sigma_1}}{\ttabs{\alpha}{\sigma_2}}}
   [\tsubPoly]
\end{array}
$$

\medskip \judgementHead{Type Checking}{$\hastype{\Gamma}{e}{\sigma}$}

$$
\begin{array}{cc}
\inference
  {  \hastype{\Gamma}{e}{\sigma_2} && \isSubType{\Gamma}{\sigma_2}{\sigma_1} 
  && \isWellFormed{\Gamma}{\sigma_1}
  }
  {\hastype{\Gamma}{e}{\sigma_1}}
  [\tsub]
& 
\inference
  {\hastype{\Gamma}{e_x}{\rtyp_x} && 
   \hastype{\Gamma, x:\rtyp_x}{e}{\rtyp} && 
   \isWellFormed{\Gamma}{\rtyp}
  }
  {\hastype{\Gamma}{\elet{x}{e_x}{e}{}}{\rtyp}}
  [\tlet]
\end{array}
$$
$$
\begin{array}{ccc}

\inference
  {x: \tref{b}{r} \in \Gamma}
  {\hastype{\Gamma}{x}{\tref{b}{\vref = x}}}
  [\tbase]

&

\inference
  {x:\rtyp \in \Gamma}
  {\hastype{\Gamma}{x}{\rtyp}} 
  [\tvariable]
&
\inference
  {}
  {\hastype{\Gamma}{c}{\tc{c}}}
  [\tconst]

\\[0.2in]

\label{tapp}
\inference
   {\hastype{\Gamma}{e_1}{\tfun{x}{\rtyp_x}{\rtyp}} 
   &&  \hastype{\Gamma}{e_2}{\rtyp_x}
   }
   {\hastype{\Gamma}{\eapp{e_1}{e_2}}{\SUBST{\rtyp}{x}{e_2}}}
   [\tapp]

&

\inference
   {\hastype{\Gamma, x:\rtyp_x}{e}{\rtyp} 
    && \isWellFormed{\Gamma}{\rtyp_x}
   }
   {\hastype{\Gamma}{\efunt{x}{\rtyp_x}{e}}{\tfun{x}{\rtyp_x}{\rtyp}}}
   [\tfunction]

& 

\inference
  {\hastype{\Gamma, \alpha}{e}{\sigma}}
  {\hastype{\Gamma}{\etabs{\alpha}{e}}{\ttabs{\alpha}{\sigma}}}
  [\tgen]


\\[0.2in]


\inference
    {\hastype{\Gamma}{e}{\tpabs{\rvar}{\rtyp}{\sigma}} && 
     \hastype{\Gamma}{\efunbar{x:\rtyp_x}{\reft'}}{\rtyp}
    }
    {\hastype{\Gamma}
             {\epapp{e}{\efunbar{x:\rtyp_x}{\reft'}}}
             {\rpinst{\sigma}{\rvar}{\efunbar{x:\rtyp_x}{\reft'}}}
    }
    [\tpinst]
&

\inference
    {\hastype{\Gamma, \rvar:\rtyp}{e}{\sigma} &&
     \isWellFormed{\Gamma}{\rtyp} 
    }
    {\hastype{\Gamma}{\epabs{\rvar}{\rtyp}{e}}{\tpabs{\rvar}{\rtyp}{\sigma}}}
    [\tpgen]

&
\inference
  {\hastype{\Gamma}{e}{\ttabs{\alpha}{\sigma}} && 
   \isWellFormed{\Gamma}{\rtyp}
  }
  {\hastype{\Gamma}{\etapp{e}{\tau}}{\SUBST{\sigma}{\alpha}{\rtyp}}}
  [\tinst]

\end{array}$$
\caption[Type checking of \boundedcorelan.]{Static Semantics: Well-formedness, Subtyping and Type Checking.}
\label{fig:rules}
\end{figure*}



Figure \ref{fig:rules} summarizes the static semantics of \corelan
as described in ~\citep{vazou13}.
Unlike~\citep{vazou13} that syntactically separates concrete ($p$)
from abstract ($\rvapp{\rvar}{x}$) refinements,
here, for simplicity, we merge both concrete and abstract refinements to
atomic refinements $\areft$.

\mypara{A type environment} $\Gamma$ is a sequence of type bindings $x:\sigma$.
We use environments to define three kinds of judgments:

\begin{itemize}
\item{\emphbf{Well-formedness judgments} (\isWellFormed{\Gamma}{\sigma})}
state that a type schema $\sigma$ is well-formed under environment
$\Gamma$. That is, the judgment states that the refinements in $\sigma$
are boolean expressions in the environment $\Gamma$.

\item{\emphbf{Subtyping judgments} (\isSubType{\Gamma}{\sigma_1}{\sigma_2})}
state that the type schema $\sigma_1$ is a subtype of the type schema
$\sigma_2$ under environment $\Gamma$. That is, the judgment states that
when the free variables of $\sigma_1$ and $\sigma_2$ are bound to values
described by $\Gamma$, the values described by $\sigma_1$ are a subset
of those described by $\sigma_2$.

\item{\emphbf{Typing judgments} (\hastype{\Gamma}{e}{\sigma})} state that
the expression $e$ has the type schema $\sigma$ under environment $\Gamma$.
That is, the judgment states that when the free variables in $e$ are bound
to values described by $\Gamma$, the expression $e$ will evaluate to a value
described by $\sigma$.
\end{itemize}

\mypara{The Well-formedness rules}
check that the concrete and abstract refinements are indeed $\tbbool$-valued
expressions in the appropriate environment. The key rule is \wtBase, which
checks that the refinement $\reft$ is boolean.

\mypara{The Subtyping rules}
stipulate when the set of values described by schema $\sigma_1$ is subsumed
by (\ie contained within) the values described by $\sigma_2$.
The rules are standard except for \tsubBase, which reduces subtyping of
basic types to validity of logical implications, by translating the
refinements $r$ and the environment $\Gamma$ into logical formulas:
%
\begin{align*}
\inter{r}      & \defeq r &
\inter{\Gamma} & \defeq \bigwedge \{ \SUBST{r}{\vref}{x}\ |\ (x, \tref{b}{r}) \in \Gamma \}
\end{align*}
%
Recall that we ensure that the refinements $r$
belong to a decidable logic so that validity
checking can be performed by an off-the-self
SMT solver.

\mypara{Type Checking Rules}
are standard except for \tpgen and \tpinst, which
pertain to abstraction and instantiation of abstract refinements.
%
The rule \tpgen is the same as \tfunction: we simply check the body
$e$ in the environment extended with a binding for the refinement
variable $\rvar$.
%
The rule \tpinst checks that the concrete refinement is of the appropriate
(unrefined) type $\tau$, and then replaces all (abstract) applications of
$\rvar$ inside $\sigma$ with the appropriate (concrete) refinement $\reft'$
with the parameters $\overline{x}$ replaced with arguments at that application.
%
In~\cite{vazou13} we prove the following soundness result for \corelan
which states that well-typed programs cannot crash:

\begin{lemma*}[Soundness of \corelan~\cite{vazou13}]
\label{theorem:core}
If   $\hastype{\emptyset}{e}{\sigma}$
then $e \not \goestostar{\ecrash}$.
\end{lemma*}

\end{comment}

\section{A Refined Relational Database}\label{sec:database}

Next, we use bounded refinements to develop a library
for relational algebra, which we use to enable generic,
type safe database queries.
%
A relational database stores data in \emph{tables},
that are a collection of \emph{rows}, which in turn 
are \emph{records} that represent a unit of data 
stored in the table.
The tables's \textit{schema} describes the types of 
the values in each row of the table.
For example, the table in Figure~\ref{fig:moviedb} organizes 
information about movies, and has the schema:
%
\begin{code}
 Title:String, Dir:String, Year:Int, Star:Double
\end{code}

\begin{figure}[t]
$$
\begin{tabular}{| l | l| r | r |}
  \hline
  \textbf{Title} & \textbf{Director} & \textbf{Year} & \textbf{Star} \\
  \hline  
  ``Birdman'' & ``I\~{n}\'{a}rritu''   & 2014 & 8.1\\
  ``Persepolis''  & ``Paronnaud'' & 2007 & 8.0 \\ 
  \hline
\end{tabular}
$$
\caption{\label{fig:moviedb} Example Table of Movies}
\end{figure}

First, we show how to write type safe extensible 
records  that represent rows, and use them to 
implement database tables~(\S~\ref{subsec:records}). 
%
Next, we show how bounds let us specify type 
safe relational operations and how they may be 
used to write safe database queries~(\S~\ref{subsec:relational}).

\subsection{Rows and Tables}\label{subsec:records}

We represent the rows of a database with dictionaries, 
which are maps from a set of keys to values.
In the sequel, each key corresponds to a column, and 
the mapped value corresponds to a valuation of the column 
in a particular row.

\paragraph{A dictionary} @Dict <r> k v@ maps a key @x@ of 
type @k@ to a value of type @v@ that satisfies the property @r x@
%
% \NV CUT
\begin{code}
  type Range k v = k -> v -> Bool
   
  data Dict k v <r :: Range k v> = D {
      dkeys :: [k]
    , dfun  :: x:{k | x Set_mem elts dkeys} -> v<r x>
    }
\end{code}      
%
Each dictionary @d@ has a domain @dkeys@ 
\ie the list of keys for which @d@ is defined 
and a function @dfun@ that is defined only on
elements @x@ of the domain @dkeys@.
%
For each such element @x@, @dfun@ returns a value that satisfies the
property @r x@.

\paragraph{Propositions about the theory of sets} can be decided
efficiently by modern SMT solvers. Hence we use such 
propositions within refinements~\citep{realworldliquid}.
% 
The measures (logical functions) @elts@ and @keys@ 
specify the set of keys in a list and a dictionary 
respectively:
%
\begin{code}
  elts        :: [a] -> Set a
  elts ([])   = Set_emp
  elts (x:xs) = {x} Set_cup elts xs

  keys        :: Dict k v -> Set k
  keys d      = elts (dkeys d) 
\end{code}

\paragraph{Domain and Range of dictionaries.}
%
In order to precisely define the domain (\eg columns) and range (\eg values)
of a dictionary (\eg row), we define the following aliases:
%
% NV CUT
\begin{code}
  type RD k v <dom :: Dom k v, rng :: Range k v>
    = {v:Dict <rng> k v | dom v}

  type Dom k v = Dict k v -> Bool 
\end{code}
%
We may instantiate @dom@ and @rng@ with predicates that precisely describe
the values contained with the dictionary.
%
For example,
%
\begin{code}
  RD < \d -> keys d == {"x"}
     , \k v-> 0 < v         > String Int
\end{code}
%
%% instantiating @dom@ with 
%% \hbox{@\d -> keys d = {"x"}@}
%% and @rng@ with the predicate
%% \hbox{@\k v -> k = "x" => v > 0@}
%
describes dictionaries with a single field @"x"@ 
whose value (as determined by @dfun@) is stricly 
greater than 0.
%
We will define schemas by appropriately 
instantiating the abstract refinements 
@dom@ and @rng@.


\paragraph{An empty dictionary} has an empty domain 
and a function that will never be called:
%
\begin{code}
  empty   :: RD <emptyRD, rFalse> k v
  empty   = D [] (\x -> error "calling empty")

  emptyRD = \d -> keys d == Set_emp
  rFalse  = \k v -> false
\end{code}
%% %
%% Thus, @empty@'s range satisfies \textit{any} predicate -- that is,
%% it satisfies @false@.
 
\paragraph{We define singleton maps} as dependent pairs 
@x := y@ which denote the mapping from @x@ to @y@:
%
\begin{code}
  data P k v <r :: Range k v> 
    = (:=) {pk :: k, pv :: v<r pk>}
\end{code}
%
Thus, @key := val@ has type \hbox{@P<r> k v@} only if 
@r key val@.

\paragraph{A dictionary may be extended} with a singleton
binding (which maps the new key to its new value). 
%
\begin{code}
  (+=)   :: bind:P<r> k v 
         -> dict:RD<pTrue, r> k v 
         -> RD <addKey (pk bind) dict, r> k v
 
  (k := v) += (D ks f) 
         = D (k:ks) 
             (\i -> if i == k then v else f i)
  
  addKey = \k d d' -> keys d' == {k} Set_cup keys d
  pTrue  = \_ -> true
\end{code}
%
Thus, @(k := v)  += d@ evaluates to 
a dictionary @d'@ that extends @d@ 
with the mapping from @k@ to @v@.
%
The type of @(+=)@ constrains the new binding @bind@, 
the old dictionary @dict@ and the returned value to have 
the same range invariant @r@.
%
The return type states that the output dictionary's 
domain is that of the domain of @dict@ extended by 
the new key @(pk bind)@.

\paragraph{To model a row in a table} \ie a schema, 
we define the unrefined (Haskell) type @Schema@, 
which is a dictionary mapping @String@s, \ie the 
names of the fields of the row, to elements of 
some universe @Univ@ containing @Int@, @String@ 
and @Double@.
%
(A closed universe is not a practical restriction; 
most databases support a fixed set of types.)
% 
\begin{code}
  data Univ   = I Int | S String | D Double

  type Schema = RD String Univ
\end{code}

\paragraph{We refine Schema} with concrete instantiations
for @dom@ and @rng@, in order to recover precise 
specifications for a particular database. For example, 
@MovieSchema@ is a refined @Schema@ that describes the 
rows of the Movie table in Figure~\ref{fig:moviedb}:
%
%
\begin{code}
type MovieSchema = RD <md, mr> String Univ

  md = \d -> 
      keys d={"year","star","dir","title"}
  mr = \k v -> 
      (k = "year"  => isI v && 1888 < toI v)
   && (k = "star"  => isD v && 0 <= toD v <= 10)
   && (k = "dir"   => isS v)
   && (k = "title" => isS v)

  isI (I _)   = True 
  isI _       = False 

  toI       :: {v: Univ | isI v} -> Int
  toI (I n) = n
...
\end{code}
%
The predicate @md@ describes the \emph{domain} of the movie schema,
restricting the keys to exactly @"year"@, @"star"@, @"dir"@, and @"title"@.
%
The range predicate @mr@ describes the types of the values in the schema:
%
a dictionary of type @MovieSchema@ must map 
@"year"@ to an @Int@,
@"star"@ to a @Double@, 
and @"dir"@ and @"title"@ to @String@s.
%
The range predicate may be used to impose additional constraints on the values
stored in the dictionary.
%
For instance, @mr@ restricts the year to be not only an integer but
also greater than @1888@.
%
%%%Because refinements in \toolname are drawn from a decidable logic,
%%%refining the range with logical predicates comes ``for free''.
%%%%
%%%A more heavyweight dependent type system, on the other hand, would
%%%require the programmer to manually thread proofs of these range
%%%predicates throughout the code.

\paragraph{We populate the Movie Schema} by extending the
empty dictionary with the appropriate pairs of fields and 
values. For example, here are the rows from the table
in Figure~\ref{fig:moviedb}
%
\begin{code}
  movie1, movie2 :: MovieSchema
  movie1 = ("title" := S "Persepolis")
        += ("dir"   := S "Paronnaud") 
        += ("star"  := D 8) 
        += ("year"  := I 2007) 
        += empty

  movie2 = ("title" := S "Birdman") 
        += ("star"  := D 8.1) 
        += ("dir"   := S "Inarritu")
        += ("year"  := I 2014) 
        += empty
\end{code}
%
Typing @movie1@ (and @movie2@) as @MovieSchema@
boils down to proving:
%
That @keys movie1 = {"year", "star", "dir", "title"}@;
and that each key is mapped to an appropriate value 
as determined by @mr@.
%
For example, declaring @movie1@'s year to be @I 1888@
or even misspelling @"dir"@ as @"Dir"@
will cause the @movie1@ to become ill-typed.
%
As the (sub)typing relation depends on logical 
implication (unlike in @HList@ based approaches 
\cite{heterogeneous}) key ordering \emph{does not} 
affect type-checking;
%
in @movie1@ the star field is added before the 
director, while @movie2@ follows the opposite order.
%%%On the contrary, with dependent types, proving that two heterogeneous
%%%collections constructed with different ordering have the same type
%%%would require an additional (manually-supplied) equality proof.

\paragraph{Database Tables} are collections of rows, 
\ie collections of refined dictionaries.
%
We define a type alias @RT s@ (Refined Table) for 
the list of refined dictionaries from the field 
type @String@ to the @Univ@erse.
%
\begin{code}
  type RT (s :: {dom::TDom, rng::TRange}) 
    = [RD <s.dom, s.rng> String Univ]

  type TDom   = Dom   String Univ
  type TRange = Range String Univ
\end{code}
%
For brevity we pack both the domain and the range 
refinements into a record @s@ that describes the 
schema refinement; \ie each row dictionary has 
domain @s.dom@ and range @s.rng@.

For example, the table from Figure~\ref{fig:moviedb}
can be represented as a type @MoviesTable@ which 
is an @RT@ refined with the domain and range @md@ 
and @mr@ described earlier (\S~\ref{subsec:records}):
%
\begin{code}
  type MoviesTable = RT {dom = md, rng = mr}
   
  movies :: MoviesTable 
  movies = [movie1, movie2]
\end{code}

\subsection{Relational Algebra}\label{subsec:relational}

Next, we describe the types of the relational algebra operators
which can be used to manipulate refined rows and tables.
For space reasons, we show the \emph{types} of the basic 
relational operators; their (verified) implementations 
can be found online~\cite{liquidhaskellgithub}.
%
\begin{code}
  union   :: RT s -> RT s -> RT s
  diff    :: RT s -> RT s -> RT s
  select  :: (RD s -> Bool) -> RT s -> RT s
  project :: ks:[String] -> RTSubEqFlds ks s 
          -> RTEqFlds ks s
  product :: ( Disjoint s1 s2, Union s1 s2 s
             , Range s1 s, Range s2 s) 
          => RT s1 -> RT s2 -> RT s
\end{code}

\paragraph{\texttt{union} and \texttt{diff}} compute the union 
and difference, respectively of the (rows of) two tables.
%
The types of @union@ and @diff@ state that the 
operators work on tables with the same schema 
@s@ and return a table with the same schema.

\paragraph{\texttt{select}} takes a predicate @p@
and a table @t@ and filters the rows of @t@ 
to those which that satisfy @p@.
%
The type of @select@ ensures that @p@ will 
not reference columns (fields) that are
not mapped in @t@, as the predicate @p@
is constrained to require a dictionary 
with schema @s@.

\paragraph{\texttt{project}} takes
a list of @String@ fields @ks@ 
and a table @t@ and projects 
exactly the fields @ks@ at 
each row of @t@.
%
@project@'s type states that for 
any schema @s@, the input table 
has type @RTSubEqFlds ks s@ 
\ie its domain should have at 
least the fields @ks@ and the 
result table has type @RTEqFlds ks s@, 
\ie its domain has exactly the elements @ks@. 
%
\begin{code}
  type RTSubEqFlds ks s
    = RT s{dom = \z -> elts ks Set_sub  keys z}

  type RTEqFlds ks s
    = RT s{dom = \z -> elts ks == keys z}
\end{code}
% 
The range of the argument and the result tables 
is the same and equal to @s.rng@.

\paragraph{\texttt{product}} takes two tables 
as input and returns their (Cartesian) 
product.
%
It takes two Refined Tables with schemata 
@s1@ and @s2@ and returns a Refined Table 
with schema @s@. Intuitively, the output
schema is the ``concatenation'' of the input
schema; we formalize this notion using bounds:
%
(1)~@Disjoint s1 s2@ says the domains of 
    @s1@ and @s2@ should be disjoint,
%
(2)~@Union s1 s2 s@ says the domain of @s@ 
    is the union of the domains of @s1@ and @s2@, 
%
(3)~@Range s1 s@ (\resp @Range s2 s2@) says 
    the range of @s1@ should imply the result 
    range @s@; together the two imply the output
    schema @s@ preserves the type of each key in 
    @s1@ or @s2@.
%
\begin{code}
  bound Disjoint s1 s2 = \x y -> 
    s1.dom x => s2.dom y => keys x Set_cap keys y == Set_emp
   
  bound Union s1 s2 s = \x y v -> 
    s1.dom x => s2.dom y 
             => keys v == keys x Set_cup keys y 
             => s.dom v

  bound Range si s = \x k v -> 
    si.dom x => k Set_mem keys x => si.rng k v 
             => s.rng k v 
\end{code}

%% We note that none of these restrictions can be 
%% expressed using unbounded Abstract Refinement Types.

Thus, bounded refinements  enable the precise 
typing of  relational algebra operations.
They let us describe precisely when union, 
intersection, selection, projection and products 
can be computed, and let us determine, at compile
time the exact ``shape'' of the resulting tables.

% \subsection{Data Base Queries}\label{subsec:dbclient}

\paragraph{We can query Databases} by writing functions 
that use the relational algebra combinators. 
%
For example, here is a query that returns the 
``good'' titles -- with more than 8 stars -- 
from the @movies@ table
\footnote{More example queries can be found online~\cite{liquidhaskellgithub}}
%
\begin{code}
  good_titles = project ["title"] $ select (\d ->
                  toDouble (dfun d $ "star") > 8
                ) movies
\end{code}
%
%% The above query selects the movies that have more
%% than 8 stars and projects only their @"title"@ field.
%

Finally, note that our entire library -- including 
records, tables, and relational combinators -- is 
built using vanilla Haskell \ie without \emph{any} 
type level computation. 
%
All schema reasoning happens at the granularity of 
the logical refinements. That is if the refinements
are erased from the source, we still have a well-typed
Haskell program but of course, lose the safety 
guarantees about operations (\eg ``dynamic'' key lookup) 
never failing at run-time.


 

%%% Local Variables: 
%%% mode: latex
%%% TeX-master: "main"
%%% End: 

\section{A Refined IO Monad}\label{sec:state}

Next, we illustrate the expressiveness of Bounded Refinements by 
showing how they enable the specification and verification of 
stateful computations. We show how to 
%
(1)~implement a refined \emph{state transformer} 
    (\RIO) monad, where the transformer is indexed by refinements 
    corresponding to \emph{pre}- and \emph{post}-conditions 
    on the state~(\S~\ref{subsec:state:definition}),
%
(2)~extend \RIO with a set of combinators for 
    \emph{imperative} programming, \ie whose types 
    precisely encode Floyd-Hoare style program 
    logics~(\S~\ref{subsec:state:examples}) and
%
(3)~use the \RIO monad to write \emph{safe scripts}
    where the type system precisely tracks capabilities
    and statically ensures that functions only access 
    specific resources~(\S~\ref{subsec:state:files}).

%%% using a capability system as described in \shill~\citep{shill}.
%%% Our method is inspired by the method of~\citep{Filliatre98} 
%%% but unlike related techniques~\cite{ynot,dijkstramonad} we
%%% require no special support from the type system, and yet 
%%% ensure \emph{decidable checking} of verification conditions
%%% and \emph{inference} of loop invariants and pre- and
%%% post- conditions via liquid typing.

\subsection{The \RIO Monad}
\label{subsec:state:definition}

\paragraph{The \RIO data type} describes stateful computations.
Intuitively, a value of type @RIO a@ denotes a computation 
that, when evaluated in an input @World@ produces a value 
of type @a@ (or diverges) and a potentially transformed 
output @World@. We implement @RIO a@ as an abstractly
refined type (as described in ~\citep{vazou13})
%%(cf. \S~\ref{sec:overview:data})
%%\nv{TODO: compare with Vac from ART} 
%
%  data World
%
\begin{code}
  type Pre    = World -> Bool 
  type Post a = World -> a -> World -> Bool 

  data RIO a <p :: Pre, q :: Post a> = RIO { 
    runState :: w:World<p> -> (x:a, World<q w x>) 
  }
\end{code}
%
That is, @RIO a@ is a function @World-> (a, World)@, where
@World@ is a primitive type that represents the state of 
the machine \ie the console, file system, \etc
%
This indexing notion is directly inspired by the method 
of~\citep{Filliatre98} (also used in \cite{ynot}).

%%% but unlike related techniques~\cite{ynot,dijkstramonad} we
%%% require no special support from the type system, and yet 
%%% ensure \emph{decidable checking} of verification conditions
%%% and \emph{inference} of loop invariants and pre- and
%%% post- conditions via liquid typing.



\paragraph{Our Post-conditions are Two-State Predicates}
that relate the input- and output- world (as in~\cite{ynot}). 
%
Classical Floyd-Hoare logic, in contrast,
uses assertions which are single-state 
predicates.
%
We use two-states to smoothly account for 
specifications for stateful procedures. 
This increased expressiveness makes the 
types slightly more complex than a direct
one-state encoding which is, of course 
also possible with bounded refinements.


\paragraph{An \texttt{RIO} computation is parameterized} by two 
abstract refinements:
%
\begin{inparaenum}[(1)]
\item @p :: Pre@, which is a predicate over the \emph{input} 
   world, \ie the input world @w@ satisfies the refinement 
   @p w@; and
%
\item @q :: Post a@, which is a predicate relating the 
   \emph{output} world with the input world and the 
   value returned by the computation, \ie the output 
   world @w'@ satisfies the refinement @q w x w'@ where 
   @x@ is the value returned by the computation.
\end{inparaenum}
% 
Next, to use @RIO@ as a monad, we define @bind@ and 
@return@ functions for it, that satisfy the monad laws.
%%like the ones defined 
%%for Haskell's state monad.
  
\paragraph{The \return operator} yields a pair of the 
supplied value @z@ and the input world unchanged:
%
\begin{code}
  return   :: z:a -> RIO <p, ret z> a
  return z = RIO $ \w -> (z, w)

  ret z    = \w x w' -> w' == w && x == z
\end{code}
% $
The type of \return states that for any 
precondition @p@ and any supplied value 
@z@ of type @a@, the expression @return z@ 
is an \RIO computation with precondition
@p@ and a post-condition @ret z@.
The postcondition states that: 
%
(1)~the output @World@ is the same as the input, and 
%
(2)~the result equals to the supplied value @z@.
%
Note that as a consequence of the equality of the two worlds
and congruence, the output world @w'@ trivially satisfies @p w'@.
%
%% CHECK (3)~the result world satisfies the precondition @p@.
 
\paragraph{The \bind Operator} is defined in the usual way.
However, to type it precisely, we require bounded refinements.
%
\begin{code}
  (>>=) :: (Ret q1 r, Seq r q1 p2, Trans q1 q2 q)
        => m:RIO <p, q1> a
        -> k:(x:a<r> -> RIO <p2 x, q2 x> b)
        -> RIO <p, q> b 

  (RIO g) >>= f = RIO $ \x -> 
    case g x of { (y, s) -> runState (f y) s } 
\end{code}
%
The bounds capture various sequencing requirements 
(c.f. the Floyd-Hoare rules of consequence).
%
First, the output of the first action @m@, 
satisfies the refinement required by the 
continuation @k@;
%
\begin{code}
  bound Ret q1 r = \w x w' -> q1 w x w' => r x 
\end{code}
%
Second, the computations may be sequenced,
\ie the postcondition of the first action 
@m@ implies the precondition of the 
continuation @k@ (which may be dependent 
upon the supplied value @x@):
% 
\begin{code}
  bound Seq q1 p2 = \w x w' -> 
        q1 w x w' => p2 x w'
\end{code}%
%
Third, the transitive composition of the two 
computations, implies the final postcondition:
%
\begin{code}
  bound Trans q1 q2 q = \w x w' y w'' -> 
        q1 w x w' => q2 x w' y w'' => q w y w''
\end{code}
  
%%% \toolname verifies that the implementation of 
%%% @return@ and @>>=@ satisfy their refined type 
%%% signatures.
%$
Both type signatures would be impossible 
to use if the programmer had to manually 
instantiate the abstract refinements 
(\ie pre- and post-conditions.) 
%
Fortunately, Liquid Type inference % automatically
generates the instantiations making it practical
to use \toolname to verify stateful computations
written using @do@-notation.

\subsection{Floyd-Hoare Logic in the \RIO Monad}
\label{subsec:state:examples}

Next, we use bounded refinements to derive an
encoding of Floyd-Hoare logic, by showing how to 
read and write (mutable) variables and
typing higher order 
@ifM@ and @whileM@ combinators.

\paragraph{We Encode Mutable Variables} as fields of 
the @World@ type. For example, we might encode
a global counter as a field:
%
\begin{code}
  data World = { ... , ctr :: Int, ... }
\end{code}
%
We encode mutable variables in the refinement logic
using McCarthy's @select@ and @update@ operators 
for finite maps and the associated axiom:
%
\begin{code}
  select :: Map k v -> k -> v
  update :: Map k v -> k -> v -> Map k v

  forall m, k1, k2, v.
       select (update m k1 v) k2
    == (if k1 == k2 then v else select m k2 v)
\end{code}
%
The quantifier free theory of @select@ and @update@ is decidable
and implemented in modern SMT solvers~\cite{SMTLIB2}.

%%% which relates the two operators with
%%% the axioms
%%% %
%%% \begin{mcode}
%%% $\forall$wxe. select (update w x e) x = e
%%% $\forall$wxye.x$\neq$y $\Rightarrow$ select (update w x e) y = select w y
%%% \end{mcode}
%%% %
%%% %
%%%  are SMT-decidable~\cite{NelsonOppen}
%%% The operators are related by the axioms:
%%% %
%%% \begin{mcode}
%%% $\forall$wxe. select (update w x e) x = e
%%% $\forall$wxye.x$\neq$y $\Rightarrow$ select (update w x e) y = select w y
%%% \end{mcode}
%
\paragraph{We Read and Write Mutable Variables} via 
suitable ``get'' and ``set'' actions. For example,
we can read and write @ctr@ via:
%
\begin{code}
  getCtr   :: RIO <pTrue, rdCtr> Int
  getCtr   = RIO $ \w -> (ctr w, w)
    
  setCtr   :: Int -> RIO <pTrue, wrCtr n> ()
  setCtr n = RIO $ \w -> ((), w { ctr = n })
\end{code}
%$
Here, the refinements are defined as:
%
\begin{code}
  pTrue = \w -> True
  rdCtr = \w x w' -> w' == w && x == select w ctr
  wrCtr n = \w _ w' -> w' == update w ctr n 
\end{code}
%
Hence, the post-condition of @getCtr@ states 
that it returns the current value of @ctr@, 
encoded in the refinement logic with McCarthy's 
@select@ operator while leaving the world unchanged.
%
The post-condition of @setCtr@ states that @World@
is updated at the address corresponding to @ctr@,
encoded via McCarthy's @update@ operator. 

\paragraph{The \texttt{ifM} combinator} 
takes as input a @cond@ action that returns a @Bool@ and,
depending upon the result, executes either
the @then@ or @else@ actions. We type it as:
%
%% name the return state v to fit it in one line
\begin{code}
  bound Pure g = \w x v  ->(g w x v => v == w)
  bound Then g p1 = \w v -> (g w True  v => p1 v)
  bound Else g p2 = \w v -> (g w False v => p2 v)

  ifM :: (Pure g, Then g p1, Else g p2)
      => RIO <p , g> Bool       -- cond
      -> RIO <p1, q> a          -- then
      -> RIO <p2, q> a          -- else
      -> RIO <p , q> a
\end{code}
%
The abstract refinements and bounds 
correspond exactly to the hypotheses in the 
Floyd-Hoare rule for the @if@ statement.
%
The bound @Pure g@ states that the @cond@ 
action may access but does not \emph{modify} 
the @World@, \ie the output is the same 
as the input @World@. (In classical Floyd-Hoare 
formulations this is done by syntactically 
separating terms into pure \emph{expressions} 
and side effecting \emph{statements}).
%
The bound @Then g p1@ and @Else g p2@ respectively
state that the preconditions of the @then@ and @else@
actions are established when the @cond@ returns @True@
and @False@ respectively. 

%%% The @Then p qg p1@ bound states that 
%%% under an environment where the @b == True@, 
%%% the post-condition of the guard computation implies 
%%% the precondition of 	@e1@, 
%%% \hbox{\ie @qg w b v => p1 v@}
%%% where \hbox{@prop b@} holds.
%%% %
%%% Similarly, 
%%% the second bound states that 
%%% under an environment where the variable @b@ is false, 
%%% the post-condition of the guard computation implies 
%%% the precondition of 	@e2@, \ie \hbox{@qg w b v => p2 v@}
%%% where \hbox{@!(prop b)@} holds.
%%% %
%%% The final bound lifts the postcondition from the world @w'@
%%% to the world @w@, much like the lifting that was required in the 
%%% bind operator 
%%% (\S~\ref{subsec:state:definition}).


%% \NV{R B: types for product and ifM use many (separately defined) bounds.}
%% \NV{  You should empasize that this is just syntactic restriction, since}
%% \NV{  the implementations are synthesized automatically.}
%% \NV{  This is in contrast with type classes, which require instances.}

\paragraph{We can use \texttt{ifM}} to implement a stateful 
computation that performs a division, after checking 
the divisor is non-zero.
%
We specify that @div@ should not be called with a zero divisor. 
Then, \toolname verifies that @div@ is called safely:
%
\begin{code}
  div :: Int -> {v:Int | v /= 0} -> Int

  ifTest :: RIO Int
  ifTest = ifM nonZero divX (return 10)
    where nonZero = getCtr >>= return . (/= 0)
          divX    = getCtr >>= return . (div 42)
\end{code}
%
Verification succeeds as the post-condition of @nonZero@
is instantiated to 
\hbox{@\_ b w -> b <=> select w ctr /= 0@}
and the pre-condition of @divX@'s is instantiated to
\hbox{@\w -> select w ctr /= 0@}, which suffices to 
prove that @div@ is only called with non-zero values.

\paragraph{The \texttt{whileM} combinator} 
formalizes loops as @RIO@ computations:
%
\begin{code}
  whileM :: (OneState q, Inv p g b, Exit p g q)  
         => RIO <p, g> Bool      -- cond 
         -> RIO <pTrue, b> ()    -- body
         -> RIO <p, q> ()
\end{code}
%
As with @ifM@, the hypotheses of the Floyd-Hoare derivation rule
become bounds for the signature.
%
Given a @cond@ition with pre-condition @p@ and 
post-condition @g@ and @body@ with a true 
precondition and post-condition @b@, the computation 
@whileM cond body@ has precondition @p@ and post-condition 
@q@ as long as the bounds (corresponding to the Hypotheses
in the Floyd-Hoare derivation rule) hold.
%
First, @p@ should be a loop invariant; \ie when 
the @cond@ition returns @True@ the post-condition 
of the body @b@ must imply the @p@:
%
\begin{code}
  bound Inv p g b = \w w' w'' ->
      p w => g w True w' => b w' () w'' => p w'' 
\end{code}
%
Second, when the @cond@ition returns @False@ the invariant @p@
should imply the loop's post-condition @q@:
%
\begin{code}
  bound Exit p g q = \w w' ->
        p w => g w False w' => q w () w'
\end{code}
%
Third, to avoid having to transitively connect the guard and the body,
we require that the loop post-condition be a one-state predicate,
independent of the input world (as in Floyd-Hoare logic):
%
\begin{code}
  bound OneState q = \w w' w'' ->
        q w () w'' => q w' () w''
\end{code}

\paragraph{We can use \texttt{whileM}} to implement a loop that repeatedly
decrements a counter while it is positive, and to then verify that
if it was initially non-negative, then
at the end the counter is equal to @0@.
%
\begin{code}
  whileTest   :: RIO <posCtr, zeroCtr> ()
  whileTest = whileM gtZeroX decr
    where gtZeroX = getCtr >>= return . (> 0)
  
  posCtr  = \w -> 0 <= select w ctr
  zeroCtr = \_ _ w' -> 0 == select w ctr 
\end{code}
%
Where the decrement is implemented by @decr@ with type:
%
\begin{code}
  decr :: RIO <pTrue, decCtr> ()
  
  decCtr = \w _ w' -> 
    w' == update w ctr ((select ctr w) - 1)
\end{code}
% $
\toolname verifies that at the end of @whileTest@ 
the counter is zero (\ie the post-condition @zeroCtr@)
by instantiating suitable (\ie inductive) refinements
for this particular use of @whileM@.
 
%%%  as the conditional  @gtZeroX@'s 
%%% post-condition is instantiated to 
%%% \hbox{@\_ b w -> 0 <= select w ctr && b <=> 0 < select ctr w@}
%%% %
%%% and @whileM@'s precondition, that assumes the guard @b@ to be false, will be solved to 
%%% \hbox{@\w -> ctr w == 0@} that gives the specified postcondition.


%%%% ZAP Note that in the above type for @whileM@
%%%% ZAP the expression's precondition is set to true.
%%%% ZAP %
%%%% ZAP In theory we could set the expression's precondition 
%%%% ZAP to any abstract precondition implied by the conditional's 
%%%% ZAP post condition. 
%%%% ZAP %
%%%% ZAP By doing so, we would increase the precision of 
%%%% ZAP @whileM@'s signature but we would also add one extra bound.
%%%% ZAP %
%%%% ZAP Generalizing tha above reasoning, the more complicated the bounds
%%%% ZAP the more precise the type signature.
%%%% ZAP %
%%%% ZAP Since complex bounds drastically increase verification time, 
%%%% ZAP it is up to the user to find the golden ratio between 
%%%% ZAP precision and practicality.




%%% Local Variables: 
%%% mode: latex
%%% TeX-master: "main"
%%% End: 
%  LocalWords:  ret runState stateful

\section{Capability Safe Scripting via \RIO}
\label{sec:files}\label{subsec:state:files}

%% Remove this figure if easy
\begin{figure}[t]
\begin{mcode}
pread, pwrite, plookup, pcontents,
pcreateD, pcreateF, pcreateFP :: Priv -> Bool

active   :: World -> Set FH 
caps     :: World -> Map FH Priv

pset p h = \w -> p (select (caps w) h) && 
                 h $\in$ active w
\end{mcode}
\caption{Privilege Specification.}
\label{fig:fstypes} 
\end{figure}

%%%%\begin{figure}
\begin{code}
copySpec h d = \w ->
  pset pcontents h w && pset plookup h     w &&
  pset pread h     w && pset pcreateFile d w &&
  pset pwrite d    w && pset pcreateF d    w &&
  pwrite (pcreateFP (select (caps w) d)))

copyRec recur s d = 
  do cs <- contents s
     forM_ cs $ \ p -> do
       x <- flookup s p
       when (isFile x) $ do
         y <- create d p
         s <- fread x
         write y s
       when (recur && (isDir x)) $ do
         y <- createDir d p
         copyRec recur x y
\end{code}
\caption{\label{fig:copysrc} Specification and Source of \texttt{copyRec}}
\end{figure}%$
% \begin{figure}[t]
% \begin{code}
% copyRec :: Bool -> s:FH -> d:FH ->
%            RIO<\w     -> CopySpec v s d,
%               ,\_ _ w -> CopySpec v s d> () 
% \end{code}
% \caption{\label{fig:copytype1} (Incorrect) Type for \texttt{copySrc}}
% \end{figure}
% \begin{figure}[t]
% \begin{code}
% bound Stable i = \w x y v -> 
%   i w => (EqP v w ||
%          (Alloc v w x && (CopyP v w x y 
%                          || DerivP v w x y)))
%       => i v 

% copyRec :: (Stable i) => 
%            Bool -> s:FH -> d:FH ->
%            RIO<\w     -> i w && CopySpec v s d,
%               ,\_ _ w -> i w && CopySpec v s d> () 
% \end{code}
% \caption{\label{fig:copytype2} (Correct) Type for \texttt{copySrc}}
% \end{figure}
%%% Local Variables: 
%%% mode: latex
%%% TeX-master: "main"
%%% End: 

%%%%\begin{figure}
\begin{code}
findSpec i f = \w ->
  i w && pset pcontents f w && pset plookup f w
eqc f g = \w ->
  select (caps w) f == select (caps w) g
  
bound Cmd q i p = \f v ->
  q f => i v => p f v
bound Next q i p = \f g v ->
  q f => p f v && g in active v && eqc f g v
      => p g v

findExec ::
  (Stable i, Cmd q i p, Next q i p)
  => f:FHandle<q>
  -> (g:FHandle -> RIO<p g, eqp> ())
  -> RIO<findSpec i f, \_ _ -> findSpec i f> () 
findExec f cmd = 
  do when (isFile f) $ cmd f
     when (isDir f) $
       do cs <- contents f
          forM_ cs $ \z ->
            do h <- flookup f z
              findExec h cmd
            
findWr d = \w -> findSpec d w && pset pwrite d w
prepend :: String -> d:FH -> RIO<findWr, True> ()
prepend s d = findExec dir (\f -> write f s)
\end{code}
\caption{\label{fig:findsrc} Specification and Source of \texttt{findExec}}
\end{figure}
%%% Local Variables: 
%%% mode: latex
%%% TeX-master: "main"
%%% End: 


% \section{Script Permissions}\label{sec:files}\label{subsec:state:files}
Next, we describe how we use the \RIO monad to reason about shell
scripting, inspired by the \shill~\citep{shill} programming language.
%

\paragraph{\shill} is a scripting language that restricts the
privileges with which a script may execute by using
\emph{capabilities} and \emph{dynamic contract checking}~\citep{shill} .
%
Capabilities are \emph{run-time values} 
that witness the right to use a particular resource 
(\eg a file).
%
A capability is associated with a set of privileges, 
each denoting the permission to use the capability 
in a particular way (such as the permission to write 
to a file).
%
A contract for a \shill procedure describes the 
required input capabilities and any output values.
%
The \shill runtime guarantees that system resources are accessed in
the manner described by its contract.

In this section, we turn to the problem of
preventing \shill runtime failures.
%
(In general, the verification of file system resource usage is a rich
topic outside the scope of this paper.)
%
That is, assuming the \shill runtime and an API as described in
\cite{shill}, how can we use Bounded Refinement Types to encode
scripting privileges and reason about them \emph{statically?}

\paragraph{We use \RIO types} to specify \shill 's API operations
%
thereby providing \emph{compile-time} guarantees 
about privilege and resource usage.
%
To achieve this, we:
%
connect the state (@World@) of the \RIO monad with a
\emph{privilege specification} denoting the set of 
privileges that a program may use~(\S~\ref{sec:privilege-spec});
%
specify the \emph{file system API} in terms of this
abstraction~(\S~\ref{sec:fs-api});
%
and use the above to specify and verify the particular 
privileges that a \emph{client} of the API uses~(\S~\ref{sec:fs-client}).

\subsection{Privilege Specification}
\label{sec:privilege-spec}
%
Figure~\ref{fig:fstypes} summarizes how we specify privileges 
inside @RIO@. 
%
We use the type @FH@ to denote a file handles, analogous to \shill's
capabilities. An abstract type @Priv@ denotes the sets of privileges
that may be associated with a particular @FH@.

\paragraph{To connect \texttt{World}s with Privileges} we assume 
a set of uninterpreted functions of type @Priv ->  Bool@ 
that act as predicates on values of type @Priv@, each 
denoting a particular privilege.
%
For example, given a value @p :: Priv@, the proposition 
@pread p@ denotes that @p@ includes the ``read'' privilege.
%
The function @caps@ associates each @World@ with a @Map FH Priv@,
a table that associates each @FH@ with its privileges.
% 
The function @active@ maps each @World@ to the @Set@ of
allocated @FH@s.
%
Given @x:FH@ and @w:World@, @pwrite (select (caps w) x)@
denotes that in the state @w@, the file @x@ 
may be written.
%
This pattern is generalized by the predicate @pset pwrite x w@.

\subsection{File System API Specification}
\label{sec:fs-api}
%
A privilege tracking file system API can be partitioned into the
privilege \emph{preserving} operations and the privilege \emph{extending}
operations.

\paragraph{To type the privilege preserving} operations, we define a predicate
@eqP w w'@ that says that the set of privileges and active handles
in worlds @w@ and @w'@ are \emph{equivalent}.
%
\begin{code}
  eqP = \w _ w' -> 
    caps w == caps w' && active w == active w'
\end{code}
%
We can now specify the privilege preserving operations that @read@ and @write@ files, 
and list the @contents@ of a directory, all of which require the 
capabilities to do so in their pre-conditions:
%
\begin{code}
  read :: {- Read the contents of h -}
    h:FH -> RIO<pset pread h, eqp> String
  
  write :: {- Write to the file h -}
    h:FH -> String -> RIO<pset pwrite h, eqp> ()
  
  contents :: {- List the children of h -}
    h:FH -> RIO<pset pcontents h, eqp> [Path]
\end{code} 

\paragraph{To type the privilege extending} operations, we define 
predicates that say that the output world is suitably 
extended. First, each such operation \emph{allocates} 
a new handle, which is formalized as:
%
\begin{mcode}
  alloc w' w x = 
    (x $\not \in$ active w) && active w' == {x} $\cup$ active w
\end{mcode}
%
which says that the active handles in (the new @World@) 
@w'@ are those of (the old @World@) @w@ extended with the
hitherto \emph{inactive} handle @x@.
%
Typically, after allocating a new handle, a script will
want to add privileges to the handle that are obtained
from existing privileges.

\paragraph{To \texttt{create} a new file} in a directory with handle @h@ we
want the new file to have the privileges \emph{derived} from
@pcreateFP (select (caps w) h)@ (\ie the create privileges of @h@). We
formalize this by defining the post-condition of @create@ as the predicate @derivP@:
%
\begin{code}
  derivP h  = \w x w' -> 
    alloc w' w x && 
    caps w' == store (caps w) x 
                  (pcreateFP (select (caps w)) h)

  create :: {- Create a file -}
    h:FH->Path->RIO<pset pcreateF h, derivP h> FH
\end{code}
%
Thus, if @h@ is writable in the old @World w@ 
(@pwrite (pcreateFP (select (caps w) h))@) and
@x@ is derived from @h@ (@derivP w' w x h@ both hold),
then we know that @x@ is writable in the new @World w'@
(@pwrite (select (caps w') x)@).

\paragraph{To \texttt{lookup} existing files} or create sub-directories,
we want to directly \emph{copy} the privileges of the parent handle. 
We do this by using a predicate @copyP@ as the post-condition for 
the two functions:
%
\begin{code}
  copyP h = \w x w' ->
    alloc w' w x && 
    caps w' == store (caps w) x 
                     (select (caps w) y)

  lookup :: {- Open a child of h -}
    h:FH->Path->RIO<pset plookup h, copyP h> FH

  createDir :: {- Create a directory -}
    h:FH->Path->RIO<pset pcreateD h, copyP h> FH
\end{code}
  
%%\rj{HEREHEREHERE}
%%
\subsection{Client Script Verification}
\label{sec:fs-client}
%
We now turn to a client script,
the program @copyRec@ % in Figure~\ref{fig:copysrc}
that copies the contents of the directory @f@ to the
directory @d@.
%
\begin{code}
  copyRec recur s d = 
    do cs <- contents s
       forM_ cs $ \ p -> do
         x <- flookup s p
         when (isFile x) $ do
           y <- create d p
           s <- fread x
           write y s
         when (recur && (isDir x)) $ do
           y <- createDir d p
           copyRec recur x y
\end{code}%$
%
@copyRec@ executes by first listing the contents of @f@, 
and then opening each child path @p@ in @f@. 
%
If the result is a file, it is copied to the directory @d@.
%
Otherwise, @copyRec@ recurses on @p@, if @recur@ is true.

In a first attempt to type @copyRec@ we give it the following type:
\begin{code}
  copyRec :: Bool -> s:FH -> d:FH ->
             RIO<copySpec s d,
                 \_ _ w -> copySpec s d w> () 

 copySpec h d = \w ->
   pset pcontents h w && pset plookup h     w &&
   pset pread h     w && pset pcreateFile d w &&
   pset pwrite d    w && pset pcreateF d    w &&
   pwrite (pcreateFP (select (caps w) d)))
\end{code}              
%
The above specification gives @copyRec@ a minimal set of privileges. 
%
Given a source directory handle @s@ and destination handle @d@, the
@copyRec@ must at least:
%
\begin{inparaenum}[(1)]
  \item list the contents of @s@ (@pcontents@),
  \item open children of @s@ (@plookup@),
  \item read from children of @s@ (@pread@),
  \item create directories in @d@ (@pcreateD@),
  \item create files in @d@ (@pcreateF@), an
  \item write to (created) files in @d@ (@pwrite@).
\end{inparaenum}
%
Furthermore, we want to restrict the privileges on newly created files
to the write privilege, since @copyRec@ does not need to read from or
otherwise modify these files.

Even though the above type is sufficient to verify
the various clients of @copySpec@ it
is insufficient to verify @copySpec@'s implementation, 
as the postcondition merely states that @copySpec s d w@ holds.
%
Looking at the recursive call in the last line of @copySpec@'s implementation,
the output world @w@ is only known to satisfy @copySpec x y w@ (having
substituted the formal parameters @s@ and @d@ with the actual @x@ and
@y@), with no mention of @s@ or @d@!
%
Thus, it is impossible to satisfy the postcondition of @copyRec@, as
information about @s@ and @d@ has been lost.

\paragraph{Framing} is introduced to address the above problem.
Intuitively, because no privileges are ever
\emph{revoked}, if a privilege for a file existed \emph{before} the
recursive call, then it exists \emph{after} as well.
%
We thus introduce a notion of \emph{framing} -- assertions about
unmodified state that hold before calling @copyRec@ must hold after
@copyRec@ returns.
%
Solidifying this intuition, we define a predicate @i@ to be @Stable@
when assuming that the predicate @i@
holds on @w@, if @i@ only depends on the allocated set of 
privileges, then @i@ will hold on a world @w'@ so long as
the set of priviliges in @w'@ contains those in @w@.
%
The definition of @Stable@ is derived precisely from the ways in which
the file system API may modify the current set of privileges:
%
\begin{code}
  bound Stable i = \x y w w' -> 
   i w => ( eqP w () w' || copyP y w x w'
           || derivP y w x w'
          ) => i w'
\end{code}
%
We thus parameterize @copyRec@ by a predicate @i@, bounded by @Stable i@, 
which precisely describes the possible world transformations under which 
@i@ should be stable:
%
\begin{code}
  copyFrame i s d = \w -> i w && copySpec s d w

  copyRec :: (Stable i) => 
             Bool -> s:FH -> d:FH ->
             RIO<copyFrame i s d,
                 \_ _ w -> copyFrame i s d w> () 
\end{code}              
%
Now, we can verify @copyRec@'s body, as
at the recursive call that appears in the last line of the implementation,
@i@ is instantiated with 
%
@\w -> copySpec s d w@.

%%%%\mypara{Higher Order Scripts}
%%%%%
%%%%Using the same mechanism as @copyRec@, we can type higher-order
%%%%privilege-typed programs.
%%%%%
%%%%In Figure~\ref{fig:findsrc} we use Bounded Refinement Types to type the 
%%%%higher-order @findExec@, which recurses on a directory tree, executing its
%%%%input @cmd@ on each encountered file.
%%%%%
%%%%The first constraint implements framing as in @copyRec@.
%%%%%
%%%%Since the input @cmd@ may require privileges beyond those mentioned in
%%%%@findSpec f w@, we denote its precondition with the abstract
%%%%refinement @p@.
%%%%%
%%%%We then partition the @FH@s for which @p@ must hold using the
%%%%predicate @q@ -- thus, the bound @Cmd@ says that if @q@ holds for some
%%%%@f@, then @i@ implies that @p@, the precondition for @cmd@, holds for
%%%%@f@.
%%%%%
%%%%Finally, to allow @cmd@ to be applied to files \emph{derived} from
%%%%@f@, the third bound, @Next@, says that for a given @f :: FH<q>@, if
%%%%for some world @w@ @p f w@ holds, then @p g w@ holds for @g :: FH@
%%%%whose privileges are equivalent to @f@ in @w@.
%%%% 
%%%%The script @prepend@ recurses on the contents of a directory @d@ and
%%%%writes a string to each encountered file by calling @findExec@.
%%%%%
%%%%At the call to @findExec@, @i@ and @p@ are instantiated to:
%%%%%
%%%%\begin{mcode}
%%%%  i $\mapsto$ \w -> findSpec d w && pset pwrite d w
%%%%  p $\mapsto$ \w -> pset pwrite d w 
%%%%\end{mcode}
%%%%%
%%%%The remaining obligations (@Stable i@, @Cmd q i p@, and @Next q i p@)
%%%%are easily dispatched by the SMT solver, thus verifying @prepend@.


%%% Local Variables: 
%%% mode: latex
%%% TeX-master: "main"
%%% End: 

\chapter{Related Work}\label{chapter:related}

\toolname combines ideas from four main lines of research fields. 
%
It is a 
refinement type checker (\S~\ref{related:refinementtypes})
that enjoys SMT-based (\S~\ref{related:smtbased}) automated type checking. 
Via Refinement Reflection we touch the expressiveness 
of fully dependently types systems (\S~\ref{related:dependenttypes}), 
getting an automated and expressive verifier for Haskell programs (\S~\ref{related:haskell}).  
%

%%%%%%%%%%%%%%%%%%%%%%%%%%%%%%%%%%%%%%%%%%%%%%%%%%%%%%%%%%%%%%%%%%%%%%%%%%%%%%%
%%%%%%%%%%%%%%% Classic refinement types %%%%%%%%%%%%%%%%%%%%%%%%%%%%%%%%%%%%%%
%%%%%%%%%%%%%%%%%%%%%%%%%%%%%%%%%%%%%%%%%%%%%%%%%%%%%%%%%%%%%%%%%%%%%%%%%%%%%%%

\section{Refinement Types}\label{related:refinementtypes}

\mypara{Standard Refinement Types}
Refinement Types were introduced by Freeman and
Pfenning~\cite{FreemanPfenning91}, with refinements limited to
restrictions on the structure of algebraic datatypes. 
%
Freeman and Pfenning carefully designed the refinement logic 
to ensure \textit{decidable type inference}
via the notion of predicate subtyping (PVS~\cite{Rushby98}).
% 
The goal of refinement types is 
to refine the type system of an existing, general purpose,  
target language so that it
\textit{rejects more programs} as ill typed, 
unlike dependent type systems, 
that aim to increase the expressiveness 
and alter the semantics of the language.

\mypara{Applications of Refinement Types}
Xi and Pfenning implemented DML~\cite{pfenningxi98}
a refinement type checker for ML 
where arrays are indexed by terms 
from Presburger arithmetic to statically eliminate array bound checking. 
%
Since then, refinement types have been implemented for various general purpose languages, 
including 
ML~\cite{GordonTOPLAS2011,LiquidPLDI08},
C~\cite{deputy,LiquidPOPL10},
Racket~\cite{RefinedRacket}
and Scala~\cite{refinedscala}
to prove various correctness properties ranging from safe memory accessing 
to correctness of security protocols.
%
All the above systems operate under CBV semantics 
that implicitly assume that all free variables are bound to values. 
%
This assumption, that breaks under Haskell's lazy semantics,
turned out to be crucial for the soundness 
of refinement type checking.
To restore soundness in \toolname we 
use a refinement type based termination checker 
to distinguish between provably terminating and potential diverging free variables. 


%%%%%%%%%%%%%%%%%%%%%%%%%%%%%%%%%%%%%%%%%%%%%%%%%%%%%%%%%%%%%%%%%%%%%%%%%%%%%%%
%%%%%%%%%%%%%%% Extensions of refinement types %%%%%%%%%%%%%%%%%%%%%%%%%%%%%%%%
%%%%%%%%%%%%%%%%%%%%%%%%%%%%%%%%%%%%%%%%%%%%%%%%%%%%%%%%%%%%%%%%%%%%%%%%%%%%%%%

\mypara{Reconciliation between Expressiveness and Decidability}
Reluctant to give up decidable type checking, 
many systems have pushed the expressiveness of refinement types
within decidable logics. 
%
Kawaguchi \etal~\cite{LiquidPLDI09} 
introduce \emph{recursive} and \emph{polymorphic} refinements for data
structure properties increasing the expressiveness but also the complexity 
of the underlying refinement system. 
%
\catalyst~\citep{catalyst} permits a form of
higher order specifications where refinements
are relations which may themselves be parameterized
by other relations.
%
However, to ensure decidable checking, \catalyst
is limited to relations that can be specified as
catamorphisms over inductive types, precluding
for example, theories like arithmetic.
%
In the same direction, Abstract and Bounded refinement types 
encode modular, higher order specifications 
using the decidable theory of uninterpreted functions. 
% 
All the above systems
only allow for ``shallow'' specifications, 
where the underlying solver can only reason about 
(decidable) abstractions of user defined functions 
and not the exact description of the function  implementations of the functions. 
%
Refinement Reflection, on the other hand, 
reflects user defined function definitions 
into the logic, allowing for ``deep'' program specifications
but requiring the user to manually provide cumbersome proof terms. 



\section{SMT-based verification}\label{related:smtbased}

Knowles and Flanagan~\cite{Knowles10} allow refinement predicates to
be arbitrary terms of the language being typechecked and present a
technique for deciding some typing obligations statically and
deferring others to runtime.
%; Gronksi \etal~\cite{Gronski06} present animplementation of such a system.

The \fstar system enables full dependent typing via
SMT solvers via a higher-order universally quantified
logic that permit specifications similar to ours
(\eg @compose@, @filter@ and @foldr@).
%% https://github.com/FStarLang/FStar/
%
While this approach is at least as expressive
as bounded refinements it has two drawbacks.
%
First, due to the quantifiers, the generated VCs
fall outside the SMT decidable theories.
This renders the type system undecidable (in theory),
forcing a dependency on the solver's unpredictable
quantifier instantiation heuristics (in practice).
%
Second, more importantly, % perhaps more importantly,
the higher order
predicates must be \emph{explicitly} instantiated,
placing a heavy annotation burden on the programmer.
%
In contrast, bounds permit decidable
checking, and are automatically instantiated
via Liquid Types.


\paragraph{SMT-Based Verification}
%
SMT solvers have been extensively used to automate
reasoning on verification languages like
Dafny~\cite{dafny}, Fstar~\cite{fstar} and Why3~\cite{why3}.
%
These languages are designed for verification,
thus have limited support for commonly used language
features like parallelism and optimized libraries
that we use in our verified implementation.
%


% We compare refinement reflection to the most closely related
% lines of work in the vast literature on program verification.

\mypara{SMT-Based Verification}
%
SMT-solvers have been extensively used to automate
program verification via Floyd-Hoare logics~\cite{Nelson81}.
%
Our work is inspired by Dafny's Verified
Calculations~\citep{LeinoPolikarpova16},
a framework for proving theorems in
Dafny~\citep{dafny}, but differs in
%
(1)~our use of reflection instead of axiomatization and
(2)~our use of refinements to compose proofs.
%
Dafny, and the related \fstar~\citep{fstar}
which like \toolname, uses types to compose
proofs, offer more automation by translating
recursive functions to SMT axioms.
However, unlike reflection, this axiomatic
approach renders typechecking and verification
undecidable (in theory) and leads to
unpredictability and divergence
(in practice)~\citep{Leino16}.
%\NV{CHECL Relational-F*, Barthe et al, from POPL 2014, and EasyCrypt}

%% In a work more closely related to
%% ours, \fstar uses refinement types
%% for program verification supporting
%% expressiveness of fully dependent types.
%% %
%% As in Dafny, \fstar directly translates
%% recursive functions to axioms in the logic
%% thus suffers from the ``butterfly effect''
%% and allows the user to explicitly write SMT tactics to control it.

%% Leino \etal~\citep{Leino16}
%% name this problem as the ``butterfly
%% effect'', in which minor modifications
%% to the program source cause significant
%% instabilities in verification and propose
%% trigger selection strategies to address it.
%% %
%% We avoid the ``butterfly effect'' by not
%% directly axiomatizing functions into logic.
%% Instead the information about the function's
%% body is exactly captured in function's result
%% type and user needs to explicitly invoke the function to push
%% the function's definition information into the logic.

\begin{comment}

\section{Dependent Type Systems}\label{related:dependenttypes}

%%%%%%%%%%%%%%%%%%%%%%%%%%%%%%%%%%%%%%%%%%%%%%%%%%%%%%%%%%%%%%%%%%%%%%%%%%%%%%%
%%%%%%%%%%%%%%% Fully dependent typing %%%%%%%%%%%%%%%%%%%%%%%%%%%%%%%%%%%%%%%%
%%%%%%%%%%%%%%%%%%%%%%%%%%%%%%%%%%%%%%%%%%%%%%%%%%%%%%%%%%%%%%%%%%%%%%%%%%%%%%%

\mypara{Dependent types}
%
Our work is also inspired by dependently typed
systems like Coq~\citep{coq-book} and
Agda~\citep{agda}.
%
Reflection shows how deep specification
and verification in the style of Coq and Agda
can be \emph{retrofitted} into existing languages
via refinement typing.
%
Furthermore, we can use SMT to significantly
automate reasoning over important theories like
arithmetic, equality and functions.
%
It would be interesting to investigate how
the tactics and sophisticated proof search
of Coq \etc can be adapted to the refinement setting.

% which allow for arbitrary expressiveness of the type system
% in the cost of automatic verification.
%
%% The syntax of \libname's operators is inspired by
%% Equational Reasoning in Agda~\citep{agda}.
%% Here we extended these equational operators
%% to support linear arithmetic and, for example, prove
%% properties of Ackermann function.
%% %
%% Unlike Adga, proof term are explicit in \libname,
%% we do not use heuristics to infer proofs.
\mypara{Higher order Logics and Dependent Type Systems}
%
including
NuPRL~\citep{Constable86},
Coq~\citep{coq-book}, Agda~\citep{norell07},
and even to some extent, \haskell~\citep{JonesVWW06, McBride02},
occupy the maximal extreme of the expressiveness spectrum.
However, in these settings, checking requires explicit
proof terms which can add considerable programmer overhead.
%
Our goal is to eliminate the programmer overhead of
proof construction by restricting specifications to
decidable, first order logics and to see how far
we can go without giving up on expressiveness.

%  Higher-order logics: Coq/HTT/F*/Agda which have explicit predicates, quantification 
A number of higher-order logics and corresponding verification tools
have been developed for reasoning about programs.
%
Example of systems of this type include NuPRL \cite{Constable86},
%F$_{<:}$ \cite{Cardelli91},
Coq \cite{coq-book}, F$^\star$ \cite{SwamyCFSBY11} and Agda \cite{norell07}
which support the development and verification of higher-order, 
pure functional programs.
%
While these systems are highly expressive, their expressiveness comes at the
cost of making logical validity checking undecidable.
%
To help automate validity checking, both built-in and user-provided
tactics are used to attempt to discharge proof obligations; however,
the user is ultimately responsible for manually proving any
obligations which the tactics are unable to discharge.



\spara{Dependent Types} are the basis of many verifiers, 
or more generally, proof assistants.
%
Verification of haskell code is possible with
``full'' dependently typed systems like Coq~\cite{coq-book}, 
Agda~\cite{norell07}, Idris~\cite{Brady13}, Omega~\cite{Sheard06}, and
 {$\lambda_\rightarrow$}~\cite{LohMS10}.
 %
 While these systems are highly expressive,
their expressiveness comes at the cost of making logical validity checking undecidable
thus rendering verification cumbersome.	
 %
 
 \spara{Dependent Types} are the basis of many verifiers, 
or more generally, proof assistants.
%
In this setting arbitrary terms may appear inside types,
so to prevent logical inconsistencies, and enable
the checking of type equivalence, all terms must
terminate.
%
``Full'' dependently typed systems like Coq~\cite{coq-book}, 
Agda~\cite{norell07}, and Idris~\cite{Brady13} typically use 
\emph{structural} checks where recursion is allowed on 
sub-terms of ADTs to ensure that \emph{all} terms terminate.
%
We differ in that, since the refinement logic is
restricted, we do not require that all functions terminate,
and hence, we can prove properties of possibly diverging 
functions like @collatz@ as well as lazy functions like @repeat@.
%
Recent languages like Aura~\citep{AURA} and Zombie~\citep{Zombie}
allow general recursion, but constrain the logic to a terminating 
sublanguage, as we do, to avoid reasoning 
about divergence in the logic.
%
In contrast to us, the above systems crucially assume 
\emph{call-by-value} semantics to ensure that binders are bound
to values, \ie cannot diverge.




\paragraph{Dependent Types}
Unlike Refinement Types, dependent type systems,
like Coq~\cite{coq-book}, Adga~\cite{agda} and Isabelle/HOL~\cite{isabelle} allow for ``deep'' specifications
which talk about program functions,
such as the program equivalence reasoning we presented.
%
Compared to (Liquid) Haskell,
these systems allow for tactics and heuristics
that automate proof term generation
but lack SMT automations and
general-purpose language features,
like non-termination, exceptions and IO.
%
Zombie~\cite{Zombie,Sjoberg2015} and Fstar~\cite{fstar} allow dependent types to
coexist with divergent and effectful programs,
but still lack the optimized libraries,
like @ByteSting@, which come
with a general purpose language
with long history, like Haskell.



\paragraph{Parallel Code Verification}

One work  closely related to ours is
SyDPaCC~\cite{SyDPaCC}, a Coq library that
automatically parallelizes list homomorphisms
by extracting parallel Ocaml versions of user provided Coq functions.
%
Unlike SyDPaCC, we are not automatically generating the parallel
function version, because of engineering limitations
(\S~\ref{sec:evaluation}).  Once these are addressed, code extraction
can be naturally implemented by turning
Theorem~\ref{theorem:two-level} into a Haskell type class with a
default parallelization method.
%
SyDPaCC used maximum prefix sum as a case study,
whose morphism verification is
much simpler than our string matching case study.
%
Finally, our implementation consists of
2K lines of Liquid Haskell, which we consider verbose and aim to
use tactics to simplify.
On the contrary, the SyDPaCC implementation
requires three different languages:
2K lines of Coq with tactics, 600 lines of Ocaml and 120 lines of C,
and is considered ``very concise''.

\mypara{Dependent Types for Non-Terminating Programs}
%
Zombie~\citep{Zombie, Sjoberg2015} integrates
dependent types in non terminating programs
and supports automatic reasoning for equality.
%
Vazou \etal have previously~\citep{Vazou14} shown
how Liquid Types can be used to check
non-terminating programs.
%
Reflection makes \toolname at least as
expressive as Zombie, \emph{without}
having to axiomatize the theory of
equality within the type system.
%
Consequently, in contrast to Zombie,
SMT based reflection lets \toolname
verify higher-order specifications
like @foldr_fusion@.

%%%%%%%%%%%%%%%%%%%%%%%%%%%%%%%%%%%%%%%%%%%%%%%%%%%%%%%%%%%%%%%%%%%%%%%%%%%%%%%
%%%%%%%%%%%%%%% alternative verification in Haskell %%%%%%%%%%%%%%%%%%%%%%%%%%%
%%%%%%%%%%%%%%%%%%%%%%%%%%%%%%%%%%%%%%%%%%%%%%%%%%%%%%%%%%%%%%%%%%%%%%%%%%%%%%%

\section{Verification in Haskell}\label{related:haskell}


\mypara{Proving Equational Properties}
% of Haskell Programs}
%
Several authors have proposed tools for proving
(equational) properties of (functional) programs.
%
%
HERMIT~\citep{Farmer15} proves equalities by rewriting
the GHC core language, guided by user specified scripts.
%
In contrast, our proofs are simply Haskell programs,
we can use SMT solvers to automate reasoning, and,
most importantly, we can connect the validity of
proofs with the semantics of the programs.


\spara{Static Contract Checkers} 
like ESCJava~\cite{ESCJava} are a classical way of verifying 
correctness through assertions and pre- and post-conditions. 
%
\cite{XuPOPL09} describes a static contract checker for 
Haskell that uses symbolic execution to unroll procedures
upto some fixed depth, yielding weaker ``bounded'' soundness
guarantees.
% 

Similarly, Zeno~\cite{ZENO} is an automatic Haskell 
prover that combines unrolling with heuristics for rewriting
and proof-search. 
%%Based on rewriting, it is sound but 
%%``Zeno might loop forever'' when faced with 
%%non-termination.
%
Finally, the Halo~\cite{halo} contract checker encodes 
Haskell programs into first-order logic by directly 
modeling the code's denotational semantics,
again, requiring heuristics for instantiating axioms 
describing functions' behavior.
%
   Haskell itself can be used to \emph{fake} ``lightweight'' dependent 
   types~\citep{ChakravartyKJ05,JonesVWW06,Weirich12}.
   In this style, the invariants are expressed in 
   a restricted~\citep{Jia10} total 
   index language and relationships (\eg $x<y$ and $y<z$) 
   are combined (\eg $x<z$) by explicitly constructing
   a term denoting the consequent from terms 
   denoting the antecedents.
   %
   On the plus side this ``constructive'' approach
   ensures soundness. 
   It is impossible to witness inconsistencies, 
   as doing so triggers diverging computations.
   %
   However, it is not easy to use restricted indices
   with explicitly constructed relations to verify 
   complex properties~\citep{LindleyM13}.


\spara{Totality Checking}
is feasible by GHC itself, via an option flag that warns of any incomplete patterns.
%
Regrettably, GHC's warnings are local, \ie
GHC will raise a warning for @head@'s partial definition, 
but not for its caller, as the programmer would desire.
%%(2)~ and preservative:
%%a warning will be raised for any incomplete pattern
%%without an attempt to reason if it is reachable or not.
%
Catch~\cite{catch}, 
a fully automated tool that tracks incomplete patterns,
addresses the above issue
%
by computing functions' pre- and post-conditions.
Moreover, catch statically analyses the code 
to track reachable incomplete patterns.
%
\toolname allows more precise analysis than catch, 
thus, by assigning the appropriate
types to $\star$Error functions (\S~\ref{sec:totality}) 
it tracks reachable incomplete patters 
%we get catch analysis
as a side-effect of verification.
 
 
 \spara{Termination Analysis}
is crucial for \toolname's soundness 
and is implemented in a technique inspired by~\cite{XiTerminationLICS01}, 
%
Various other authors have proposed techniques to verify termination of
recursive functions, either using the ``size-change
principle''~\cite{JonesB04,Sereni05}, or by annotating types with size indices
and verifying that the arguments of recursive calls have smaller
indices~\cite{HughesParetoSabry96,BartheTermination}.
%
To our knowledge, none of the above analyses have been empirically
evaluated on large and complex real-world libraries.

AProVE~\cite{Giesl11} implements a powerful, fully-automatic
termination analysis for Haskell based on term-rewriting.
%
Compared to AProVE,
encoding the termination proof via 
refinements provides advantages that are crucial in 
large, real-world code bases. 
Specifically, refinements
let us
%
(1) prove termination over a subset 
    (not all) of inputs; many functions (\eg @fac@) 
    terminate only on @Nat@ inputs and not all @Int@s,
%
(2) encode pre-conditions, 
    post-conditions, and auxiliary invariants that 
    are essential for proving termination, (\eg @qsort@),
%
(3) easily specify non-standard 
    decreasing metrics and prove termination, (\eg @range@).
%
In each case, the code could be (significantly) 
\emph{rewritten} to be amenable to AProVE but this defeats
the purpose of an automatic checker.
 
%%%%%%%%%%%%%%%%%%%%%%%%%%%%%%%%%%%%%%%%%%%%%%%%%%%%%%%%%%%%%%%%%%%%%%%%%%%%%%%
%%%%%%%%%%%%%%% dependent types in Haskell %%%%%%%%%%%%%%%%%%%%%%%%%%%%%%%%%%%%
%%%%%%%%%%%%%%%%%%%%%%%%%%%%%%%%%%%%%%%%%%%%%%%%%%%%%%%%%%%%%%%%%%%%%%%%%%%%%%%
 
 
 \mypara{Dependent Types in Haskell}
%
Integration of dependent types into Haskell
has been a long standing goal that dates back
to Cayenne~\citep{cayenne}, a Haskell-like,
fully dependent type language with undecidable
type checking.
%
In a recent line of work~\citep{EisenbergS14}
Eisenberg \etal aim to allow fully dependent
programming within Haskell, by making
``type-level programming ... at least as
  expressive as term-level programming''.
%
Our approach differs in two significant ways.
%
First, reflection allows SMT-aided verification,
which drastically simplifies proofs over key theories
like linear arithmetic and equality.
%
Second, refinements are completely erased at run-time.
That is, while both systems automatically lift Haskell
code to either uninterpreted logical functions
or type families, with refinements, the logical
functions are not accessible at run-time and
promotion cannot affect the semantics of
the program.
%
As an advantage (resp. disadvantage), refinements
cannot degrade (resp. optimize)
the performance of programs.

 \mypara{Our Relational Algebra Library} builds on a long
line of work on type safe database access.
%
The HaskellDB~\citep{haskellDB}
showed how phantom types could be used to eliminate
certain classes of errors.
%
Haskell's HList library~\citep{heterogeneous}
extends this work with type-level computation
features to encode heterogeneous lists, which
can be used to encode database schema, and
(unlike HaskellDB) statically reject accesses
of ``missing'' fields.
%
The HList implementation is non-trivial,
requiring new type-classes for new operations
(\eg @append@ing lists); \citep{thepipower}
shows how a dependently typed language greatly
simplifies the implementation.
%
Much of this simplicity can be recovered in
Haskell using the @singleton@ library~\citep{Weirich12}.
%
Our goal is to show that bounded refinements
are expressive enough to permit the construction
of rich abstractions like a relational algebra
and generic combinators for safe database access
while using SMT solvers to provide decidable
checking and inference. Further, unlike the
HList based approaches, refinements they can
be used to \emph{retroactively} or \emph{gradually}
verify safety; if we erase the types we still
get a valid Haskell program operating over
homogeneous lists.
We do not extende Haskell's expressivity 
 
 Haskell itself can be considered a dependently-typed language,
 as type level computation is allowed via 
 Type Families~\cite{McBride02},
 Singleton Types\cite{Weirich12}, 
 Generalized Algebraic  Datatypes (GADTs)~\cite{JonesVWW06, SchrijversJSV09}, 
 and type-level functions~\cite{ChakravartyKJ05}.
 %
Again, 
verification in haskell itself turns out to be quite painful~\cite{LindleyM13}.















\spara{Tracking Divergent Computations}
The notion of type stratification to track potentially 
diverging computations dates to at least~\citep{ConstableS87} 
which uses $\bar{\typ}$ to encode diverging terms, and types 
$\efix{}$ as $(\bar{\typ}\rightarrow\bar{\typ}) \rightarrow \bar{\typ}$).
%
More recently, \cite{Capretta05} tracks diverging 
computations within a \emph{partiality monad}.
%
Unlike the above, we use refinements to 
obtain terminating fixpoints (\etfix{}), which let us prove 
the vast majority (of sub-expressions) in real world libraries 
as non-diverging, avoiding the restructuring that would
be required by the partiality monad.

\spara{Termination Analyses}
Various authors have proposed techniques to verify termination 
of recursive functions, either using the ``size-change principle'' 
\cite{JonesB04,Sereni05}, or by annotating types with size indices 
and verifying that the arguments of recursive calls have smaller 
indices~\cite{HughesParetoSabry96,BartheTermination}.
%
Our use of refinements to encode terminating fixpoints is most 
closely related to~\cite{XiTerminationLICS01}, but this work 
also crucially assumes CBV semantics for soundness.

AProVE~\cite{Giesl11} implements a powerful, fully-automatic
termination analysis for Haskell based on term-rewriting.
%
While we could use an external analysis like AProVE,
we have found that encoding the termination proof via 
refinements provided advantages that are crucial in 
large, real-world code bases. Specifically, refinements
let us
%
(1) prove termination over a subset 
    (not all) of inputs; many functions (\eg @fac@) 
    terminate only on @Nat@ inputs and not all @Int@s,
%
(2) encode pre-conditions, 
    post-conditions, and auxiliary invariants that 
    are essential for proving termination, (\eg @gcd@),
%
(3) easily specify non-standard 
    decreasing metrics and prove termination, (\eg @range@).
%
In each case, the code could be (significantly) 
\emph{rewritten} to be amenable to AProVE but this defeats
the purpose of an automatic checker.
%
Finally, none of the above analyses have been empirically
evaluated on large and complex real-world libraries.


\spara{Static Contract Checkers} 
like ESCJava~\cite{ESCJava} are a classical way of verifying 
correctness through assertions and pre- and post-conditions. 
%
Side-effects like modifications of global variables are a 
well known issue for static checkers for imperative languages;
the standard approach is to use an effect analysis to determine
the ``modifies clause'' \ie the set of globals modified by a procedure.
%
Similarly, one can view our approach as implicitly computing 
the non-termination effects.
%
%
\cite{XuPOPL09} describes a static contract checker for 
Haskell that uses symbolic execution to unroll procedures
upto some fixed depth, yielding weaker ``bounded'' soundness
guarantees.
% 

%
Similarly, Zeno~\cite{ZENO} is an automatic Haskell 
prover that combines unrolling with heuristics for rewriting
and proof-search. 
Based on rewriting, it is sound but 
``Zeno might loop forever'' when faced with 
non-termination.
%
Finally, the Halo~\cite{halo} contract checker encodes 
Haskell programs into first-order logic by directly 
modeling the code's denotational semantics,
again, requiring heuristics for instantiating axioms 
describing functions' behavior. Halo's translation of Haskell
programs directly encodes constructors as uninterpreted functions,
axiomatized to be injective (as the denotational semantics requires).
This heavyweight encoding is more precise than predicate abstraction 
but leads to model-theoretic problems (outlined in the Halo paper) and 
affects the efficiency of the encoding when scaling to larger programs 
(see also \ref{sec:refinedhaskell:conclusion}, paragraph B) in the lack of specialized 
decisions procedures.
%
Unlike any of the above, our type-based approach does 
not rely on heuristics for unrolling recursive procedures, 
or instantiating axioms. 
%
Instead we are based on decidable SMT validity 
checking and abstract interpretation~\cite{LiquidPLDI08} 
which makes the tool predictable and the overall workflow
scale to the verification of large, real-world
code bases.

\end{comment}


\begin{comment}
\paragraph{Parallel Code Verification}
Dependent type theorem provers have been used before to
verify parallel code.
%
BSP-Why~\cite{bspwhy} is an extension to Why2 that is using both Coq and SMTs
to discharge user specified verification conditions.
%
Daum~\cite{daum07} used Isabelle to formalize the semantics
of a type-safe subset of C, 
by extending Schirmer's~\cite{schirmer06}
formalization of sequential imperative languages.
%
Finally, Swierstra~\cite{wouter10} formalized mutable arrays in Agda
and used them to reason about distributed maps and sums.

\mypara{Deterministic Parallelism}
%
Deterministic parallelism has plenty of theory but relatively few practical
implementations.  Early discoveries were based on limited producer-consumer
communication, such as single-assignment variables \cite{Tesler-1968,IStructures}, Kahn
process networks~\cite{kahn-1974}, and synchronous dataflow~\cite{lee-sdf}.
Other models use synchronous updates to shared state, as in
Esterel~\cite{synchronous-overview} or PRAM.  Finally, work on type systems for
permissions management \cite{permission-types,habanero-java-permissions},
supports the development of {\em non-interfering} parallel programs that access
disjoint subsets of the heap in parallel.  Parallel functional programming is
also non-interfering~\cite{manticore,multicore-ghc}.
%
Irrespective of which theory is used to support deterministic parallel
programming, practical implementations such as Cilk~\cite{cilk} or Intel
CnC~\cite{cnc} are limited by host languages with type systems insufficient to
limit side effects, much less prove associativity.  Conversely, dependently
typed languages like Agda and Idris do not have parallel programming APIs and
runtime systems.

% synchronous languages such as Esterel
\mypara{Our Approach for Verifying Stateful Computations} using monads
indexed by pre- and post-conditions is inspired by the method of
Filli\^atre~\citep{Filliatre98}, which was later enriched with
separation logic in Ynot~\citep{ynot}. In future work it would
be interesting to use separation logic based refinements to specify
and verify the complex sharing and aliasing patterns allowed by Ynot.
%
\fstar encodes stateful computations in a special Dijkstra
Monad~\citep{dijkstramonad} that replaces the two assertions with
a single (weakest-precondition) predicate transformer which
can be composed across sub-computations to yield a transformer
for the entire computation.
%
Our \RIO approach uses the idea of indexed monads but
has two concrete advantages.
%
First, we show how bounded refinements alone suffice to
let us fashion the \RIO abstraction from scratch.
%
Consequently, second, we automate inference of pre- and
post-conditions and loop invariants as refinement instantiation
via Liquid Typing~\citep{LiquidPLDI08}.


Systems~\citep{sousa16} and \citep{KobayashiRelational15}
extend classical safety verification algorithms,
respectively based on Floyd-Hoare logic and Refinement Types,
to the setting of relational or $k$-safety properties
that are assertions over $k$-traces of a program.
%
Thus, these methods can automatically prove that
certain functions are associative, commutative \etc.
but are restricted to first-order properties and
are not programmer-extensible.
%
Zeno~\citep{ZENO} generates proofs by term
rewriting and Halo~\citep{halo} uses an axiomatic
encoding to verify contracts.
%
Both the above are automatic, but unpredictable and not
programmer-extensible, hence, have been limited to
far simpler properties than the ones checked here.

\end{comment}

\section*{Acknowledgments}
We thank the anonymous reviewers and Colin Gordon for providing invaluable feedback
about earlier drafts of this paper.

{
\bibliographystyle{plain}
\bibliography{sw}
}

\end{document}
