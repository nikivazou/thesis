\section{Related Work}\label{sec:related}

%% Our work is related to the vast literature on program logics,
%% dependent types and decision procedures. Next, we discuss the
%% most closely related lines of work.

\paragraph{Higher order Logics and Dependent Type Systems}
%
including
NuPRL~\citep{Constable86},
Coq~\citep{coq-book}, Agda~\citep{norell07},
and even to some extent, \haskell~\citep{JonesVWW06, McBride02},
occupy the maximal extreme of the expressiveness spectrum.
However, in these settings, checking requires explicit
proof terms which can add considerable programmer overhead.
%
Our goal is to eliminate the programmer overhead of
proof construction by restricting specifications to
decidable, first order logics and to see how far
we can go without giving up on expressiveness.
%
The \fstar system enables full dependent typing via
SMT solvers via a higher-order universally quantified
logic that permit specifications similar to ours
(\eg @compose@, @filter@ and @foldr@).
%% https://github.com/FStarLang/FStar/
%
While this approach is at least as expressive
as bounded refinements it has two drawbacks.
%
First, due to the quantifiers, the generated VCs
fall outside the SMT decidable theories.
This renders the type system undecidable (in theory),
forcing a dependency on the solver's unpredictable
quantifier instantiation heuristics (in practice).
%
Second, more importantly, % perhaps more importantly,
the higher order
predicates must be \emph{explicitly} instantiated,
placing a heavy annotation burden on the programmer.
%
In contrast, bounds permit decidable
checking, and are automatically instantiated
via Liquid Types.


\paragraph{Our notion of Refinement Types}
%
has its roots in the predicate subtyping
of PVS~\cite{Rushby98} and \emph{indexed types}
(DML~\cite{pfenningxi98}) where types are constrained
by predicates drawn from a logic.
%
To ensure decidable checking several refinement
type systems including~\citep{pfenningxi98,Dunfield07,LiquidICFP14}
restrict refinements to decidable, quantifier free logics.
%
While this ensures predictable checking and inference~\cite{LiquidPLDI08}
it severely limits the language of specifications, and makes it hard to
fashion simple higher order abstractions like @filter@ (let alone the more
complex ones like relational algebras and state transformers.)

%%% All these systems start from an expressive higher order logic but
%%% requires explicit, user provided proofs.
%%% %
%%% The burden of manually providing proofs is mitigated by providing
%%% proof tactics that attempt to automatically discharge proof
%%% obligations.
%%%
%%% We follow the opposite approach, instead pushing the expressiveness
%%% limits of an automatic, decidable system.
%%% %
%%% We argue that our case studies demonstrate that bounded refinement
%%% types achieve a high degree of expressiveness without sacrificing
%%% automation or decidability.

\paragraph{To Reconcile Expressiveness and Decidability}
%
\catalyst~\citep{catalyst} permits a form of
higher order specifications where refinements
are relations which may themselves be parameterized
by other relations, which allows for example, a
way to precisely type @filter@ by suitably
composing relations.
%
However, to ensure decidable checking, \catalyst
is limited to relations that can be specified as
catamorphisms over inductive types, precluding
for example, theories like arithmetic.
More importantly, (like \fstar), \catalyst provides
no inference: higher order relations must be
\emph{explicitly} instantiated.
%
Bounded refinements build directly upon
abstract refinements~\citep{vazou13},
a form of refinement polymorphism
analogous to parametric polymorphism.
%
While \cite{vazou13} adds expressiveness via
abstract refinements, without bounds we cannot
specify any \emph{relationships between} the
abstract refinements. The addition of bounds
makes it possible to specify and verify the examples
shown in this paper,
while preserving decidability and inference.
%%While this technique adds expressiveness, it prevents
%%the ability to specify relationships between the abstract
%%(or higher order) refinements, which precludes the
%%verification of \emph{any} of our examples.
%%%
%%Our contribution is to show that by adding a form of
%%bounded quantification for abstract refinements we
%%can drastically expand expressiveness

\paragraph{Our Relational Algebra Library} builds on a long
line of work on type safe database access.
%
The HaskellDB~\citep{haskellDB}
showed how phantom types could be used to eliminate
certain classes of errors.
%
Haskell's HList library~\citep{heterogeneous}
extends this work with type-level computation
features to encode heterogeneous lists, which
can be used to encode database schema, and
(unlike HaskellDB) statically reject accesses
of ``missing'' fields.
%
The HList implementation is non-trivial,
requiring new type-classes for new operations
(\eg @append@ing lists); \citep{thepipower}
shows how a dependently typed language greatly
simplifies the implementation.
%
Much of this simplicity can be recovered in
Haskell using the @singleton@ library~\citep{Weirich12}.
%
Our goal is to show that bounded refinements
are expressive enough to permit the construction
of rich abstractions like a relational algebra
and generic combinators for safe database access
while using SMT solvers to provide decidable
checking and inference. Further, unlike the
HList based approaches, refinements they can
be used to \emph{retroactively} or \emph{gradually}
verify safety; if we erase the types we still
get a valid Haskell program operating over
homogeneous lists.


%%% Unlike all the above systems in which safety
%%% of relational operations crucially depends on
%%% heterogeneous collections, we prove database
%%% operations safe using homogeneous collections,
%%% \ie lists of dictionaries, but by indexing the
%%% homogeneous types with Abstract Refinements, thus
%%% composing heterogeneous \textit{refined} types.



\paragraph{Our Approach for Verifying Stateful Computations} using monads
indexed by pre- and post-conditions is inspired by the method of
Filli\^atre~\citep{Filliatre98}, which was later enriched with
separation logic in Ynot~\citep{ynot}. In future work it would
be interesting to use separation logic based refinements to specify
and verify the complex sharing and aliasing patterns allowed by Ynot.
%
\fstar encodes stateful computations in a special Dijkstra
Monad~\citep{dijkstramonad} that replaces the two assertions with
a single (weakest-precondition) predicate transformer which
can be composed across sub-computations to yield a transformer
for the entire computation.
%
Our \RIO approach uses the idea of indexed monads but
has two concrete advantages.
%
First, we show how bounded refinements alone suffice to
let us fashion the \RIO abstraction from scratch.
%
Consequently, second, we automate inference of pre- and
post-conditions and loop invariants as refinement instantiation
via Liquid Typing.
%~\citep{LiquidPLDI08}.


%% Later, Eisenberg and Weirich used Singleton Types~\citep{Weirich12}
%% to translate Oury Swiersta's database example in Haskell,
%% demonstrating the power of singleton types.
%
%
% provides elegant
% types for relational operators, while using implicit
% parameter to lower the proof burden.


% Oury and Swiersta, used The Power of Pi~\citep{thepipower} to
% build a more comprehensible relational algebra in Agda.
% %
% Once again, they model database schemata as heterogeneous collections,
% but this time they rely on Agda inference to provide elegant types for
% relational operators and reduce the user's proof burden.
% %



%%%%% Adopting concepts from HaskellDB, Kiselyov \etal
%%%%% used dependently typed programming in Haskell to
%%%%% model databases as heterogeneous collections~\citep{heterogeneous},
%%%%% achieving even stronger safety guarantees.
%%%%% %
%%%%% As an example a query that projects from a table a non existing field,
%%%%% will be accepted as safe from haskellDB but easily marked as unsafe
%%%%% by heterogeneous collections.
%%%%% %
%%%%% Even though being extensively expressive,
%%%%% heterogeneous collections heavily rely on
%%%%% Haskell's dependent type system -- \eg each
%%%%% collection operations, like append, requires
%%%%% the definition of a type class -- which renders
%%%%% then incomprehensible.
%


%%% develop an \RIO monad using a mor
%%% Our $\RIO$ monad has the same expressive power as the Dijkstra Monad,
%%% but unlike them, our inference mechanism relies on the abstract
%%% interpretation framework of liquid types, allowing \toolname to
%%% compute loop invariants as well as function pre- and postconditions.
%%%
%%% for the first time combines verification of  higher order programs with inference.
%%% %
%%% The key idea is that instead using both a pre- and post-condition,
%%% the Dijkstra Monad $\texttt{DST}\ \ta\ wp$
%%% is indexed by the weakest precondition predicate transformer $wp$
%%% and does type inference by composing weakest preconditions.



%%%\paragraph{Verification of Stateful Programs} with type theory
%%%was initiated by
%%%when he verified imperative programs in Coq showing for first time
%%%how proofs in Hoare logic can be interpreted in type theory.
%%%%
%%%Ynot~\citep{ynot} is a Coq extension that applies
%%%Hoare Type Theory~\citep{nanevski06} to
%%%higher order programs.
%%%%
%%%Much like our Refined Input/Output monad \ynot defines the Hoare Monad
%%%$\HIO\ p\ \ta\ q$ that represents a computation
%%%with precondition $p$ and postcondition $q$
%%%that if terminates it will return a value of type $\ta$.
%%%%
%%%% However, being in the fully dependently typed language of Coq
%%%% the $p$ and $q$ can be arbitrary predicates
%%%% -- such as Separation Logic assertions --
%%%% and thus no inference is performed.
%%%% %
%%%% Borgstr{\"o}m~\citep{Gordon09} \etal present an application on
%%%% the Hoare Monad in F7~\citep{GordonRefinement09} but still without
%%%% any inference available.

%%% ZAP %%Our work on stateful verification is similar to the Dijkstra Monad in that
%%% ZAP %%bounds on abstract refinements can be seen an predicate transformers
%%% ZAP %%since the specify the post-condition given the precondition
%%% ZAP %%RIO <p, q> a {q => p'}
%%% ZAP %%%
%%% ZAP   we have two states post-condition
%%% ZAP   so no need for separation logic or precondition transformers,
%%% ZAP   thought bounds clearly transforms pre to post.
%%% ZAP
%%% ZAP - Polymorphism and Seperation in Hoare Type Theory 06
%%% ZAP Hoare’s original logic [12]
%%% ZAP or the more recent forms of Separation Logic [30, 36, 31], which
%%% ZAP - \citep{Borgstrom11} use substructural state on top of a Hoare-like monad to model local state

%%%%%%%%%%%%%%% DUMP
%%% Local Variables:
%%% mode: latex
%%% TeX-master: "main"
%%% End:
