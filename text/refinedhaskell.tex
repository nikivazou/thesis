\chapter{Refinement Types in Practice}\label{chapter:tool}
\makequote
{Everything should be made as simple as possible, but no simpler.}
{Albert Einstein}


\begin{comment}
“One day I will find the right words, and they will be simple.” 
― Jack Kerouac, The Dharma Bums


“Life is really simple, but we insist on making it complicated.” ~ Confucius

“Knowledge is a process of piling up facts; wisdom lies in their simplification.” ~ Martin H. Fischer


“If you can’t explain it to a six year old, you don’t understand it yourself.” ~ Albert Einstein
\end{comment}

% \section{Introduction}\label{sec:introduction}
Refinement types enable specification of complex invariants 
by extending the base type system with \emph{refinement predicates} 
drawn from decidable logics. For example,
%
\begin{code}
  type Nat = {v:Int | 0 <= v}
  type Pos = {v:Int | 0 < v}
\end{code}
%
are refinements of the basic type @Int@ with a logical predicate 
that states the \emph{values} @v@ being described must be 
\emph{non-negative} and \emph{postive} respectively. 
%
We can specify \emph{contracts} of functions by refining function types. 
For example, the contract for @div@
%
\begin{code}
  div :: n:Nat -> d:Pos -> {v:Nat | v <= n}
\end{code}
%
states that @div@ \emph{requires} a non-negative dividend @n@ and a positive
divisor @d@ and \emph{ensures} that the result is less than the dividend.
%
If a program (refinement) type checks, we can be sure that @div@ will never 
throw a divide-by-zero exception.

Refinement types \citep{ConstableS87,Rushby98} 
have been implemented for several languages like
ML~\cite{pfenningxi98,GordonTOPLAS2011,LiquidPLDI08},
C~\cite{deputy,LiquidPOPL10},
TypeScript~\cite{Vekris16},
Racket~\cite{RefinedRacket} and Scala~\cite{refinedscala}.
%
Here we present \toolname,
a refinement type checker for Haskell.
%
In this chapter we start with an example driven informal and practical overview
of \toolname.
%
In particular, we try to answer the following questions:
%
\begin{enumerate}
  \item What properties can be specified with refinement types?
  \item What inputs are provided and what feedback is received?
  \item What is the process for modularly verifying a library?
  \item What are the limitations of refinement types? 
\end{enumerate}

We attempt to investigate these questions, by using the
refinement type checker \toolname, to specify and verify a variety of 
properties of over 10,000 lines of Haskell code from popular 
libraries, including @containers@, \hbox{@hscolor@,} @bytestring@, @text@, 
@vector-algorithms@ and @xmonad@. 
%
\begin{itemize}
\item First (\S~\ref{sec:liquidhaskell}), 
we present a high-level overview of \toolname, through a tour 
of its features.
%

\item Second, we present a qualitative discussion of the kinds of properties
that can be checked -- ranging from generic application independent 
criteria like totality (\S~\ref{sec:totality}), 
\ie that a function is defined for all inputs (of a given type)
and termination, 
(\S~\ref{sec:termination}) 
\ie that a recursive function cannot diverge,
to application specific concerns like memory safety (\S~\ref{sec:memory-safety}) 
and functional correctness properties (\S~\ref{sec:structures}).
%
\item Finally (\S~\ref{sec:realworld:evaluation}), we present a quantitative evaluation of the approach, with a view
towards measuring the efficiency and programmer's effort required for
verification, 
and we discuss various limitations of the approach which could
provide avenues for further work.
\end{itemize}

\section{\toolname}\label{sec:liquidhaskell}
\begin{figure}[t!]
\centering
\captionsetup{justification=centering}
\noindent\makebox[\textwidth]{\includegraphics[width=\textwidth]{text/realworldhaskell/liquidHaskell}}
\caption{\toolname Workflow.}
	\label{fig:internals}
\end{figure}
% 
% link for workflow
% https://www.draw.io/#G0Bwp_mIorSVqJb2RENnNVWVlQTmc
%
We start with a short description of the \toolname workflow,
summarized in Figure~\ref{fig:internals} and continue with an 
example driven overview of how properties are specified
and verified using the tool. 

% \mypara{Usage} 
\mypara{Source}
\toolname can be run from the command-line\footnote{\url{https://hackage.haskell.org/package/liquidhaskell}}
or within a web-browser\footnote{\url{http://goto.ucsd.edu/liquid/haskell/demo/}}.
It takes as \emph{input}:
%
(1)~a single Haskell \emph{source} file with code and refinement
    type specifications including refined datatype definitions, 
    measures (\S~\ref{sec:tool:measures}), predicate and type 
    aliases, and function signatures;
%
(2)~a set of directories containing \emph{imported modules} 
    (including the \verb+Prelude+) which may themselves 
    contain specifications for exported types and functions; and
%
(3)~a set of predicate fragments called \emph{qualifiers},
    which are used to infer refinement types. This set is 
    typically empty as the default set of qualifiers extracted 
    from the type specifications suffices for inference.

\mypara{Core}
\toolname uses GHC to reduce the source to the Core IL~\cite{SulzmannCJD07}
and, to facilitate source-level error reporting, creates a map from Core 
expressions to locations in the Haskell source.

\mypara{Constraints}
Then, it uses the abstract interpretation framework of Liquid Typing~\cite{LiquidPLDI08}, 
modified to ensure soundness under lazy evaluation~\ref{chapter:refinedhaskell}
and extended with Abstract~\ref{chapter:abstractrefinements}
and Bounded~\ref{boundedrefinements} Refinement Types
and Refinement Reflection~\ref{refinementrflection},
to generate logical constraints from the Core IL.
     
\mypara{Solution}
Next, it uses a fixpoint algorithm (from~\citep{LiquidPLDI08})
combined with an SMT solver to solve the constraints, and hence 
infers a valid refinement typing for the program. 
%
\toolname can use any solver that implements the SMT-LIB2
standard~\cite{SMTLIB2}, including Z3~\citep{z3}, CVC4~\citep{CVC4}, and
MathSat~\citep{MathSat}.

 
\mypara{Types \& Errors}
% \NV{satisfiability and validity refer to different things here, 
% which is confusing...}
If the set of constraints is satisfiable, then \toolname outputs 
\textsc{Safe}, meaning the program is verified.
If instead, the set of constraints is not satisfiable, then \toolname
outputs \textsc{Unsafe}, and uses the invalid constraints to 
report refinement type errors at the \emph{source positions}
that created the invalid constraints, using the location 
information to map the invalid constraints to source positions.
%
In either case, \toolname produces as output a source map
containing the \emph{inferred} types for each program 
expression, which, in our experience, is crucial for 
debugging the code and the specifications.

%\mypara{Optional Typing}
%
\toolname is best thought of as an \emph{optional} type checker
for Haskell. By optional we mean that the refinements have \emph{no} 
influence on the dynamic semantics, which makes it easy to apply 
\toolname to \emph{existing} libraries.
%
To emphasize the optional nature of refinements and preserve 
compatibility with existing compilers, all specifications 
appear within comments of the form \verb|{-@ ... @-}|, 
which we omit below for brevity.

\subsection{Specifications}

A refinement type is a Haskell type where each component
of the type is decorated with a predicate from a (decidable)
refinement logic. We use the quantifier-free logic of equality, 
uninterpreted functions and linear arithmetic (QF-EUFLIA)~\cite{Nelson81}. 
For example,
%
\begin{code}
   {v:Int | 0 <= v && v < 100}
\end{code}
%
describes @Int@ values between @0@ and @100@.

\mypara{Type Aliases} For brevity and readability, it is often convenient 
to define abbreviations for particular refinement predicates and types.
For example, we can define an alias for the above predicate
%
\begin{code}
  predicate Btwn Lo N Hi = Lo <= N && N < Hi
\end{code}
%
and use it to define a \emph{type alias}
%
\begin{code}
  type Rng Lo Hi = {v:Int | Btwn Lo v Hi} 
\end{code}
%
We can now describe the above integers as @(Rng 0 100)@.

\mypara{Contracts} 
To describe the desired properties of a function, we need
simply refine the input and output types with predicates 
that respectively capture suitable pre- and post-conditions. 
For example,
%
\begin{code}
  range :: lo:Int -> hi:{Int | lo <= hi} -> [(Rng lo hi)]
\end{code}
%
states that @range@ is a function that takes two @Int@s 
respectively named @lo@ and @hi@ and returns a list of @Int@s 
between @lo@ and @hi@. There are three things worth
noting.
%
First, we have binders to name the function's \emph{inputs} 
(\eg @lo@ and @hi@) and can use the binders inside the 
function's \emph{output}.
%
Second, the refinement in the \emph{input} type describes the 
\emph{pre-condition} that the second parameter @hi@ cannot 
be smaller than the first @lo@.
%
Third, the refinement in the \emph{output} type describes the
\emph{post-condition} that all returned elements are between 
the bounds of @lo@ and @hi@.


\subsection{Verification}\label{sec:tool:verification}

Next, consider the following implementation for @range@:
%
\begin{code}
  range lo hi 
    | lo <= hi  = lo : range (lo + 1) hi
    | otherwise = []
\end{code}
%
When we run \toolname on the above code, it reports an 
error at the definition of @range@. This is unpleasant! 
One way to debug the error is to determine what type has
been \emph{inferred} for @range@, \eg by hovering the 
mouse over the identifier in the web interface. 
In this case, we see that the output type is essentially:
%
\begin{code}
  [{v:Int | lo <= v && v <= hi}]
\end{code}
%
which indicates the problem. There is an \emph{off-by-one} 
error due to the problematic guard. If we replace the second @<=@ 
with a @<@ and re-run the checker, the function is verified.

\mypara{Holes} It is often cumbersome to specify the Haskell
types, as those can be gleaned from the regular type signatures 
or via GHC's inference. Thus, \toolname allows the user to leave 
holes in the specifications. Suppose @rangeFind@ has type 
%
\begin{code}
  (Int -> Bool) -> Int -> Int -> Maybe Int
\end{code}
%
where the second and third parameters define a range. 
We can give @rangeFind@ a refined specification:
%
\begin{code}
  _ -> lo:_ -> hi:{Int | lo <= hi} -> Maybe (Rng lo hi)
\end{code}
%
where the @_@ is the unrefined Haskell type for the 
corresponding position in the type.

\mypara{Inference} Next, consider the implementation
%
\begin{code}
  rangeFind f lo hi = find f $ range lo hi 
\end{code}
%$
where @find@ from @Data.List@ has the (unrefined) type
%
\begin{code}
  find :: (a -> Bool) -> [a] -> Maybe a
\end{code}
%
\toolname uses the abstract interpretation framework of 
Liquid Typing~\cite{LiquidPLDI08} to infer that the type
parameter @a@ of @find@ can be instantiated with @(Rng lo hi)@
thereby enabling the automatic verification of @rangeFind@.

Inference is crucial for automatically synthesizing types
for polymorphic instantiation sites -- note there is another
instantiation required at the use of the apply operator 
@dollar@ --  and to relieve the programmer of the tedium of %$
specifying signatures for all functions. 
%
Of course, for functions exported by the module,
we must write signatures to specify preconditions -- otherwise, 
the system defaults to using the trivial (unrefined) Haskell 
type as the signature \ie, checks the implementation assuming 
arbitrary inputs.

\subsection{Measures}\label{sec:tool:measures}
So far, the specifications have been limited to comparisons and 
arithmetic operations on primitive values. 
We use \emph{measure functions}, or just measures, to 
specify \emph{inductive properties} of algebraic data types. 
%
For example, we define a measure @len@ to write properties about the number
of elements in a list.
%
\begin{code}
  measure len :: [a] -> Int
  len []      = 0
  len (x:xs)  = 1 + (len xs)
\end{code}
%
Measure definitions are \emph{not} arbitrary Haskell code but a very 
restricted subset~\ref{sec:measures}.
Each measure has a single equation per constructor that defines the
value of the measure for that constructor. The right-hand side of the 
equation is a term in the restricted refinement logic. Measures are 
interpreted by generating refinement types for the corresponding 
data constructors.
%
For example, from the above, \toolname derives the 
following types for the list data constructors:
%
\begin{code}
  []  :: {v:[a]| len v = 0}
  (:) :: _ -> xs:_ -> {v:[a]| len v = 1 + len xs}
\end{code}
%
Here, @len@ is an \emph{uninterpreted function} in the refinement logic.
We can define multiple measures for a type; \toolname simply conjoins
the individual refinements arising from each measure to obtain a single
refined signature for each data constructor.

\mypara{Using Measures}
We use measures to write specifications about algebraic types. 
For example, we can specify and verify that: 
%
\begin{code}
  append :: xs:[a] -> ys:[a] 
         -> {v:[a]| len v = len xs + len ys}

  map    :: (a -> b) -> xs:[a] 
         -> {v:[b]| len v = len xs} 

  filter :: (a -> Bool) -> xs:[a] 
         -> {v:[a]| len v <= len xs}
\end{code}

\mypara{Propositions} 
%%In addition to allowing the specification of structural features like
%%lengths, heights and so on, 
Measures can be used to encode sophisticated 
invariants about algebraic data types.
%
To this end, the user can write a measure whose output has a special type 
@Prop@ denoting propositions in the refinement logic. For instance, we can
describe a list that contains a @0@ as:
%
\begin{code}
  measure hasZero :: [Int] -> Prop
  hasZero []      = false
  hasZero (x:xs)  = x == 0 || hasZero xs
\end{code}
%
We can then define lists containing a @0@ as:
%
\begin{code}
  type HasZero = {v : [Int] | hasZero v } 
\end{code}
%
Using the above, \toolname will accept 
%
\begin{code}
  xs0 :: HasZero 
  xs0 = [2,1,0,-1,-2]
\end{code}
%
but will reject
%
\begin{code}
  xs' :: HasZero 
  xs' = [3,2,1]
\end{code}



\subsection{Refined Data Types}

Often, we require that \emph{every} instance of a type satisfies some invariants. 
For example, consider a @CSV@ data type, that represents tables:
%
\begin{code}
  data CSV a = CSV { cols :: [String]
                   , rows :: [[a]]    }
\end{code}
%
% With \toolname we can enforce the invariant that for every @CSV@ table, 
% with a number of columns given by @dim@,
% each row has @dim@ elements,
% with the below refined data type definition
%%With \toolname we can enforce the invariant that every @CSV@ table 
%%has the number of columns given by @dim@, and that each row has 
%%@dim@ elements with a refined data type definition, such as:
With \toolname we can enforce the invariant that every row in a @CSV@ table
should have the same number of columns as there are in the header
%
\begin{code}
  data CSV a = CSV { cols :: [String]  
                   , rows :: [ListL a cols] }
\end{code}
%
using the alias
%
\begin{code}
  type ListL a X = {v:[a]| len v = len X}
\end{code}
%
A refined data definition is \emph{global} in that \toolname 
will reject any @CSV@-typed expression that does not respect 
the refined definition. For example, both of the below 
%
\begin{code}
  goodCSV = CSV [  "Month", "Days"] 
                [ ["Jan"  , "31"]
                , ["Feb   , "28"]
                , ["Mar"  , "31"] ]

  badCSV  = CSV [  "Month", "Days"] 
                [ ["Jan"  , "31"]
                , ["Feb   , "28"]
                , ["Mar"        ] ]
\end{code}
%
are well-typed Haskell, but the latter is rejected by \toolname.
%
Like measures, the global invariants are enforced by refining 
the constructors' types. 

\subsection{Refined Type Classes}\label{sec:type-classes}

Next, let us see how \toolname allows verification of
programs that use ad-hoc polymorphism via type classes.
%
While the implementation of each typeclass instance is 
different, there is often a common interface that 
all instances should satisfy.

\mypara{Class Measures}
As an example, consider the class definition
%
\begin{code}
  class Indexable f where
    size :: f a -> Int
    at   :: f a -> Int -> a
\end{code}
%
For safe access, we might require that @at@'s second 
parameter is bounded by the @size@ of the container.
To this end, we define a \emph{type-indexed} 
measure, using the @class measure@ keyword
%
\begin{code}
  class measure sz :: a -> Nat
\end{code}
%
Now, we can specify the safe-access precondition  
independent of the particular instances of @Indexable@:
%
\begin{code}
  class Indexable f where
    size :: xs:_ -> {v:Nat | v = sz xs}
    at   :: xs:_ -> {v:Nat | v < sz xs} -> a
\end{code}

\mypara{Instance Measures}
For each concrete type that instantiates a class, we require 
a corresponding definition for the measure. 
For example, to define lists as an instance of @Indexable@, 
we require the definition of the @sz@ instance for lists:
%
\begin{code}
  instance measure sz :: [a] -> Nat
    sz []     = 0
    sz (x:xs) = 1 + (sz xs)
\end{code}
%
Class measures work just like regular measures in that the above 
definition is used to refine the types of the list data constructors.
After defining the measure, we can define the type instance as:
%
\begin{code}
  instance Indexable [] where
    size []        = 0
    size (x:xs)    = 1 + size xs

    (x:xs) `at` 0  = x
    (x:xs) `at` i  = index xs (i-1)
\end{code}
%
\toolname uses the definition of @sz@ for lists to check that @size@ 
and @at@ satisfy the refined class specifications. 
% NV the dictionary relevant this were removed
% , and hence, that 
% the above creates a valid instance dictionary for @Indexable@.

\mypara{Client Verification}
At the clients of a type-class we use the refined 
types of class methods. Consider a client of @Indexable@s:
%
\begin{code}
  sum :: (Indexable f) => f Int -> Int
  sum xs = go 0 
    where
      go i | i < size xs = xs `at` i + go (i+1)
           | otherwise   = 0
\end{code}
%
\toolname proves that each call to @at@ is safe, by using the refined
class specifications of @Indexable@. 
Specifically, each call to @at@ is guarded by a check @i < size xs@
and @i@ is  increasing 
from 0, so \toolname proves that @xs `at` i@ will always be safe.

\begin{comment}
\subsection{Abstracting Refinements}

So far, all the specifications have used \emph{concrete} refinements. Often it is
useful to be able to \emph{abstract} the refinements that appear in a
specification. For example, consider a monomorphic variant of @max@
%
\begin{code}
  max     :: Int -> Int -> Int 
  max x y = if x > y then x else y
\end{code}
%
We would like to give @max@ a specification that lets us verify:
%
\begin{code}
  xPos  :: {v: _ | v > 0}
  xPos  = max 10 13

  xNeg  :: {v: _ | v < 0}
  xNeg  = max (-5) (-8)

  xEven :: {v: _ | v mod 2 == 0} 
  xEven = max 4 (-6)
\end{code}
%
To this end, \toolname allows the user to \emph{abstract refinements} over
types~\cite{vazou13}, for example by typing @max@ as:
%
\begin{code}
 max :: forall <p :: Int -> Prop>. 
          Int<p> -> Int<p> -> Int<p>
\end{code}
%
The above signature states that for any refinement @p@, if the two
inputs of @max@ satisfy @p@ then so does the output. \toolname uses
Liquid Typing to automatically instantiate @p@ with suitable concrete
refinements, thereby checking @xPos@, @xNeg@, and @xEven@.


\mypara{Dependent Composition}
Abstract refinements turn out to be a surprisingly expressive and 
useful specification mechanism. For example, consider the function 
composition operator:
%
\begin{code}
  (.) :: (b -> c) -> (a -> b) -> a -> c
  (.) f g x = f (g x)  
\end{code}
%
Previously, it was not possible to check, \eg that:
%
\begin{code}
  plus3 :: x:_ -> {v:_ | v = x + 3}
  plus3 = (+ 1) . (+ 2)
\end{code}
%
as the above required tracking the dependency between @a@, @b@ and @c@,
which is crucial for analyzing idiomatic Haskell.
With abstract refinements, we can give the @(.)@ operator the type:
%
\begin{code}
  (.) :: forall < p :: b -> c -> Prop
                , q :: a -> b -> Prop>.
           f:(x:b -> c<p x>) 
        -> g:(x:a -> b<q x>) 
        -> y:a 
        -> exists[z:b<q y>].c<p z>
\end{code}
%
which gets automatically instantiated at usage sites, allowing \toolname
to precisely track invariants through the use of the ubiquitous 
higher-order operator.

\mypara{Dependent Pairs}
Similarly, we can abstract refinements over the definition of datatypes.
% Similarly, we can abstract refinements over the definition of datatypes.
For example, we can express dependent pairs in \toolname by refining the 
definition of tuples as:
%
\begin{code}
  data Pair a b <p :: a -> b -> Prop> 
    = Pair { fst :: a, snd :: b<p fst>}
\end{code}
%
That is, the refinement @p@ relates the @snd@ element with the @fst@.
Now we can define increasing and decreasing pairs
%
\begin{code}
  type IncP = Pair <{\x y -> x < y}> Int Int
  type DecP = Pair <{\x y -> x > y}> Int Int
\end{code}
%
and then verify that:
%
\begin{code}
  up :: IncP
  up = Pair 2 5
  
  dn :: DecP
  dn = Pair 5 2
\end{code}
%
Now that we have a bird's eye view of the various specification mechanisms
supported by \toolname, let us see how we can profitably apply them to
statically check a variety of correctness properties in real-world codes.
\end{comment}
%%% Local Variables: 
%%% mode: latex
%%% TeX-master: "main"
%%% End: 

\section{Totality}\label{sec:totality}
%% OK \NV{DONE intro (R1)
%% OK Sections 3 and 4 cover "totality" and "termination" respectively. It would be helpful to explain these terms at the start of section 3, so that the two are distinguished appropriately.
%% OK }
Well typed Haskell code can go very wrong:
%
\begin{code}
  *** Exception: Prelude.head: empty list
\end{code}
%
As our first application, let us see how to use 
\toolname to statically guarantee the absence
of such exceptions, \ie, to prove various 
functions \emph{total}.

\subsection{Specifying Totality}

First, let us see how to specify the notion of
totality inside \toolname. Consider the source of 
the above exception:
%
\begin{code}
  head :: [a] -> a
  head (x:_) = x
\end{code}
%
Most of the work towards totality checking is done by 
the translation to GHC's Core, in which every function 
\emph{is} total, but may explicitly call an \emph{error} 
function that takes as input a string that describes the 
source of the pattern-match failure and throws an exception.
%
For example @head@ is translated into
%
\begin{code}
  head d = case d of 
             x:xs -> x
             []   -> patError "head"
\end{code}

Since every core function is total, but may explicitly 
call error functions, to prove that the source function is 
total, it suffices to prove that @patError@ 
will \emph{never} be called.
%
We can specify this requirement by giving the error 
functions a @false@ pre-condition:
%
\begin{code}
  patError :: {v:String | False } -> a
\end{code}
%
The pre-condition states that the input type is \emph{uninhabited}
and so an expression containing a call to @patError@ will only type 
check if the call is \emph{dead code}.


\subsection{Verifying Totality}

The (core) definition of @head@ does not typecheck
as is; but requires a pre-condition that states that the function
is only called with non-empty lists. Formally, we do so by 
defining the alias
%
\begin{code}
  predicate NonEmp X = 0 < len X 
\end{code}
%
and then stipulating that 
%
\begin{code}
  head :: {v : [a] | NonEmp v} -> a
\end{code}
%
To verify the (core) definition of @head@, \toolname uses the above signature
to check the body in an environment
%
\begin{code}
  d :: {0 < len d}
\end{code}
%
When @d@ is matched with @[]@, the environment is 
strengthened with the corresponding refinement from 
the definition of @len@, \ie,
%
\begin{code}
  d :: {0 < (len d) && (len d) = 0}
\end{code}
%
Since the formula above is a contradiction, \toolname concludes that the
call to @patError@ is dead code, and thereby verifies the totality 
of @head@. Of course, now we have pushed the burden of proof onto clients
of @head@ -- at each such site, \toolname will check that the argument 
passed in is indeed a @NonEmp@ list, and if it successfully does so, then
we, at any uses of @head@, can rest assured that @head@ will never throw an 
exception. 

\mypara{Refinements and Totality} 
While the @head@ example is quite simple, in general, refinements make
it easy to prove totality in complex situations, where we must track
dependencies between inputs and outputs. For example, consider the @risers@
function from \cite{catch}:
%
\begin{code}
  risers []       = []
  risers [x]      = [[x]]
  risers (x:y:zs) 
    | x <= y      = (x:s) : ss 
    | otherwise   = [x] : (s:ss) 
    where 
      s:ss    = risers (y:etc)
\end{code}
%
The pattern match on the last line is partial; its core translation is
%
\begin{code}
  let (s, ss) = case risers (y:etc) of
                  s:ss -> (s, ss)
                  []   -> patError "..."
\end{code}
%
What if @risers@ returns an empty list? 
Indeed, @risers@ \emph{does}, on occasion, return an empty list per its
first equation. However, on close inspection, it turns out that 
\emph{if} the input is non-empty, \emph{then} the output is also
non-empty. Happily, we can specify this as:
%
\begin{code}
  risers :: l:_ -> {v:_ | NonEmp l => NonEmp v} 
\end{code}

\toolname verifies that @risers@ meets the above specification, 
and hence that the @patError@ is dead code as at that 
site, the scrutinee is obtained from calling @risers@ with a
@NonEmp@ list.

\mypara{Non-Emptiness via Measures}
Instead of describing non-emptiness indirectly using @len@, a 
user could a special measure:
%
\begin{code}
  measure nonEmp  :: [a] -> Prop
    nonEmp (x:xs)   = True
    nonEmp []       = False

  predicate NonEmp X = nonEmp X
\end{code}
%
After which, verification would proceed analagous to the above.

\mypara{Total Totality Checking} 
@patError@ is one of many possible errors thrown by non-total functions.  
@Control.Exception.Base@ has several others including @recSelError@, @irrefutPatError@, \etc which serve the purpose of making 
core translations total.
%
Rather than hunt down and specify @False@ preconditions one
by one, the user may automatically turn on totality checking 
by invoking \toolname with the \cmdtotality command line option, 
at which point the tool systematically checks that all the above 
functions are indeed dead code, and hence, that all definitions are total.

\subsection{Case Studies}

We verified totality of two libraries: \lbhscolour and \lbmap, earlier versions
of which had previously been proven total by \texttt{catch}~\citep{catch}.

\mypara{\lbmap} 
is a widely used library for (immutable) key-value maps, implemented
as balanced binary search trees.
Totality verification of \lbmap was quite straightforward.
We had already verified termination and the crucial 
binary search invariant~\ref{chapter:abstractrefinements}. To verify 
totality it sufficed to simply re-run verification with
the \cmdtotality argument.
%
All the important specifications were already captured by the types, 
and no additional changes were needed to prove totality.
%
%% \RJ{was it trivially total? i.e. is it total if you strip out all refinements
%% from specs?}
%% \NV{No, it fails in 6 functions all of which can trivially be reasoned to be total}
%% \NV{hedgeUnion, hedgeDiff, hedgeMerge, submap', join, merge}
%% \NV{The interesting story is that during verification \emph{we accidentally modified}
%% turn a function to partial, see my commit 041f1f0fea4d34ee41f50dbf7ce43e3c084c2743}
%

This case study illustrates an advantage of \toolname over specialized provers 
(\eg, \texttt{catch}~\citep{catch}): it can be used to prove totality, termination and
functional correctness at the same time, facilitating a nice reuse of
specifications for multiple tasks.

%% DONE \NV{(R3)
%% DONE Before discussing HsColour, I'd give a brief explanation of what it is.
%% DONE }
\mypara{\lbhscolour} is a library for generating syntax-highlighted LATEX and HTML from
Haskell source files.
Checking \lbhscolour was not so easy, as in some cases assumptions are used about the 
structure of the input data:
%
For example, @ACSS.splitSrcAndAnnos@ handles an
input list of @String@s and assumes that whenever
a specific @String@ (say @breakS@) appears then 
at least two @String@s (call them @mname@ and @annots@)
follow it in the list.
Thus, for a list @ls@ that starts with @breakS@ 
the irrefutable pattern  @(_:mname:annots)@ @=@ @ls@
should be total.
%
Though possible, it is currently it is somewhat cumbersome to specify such 
properties. 
%
As an easy and practical solution, 
to prove totality, we added a dynamic check that 
validates that the length of the input @ls@ exceeds @2@.

%% measure follows a b c = \case 
%%   []   -> true
%%   x:xs -> if x == a then first2 b c xs else follows a b c xs
%% 
%% measure first2 b c = \case
%%   []   -> false
%%   x:xs -> x == b && first1 c xs
%% 
%% measure first1 c = \case
%%   []   -> false
%%   x:xs -> x == c
%% 
%% Worse, \toolname has no way to express such an invariant: 
%% \toolname naturally describes invariants that recursively 
%% hold for every list element and 
%% reaches its limitations when reasoning about non-recursive
%% properties.

In other cases assertions were imposed via monadic checks, \eg @HsColour.hs@ reads the input arguments and 
checks their well-formedness using 
%
\begin{code}
  when (length f > 1) $ errorOut "..."
\end{code} %$
%
Currently \toolname does not support monadic reasoning that 
allows assuming that @(length f <= 1)@
holds when executing the action \emph{following} the @when@ check. 
%
Finally, code modifications were required to capture properties 
that are cumbersome to express with \toolname.
%
For example, @trimContext@ checks if there is an element that 
satisfies @p@ in the list @xs@; if so it defines 
%
@ys = dropWhile (not . p) xs@
%
and computes @tail ys@.
%
By the check we know that @ys@ has at least one element, the 
one that satisfies @p@. 
%
Due to the complexity of this property, we preferred to rewrite the specific code 
in a more verification friendly version. 


%%% \mynote{Bug}
%%% %
%%% \RJ{WHY? Seems like a simple GHC CHECK?}
%%% %
%%% On the positive side, totality verification revealed a subtle bug:
%%% %
%%% The instance @Enum@ of @Highlight@ does not define the @toEnum@ 
%%% method. In core, this reduces to a call to the error function 
%%% @noMethodBinding@.
%%% %
%%% Even though this totality bug can be tracked by GHC compilation,
%%% it exposes the strengths of our totality checker.

On the whole, while proving totality can be cumbersome 
(as in \lbhscolour) it is a nice side benefit of refinement
type checking and can sometimes be a fully automatic corollary
of establishing more interesting safety properties (as in \lbmap).

\input{text/realworldhaskell/termination}
\input{text/realworldhaskell/memory-safety}
\newcommand\lbxmonad{\texttt{xmonad}\xspace}

\section{Functional Correctness Invariants}\label{sec:structures}

So far, we have considered a variety of general, application independent
correctness criteria. Next, let us see how we can use \toolname to specify 
and statically verify critical, application specific correctness properties,
using two illustrative case studies: red-black trees and the stack-set data
structure introduced in the \lbxmonad system.

\subsection{Red-Black Trees}\label{sec:redblack}

Red-Black trees have several non-trivial invariants that are ideal for 
illustrating the effectiveness of refinement types and contrasting with
existing approaches based on GADTs~\cite{Kahrs01}.
%
The structure can be defined via the following Haskell type:
%
\begin{code}
  data Col    = R | B
  data Tree a = Leaf 
              | Node Col a (Tree a) (Tree a)
\end{code}
%
However, a @Tree@ is a valid Red-Black tree only if it 
satisfies three crucial invariants:
%
\begin{itemize}
  \item{\emphbf{Order:}} 
    The keys must be binary-search ordered, \ie the key at each node must
    lie between the keys of the left and right subtrees of the node,
  \item{\emphbf{Color:}}
    The children of every \emph{red} @Node@ must be colored \emph{black}, 
    where each @Leaf@ can be viewed as black,
  \item{\emphbf{Height:}}
    The number of black nodes along any path from each @Node@ to its @Leaf@s 
    must be the same.
\end{itemize}

Red-Black trees are especially tricky as various operations create 
trees that can temporarily violate the invariants. Thus, while 
the above invariants can be specified with singletons and GADTs, 
encoding all the properties (and the temporary violations) results
in a proliferation of data constructors that can somewhat obfuscate 
correctness. In contrast, with refinements, we can specify and verify
the invariants in isolation (if we wish) and can trivially compose
them simply by \emph{conjoining} the refinements.

\mypara{Color Invariant}
To specify the color invariant, we define a \emph{black-rooted tree} as:
%
\begin{code}
  measure isB           :: Tree a -> Prop 
    isB (Node c x l r)  = c == B
    isB (Leaf)          = True
\end{code}
%
and then we can describe the color invariant simply as:
%
\begin{code}
  measure isRB          :: Tree a -> Prop
    isRB (Leaf)         = True
    isRB (Node c x l r) = isRB l && isRB r &&
                          c = R => (isB l && isB r)
\end{code}
%
The insertion and deletion procedures create intermediate \emph{almost} 
red-black trees where the color invariant may be violated at the root. 
Rather than create new data constructors we define almost red-black 
trees with a measure that just drops the invariant at the root:
%
\begin{code}
  measure almostRB          :: Tree a -> Prop
    almostRB (Leaf)         = True
    almostRB (Node c x l r) = isRB l && isRB r
\end{code}

\mypara{Height Invariant}
To specify the height invariant, we define a black-height measure:
%
\begin{code}
  measure bh          :: Tree a -> Int
    bh (Leaf)         = 0
    bh (Node c x l r) = bh l + if c = R then 0 else 1
\end{code}
%
and we can now specify black-height balance as:
%
\begin{code}
  measure isBal          :: Tree a -> Prop
    isBal (Leaf)         = true
    isBal (Node c x l r) = bh l = bh r 
                         && isBH l && isBH r 
\end{code}
%
Note that @bh@ only considers the left sub-tree, 
but this is legitimate, because @isBal@ will 
ensure the right subtree has the same @bh@.

\mypara{Order Invariant}
We refine the data definition of @Tree@ 
to encode the ordering property:
%
\begin{code}
  data Tree a
    = Leaf
    | Node { c   :: Col
           , key :: a
           , lt  :: Tree {v:a | v < key }
           , rt  :: Tree {v:a | key < v } }
\end{code}
%

\mypara{Composing Invariants}
Finally, we can compose the invariants and define a 
Red-Black tree with the alias:
%
\begin{code}
  type RBT a = {v:Tree a | isRB v && isBal v}
\end{code}
%
An almost Red-Black tree is the above with @isRB@ 
replaced with @almostRB@, \ie does not require any 
new types or constructors.
If desired, we can ignore a particular invariant 
simply by replacing the corresponding refinement 
above with @true@.
Given the above -- and suitable signatures \toolname 
verifies the various insertion, deletion and rebalancing
procedures for a Red-Black Tree library.

\subsection{Stack Sets in XMonad}\label{sec:xmonad}

\lbxmonad is a dynamically tiling \texttt{X11} 
window manager that is written and configured in Haskell. 
The set of windows managed by XMonad is organized into a
hierarchy of types. At the lowest level we have a 
\emph{set} of windows @a@ represented as a @Stack a@
%
\begin{code}
  data Stack a = Stack { focus :: a   
                       , up    :: [a] 
                       , down  :: [a] }
\end{code}
%
The above is a zipper~\cite{zipper} where @focus@ is the 
``current" window and @up@ and @down@ the windows ``before"
and ``after" it.
%
Each @Stack@ is wrapped inside a @Workspace@ that also has
information about layout and naming:
%
\begin{code}
  data Workspace i l a = Workspace 
     { tag    :: i
     , layout :: l
     , stack  :: Maybe (Stack a) }
\end{code}
%
which is in turn, wrapped inside a @Screen@:
%
\begin{code}
  data Screen i l a sid sd = Screen 
    { workspace    :: Workspace i l a
    , screen       :: sid
    , screenDetail :: sd }
\end{code}
%
The set of all screens is represented by the top-level zipper:
%
\begin{code}
  data StackSet i l a sid sd = StackSet 
    { cur :: Screen i l a sid sd  
    , vis :: [Screen i l a sid sd]
    , hid :: [Workspace i l a]   
    , flt :: M.Map a RationalRect } 
\end{code}


\mypara{Key Invariant: Uniqueness of Windows}
The key invariant for the @StackSet@ type is that each window @a@
should appear at most once in a @StackSet i l a sid sd@. That is,
a window should \emph{not be duplicated} across stacks or workspaces.
Informally, we specify this invariant by defining a measure for the 
\emph{set of elements} in a list, @Stack@, @Workspace@ and @Screen@,
and then we use that measure to assert that the relevant sets are 
disjoint.

\mypara{Specification: Unique Lists} To specify that the set of elements
in a list is unique, \ie there are no duplicates in the list we first define
a measure denoting the set using Z3's~\citep{z3} built-in theory of sets:
%
\begin{code}
  measure elts  :: [a] -> Set a 
    elts ([])   = emp  
    elts (x:xs) = cup (sng x) (elts xs) 
\end{code}
%
Now, we can use the above to define uniqueness:
%
\begin{code}
  measure isUniq  :: [a] -> Prop 
    isUniq ([])   =  true 
    isUniq (x:xs) =  notIn x xs && isUniq xs
\end{code}
%
where @notIn@ is an abbreviation: 
%
\begin{code}
  predicate notIn X S = not (mem X (elts S))
\end{code}

\mypara{Specification: Unique Stacks}
We can use @isUniq@ to define unique, \ie, duplicate free,
@Stack@s as:
%
\begin{code}
  data Stack a = Stack 
   { focus :: a   
   , up    :: {v:[a] | Uniq1 v focus}
   , down  :: {v:[a] | Uniq2 v focus up} }
\end{code}
%
using the aliases
%
\begin{code}
  predicate Uniq1 V X    = isUniq V  && notIn X V
  predicate Uniq2 V X Y  = Uniq1 V X && disjoint Y V
  predicate disjoint X Y = cap (elts X) (elts Y) = emp
\end{code}
%
\ie the field @up@ is a unique list of elements different 
from @focus@, and the field @down@ is additionally disjoint 
from @up@.

\mypara{Specification: Unique StackSets}
It is straightforward to lift the @elts@ measure to 
the @Stack@ and the wrapper types @Workspace@ and 
@Screen@, and then correspondingly lift @isUniq@ to 
@[Screen]@ and \hbox{@[Workspace]@.}
%
Having done so, we can use those measures to refine 
the type of @StackSet@ to stipulate that there are 
no duplicates:
%
\begin{code}
  type UniqStackSet i l a sid sd 
    = {v: StackSet i l a sid sd | NoDups v} 
\end{code}
%
using the predicate aliases
%
\begin{code}
  predicate NoDups V 
    =  disjoint3 (hid V) (cur V) (vis V) 
    && isUniq (vis V) && isUniq (hid V)
  
  predicate disjoint3 X Y Z 
    =  disjoint X Y && disjoint Y Z && disjoint X Z
\end{code}
%
\toolname automatically turns the record selectors of 
refined data types to measures that return the values 
of appropriate fields, hence @hid x@ (resp. @cur x@, @vis x@)
are the values of the \hbox{@hid@,} @cur@ and @vis@ fields of
a @StackSet@ named @x@.

%%%% \begin{code}
%%%% data Stack a = Stack { focus :: a   
%%%%                      , up    :: ULNEq a focus
%%%%                      , down  :: ULNEq a focus }
%%%% 
%%%% data StackSet i l a sid sd = StackSet 
%%%%    { lcurrent  ::  Screen i l a sid sd   
%%%%    , lvisible  :: [Screen i l a sid sd]
%%%%    , lhidden   :: [Workspace i l a]
%%%%    , lfloating :: M.Map a RationalRect     
%%%%    }
%%%% \end{code}
%%%% %
%%%% data Stack a = Stack { focus  :: !a   
%%%%                      , up     :: [a]   
%%%%                      , down   :: [a] } 
%%%%                                   
%%%% data Stack a = Stack { focus :: a   
%%%%                      , up    :: UListDif a focus
%%%%                      , down  :: UListDif a focus }
%%%% 
%%%% data Workspace i l a = Workspace  { tag :: !i, layout :: l, stack :: Maybe (Stack a) }
%%%% 
%%%% data Workspace i l a = Workspace  { tag :: i, layout :: l, stack :: Maybe (UStack a) }
%%%% 
%%%% type UStack a = {v:(Stack a) | (ListDisjoint (getUp v) (getDown v))}
%%%% 
%%%% 
%%%% data Screen i l a sid sd = Screen { workspace     :: !(Workspace i l a)
%%%%                                   , screen        :: !sid
%%%%                                   , screenDetail  :: !sd }
%%%% 
%%%% data StackSet i l a sid sd =
%%%%     StackSet { current  :: !(Screen i l a sid sd)    -- ^ currently focused workspace
%%%%              , visible  :: [Screen i l a sid sd]     -- ^ non-focused workspaces, visible in xinerama
%%%%              , hidden   :: [Workspace i l a]         -- ^ workspaces not visible anywhere
%%%%              , floating :: M.Map a RationalRect      -- ^ floating windows
%%%%              } 
%%%% 
%%%% A workspace is just a @Stack@ of virtual workspaces 
%%%% tagged with a tag @i@ and its layout @l@
%%%% @Workspace i l a@.
%%%% 
%%%% To view a workspace on a physical screen one needs to 
%%%% associate the workspace with physical screen's id @sid@
%%%% and details @sd@, 
%%%% forming the new data structure @Screen i l a sid sd@.
%%%% 
%%%% Xinerama in X11 allows viewing multiple virtual workspaces
%%%% simultaneously. 
%%%% %
%%%% While only one the current one will ever be in focus (i.e. will
%%%% receive keyboard events), other workspaces may be passively
%%%% viewable.  
%%%% %
%%%% We thus need to track which virtual workspaces are
%%%% associated (viewed) on which physical screens.  
%%%% %
%%%% To keep track of
%%%% this, \lbxmonad's main data structure  @StackSet i l a sid sd@ 
%%%% keeps, apart from the current screen,
%%%% separate lists of visible but non-focused
%%%% workspaces (@Screen@) , and non-visible workspaces (@Workspace@).

%%%% \mypara{Unique Stack} 
%%%% Assume the existence of a predicate 
%%%% @ULNeq (X::a) (XS::[a])@ that ensures that 
%%%% (1)~the list @XS@ has no duplicates and
%%%% (2)~@X@ is not an element of @XS@.
%%%% %
%%%% With that magical predicate we define an \emph{almost unique}
%%%% stack: 
%%%% \begin{code}
%%%% data Stack a = Stack { focus :: a   
%%%%                      , up    :: ULNEq a focus
%%%%                      , down  :: ULNEq a focus }
%%%% \end{code}
%%%% The stack has a @focus@ element @a@ and 
%%%% and two unique lists of @a@'s @up@ and @down@ of the focus.
%%%% 
%%%% The above Stack is almost unique, as an element 
%%%% may appear both in the @up@ and the @down@ lists.
%%%% %
%%%% We define a type alias for @U@nique@Stack@
%%%% that rejects with possibility:
%%%% 
%%%% \begin{code}
%%%% type UStack a = {v:(Stack a) |  (LDisjoint (up v) (down v))}
%%%% 
%%%% predicate LDisjoint X Y = 
%%%%   (Set_emp (Set_cap (elts X) (elts Y)))
%%%% \end{code}
%%%% 
%%%% 
%%%% Note that the above definitions crucially depend on set theoretic properties.
%%%% Thus, verification of \lbxmonad is achieved using an SMT back-end 
%%%% that supports set theory (like Z3~\citep{z3}).
%%%% 
%%%% We slightly modify the above measure definition
%%%% to define @dups@, a measure that returns the 
%%%% duplicate elements of a list.
%%%% %
%%%% \begin{code}
%%%%   measure dups :: [a] -> Set a
%%%%   dups ([])  = emp v
%%%%   dups(x:xs) = if mem x (elts xs) 
%%%%                  then cup (sng x) (dups xs)
%%%%                  else (dups xs)
%%%% \end{code}
%%%% 
%%%% Using @dups@ we define the magical list type @ULNEq a N@ 
%%%% as a list @v@ that has no duplicates, \ie the set @dups v@
%%%% is empty, and @N@ does not belong to the set @elts v@.
%%%% 
%%%% \begin{code}
%%%% type ULNEq a N = {v:[a] | ( (UL v) && (not (LElt N v))}
%%%% predicate LElt N LS  = (Set_mem N (elts LS)) 
%%%% predicate UL     LS  = (Set_emp (dups LS)   )
%%%% \end{code}
%%%% 
%%%% Throughout verification
%%%% we need to establish and use the invariant 
%%%% that each Stack is Unique.
%%%% %
%%%% This is achieved by the following annotation
%%%% \begin{code}
%%%% using (Stack a) as (UStack a)
%%%% \end{code}
%%%% that allows \toolname to 
%%%% (1)~ use the invariant:
%%%% each time a Stack value is retrieved from the environment
%%%% \toolname strengthens its type with the disjointness information;
%%%% (2)~ prove the invariant:
%%%% each time @Stack@ data constructor is used,
%%%% an disjoint constraint should be proved.
%%%% Failure to prove this constraint will raise an 
%%%% ``Invariant Check'' error.
%%%% 
%%%% \mypara{Unique StackSet} 
%%%% Establishing Uniqueness on StackSets is a generalization of the above procedure.
%%%% 
%%%% The definition of a @StackSet@ includes the @current@ Screen, the list of @visible@ screens,
%%%% and the list of @hidden@ workspaces.
%%%% \begin{code}
%%%% data StackSet i l a sid sd = StackSet 
%%%%    { lcurrent  ::  Screen i l a sid sd   
%%%%    , lvisible  :: [Screen i l a sid sd]
%%%%    , lhidden   :: [Workspace i l a]
%%%%    , lfloating :: M.Map a RationalRect     
%%%%    }
%%%% \end{code}
%%%% %
%%%% \toolname automatically turns the record selectors of refined data types
%%%% to measures that return the appropriate fields.
%%%% %
%%%% Thus we infix the refined selectors with an @l@
%%%% to distinguish between the haskell (\eg, @current@)
%%%% and the logical (\eg, @lcurrent@) selectors. 
%%%% 
%%%% To prove absence of duplicates we need to @use@
%%%% only @StackSet@s that have no duplicates:  
%%%% \begin{code}
%%%% using (StackSet i l a sid sd) 
%%%%  as  {v:StackSet i l a sid sd|(NoDuplicates v)}
%%%% \end{code}
%%%% 
%%%% @NoDuplicates@ ensures that the elements of
%%%% @hidden@, @current@, and @visible@ 
%%%% workspaces are mutually disjoint
%%%% and that the @visible@ and @hidden@ workspaces
%%%% have no duplicates.
%%%% %
%%%% \begin{code}
%%%% predicate NoDuplicates SS = 
%%%%     (Disjoint3  (workspacesElts (lhidden  SS)) 
%%%%                 (screenElts     (lcurrent SS)) 
%%%%                 (screensElts    (lvisible SS))) 
%%%%   &&
%%%%     (Set_emp (screensDups    (lvisible SS))) 
%%%%   &&
%%%%     (Set_emp (workspacesDups (lhidden  SS)))
%%%% \end{code}
%%%% %
%%%% Again we used recursively defined measures to grap 
%%%% the elements and the duplicates of the structures.
%%%% For example, @screenElts@ returns 
%%%% the elements of the stack of the workspace of the screen, 
%%%% and is used by @screensDup@ to grap the duplicates of 
%%%% a list of Screens.
%%%% 
\mypara{Verification}
\toolname uses the above refined type to verify the key invariant,
namely, that no window is duplicated.
%
% However, the verification process, while eventually successful, 
% revealed several limitations of our approach.
%
Three key actions of the, eventually successful, verification process
can be summarized as follows:
\begin{itemize}
\item\emph{Strengthening library functions.} 
  \lbxmonad repeatedly concatenates the lists of a Stack. %edit to fix overful box
  %
  To prove that for some \hbox{@s:Stack a@,} @(up s ++ down s)@ is a unique list,
  the type of @(++)@ needs to capture that concatenation of two unique and
  disjoint lists is a unique list.
  %
  For verification, we assumed that Prelude's @(++)@ satisfies this property.
  %
  But, not all arguments of @(++)@ are unique disjoint lists:
  @"StackSet" ++ "error"@ is a trivial example that does not satisfy
  the assumed preconditions of @(++)@ thus creating a type error.
  % 
  Currently, \toolname does not support intersection types, 
  thus we used an unrefined @(++.)@ variant of @(++)@ for such cases.
     
\item\emph{Restrict the functions' domain.}
%% \RJ{Seems like you HAVE to do this -- has nothing to do with LH?}
  @modify@ is a @maybe@-like function that given a default value @x@,
  a function @f@, and a StackSet @s@, applies @f@ on the @Maybe@
  values inside @s@. 
  %
\begin{code}
modify :: x:{v:Maybe (Stack a) | isNothing v}
       -> (y:Stack a -> Maybe {v:Stack a | SubElts v y})
       -> UniqStackSet i l a s sd 
       -> UniqStackSet i l a s sd
\end{code}
	%
        Since inside the StackSet @s@ each @y:Stack a@ could be replaced
    with either the default value @x@ or with @f y@, we need to
    ensure that both these alternatives will not insert duplicates.
	%
	This imposes the curious precondition that the default
	value should be @Nothing@.

			
	\item\emph{Code inlining}
    %
    Given a tag @i@ and a StackSet @s@,  @view i s@ will set the current Screen 
    to the screen with tag @i@, if such a screen exists in @s@.
    %
    Below is the original definition for @view@ in case when a screen with tag 
    @i@ exists in visible screens
    %
\begin{code}
view :: (Eq s, Eq i) => i 
     -> StackSet i l a s sd 
     -> StackSet i l a s sd
view i s    
  | Just x <- find ((i==).tag.workspace) (visible s)
  = s { current = x
      , visible = current s 
                : deleteBy (equating screen) x (visible s) } 
\end{code}
    %
    Verification of this code is difficult as we cannot suitably type @find@. 
    %
    Instead we \emph{inline} the call to @find@ and the field update into a 
    single recursive function @raiseIfVisible i s@ that in-place replaces @x@ 
    with the current screen.  
\end{itemize}

Finally, \lbxmonad comes with an extensive suite of QuickCheck properties,
that were formally verified in Coq~\cite{Swierstra2012}. In future work~\ref{chapter:conclusion},
it would be interesting to do a similar verification with \toolname, 
to compare the refinement types to proof-assistants.

%%% TODO \subsection{QuickCheck Properties}
%%% TODO 
%%% TODO \lbxmonad is tested against $113$ \texttt{quickcheck} properties.
%%% TODO %
%%% TODO Of those $15$ check the uniqueness invariant 
%%% TODO and the rest $113$ check various functional properties.
%%% TODO %
%%% TODO We started the endeavour of verifying these properties with \toolname.
%%% TODO %
%%% TODO We looked at a sample of $15$ properties to conclude that
%%% TODO which we categorized as follows:
%%% TODO \begin{itemize}
%%% TODO \item\emph{Easy to be proved.}
%%% TODO Consider the \texttt{quickcheck} property that checks that @view@ing 
%%% TODO a @StackSet a@ is idempotent:
%%% TODO \begin{code}
%%% TODO prop_view_idem (x :: T) (i :: NonNegative Int) 
%%% TODO   = i `tagMember` x ==> view i (view i x) == (view i x)
%%% TODO \end{code}
%%% TODO %
%%% TODO The above property directly translated to a haskell function
%%% TODO \begin{code}
%%% TODO type Valid     = {v:Bool | (Prop v) }
%%% TODO 
%%% TODO prop_view_idem :: StackSet i l a sid sd -> i -> Valid
%%% TODO prop_view_idem x i 
%%% TODO   | i `tagMember` x = view i (view i) == v
%%% TODO   | otherwise       = True
%%% TODO \end{code}
%%% TODO %
%%% TODO By the above type signature,
%%% TODO \ie by the result type @Valid@, 
%%% TODO we specify that the function should always returns True.
%%% TODO %
%%% TODO When typechecking the above function,
%%% TODO \toolname proves that the property holds.
%%% TODO %
%%% TODO \toolname is able to verify this property as the result type of @view@
%%% TODO is strengthens with a refinement 
%%% TODO (in this case @EqTag x i => x = v@) 
%%% TODO that directly implies this property.
%%% TODO 
%%% TODO The above is generalizing to (10/17) properties that we checked:
%%% TODO strengthening the function types by refinements that \toolname can prove
%%% TODO is sufficient to verify these properties.
%%% TODO 
%%% TODO \item\emph{Can be estimated.}
%%% TODO In some other properties (like checking that @view@ing is reversible),
%%% TODO two StackSets (@s1@ and @s2@) were normalized before being compared,
%%% TODO that is their elements were first sorted.
%%% TODO %
%%% TODO In our logic we do not support any operation that can normalize structures in such a way.
%%% TODO %
%%% TODO Thus we cannot prove this exact property.
%%% TODO %
%%% TODO Instead we approximated it, by proving that proving that @s1@ and @s2@
%%% TODO have the same sets of elements.
%%% TODO %
%%% TODO We approximated (3/17) of the properties.
%%% TODO 
%%% TODO \item\emph{Their proof cannot be supported, currently.} [1]
%%% TODO One \texttt{quickcheck} property checks 
%%% TODO that @i@ cannot belong to an empty stackset.
%%% TODO %
%%% TODO We used abstract refinements to encode empty stacksets.
%%% TODO %
%%% TODO Proving the above property would be easy 
%%% TODO if we could mix abstract and concrete refinements in logical formulas
%%% TODO or if \toolname supported sum types.
%%% TODO %
%%% TODO Both the alternatives constitutes features that
%%% TODO we would like to extend \toolname with in the near future.
%%% TODO %
%%% TODO Still, currently we are not able to prove such kind of properties.
%%% TODO \item\emph{Cannot be expressed}
%%% TODO Other properties %, like @prop_focus_left_master@ 
%%% TODO check that the order of the windows is not affected by certain operations.
%%% TODO %
%%% TODO Though not infeasible we acknowledge that \toolname 
%%% TODO is not appropriate for reasoning about order preserving
%%% TODO and verification of such properties 
%%% TODO would require many code modifications.
%%% TODO %
%%% TODO (3/17) are order preserving properties.
%%% TODO \end{itemize}

\input{text/realworldhaskell/evaluation}

\mypara{Acknowledgments}
The material of this chapter are adapted from the following publication:
\noindent N. Vazou, E. Seidel, and R. Jhala,
``LiquidHaskell: Experience with Refinement Types in the Real World'', 
Haskell, 2014.


\chapter{Soundness Under Lazy Evaluation}\label{chapter:refinedhaskell}
\makequote{Laziness may appear attractive, but work gives satisfaction.}{Anne Frank}

\begin{comment}
Must program verifiers always choose between expressiveness
and automation?
%
On the one hand, tools based on higher order logics
and full dependent types impose no limits on expressiveness,
but require user-provided (perhaps, tactic-based) proofs.
%
On the other hand, tools based on Refinement Types~\cite{Rushby98,pfenningxi98}
trade expressiveness for automation. For example, the refinement types
%
\begin{code}
  type Pos     = {v:Int | 0 < v}
  type IntGE x = {v:Int | x <= v}
\end{code}
%
specify subsets of @Int@ corresponding to values
that are positive or larger than some other value @x@
respectively. By limiting the refinement predicates to
SMT-decidable logics~\cite{Nelson81}, refinement type
based verifiers eliminate the need for explicit proof terms,
and thus automate verification.

% We can specify contracts like pre- and post-conditions by
% suitably refining the input and output types of functions.

This high degree of automation has enabled the
use of refinement types for a variety of verification
tasks, ranging from array bounds checking~\cite{LiquidPLDI08},
termination and totality checking~\cite{Vazou14},
protocol validation~\cite{GordonTOPLAS2011,FournetCCS11},
and securing web applications~\cite{SwamyOAKLAND11}.
%
Unfortunately, this automation comes at a price.
To ensure predictable and decidable type checking, we must
limit the logical formulas appearing in specification types
to decidable (typically quantifier free) first order theories,
thereby precluding \emph{higher-order} specifications that
are essential for \emph{modular} verification.
\end{comment}

In this chapter we introduce \emph{Bounded Refinement Types} which enable 
\emph{bounded quantification} over refinements. 
%
Previously (Chapter~\ref{chapter:abstractrefinements}),
we developed Abstract Refinement Types, a mechanism
for quantifying type signatures over abstract refinement parameters.
%
We preserved decidability of checking and inference
by encoding abstractly refined types with uninterpreted functions
obeying the decidable axioms of congruence~\cite{Nelson81}. 
%
While useful,
refinement quantification was not enough to enable higher order abstractions
requiring fine grained \emph{dependencies between} abstract refinements.
%
In this chapter, we solve this problem by enriching signatures
with bounded quantification. 
%
The \emph{bounds} correspond to Horn
implications between abstract refinements, which, as in the classical
setting, correspond to subtyping constraints that must be satisfied 
by the concrete refinements used at any call-site. This
addition proves to be remarkably effective.

\begin{itemize}
\item
First, we demonstrate via a series of short examples how bounded refinements
enable the specification and verification of diverse textbook higher order
abstractions that were hitherto beyond the scope of decidable refinement
typing~(\S~\ref{sec:boundedrefinementtypes:overview}).

\item
Second, we formalize bounded types and show how bounds are translated
into ``ghost'' functions, reducing type checking and inference to the
``unbounded'' setting of chapter~\ref{chapter:abstractrefinements}, 
thereby ensuring that checking
remains decidable. Furthermore, as the bounds are Horn constraints, we
can directly reuse the abstract interpretation of Liquid Typing~\citep{LiquidPLDI08}
to automatically infer concrete refinements at instantiation
sites~(\S~\ref{sec:check}).

\item
Third, to demonstrate the expressiveness of bounded refinements, we
use them to build a typed library for extensible dictionaries, to
then implement a relational algebra library on top of those
dictionaries, and to finally build a library for type-safe
database access~(\S~\ref{sec:database}).

\item
Finally, we use bounded refinements to develop a \emph{Refined State Transformer}
monad for stateful functional programming, based upon Filli\^atre's method
for indexing the monad with pre- and post-conditions~\citep{Filliatre98}.
%
We use bounds to develop branching and looping combinators whose types
signatures capture the derivation rules of Floyd-Hoare logic, thereby
obtaining a library for writing verified stateful computations~(\S~\ref{sec:state}).
%
We use this library to develop a refined IO monad that tracks capabilities
at a fine-granularity, ensuring that functions only access specified
resources~(\S~\ref{sec:files}).
\end{itemize}

We have implemented Bounded Refinement Types in \toolname.
The source code of the examples (with slightly more verbose concrete syntax)
is at \cite{liquidhaskellgithub}.
%
While the construction of these verified abstractions is possible with full
dependent types, bounded refinements
%
keep checking automatic and decidable,
%
use abstract interpretation to automatically synthesize
refinements (\ie pre- and post-conditions and loop invariants),
and most importantly
%
enable retroactive or \emph{gradual} verification as when
erase the refinements, we get valid programs in the
host language.
%
Thus, bounded refinements point a way towards 
both automated and expressive verification. 
%
%%% Local Variables:
%%% mode: latex
%%% TeX-master: "main"
%%% End:

\section{Overview}\label{sec:refinedhaskell:overview}

We start with an informal overview of a sound refinement type 
system for Haskell. 
%
After recapitulating the basics of refinement types
we illustrate why the classical approach based on 
verification conditions (VCs) is unsound due to 
lazy evaluation.  
%
Next, we step back to understand precisely how the 
VCs arise from refinement subtyping and how subtyping
is different under eager and lazy evaluation. 
In particular, we demonstrate that under lazy, but 
\emph{not} eager, evaluation, the refinement type 
system, and hence the VCs, must account for divergence.
%
Consequently, we develop a type system that accounts 
for divergence in a modular and syntactic fashion 
and illustrate its use via several small examples.
%
Finally, we show how a refinement-based termination
analysis is used to improve precision, yielding 
a highly effective SMT-based verifier for Haskell.

\subsection{Standard Refinement Types: From Subtyping to VC}\label{subsec:vc}


\begin{figure}[!t]
\centering
\captionsetup{justification=centering}
$$
\begin{array}{rrcl}
\emphbf{Refinements} & \refi     & ::= & \dots \text{varies} \dots \\[0.05in] 
\emphbf{Basic Types} & \base  & ::= & \tref{v}{\tint}{}{\refi} \spmid \dots \\[0.05in] 
\emphbf{Types}       & \typ   & ::= & \base \spmid \tfunref{\x}{\typ}{\typ}{v}{\refi} \\[0.05in] 
\emphbf{Environment} & \env   & ::= & \emptyset \spmid \bind{x}{\typ}, \env \\[0.1in]

\multicolumn{3}{l}{\emphbf{Subtyping}}   & \issubtype{\env}{\typ_1}{\typ_2} \\[0.1in]

\multicolumn{4}{l}{\mbox{\emphbf{Abbreviations}}} \\[0.05in]
  \multicolumn{2}{r}{\bind{x}{\ttref{\refi}}} & \doteq & \bind{x}{\tref{x}{\tint}{}{\refi}} \\[0.05in]
  \multicolumn{2}{r}{\ttreftsimple{\x}{\refi}}   & \doteq & {\tref{x}{\tint}{}{\refi}} \\[0.05in]
  \multicolumn{2}{r}{\ttref{\refi}}             & \doteq & {\tref{v}{\tint}{}{\refi}} \\[0.05in]
  \multicolumn{2}{r}{\reft{\x}{\reft{\y}{\tint}{\refi_\y}}{\refi_\x}} & \doteq & 
  					{\reft{\x}{\tint}{\refi_\x \land \refi_\y\sub{\y}{\x}}} \\[0.05in]


\multicolumn{4}{l}{\mbox{\emphbf{Translation}}} \\[0.05in]
\multicolumn{2}{r}{\embed{\issubtype{\env}{\base_1}{\base_2}}} & \defeq& \isvalidvc{\embed{\env}}{\embed{\base_1}}{\embed{\base_2}}\\[0.05in]
  \multicolumn{2}{r}{\embed{\tttref{\x}{\tint}{}{\refi}}} & \defeq & \lref\\[0.05in]
  \multicolumn{2}{r}{\embed{\ttbind{\x}{\tttref{\vv}{\tint}{}{\refi}}}} &\defeq & \ \mbox{``$\mathtt{\x}$ is a value"}  \Rightarrow \SUBST{\lref}{\vv}{\x}\\[0.05in]
  \multicolumn{2}{r}{\embed{\bind{\x}{(\tfun{\y}{\typ_\y}{\typ})}}} &\defeq & \tttrue\\[0.05in]
  \multicolumn{2}{r}{\embed{\ttbind{\x_1}{\typ_1},\ldots,\ttbind{\x_\n}{\typ_\n}}} & \defeq & \embed{\ttbind{\x_1}{\typ_1}} \wedge \ldots \wedge \embed{\ttbind{\x_\n}{\typ_\n}} \\
\end{array}
$$
\caption[Summary of Informal Notation.]{Informal Notation: Types, Subtyping, and VCs.}
\label{fig:overview:syntax}
\end{figure}


First, let us see how standard refinement type systems 
\cite{LiquidPLDI08, Knowles10} will 
use
the aforementioned refinement type aliases @Pos@ and @Nat@ 
and the specification for @div@
%
to \emph{accept} @good@ and
\emph{reject} @bad@.
We use the syntax of Figure~\ref{fig:overview:syntax}, where $r$ is a
\emph{refinement expression}, or just \emph{refinement} for short.
We will vary the expressiveness of the language of refinements in
different parts of this document.
%
\begin{code}
    good     :: Nat -> Nat -> Int
    good x y = let z = y + 1 in x `div` z

    bad      :: Nat -> Nat -> Int
    bad x y  = x `div` y
\end{code}

\spara{Refinement Subtyping} 
To analyze the body of @bad@, the refinement type system will 
check that the second parameter @y@ has type @Pos@ at the call 
to @div@; formally, that %. Formally, the system will check that the type of
the actual parameter @y@ is a \emph{subtype} of the type of 
@div@'s second input, via a subtyping query:
%
$$
 \ttbind{x}{\tref{x}{Int}{}{x \geq 0}},\ 
 \ttbind{y}{\tref{y}{Int}{}{y \geq 0}}
 \ \vdash\
 \subt{\tttref{y}{Int}{}{y \geq 0}}{\tttref{v}{Int}{}{v > 0}}$$
%
We use the Abbreviations of Figure~\ref{fig:overview:syntax} to simplify the 
syntax of the queries.
%
So the above query simplifies to:
$$\ttbind{x}{\ttref{x \geq 0}},\ \ttbind{y}{\ttref{y \geq 0}}\ \vdash\ \subtref{v \geq 0}{v > 0}$$

\spara{Verification Conditions}
To discharge the above subtyping query, a refinement type system
generates a \emph{verification condition} (VC), a logical formula 
that stipulates that under the assumptions corresponding to the 
environment bindings, the refinement in the sub-type \emph{implies} 
the refinement in the super-type.
%
We use the translation \embed{\cdot} shown in Figure~\ref{fig:overview:syntax}
to reduce a subtyping query to a verification condition.
%
The translation of a basic type into logic is the refinement of the type.
The translation of an environment is the conjunction of its bindings.
Finally, the translation of a binding \ttbind{\x}{\typ} is the embedding of \typ 
guarded by a predicate denoting that ``\x is a value''.
%
For now, let us ignore this guard and see how the subtyping query for 
@bad@ reduces to the \emph{classical} VC:
%%
$$\mathvc{\mathvc{(\ttx \geq 0)}} \wedge \mathvc{(\tty \geq 0)}\ \Rightarrow\ \mathvc{(\ttv \geq 0)} \Rightarrow \mathvc{(\ttv > 0)}$$
%%
Refinement type systems are carefully engineered (\Sref{sec:typing}) 
so that (\emph{unlike} with full dependent types) the logic of 
refinements \emph{precludes} arbitrary functions and only includes 
formulas from efficiently decidable logics, \eg the quantifier-free 
logic of linear arithmetic and uninterpreted functions (\logiclang).
Thus, VCs like the above can be efficiently validated by SMT 
solvers \cite{z3}. 
%
In this case, the solver will reject the above VC as \emph{invalid} meaning the 
implication, and hence, the relevant subtyping requirement does not hold.
So the refinement type system will \emph{reject} @bad@.

On the other hand, a refinement system \emph{accepts} @good@.
Here, @+@'s type exactly captures its behaviour into the logic:
\begin{code}
  (+) :: x:Int -> y:Int -> {v:Int | v = x + y} 
\end{code}
Thus, we can conclude that the divisor @z@ is a positive number. 
The subtyping query for the argument to @div@ is
%
\begin{align*}
\ttbind{x}{\ttref{x \geq 0}},\
\ttbind{y}{\ttref{y \geq 0}},\ 
\ttbind{z}{\ttref{z = y + 1}}\ 
\vdash\ & \subtref{v = y + 1}{v > 0}
%\label{sub:good}
\intertext{which reduces to the \emph{valid} VC}
	\mathvc{(\ttx \geq 0)} \wedge \mathvc{(\tty \geq 0)}\wedge 
	\mathvc{(\ttz = \tty+1)} 
   \Rightarrow\ & \mathvc{(\ttv = \tty+1)} 
   \Rightarrow \mathvc{(\ttv > 0)} \nonumber
\end{align*}

\subsection{Lazy Evaluation Makes VCs Unsound} \label{sec:overview:unsound}

To generate the classical VC, we ignored the ``\x is a value'' guard 
that appears in the embedding of a binding \embed{\ttbind{\x}{\typ}} (Figure~\ref{fig:overview:syntax}). 
%
Under lazy evaluation, ignoring this ``is a value'' guard can lead to unsoundness.
%
Consider 
%
\begin{code}
   diverge   :: Int -> {v:Int | false}
   diverge n = diverge n
\end{code}
%
The output type captures the \emph{post-condition} 
that the function returns an @Int@ satisfying @false@. 
This counter-intuitive specification states, in essence, 
that the function \emph{does not terminate}, \ie does not
return \emph{any} value. 
%
Any standard refinement type checker (or Floyd-Hoare
verifier like Dafny~\cite{dafny}) 
will verify the given signature for @diverge@ 
via the classical method of inductively \emph{assuming} 
the signature holds for @diverge@ and then 
\emph{guaranteeing} the signature~\cite{Hoare71,Nipkow02}.
%
Next, consider the call to @div@ in @explode@:
%
\begin{code}
   explode   :: Int -> Int
   explode x = let {n = diverge 1; y = 0}
               in  x `div` y
\end{code}
%
To analyze @explode@, the refinement type system will check 
that @y@ has type @Pos@ at the call to @div@, \ie will check 
that 
%
\begin{align}
   \ttbind{\ttn}{\ttref{\ttfalse}},\ \ttbind{\tty}{\ttref{\tty = 0}}\ \vdash\ & \subtref{\ttv = 0}{\ttv > 0} 
   \label{sub:explode}
\intertext{In the environment, \texttt{n} is 
  bound to the type corresponding to the \emph{output} 
  type of \texttt{diverge} and \texttt{y} is bound to 
  the type stating \texttt{y} equals \texttt{0}.
  In this environment, we must prove that actual parameter's 
  type -- \ie that of \texttt{y} -- is a subtype of \texttt{Pos}. 
  The subtyping, using the embedding of Figure~\ref{fig:overview:syntax} 
  and ignoring the ``is a value'' guard, 
  reduces to the VC:}
    \mathvc{\ttfalse} \wedge \mathvc{\tty = 0}\ \Rightarrow\ & \mathvc{(\ttv = 0)} \Rightarrow \mathvc{(\ttv > 0)}
    \label{vc:explode}
\end{align}
%
The SMT solver proves this VC valid using the 
contradiction in the antecedent, thereby unsoundly 
proving the call to @div@ safe!

\mypara{Eager vs. Lazy Verification Conditions}
%
We pause to emphasize that the problem lies 
in the fact that the classical technique for encoding subtyping 
(or generally, Hoare's ``rule of consequence" \cite{Hoare71}) 
with VCs is \emph{unsound under lazy evaluation}.
%
To see this, observe that the VC~(\ref{vc:explode}) is perfectly 
\emph{sound} under eager (strict, call-by-value) evaluation.
%
In the eager setting, the program is safe in that
@div@ is never called with the divisor @0@, as it 
is not called at all!
%
The inconsistent antecedent in the VC logically encodes the 
fact that, under eager evaluation, the call to @div@ is 
\emph{dead code}.
%
Of course, this conclusion is spurious under Haskell's lazy 
semantics. As @n@ is not required, the program will dive 
headlong into evaluating the @div@ and hence crash, 
rendering the VC meaningless.


\mypara{The Problem is Laziness} Readers familar 
with fully dependently typed languages, for instance 
Cayenne~\cite{cayenne}, Agda~\cite{agda}, 
Coq~\cite{coq-book}, and Idris~\cite{Brady13}, may be
tempted to attribute the unsoundness to the presence 
of arbitrary recursion and hence non-termination 
(\eg @diverge@).
While it \emph{is} possible to define a sound semantics 
for dependent types that mention potentially non-terminating
expressions~\citep{Knowles10}, it is not clear how to reconcile
such semantics with decidable type checking. 

Refinement type systems avoid this situation by carefully restricting 
types so that they do not contain arbitrary terms (even through 
substitution), but rather only terms from restricted logics that
preclude arbitrary user-defined functions~\cite{pfenningxi98,Dunfield07,fstar}.
Very much like previous work, for now, we enforce the same restriction
with a \emph{well-formedness condition} on 
refinements (\rwbased in Fig.~\ref{fig:declang:typing}).
%
In Chapter~\ref{refinementrflection} we present how 
our logic is extended with provably terminating arbitrary terms, 
while preserving both soundness and decidability.

However, we show that restricting the logic of refinements
\emph{is plainly not sufficient for soundness} 
when laziness is combined with non-termination, 
as binders can be bound to diverging expressions. 
Unsurprisingly, in a strongly
normalizing language the question of lazy or strict semantics is irrelevant for soundness, and hence 
an ``easy'' way to solve the problem would be to completely eliminate non-termination and rely on the soundness
of previous refinement or dependent type systems! Instead, we show here how to 
recover soundness for a lazy language \emph{without} imposing such a drastic requirement. 


\subsection{Semantics, Subtyping \& Verification Conditions} \label{sec:den-sem}

To understand the problem, let us take a step 
back to get a clear view of the relationship 
between the operational semantics, subtyping,
and verification conditions.
%
We use the formulation of evaluation-order 
independent refinement subtyping developed 
for \hlang~\cite{Knowles10} in which 
refinements $r$ are \emph{arbitrary} expressions $e$ from the source language.
We define a denotation for types and use it 
to define subtyping declaratively.

\mypara{Denotations of Types and Environments}
%
Recall the type @Pos@ defined as {@{v:Int | 0 < v}@}.
Intuitively, @Pos@ denotes the \emph{set of} @Int@ 
expressions which evaluate to values greater than @0@.
%
We formalize this intuition by defining the denotation
of a type as:
$$\interp{\tttref{\x}{\typ}{}{\refi}} \defeq \{e \mid \hastype{\emptyset}{e}{\typ}, \mbox{ if } \goesto{e}{\val} \mbox{ then } \goesto{\SUBST{\refi}{\x}{\val}}{\etrue} \}$$
%
That is, the type denotes the set of expressions \mie that have
the corresponding base type \typ which \emph{diverge or} reduce 
to values that make the refinement reduce to $\etrue$. 
%
The guard \goesto{\mie}{\val} is crucially required to prove soundness in the presence of recursion.
Thus, quoting~\cite{Knowles10}, ``refinement types specify partial and not total correctness''. 

An \emph{environment} $\env$ is a sequence of type bindings
and a \emph{closing substitution} \sto\xspace is a sequence of expression bindings:
$$\env \defeq \ttbind{\x_1}{\typ_1},\ldots\ttbind{\x_\n}{\typ_\n} \qquad \qquad
  \sto \defeq \tgbind{\x_1}{\mie_1}, \ldots, \tgbind{\x_\n}{\mie_\n}$$ %%
%% 
Thus, we define the denotation of $\env$ as the set of substitutions:
%%
$$\interp{\env} \defeq \{\sto \mid \forall \ttbind{\x}{\typ} \in \env. \sto(\x) \in \interp{\thetasub{\sto}{\typ}} \}$$

\mypara{Declarative Subtyping}
Equipped with interpretations for types and environments, 
we define the \emph{declarative subtyping} \rtdsub 
(over basic types $\base$, shown in Figure~\ref{fig:overview:syntax}) 
to be containment between the types' denotations:
$$
\inference
  {\forall \sto\in \interp{\env}.\  
  		 \interp{\thetasub{\sto}{\tttref{\vv}{\Base}{}{\refi_1}}} 
  		\subseteq   \interp{\thetasub{\sto}{\tttref{\vv}{\Base}{}{\refi_2}}}}
  {\issubtype{\env}{\tttref{\vv}{\Base}{}{\refi_1}}{\tttref{\vv}{\Base}{}{\refi_2}}}
  \rtdsub
$$
%
Let us revisit the @explode@ example from \Sref{sec:overview:unsound}; 
recall that the function is safe under eager evaluation but unsafe under 
lazy evaluation. Let us see how the declarative subtyping allows us to 
reject in the one case and accept in the other.

\mypara{Declarative Subtyping with Lazy Evaluation}
Let us revisit the query (\ref{sub:explode}) to see whether it
holds under the declarative subtyping rule \rtdsub. The denotation
containment
%
\begin{align}
\label{con:explode}
   \forall \sto \in &\interp{
   		\ttbind{n}{\ttref{\ttfalse}},\ \ttbind{y}{\ttref{y = 0}}
   		}. 
     \interp{{{\sto}\ {\ttref{v = 0}}}} \subseteq \interp{{\sto}\ {\ttref{v > 0}}} 
\end{align}
%
\emph{does not} hold. To see why, consider a $\sto$ that maps 
$n$ to any diverging expression of type $\ttInt$ and $y$ 
to the value $0$.
%
Then, $0 \in \interp{\sto\ \ttref{v = 0}}$ but $0 \not \in \interp{\sto\ \ttref{v > 0}}$, 
thereby showing that the denotation containment does not hold.


\mypara{Declarative Subtyping with Eager Evaluation}
Since denotational containment (\ref{con:explode}) does not hold,
\hlang cannot verify @explode@ under eager evaluation.
%
However, Belo \etal~\cite{Greenberg11} note that under eager (call-by-value) 
evaluation, each binder in the environment is only added \emph{after} 
the previous binders have been reduced to \emph{values}. 
%
Hence, under eager evaluation we can \emph{restrict the range} of 
the closing substitutions to values (as opposed to expressions).
%
Let us reconsider (\ref{con:explode}) in this new light: 
%
there \emph{is no value} that we can map \ttn to, so the set of
denotations of the environment is empty. Hence, the 
containment~(\ref{con:explode}) vacuously holds under
eager evaluation, which proves the program safe.
%
Belo's observation 
is implicitly used by refinement types for eager languages
to prove that the standard (\ie call-by-value)
reduction from subtyping to VC is sound.

\mypara{Algorithmic Subtyping via Verification Conditions}
The above subtyping (\rtdsub) 
rule allows us to prove preservation and progress~\cite{Knowles10} 
but quantifies over evaluation of arbitrary expressions, 
and so is undecidable.
%
To make checking \emph{algorithmic} we approximate the 
denotational containment using \emph{verification conditions} (VCs), 
formulas drawn from a decidable logic, that are valid 
only if the undecidable containment holds.
%
As we have seen, the classical VC is sound only under 
eager evaluation. Next, let us use the distinctions 
between lazy and eager declarative subtyping, 
to obtain both sound and decidable VCs for the lazy setting. 

\mypara{Step 1: Restricting Refinements To Decidable Logics}
%
Given that \hlang refinements can be \emph{arbitrary} expressions,
the first step towards obtaining a VC, regardless of evaluation order,
is to restrict the refinements to a \emph{decidable} logic. 
%
We choose the quantifier free logic of equality, uninterpreted functions 
and linear arithmetic (\logiclang). 
%
Our typing rules ensure that for any valid derivation, all
refinements belong in this restricted language.

\mypara{Step 2: Translating Containment into VCs}
Our goal is to encode the denotation containment antecedent of \rtdsub
%
\begin{align}\label{denotcont}
\forall \sto\in \interp{\env}.\  
  		 \interp{\thetasub{\sto}{\tttref{\vv}{\Base}{}{\refi_1}}} 
  		\subseteq   \interp{\thetasub{\sto}{\tttref{\vv}{\Base}{}{\refi_2}}}
\end{align}
as a logical formula, that is valid \emph{only when} the above holds.
Intuitively, we can think of the closing substitutions $\sto$ as 
corresponding to \emph{assignments} or \emph{interpretations} 
$\embed{\sto}$ 
of variables $X$ of the VC.
%
We use the variable \x to approximate denotational containment 
by stating that  
if \x belongs to the type $\tttref{\vv}{\Base}{}{\refi_1}$
then \x belongs to the type $\tttref{\vv}{\Base}{}{\refi_2}$:
$$
\forall X \in \mathit{dom}(\Gamma), \x. 
    \embed{\Env}\Rightarrow 
  		 {\embed{\ttbind{\x}{\tttref{\vv}{\Base}{}{\refi_1}}}} 
  		\Rightarrow \embed{\ttbind{\x}{\tttref{\vv}{\Base}{}{\refi_2}}}
$$
where $\embed{\env}$ and $\embed{\ttbind{\x}{\typ}}$ are respectively the translation 
of the environment and bindings into logical formulas 
that are only satisfied by assignments $\embed{\sto}$
as shown in Figure~\ref{fig:overview:syntax}.
%
Using the translation of bindings, 
and by renaming \x to \vv,
we rewrite the condition as
$$
\forall X \in \mathit{dom}(\Gamma), \vv. 
    \embed{\env}
    \Rightarrow
     ({\text{``}\vv\ \text{is a value"}}  \Rightarrow \lref_1)
    \Rightarrow
     ({\text{``}\vv\ \text{is a value"}}  \Rightarrow \lref_2)
$$
Type refinements are carefully chosen to belong to the 
decidable logical sublanguage \logiclang, so we directly translate
type refinements into the logic.
%
Thus, what is left is to translate into logic the environment and the ``is a value'' guards.
%
We postpone translation of the guards as we approximate the above formula by 
a \emph{stronger}, \ie sound with respect to \ref{denotcont},
VC that just omits the guards:
$$
\forall X \in \mathit{dom}(\Gamma), \vv. 
    \embed{\env}
    \Rightarrow
      \lref_1 \Rightarrow \lref_2
$$
%
To translate environments, we conjoin their bindings' translations:
\begin{align*}
%
\embed{\ttbind{\x_1}{\typ_1},\ldots,\ttbind{\x_\n}{\typ_\n}} \defeq & 
\embed{\ttbind{\x_1}{\typ_1}} \wedge \ldots \wedge \embed{\ttbind{\x_\n}{\typ_\n}} \\
%
\intertext{However, since types denote \emph{partial correctness},
   the translations must also explicitly account for possible divergence:}
\embed{\ttbind{\x}{\tlref{\vv}{\tint}{}{\refi}}} \defeq & \ \mbox{``$\x$ is a value"}  \Rightarrow \SUBST{\lref}{\vv}{\x}
\end{align*}
%
That is, we \emph{cannot} assume that each \x satisfies its 
refinement \refi; we must \emph{guard} that assumption with a 
predicate stating that \x is bound to a value (not a diverging term).

The crucial question is: \emph{how} can one discharge these guards
to conclude that \x indeed satisfies \refi?
%
One natural route is to enrich the refinement logic with 
a predicate that states that ``\x is a value", and then 
use the SMT solver to \emph{explicitly} reason about 
this predicate and hence, divergence.
%
Unfortunately, we show in \Sref{sec:refinedhaskell:conclusion}, that 
such predicates lead to three-valued logics,
which fall outside the scope of the efficiently 
decidable theories supported by current solvers.
%
Hence, this route is problematic if we want to use 
existing SMT machinery to build automated verifiers 
for Haskell.

\subsection{Our Answer: Implicit Reasoning About Divergence}

One way forward is to \emph{implicitly} reason about divergence 
by \emph{eliminating} the ``\x is a value" guards (\ie \emph{value guards}) from the VCs.

\mypara{Implicit Reasoning: Eager Evaluation}
Under eager evaluation the domain of the closing substitutions 
can be restricted to values~\cite{Greenberg11}.
%
Thus, we can trivially eliminate the value guards, 
as they are guaranteed to hold by virtue of the 
evaluation order.
%
Returning to @explode@, we see that after eliminating 
the value guards, we get the VC (\ref{vc:explode})
which is, therefore, sound under eager evaluation.

\mypara{Implicit Reasoning: Lazy Evaluation}
However, with lazy evaluation, we cannot just 
eliminate the value guards, as the closing 
substitutions are not restricted to just values.
%
Our solution is to take this reasoning out of the 
hands of the SMT logic and place it in the hands of 
a \emph{stratified type system}.
%
We use a non-deterministic $\beta$-reduction 
(formally defined in \Sref{sec:language})
to label each type as:
%
A \Div-type, written $\DivTy{\typ}$,
which are the default types given to binders 
that \emph{may diverge}, or,
%
a \Wnf-type, written $\WnfTy{\typ}$, 
which are given to binders that are guaranteed to 
reduce, in a finite number of steps, to \emph{Haskell values}
in Weak Head Normal Form (WHNF).
%
Up to now we only discussed \tint basic types, but 
our theory supports user-defined algebraic data types.
%
An expression like @0 : repeat 0@ is an infinite Haskell value.
%
As we shall discuss, such infinite values cannot be represented in the logic.
%
To distinguish infinite from finite values, we use a \Fin-type, written $\FinTy{\typ}$,
to label binders of expressions that are guaranteed to reduce to \emph{finite values} with no redexes.
%%%
This stratification lets us generate VCs that are sound 
for lazy evaluation. 
%
Let \Base be a basic labelled type. 
%
The key piece is the translation of environment bindings:
$$
\embed{\ttbind{\x}{\tttref{\vv}{\Base}{}{\refi}}} \defeq 
  \begin{cases}
    \tttrue,         & \mbox{if $\Base$ is a \Div type} \\ 
    \SUBST{\lref}{\vv}{\x}, & \mbox{otherwise}
  \end{cases}
$$
That is, if the binder may diverge, we simply \emph{omit} 
any constraints for it in the VC, and otherwise the translation 
directly states (\ie without the value guard) that the 
refinement holds.
%
Returning to @explode@, the subtyping query (\ref{sub:explode}) 
yields the \emph{invalid} VC
$$
\tttrue \Rightarrow \ttv = 0 \Rightarrow \ttv > 0
$$
and so @explode@ is soundly rejected under lazy evaluation.

As binders appear in refinements and binders may refer 
to potentially infinite computations (\eg @[0..]@), we must 
ensure that refinements are well defined (\ie do not diverge).
%
We achieve this via stratification itself, \ie by ensuring that 
all refinements have type $\FinTy{\tbool}$. By
Corollary~\ref{cor:stratification}, this suffices to ensure that 
all the refinements are indeed well-defined and converge.



\subsection{Verification With Stratified Types}\label{sec:overview:examples}
While it is reassuring that the lazy VC soundly \emph{rejects} 
unsafe programs like @explode@, we demonstrate by example
that it usefully \emph{accepts} safe programs.
% 
First, we show how the basic system -- all terms
have \Div types -- allows us to prove ``partial correctness''
properties without requiring termination.
%
Second, we extend the basic system by 
using Haskell's pattern matching semantics to assign 
the pattern match scrutinees \Wnf types, thereby 
increasing the expressiveness of the verifier.
%
Third, we further improve the precision 
and usability of the system by using a termination 
checker to assign various terms \Fin types.
%
Fourth, we close the loop, by illustrating how the 
termination checker can itself be realized using 
refinement types.
%
Finally, we use the termination checker to ensure that
all refinements are well-defined (\ie do converge).

\mypara{Example: VCs and Partial Correctness}
The first example illustrates how, unlike Curry-Howard based
systems, refinement types \emph{do not require} termination. 
%
That is, we retain the Floyd-Hoare notion of ``partial correctness''
and can verify programs where \emph{all} terms have \Div-types.
%
Consider @ex1@ which uses the result of @collatz@ as a divisor.
%
\begin{code}
  ex1   :: Int -> Int 
  ex1 n = let x = collatz n in 10 `div` x 

  collatz :: Int -> {v:Int | v = 1}
  collatz n 
    | n == 1    = 1 
    | even n    = collatz (n / 2)
    | otherwise = collatz (3*n + 1)
\end{code}
%
The jury is still out on \emph{whether} the @collatz@ 
function terminates\footnote{
Collatz Conjecture: \url{http://en.wikipedia.org/wiki/Collatz\_conjecture}}, 
but it is easy
to verify that its output is a \Div @Int@ equal to @1@.
%
At the call to @div@ the parameter @x@ has the output type 
of @collatz@, yielding the subtyping query:
%
\begin{align*}
   \tbind{\ttx}{\tref{\ttv}{\ttInt}{}{\ttv=1}} \vdash\ & \subtref{v = 1}{v > 0} 
\end{align*}
% 
where the sub-type is just the type of $\ttx$. 
As $\ttInt$ is a \Div type, the above reduces to the VC 
${(\etrue \Rightarrow\  \ttv = 1 \Rightarrow\ \ttv > 0)}$
which the SMT solver proves valid, verifying @ex1@.

\mypara{Example: Improving Precision By Forcing Evaluation}
\label{sec:overview:pattern-match}
If all binders in the environment have \Div-types then, effectively, 
the verifier can make \emph{no} assumptions about the context in 
which a term evaluates, which leads to a drastic loss of precision. 
Consider:
\begin{code}
  ex2 = let {x = 1; y = inc x} in 10 `div` y

  inc :: z:Int -> {v:Int | v > z }
  inc = \z -> z + 1
\end{code}
%
The call to @div@ in @ex2@ is obviously safe, but the system 
would reject it, as the call yields the subtyping query:
%
$$
   \tbind{\ttx}{\tref{\ttx}{\ttInt}{}{\ttx = 1}},\
   \tbind{\tty}{\tref{\tty}{\ttInt}{}{\tty > \ttx}}\ 
   \vdash\ \subtref{v > \ttx}{v > 0} 
$$
Which, as $\ttx$ is a \Div type, reduces to the invalid VC:
$$
   \tttrue \Rightarrow\ \ttv > \ttx \Rightarrow \ttv > 0
$$
%
We could solve the problem by forcing evaluation of @x@.
%
In Haskell the @seq@ operator or a bang-pattern can be used to force evaluation.
%
In our system the same effect is achieved by the @case-of@ primitive:
inside each case the matched 
binder is guaranteed to be a Haskell value in WHNF.  
%
This intuition is formalized by the typing rule (\rtcased), which checks each 
case after assuming the scrutinee and the match binder have \Wnf types. 
%

If we force @x@'s evaluation, using the case primitive, 
the call to @div@ yields the subtyping query:
\begin{align}
   \ttbind{\ttx}{\ttreft{\ttx}{\WnfTy{\ttInt}}{\ttx = 1}},\
   \ttbind{\tty}{\ttreft{\tty}{\ttInt}{\tty > \ttx}}      	
	& \vdash\ \subtref{v > \ttx}{v > 0}\label{sub:pm:ex3:good}
\intertext{As $\ttx$ is \Wnf, we accept \texttt{ex2} by proving the validity of the VC:} 
   \ttx = 1 \Rightarrow\  \ttv > \ttx & \Rightarrow \ttv > 0
\label{vc:pm:ex3:good}
\end{align}

\mypara{Example: Improving Precision By Termination}
While forcing evaluation allows us to ensure that certain 
environment binders have non-\Div types, 
it requires program rewriting using case-splitting 
or the @seq@ operator which leads to non-idiomatic code.

Instead, our next key optimization is based on 
the observation that in practice, \emph{most terms don't diverge}.
%
Thus, we can use a termination analysis to 
aggressively assign terminating expressions 
\Fin types, which lets us strengthen the 
environment assumptions needed to prove 
the VCs.
%
For example, in the @ex2@ example the term @1@ obviously terminates.
Hence, we type @x@ as $\FinTy{\ttInt}$, yielding
the subtyping query for @div@ application:
%
\begin{align}
   \ttbind{\ttx}{\ttreft{\ttx}{\FinTy{\ttInt}}{\ttx = 1}},\
   \ttbind{\tty}{\ttreft{\tty}{\ttInt}{\tty > \ttx}}      	
	& \vdash\ \subtref{v > \ttx}{v > 0}\label{sub:ex3:good}
\intertext{As $\ttx$ is \Fin, we accept \texttt{ex2} by proving the validity of the VC:} 
   \ttx = 1 \Rightarrow\ \ttv > \ttx & \Rightarrow \ttv > 0
\label{vc:ex3:good}
\end{align}

\mypara{Example: Verifying Termination With Refinements}
%
While it is straightforward to conclude that the term @1@ 
does not diverge, how do we do so in general?
%
For example:
%
\begin{code}
  ex4 = let {x = f 9; y = inc x} in 10 `div` y
  
  f   :: Nat -> {v:Int | v = 1}
  f n = if n == 0 then 1 else f (n-1)
\end{code}
%
We check the call to @div@ via subtyping query (\ref{sub:ex3:good}) and 
VC (\ref{vc:ex3:good}), which requires us to prove that @f@ terminates on
\emph{all} $\FinTy{\mathtt{Nat}}$ inputs.

We solve this problem by showing how
refinement types may themselves be 
used to prove termination, by following
the classical recipe of proving termination 
via decreasing metrics~\cite{Turing36}
as embodied in sized types~\cite{HughesParetoSabry96,XiTerminationLICS01}.
%
The key idea is to show that each 
recursive call is made with arguments 
of a \emph{strictly smaller} size, 
where the size is itself a well 
founded metric, \eg a natural number.

We formalize this intuition by type checking
recursive procedures in a {termination-weakened environment}
where the procedure itself may only be called with arguments 
that are strictly smaller than the current parameter 
(using terminating fixpoints of \Sref{sec:typing:stratify}).
%
For example, to prove @f@ terminates, we check its body in an environment 
%
$${\ttn} : {\FinTy{\ttNat}}, \qquad {\ttf} : {\ttreft{\ttnp}{\FinTy{\ttNat}}{\ttnp < \ttn} \rightarrow \ttref{v=1}}$$
%
where we have weakened the type of $\ttf$ to stipulate that it 
\emph{only} be (recursively) called with $\ttNat$ values $\ttnp$ that are 
\emph{strictly less than} the (current) parameter $\ttn$. 
The argument of \ttf exactly captures these constraints, 
as using the Abbreviations of Figure~\ref{fig:overview:syntax}
the argument of \ttf is expanded to 
$\ttreft{\ttnp}{\FinTy{\tint}}{\ttnp < \ttn \land \ttnp >= 0}$.
The body
type-checks as the recursive call generates the valid VC:
%
$$0 \leq \ttn \wedge \lnot (0 = \ttn)  \Rightarrow \ttv = \ttn - 1 \Rightarrow (0 \leq \ttv < \ttn)$$

\mypara{Example: Diverging Refinements}
Finally, we discuss why refinements should always converge
and how we statically ensure convergence.
%
Consider the invalid specification
\begin{code}
  diverge 0 :: {v:Int | v = 12}
\end{code}
that states that the value of a diverging integer is @12@.
%
The above specification should be rejected, 
as the refinement $\mathtt{v = 12}$ does not evaluate to \etrue 
(\goesto{\mathtt{diverge\ 0 = 12} \not}{\etrue}),
instead it diverges.

We want to check the validity of the formula $\mathtt{v = 12}$
under a model that maps \ttv to the diverging integer $\mathtt{diverge \ 0}$.
%
Any system that decides this formula to be true
will be unsound, \ie the VCs will not soundly approximate subtyping.
%
For similar reasons, the system should not decide that this formula is false.
%
To reason about diverging refinements one needs three valued logic, 
where logical formulas can be solved to true, false, or diverging.
%
Since we want to discharge VC using SMT solvers that currently do not support 
three valued reasoning, we exclude diverging refinements from types.
%
To do so, we restrict $\mathtt{=}$ to finite integers
$$ (\mathtt{=}) :: {\tint^{\finite}}\rightarrow{\tint^\finite}\rightarrow{\tbool^\finite}$$
and we say that \ttreft{\vv}{\Base}{\refi} 
is well-formed \emph{iff} \refi has a $\tbool^\finite$ type (Corollary~\ref{cor:stratification}).
%
Thus the initial invalid specification will be rejected as non well-formed.
%%\DV{This paragraph is confusing. I thought
%%$len :: [a]^\Downarrow -> Int^\Downarrow$ no? But then we are not able to apply len to anything else? Is len (repeat 35) ill typed? This does not look good. 
%%Or is this typing only affecting refinements and not actual programs?}
%%\NV{
%%It can be either 
%%$len :: [a]^\Downarrow -> Int^\Downarrow$
%%or 
%%$len :: [a]^\Downarrow -> Int$.
%%Either ways len (repeat 35) should be ill typed, when len is required to terminate, 
%%for example when the result of len is used inside the refinements.
%%Though, in actual programs we can have diverging expressions, so 
%%length (repeat 35) can be well-typed, with 
%%$length :: [a] -> Int$.
%%}

%
%%% In \Sref{sec:typing} we formalize our system for proving termination 
%%% with refinements, and then show, in \S~\ref{sec:haskell}, how to generalize
%%% it to handle complex data types.
%%% The proof of the pudding is in \S~\ref{sec:evaluation} where we demonstrate 
%%% empirically the effectiveness of \toolname's approach of proving termination
%%% \emph{for} and \emph{by} refinements, on a large corpus of 10KLOC of widely 
%%% used, complex, real-world Haskell libraries.


%% JUNK
%%         take 0 _  = []
%%         take n xs = case xs of
%%                       (x:xs') -> x : take (n-1) xs'
%%                       []      -> liquidError 
%%
%%
%%
%%         findPos (x:xs) 
%%           | x > 0       = Just x 
%%           | otherwise   = find p xs
%%         findPos []      = Nothing
%% 
%%         ex0 ys = 24 `div` (findPos ys)
%% 
%%         x:Nat -> {y | x < y} -> Int
%%         ex1 = 24 `div` y

%% \subsection{An Optimization: Termination}
%%   x:Nat -> {y | x < y} -> Int
%%   ex1 = 24 `div` y
%%
%% fail      :: n:Nat -> {m:Nat | n < m} -> Int
%% fail n m  = 10000 `div` m
%% 
%% fail'     :: n:Nat -> {m:Nat | n < m} -> Int
%% fail' n m = case n of _ -> 10000 `div` m
%% 
%% bar     :: Int -> Int -> Int
%% bar     = foo x y 
%%   where 
%%     x   = fac 10
%%     y   = x + 1


%%% THE FOLLOWING FAILS IN APROVE BUT COMPLETES AUTOMATICALLY IN LH
%%% 
%%% foo   :: Int -> Int
%%% foo 0 = 0
%%% foo n = 0 + foo (n-1)
%%% 
%%% top   :: Int -> Int
%%% top n = if n < 0 then 1 else foo (foo n)


%%% Local Variables: 
%%% mode: latex
%%% TeX-master: "main"
%%% End: 

\newcommand\hnull{\ensuremath{\text{[]}}\xspace}

\subsection{Measures: From Integers to Data Types}\label{sec:measures}

\begin{figure}
\centering
\captionsetup{justification=centering}
$$
\begin{array}{lrcl}
{\emphbf{Definition}} &
  \mathit{def} & ::=  &  \mathtt{measure} \ f :: \tau \\
              & &      &  \quad eq_1 \ldots eq_n       \\[0.05in]

{\emphbf{Equation}}   & 
  \mathit{eq}  & ::=  &   f\ (D\ \overline{x}) = r    \\[0.15in] 

{\emphbf{Equation to Type}} &
\quad \embed{f\ (D\ \overline{x}) = r} & \defeq & D :: \overline{\tbind{x}{\tau}} \rightarrow \tref{\mathtt{v}}{\tau}{}{f\ \mathtt{v} = r}
\end{array}
$$
\caption{Syntax of Measures.}
\label{fig:measures}
\end{figure}



So far, all our examples have used only integer and boolean expressions in refinements.
To describe properties of algebraic data types, we use \emph{measures},
introduced in prior work on Liquid Types~\cite{LiquidPLDI09}.
%
Measures are inductively defined functions that can be used in refinements and
provide an efficient way to axiomatize properties of data types.
%
For example, @emp@ determines whether a list is empty:
%
\begin{code}
  measure emp  :: [Int] -> Bool
    emp []     = true
    emp (x:xs) = false
\end{code}
The syntax for measures deliberately looks like Haskell, but it is \emph{far} more
restricted, and should really be considered as a separate language.
A measure has exactly one argument and is defined by a list of equations,
each of which has a simple pattern on the left hand side (Figure~\ref{fig:measures}).
The right-hand side of the equation is a refinement expression $r$.
Measure definitions are typechecked in the usual way; we omit the typing rules which are standard.
(Our metatheory does not support type polymorphism,
so here we simply reason about lists of integers;
however, our implementation supports polymorphism).

\paragraph{Denotational semantics}
The denotational semantics of types in \hlang in \Sref{sec:den-sem} is readily extended to
support measures.  In \hlang a refinement $r$ is an arbitrary expression and
calls to a measure are evaluated in the usual way by pattern matching.
For example, with the above definition of @emp@ it is straightforward to show that
\begin{align}
  \mathtt{[1, 2, 3]} \dcolon \tref{\mathtt{v}}{[\tint]}{}{\mathtt{not}\ (\mathtt{emp}\ \mathtt{v})} \label{type:len}
\end{align}
as the refinement @not (emp ([1, 2, 3]))@ evaluates to $\tttrue$.

\mypara{Measures as Axioms}
How can we reason about invocations of measures in the decidable logic of VCs?
A natural approach is to treat a measure like @emp@ as an uninterpreted function
and add logical axioms that capture its behaviour. This looks easy: each equation 
of the measure definition corresponds to an axiom, thus:
%
\begin{align*}
\ttemp\ \hnull &= \tttrue\\
\forall \ttx, \ttxs.\, \ttemp\ (\ttx:\ttxs) &= \ttfalse
\end{align*}
%
Under these axioms the judgement~\ref{type:len} is indeed valid. 
% % Measures as data constructor refinements

\mypara{Measures as Refinements in Types of Data Constructors}
Axiomatizing measures is \emph{precise}; that is, 
the axioms exactly capture the meaning of measures.
Alas, axioms render SMT solvers \emph{inefficient}, and render the VC mechanism \emph{unpredictable}, 
as one must rely on various brittle syntactic matching and instantiation heuristics~\cite{simplifyj}.

Instead, we use a different approach that is \emph{both} precise \emph{and} efficient.
The key idea is this: \emph{instead of translating each measure equation into an axiom, 
we translate each equation into a refined type for the corresponding data constructor}~\citep{LiquidPLDI09}.
This translation is given in Figure~\ref{fig:measures}.
For example, the definition of the measure @emp@ yields the following refined types for the list data constructors:
$$
\begin{array}{lcl}
\hnull  & :: & \ttreft{v}{[\tint]}{emp\ v = true}\\
{:}  & :: & \tfun{\ttx}{\tint}{\tfun{\ttxs}{[\tint]}{\ttreft{v}{[\tint]}{emp\ v = false}}}
\end{array}
$$
These types ensure that:
%
~(1) each time a list value is \emph{constructed}, 
its type carries the appropriate emptiness information. 
Thus our system is able to statically decide that 
(\ref{type:len}) is valid and
~(2) each time a list value is \emph{matched}, 
the appropriate emptiness information is used to 
improve precision of pattern matching, as we see next.

\mypara{Using Measures}
\label{sec:pattern-match}
As an example, we use the measure @emp@ to 
provide an appropriate type for the @head@ function:
%
\begin{code}
  head    :: {v:[Int] | not (emp v)} -> Int 
  head xs = case xs of
              (x:_) -> x
              []    -> error "yikes"  

  error   :: {v:String | false} -> a
  error   = undefined
\end{code}
%
@head@ is safe as its input type stipulates that it will only 
be called with lists that are \emph{not} @[]@, and so
@error "..."@ is dead code.
%
The call to @error@ generates the subtyping query
%
\begin{align*}
   \tbind{\ttxs}{\tref{\ttxs}{[\tint]}{\trivial}{\lnot (\ttemp\ \ttxs)}}, \
   \tbind{\ttb}{\tttref{\ttb}{[\tint]}{\trivial}{(\ttemp\ \ttxs)= true}} 	
	 & \vdash \subtref{\tttrue}{\ttfalse} 
\end{align*}
%
The match-binder $\ttb$ holds the result of the 
match~\cite{SulzmannCJD07}. In the \texttt{[]} case,
we assign it the refinement of the type of \texttt{[]} 
which is $(\ttemp\ \ttxs) = \tttrue$. %~\cite{LiquidPLDI09}.
%
Since the call is done inside a @case-of@ expressions 
both @xs@ and @b@ are in WHNF,
thus they have \Wnf types. 
  
The verifier \emph{accepts} the program as the above subtyping reduces to the valid VC:
\begin{align*}
\lnot (\ttemp\ \ttxs) \wedge ((\ttemp\ \ttxs)= \tttrue) \Rightarrow\ & \tttrue \Rightarrow\ \ttfalse
\end{align*}
%
Thus, our system supports idiomatic 
Haskell, \eg taking the @head@ of an infinite list:
%
\begin{code}
  ex x     = head (repeat x)
  
  repeat   :: Int -> {v:[Int] | not (emp v)}
  repeat y = y : repeat y
\end{code}
%

\mypara{Multiple Measures}
If a type has multiple measures, we simply refine each data constructor's type
with the \emph{conjunction} of the refinements from each measure.
%
For example, consider a measure that computes the length of a list:
\begin{code}
  measure len  :: [Int] -> Int
    len ([])   = 0
    len (x:xs) = 1 + len xs
\end{code}
%
Using the translation of Figure~\ref{fig:measures},
we get the following types for list's data constructors.
%
\begin{align*}
\text{[]}  & ::  \ttreft{v}{[\tint]}{len\ v = 0}\\
{:}  & ::  \tfun{\ttx}{\tint}{\tfun{\ttxs}{[\tint]}{\ttreft{v}{[\tint]}{len\ v = 1 + (len\ xs)}}}\\
\intertext{The final types for list data are the 
conjunction of the refinements from $\mathtt{len}$ and $\mathtt{emp}$:}\\
\text{[]}  & ::  \ttreft{v}{[\tint]}{emp\ v = true \land len\ v = 0}\\
{:}  & ::  \tfun{\ttx}{\tint}{\tfun{\ttxs}{[\tint]}
           {\ttreft{v}{[\tint]}{emp\ v = false \land len\ v = 1 + (len\ xs)}}}
\end{align*}



\input{text/refinedhaskell/language}
\input{text/refinedhaskell/typing}
\input{text/refinedhaskell/haskell}
\input{text/refinedhaskell/evaluation}
\input{text/refinedhaskell/conclusions}

\mypara{Acknowledgments}
The material of this chapter are adapted from the following publication:
\noindent N. Vazou, E. Seidel, R. Jhala, D. Vytiniotis, and S. Peyton-Jones,
``Refinement Types for Haskell'', 
ICFP, 2014.
