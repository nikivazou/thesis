\section{Declarative Typing: \undeclang}\label{sec:language}\label{sec:undec}

Next, we formalize our stratified refinement type system, in two steps.
%
First, in this section, we present a core calculus \undeclang, 
with a general $\beta$-reduction semantics. We describe the syntax,
operational semantics, and sound but undecidable declarative typing 
rules for \undeclang. 
%
Second, in \Sref{sec:typing}, we describe \logiclang, a subset 
of \undeclang that forms a decidable logic of refinements, and 
use it to obtain \declang with decidable SMT-based algorithmic typing.

\subsection{Syntax}\label{sec:undec:syntax} 

\begin{figure}[!t]
\centering
\captionsetup{justification=centering}
$$
\begin{array}{rrcl}

\emphbf{Constants} \quad 
  & c & ::=    & 0,1,-1,\ldots \spmid \etrue, \efalse \\
  &   & \spmid & +,-,\ldots \spmid =, <, \ldots \spmid \ecrash 
  \\[0.05in]

\emphbf{Values} \quad 
  & \val & ::= &  c \spmid \efun{x}{\typ}{e} \spmid \edapp{D}{e}
  \\[0.05in] 

\emphbf{Expressions} \quad 
  & e & ::=    & \val \spmid x \spmid \eapp{e}{e} \spmid \elet{x}{e}{e} \\ 
  &   & \spmid & \ecase{e}{D}{\overline{x}}{e}{x} \\[0.05in] 

\emphbf{Refinements} \quad 
  & r & ::= &   e \\[0.05in] 

\emphbf{Basic Types} \quad 
  & \Base & ::= & \tint \spmid \tbool \spmid \ttct \\[0.05in] 

\emphbf{Types} \quad 
  & \typ & ::= & \tref{v}{\Base}{}{r} \spmid \tfunref{x}{\typ}{\typ}{v}{e} \\[0.1in]
% \hrule width 0.48\textwidth
\emphbf{Contexts} \quad 
  & C
  & ::= 
  &   	 \bullet 
  \spmid \eapp{C}{e} 
  \spmid \eapp{c}{C} 
  \spmid D\ \overline{e}\ C\ \overline{e}\\
  &&\spmid &
  \ecase{C}{D}{\overline{y}}{e}{x}
  \\[0.05in] 
\end{array}
$$

\judgementHead{Reduction}{\eval{e}{e}}

$$
\begin{array}{rcl}
\eval{C[e]&}{&C[e']} \qquad \text{if}\ \eval{e}{e'} \\
	\eval{\eapp{c}{v}&}{& \ceval{c}{v}}\\
\eval{\eapp{(\efun{x}{\tau_x}{e})}{e_x}&}{&e\sub{x}{e_x}}\\
	\eval{\elet{x}{e_x}{e}&}{&e\sub{x}{e_x}} \\
	\eval{\ecase{D_j\ \overline{e}}{D_i}{\overline{y_i}}{e_i}{x}&}
	{&e_j\sub{x}{D_j\ \overline{e}}\sub{\overline{y_j}}{\overline{e}}} \\
\end{array}
$$

\caption{Syntax and Operational Semantics of $\protect \undeclang$.}
\label{fig:undeclang}
\label{fig:operational}
\end{figure}


Figure~\ref{fig:undeclang} summarizes the syntax of \undeclang, 
which is essentially the calculus \hlang~\cite{Knowles10} 
\emph{without} the dynamic checking features (like casts), but 
\emph{with} the addition of data constructors.
%
In \undeclang, as in \hlang, refinement expressions $r$ are not drawn from a decidable 
logical sublanguage, but can be arbitrary expressions $e$
(hence $r ::= e$ in Figure~\ref{fig:undeclang}). 
This choice allows us to prove preservation and progress, 
but renders typechecking undecidable. 
 
%The syntactic elements of \undeclang are layered into 
%primitive constants, values, and expressions.

\spara{Constants}
The primitive constants of \undeclang include  
$\tttrue$, $\ttfalse$, $\mathtt{0}$, $\mathtt{1}$, $\mathtt{-1}$, \etc,
and arithmetic and logical operators like $\mathtt{+}$, $\mathtt{-}$, 
$\mathtt{\leq}$,$\mathtt{/}$, $\mathtt{\land}$, $\mathtt{\lnot}$.
%
In addition, we include a special \emph{untypable} constant $\ecrash$ 
that models ``going wrong''. Primitive operations return a $\ecrash$
when invoked with inputs outside their domain, \eg when $\mathtt{/}$ 
is invoked with $\mathtt{0}$ as the divisor, or when $\mathtt{assert}$ is 
applied to $\mathtt{false}$.

\spara{Data Constructors}
We encode data constructors as special constants. 
Each data type has an arity $\arity{T}$ that represents
the exact number of data constructors that return a value of 
type $T$.
%
For example the data type \tintlist, which represents 
lists of integers, has two data constructors: $\dnull$ and $\dcons$,
\ie has arity $2$.
%%$D^\tintlist_1 \defeq \dnull$ and
%% $D^\tintlist_2 \defeq \dcons$.


\spara{Values \& Expressions}
The values of \undeclang include constants, 
$\lambda$-abstractions $\efun{x}{\typ}{e}$, and 
fully applied data constructors $D$ that wrap expressions.
%
The expressions of \undeclang include values, as well as
variables $x$, 
applications $\eapp{e}{e}$, 
and the $\mathtt{case}$ 
and $\mathtt{let}$ expressions.

\subsection{Operational Semantics}

Figure~\ref{fig:operational} summarizes the small 
step contextual $\beta$-reduction semantics for 
\undeclang.
%
Note that we allow for reductions under data constructors, 
and thus, values may be further reduced.
%
We write \evalj{e}{e'}{j} if there exist $e_1,\ldots,e_j$ such that
$e$ is $e_1$, $e'$ is $e_j$ and $\forall i,j, 1 \leq i < j$, we have
\eval{e_i}{e_{i+1}}.
%
We write \evals{e}{e'} if there exists some (finite) $j$ such that
$\evalj{e}{e'}{j}$.

\spara{Constants} Application of a constant requires the
argument be reduced to a value; in a single step the 
expression is reduced to the output of the primitive 
constant operation. 
%
For example, consider $=$, the primitive equality operator 
on integers. We have $\ceval{=}{n} \defeq =_n$
where $\ceval{=_n}{m}$ equals \etrue iff $m$ is the same as $n$.
%%as follows:
%%%
%%$$
%%\ceval{=}{n} \defeq =_n \qquad \ceval{=_n}{m} \defeq \begin{cases} \etrue, & \mbox{if}\ m = n \\
%%                                                                     \efalse, & \mbox{otherwise}
%%                                                       \end{cases}
%%$$

%% \begin{align*}
%% \interp{=}(n) \defeq & =_n \\
%% \interp{=_n}(m) \defeq & \begin{cases} \etrue, & \mbox{if}\ m = n \\
%%                                        \efalse, & \mbox{otherwise}
%%                          \end{cases}
%% \end{align*}
%%
%%\spara{Data Constructors} \RJ{This para is cancelled with $\beta$-reduction, right?}
%%%
%%There is no rule to evaluate application of data constructors.
%%Like Haskell, our semantics do not force evaluation of expressions 
%%wrapped in data constructors, thus, $D \ e_1\ \dots \ e_n$ is a 
%%value, without requiring $e_i$s to be values.


\subsection{Types}

\undeclang types include basic types, which are \emph{refined} with predicates, 
and dependent function types.
%
\emph{Basic types} $\tbase$ comprise integers, booleans, and a family of data-types 
$T$ (representing lists, trees \etc.)
%
For example the data type \tintlist represents lists of integers.
%
We refine basic types with predicates (boolean valued expressions $e$) to obtain
\emph{basic refinement types} $\tref{v}{\tbase}{}{e}$.
%
Finally, we have dependent \emph{function types} $\tfun{x}{\typ_x}{\typ}$ 
where the input $x$ has the type $\typ_x$ and the output $\typ$ may
refer to the input binder $x$.

\spara{Notation} We write $\tbase$ to abbreviate $\tref{v}{\tbase}{}{\etrue}$, 
and \tfunbasic{\typ_x}{\typ} to abbreviate \tfun{x}{\typ_x}{\typ} if 
$x$ does not appear in $\typ$. 
% We use $p$, $q$, and $r$ for 
% refinements, and 
We use $\_$ for unused binders.
We write $\tref{v}{\tnat}{l}{r}$ to abbreviate $\tref{v}{\tint}{l}{0 \leq v \wedge r}$.


\spara{Denotations}
%
Each type $\typ$ \emph{denotes} a set of expressions $\interp{\typ}$,
that are defined via the dynamic semantics~\cite{Knowles10}.
%
Let \erase{\typ} be the type we get if we erase all refinements 
from $\typ$ and $\hastypebasesmall{\Env}{e}{\erase{\typ}}$ be the 
standard typing relation for the typed lambda calculus.
%
Then, we define the denotation of types as: 
\begin{align*}
\interp{\tref{x}{\tbase}{}{r}} \defeq & 
    \{e \mid  \hastypebasesmall{\emptyset}{e}{\tbase},
              \mbox{ if } \evals{e}{w} 
              \mbox{ then } \evals{\SUBST{r}{x}{w}}{\etrue} \}\\
\interp{\tfun{x}{\typ_x}{\typ}} \defeq & 
    \{e \mid  \hastypebasesmall{\emptyset}{e}{\erase{\tfunbasic{\typ_x}{\typ}}}, 
              \forall e_x \in \interp{\typ_x}.\ \eapp{e}{e_x} \in \interp{\typ\sub{x}{e_x}}
    \}
\end{align*}

%% The meaning of the refined type $\tref{v}{b}{}{e_r}$
%% is all the closed expressions $e$ of basic type $b$ for 
%% which $e_r$ holds, 
%% in the sense that if \evals{e}{v} for some value $v$, 
%% then $e_r\sub{v}{e}$ evaluates to \etrue.
%% %
%% A dependent function type $\tfunref{x}{\typ_x}{\typ}{}{}$ 
%% is interpreted as the set of closed expressions of
%% simple type \tfunbasic{\typ_x}{\typ}. 
%% that give output in $\interp{\typ\sub{x}{e_x}}$ 
%% %whenever their input $e_x$ is in \interp{\typ_x}

\spara{Constants}
For each constant $c$ we define its type \constty{c}
such that $c \in \interp{\constty{c}}$. 
%
For example,
%Each constant $c$ we define is in the denotation of its type \constty{c}:
%
%Each constant $c$ is has type \constty{c} such that
%$c \in \interp{\constty{c}}$. 
%
%% For example,
%
$$
\begin{array}{lcl}
\constty{3} &\doteq& \tttref{v}{\tint}{}{v = 3}\\
\constty{+} &\doteq& \tfun{\ttx}{\tint}{\tfun{\tty}{\tint}{\tttref{v}{\tint}{}{v = x + y}}}\\
\constty{/} &\doteq& \tfunbasic{\tint}{\tfunbasic{\tttref{v}{\tint}{}{v > 0}}{\tint}}\\
\constty{\eerror{\typ}} &\doteq& \tfunbasic{\tttref{v}{\tint}{}{\efalse}}{\typ}
\end{array}
$$
%
So, by definition we get the constant typing lemma
%
\begin{lemma}{[Constant Typing]}\label{lemma:constants}
Every constant $c \in \interp{\constty{c}}$.
\end{lemma}
%
Thus, if $\constty{c} \defeq \tfun{x}{\typ_x}{\typ}$, then for every value 
$w \in \interp{\typ_x}$, we require that $\ceval{c}{w} \in \interp{\typ\sub{x}{w}}$.
%
For every value $w \not \in \interp{\typ_x}$, it suffices to define $\ceval{c}{w}$
as \ecrash, a special untyped value.

\spara{Data Constructors}
%As discussed in ~\Sref{sec:measures}, 
The types of data constructor constants are refined 
with predicates that track the semantics of the 
\emph{measures} associated with the data type.
%
For example, as discussed in \Sref{sec:measures} 
we use @emp@ to refine the list data constructors' types:
$$
\begin{array}{lcl}
\constty{\dnull}  & \defeq & \tttref{v}{\tintlist}{}{\eisNull{v}}\\
\constty{\dcons}  & \defeq & \tfunbasic{\tint}{\tfunbasic{\tintlist}{\tttref{v}{\tintlist}{}{\lnot (\eisNull{v})}}}
\end{array}
$$
%
By construction it is easy to prove that Lemma~\ref{lemma:constants}
holds for data constructors.
%
For example, $\ttemp\ \dnull$ goes to $\tttrue$.
%%We \emph{compose} multiple measures for a type by 
%%refining the constructors with the \emph{conjunction} 
%%of each measure's refinements.
%


\subsection{Type Checking}\label{subsec:typing}

\newcommand\restrictdecidable[2]{#2}
\begin{figure}[t!]
\centering
\captionsetup{justification=centering}

\judgementHead{Well-Formedness}{\undeciswellformed{\Gamma}{\tau}}
$$
\inference
   {\undechastype{\Gamma, \tbind{v}{\Base}}
                 {\restrictdecidable{p}{r}}{\tbool}
   }
   {\undeciswellformed{\Gamma}{\tref{v}{\Base}{}{\restrictdecidable{p}{r}}}}
   [\rwbase]
\qquad
\inference{
	\undeciswellformed{\Gamma}{\tau_x} &&
	\undeciswellformed{\Gamma, \tbind{x}{\tau_x}}{\tau}
}{
	\undeciswellformed{\Gamma}{\tfunref{x}{\tau_x}{\tau}{v}{e}}
}[\rwfun]
$$

\judgementHead{Subtyping}{\undecissubtype{\Gamma}{\tau_1}{\tau_2}}

$$
\inference{
  {\forall \sto\in \interp{\Env}. 
  		 \interp{\thetasub{\sto}{\tref{v}{\Base}{}{\restrictdecidable{p_1}{r_1}}}} 
  		\subseteq   \interp{\thetasub{\sto}{\tref{v}{\Base}{}{\restrictdecidable{p_2}{r_2}}}}}
}{
	\undecissubtype{\Gamma}
		{\tref{v}{\Base}{}{\restrictdecidable{p_1}{r_1}}}
		{\tref{v}{\Base}{}{\restrictdecidable{p_2}{r_2}}}
}[\rsubbase]
$$
$$
\inference{
	\undecissubtype{\Gamma}{\tau'_x}{\tau_x} &&
	\undecissubtype{\Gamma, \tbind{x}{\tau'_x}}{\tau}{\tau'}
}{
	\undecissubtype{\Gamma}{\tfunref{x}{\tau_x}{\tau}{v}{e_1}}{\tfunref{x}{\tau'_x}{\tau'}{v}{e_2}}
}[\rsubfun]
$$

\judgementHead{Typing}{\undechastype{\Gamma}{e}{\tau}}
$$
\inference{
	(x,\tau) \in \Gamma 
}{
	\undechastype{\Gamma}{x}{\tau}
}[\rtvar]
\qquad
\inference{
}{
	\undechastype{\Gamma}{c}{\constty{c}}
}[\rtconst]
$$
$$
\inference{
	\undechastype{\Gamma}{e}{\tau'} &&
	\undecissubtype{\Gamma}{\tau'}{\tau} &&
	\undeciswellformed{\Gamma}{\tau} &&
}{
	\undechastype{\Gamma}{e}{\tau}
}[\rtsub]
$$
$$
\inference{
	\undechastype{\Gamma, \tbind{x}{\tau_x}}{e}{\tau} &&
	\undeciswellformed{\Gamma}{\tau_x}
}{
	\undechastype{\Gamma}{\efun{x}{\tau_x}{e}}{(\tfun{x}{\tau_x}{\tau})}
}[\rtfun]
$$
$$
\inference{
	\undechastype{\Gamma}{e_1}{(\tfunref{x}{\tau_{x}}{\tau}{v}{e_v})} &&
	\undechastype{\Gamma}{\restrictdecidable{y}{e_2}}{\tau_{x}}
}{
	\undechastype{\Gamma}{\eapp{e_1}{\restrictdecidable{y}{e_2}}}{\tau\sub{x}{\restrictdecidable{y}{e_2}}}
}[\rtapp]
$$
$$
\inference{
	\undechastype{\Gamma}{e_x}{\tau_{x}} &&
	\undechastype{\Gamma,\tbind{x}{\tau_x}}{e}{\tau} &&
	\undeciswellformed{\Gamma}{\tau}
}{
	\undechastype{\Gamma}{\elet{x}{e_x}{e}}{\tau}
}[\rtlet]
$$

$$\inference{
	\undechastype{\Gamma}{e}{\tref{v}{T}{}{r}} &&
	 \undeciswellformed{\Gamma}{\tau}\\
     \forall i. \constty{D_i} = \overline{\tbind{y_j}{\tau_j}} \rightarrow \tref{v}{T}{}{r_i} &&
      \undechastype{\Gamma, \overline{\tbind{y_j}{\tau_j}}, \tbind{x}{\tref{v}{T}{}{r \land r_i}}}{e_i}{\tau}  \\
}{
	\undechastype{\Gamma}{\ecase{e}{D_i}{\overline{y_j}}{e_i}{x}}{\tau}
}[\rtcase]$$
\caption{Type-checking for \undeclang}
\label{fig:typing}
\end{figure}



Next, we present the type-checking judgments and rules of \undeclang. 

\spara{Environments and Closing Substitutions}
A \emph{type environment} $\Env$ is a sequence of type bindings 
$\tbind{x_1}{\typ_1},\ldots,\tbind{x_n}{\typ_n}$. An environment
denotes a set of \emph{closing substitutions} $\sto$ which are 
sequences of expression bindings: 
$\gbind{x_1}{e_1}, \ldots, \gbind{x_n}{e_n}$ such that:
$$
\interp{\Env} \defeq  \{\sto \mid \forall \tbind{x}{\typ} \in \Env. 
                                    \sto(x) \in \interp{\thetasub{\sto}{\typ}} \}
$$

\spara{Judgments}
We use environments to define three kinds of
rules: Well-formedness, Subtyping, 
and Typing~\cite{Knowles10,GordonTOPLAS2011}.
%
%\spara{Well-formedness}
A judgment \undeciswellformed{\Env}{\typ} states that 
the refinement type $\typ$ is well-formed in 
the environment $\Env$.
%
Intuitively, the type $\typ$ is well-formed if all
the refinements in $\typ$ are $\tbool$-typed in $\Env$.
%
%\spara{Subtyping} 
A judgment \undecissubtype{\Env}{\typ_1}{\typ_2} states 
that the type $\typ_1$ is a subtype of %the type 
$\typ_2$ in the environment $\Env$.
%
Informally, $\typ_1$ is a subtype of $\typ_2$ if, when 
the free variables of $\typ_1$ and $\typ_2$ 
are bound to expressions described by $\Env$,
the denotation of $\typ_1$ 
is \emph{contained in} the denotation of $\typ_2$. 
%
Subtyping of basic types reduces to denotational containment checking.
%
%\spara{Implication} 
%%A judgment \issubref{\Env}{p_1}{p_2} states 
%%that the predicate $p_1$ \emph{implies} 
%%the predicate $p_2$ in the environment $\Env$.
%
That is, for any closing substitution $\sto$
in the denotation of $\Env$, for every expression $e$, 
if $e \in \interp{\thetasub{\sto}{\typ_1}}$ then 
$ e \in \interp{\thetasub{\sto}{\typ_2}}$.
%
%\spara{Typing}
A judgment \undechastype{\Env}{e}{\typ} states that
the expression $e$ has the type $\typ$ in 
the environment $\Env$.
That is, when the free variables in $e$ are 
bound to expressions described by $\Env$, the 
expression $e$ will evaluate to a value 
described by $\typ$.

\mypara{Soundness}
Following \hlang~\cite{Knowles10}, we use the (undecidable) \rsubbase to show that each step 
of evaluation preserves typing, and that if an expression
is not a value, then it can be further evaluated:
%
\begin{itemize}
\item\textbf{Preservation:} 
	If \undechastype{\emptyset}{e}{\typ} and \eval{e}{e'}, 
	then \undechastype{\emptyset}{e'}{\typ}. 
\item\textbf{Progress:}
	If \undechastype{\emptyset}{e}{\typ} and $e \not = w$,
	then \eval{e}{e'}. 
\end{itemize}
%
We combine the above to prove that evaluation preserves 
typing, and that a well typed term will not \ecrash.
%
\begin{theorem}{[Soundness of \undeclang]}\label{thm:safety}
\begin{itemize}
\item\textbf{Type-Preservation:} If \undechastype{\emptyset}{e}{\typ}, %NV with the v -> w edit this didn't fit in 1 line
       $\evals{e}{w}$ then $\undechastype{\emptyset}{w}{\typ}$.
\item\textbf{Crash-Freedom:} If \undechastype{\emptyset}{e}{\typ} 
        then $\evals{e\not}{\ecrash}$.
\end{itemize}
\end{theorem}

We prove the above following the overall recipe of~\cite{Knowles10}. 
Crash-freedom follows from type-preservation and as \ecrash has no type.
%
The Substitution Lemma, in particular, follows from a connection between
the typing relation and type denotations:

\begin{lemma}{[Denotation Typing]}\label{lem:denotation}
If $\undechastype{\emptyset}{e}{\typ}$ then $e \in \interp{\typ}$.
\end{lemma} 

%%% Local Variables: 
%%% mode: latex
%%% TeX-master: "main"
%%% End: 
