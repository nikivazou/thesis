%\documentclass{llncs}
\documentclass{sigplanconf}
%%% \usepackage[top=1.5cm,bottom=1.5cm,left=2.3cm,right=2cm]{geometry}

\pagestyle{plain}
\usepackage{times}
%\usepackage[nocompress]{cite} % AR: comment this out if you wish hyperrefs 
\usepackage{hyperref}

\usepackage{amsmath,amssymb, latexsym}

\usepackage{amsmath}
\usepackage{amsthm}

\usepackage{comment}
\usepackage{flushend}

\newtheorem{lemma}{Lemma}
\newtheorem{corollary}{Corollary}
\newtheorem{theorem}{Theorem}
\newtheorem*{theorem*}{Theorem}
\newtheorem{definition}{Definition}
\newtheorem*{lemma*}{Lemma}

\usepackage{amsfonts}
\usepackage{amssymb}
\usepackage{mathtools}
\usepackage{commands}
\usepackage{liquidHaskell}
\usepackage[inference]{semantic}

\usepackage{enumerate}
%\def\url{}
\usepackage{xspace}
\usepackage{epsfig}
\usepackage{booktabs}
\usepackage{listings}
\usepackage{comment}
\usepackage{ifthen}

\newcommand{\isTechReport}{false} % true or false
\newcommand\includeProof[1]{%
  \ifthenelse{\equal{\isTechReport}{true}}
    {{#1}}
    {\ignorespaces}
\xspace}
\newcommand\includeBytestring[1]{}

%%%%%\usepackage{fancyvrb}
%%%%%\DefineVerbatimEnvironment{code}{Verbatim}{fontsize=\small}
%%%%%\DefineVerbatimEnvironment{example}{Verbatim}{fontsize=\small}
%%%%%\newcommand{\ignore}[1]{}

% command to end a proof or definition:
%\def\qed{\rule{0.4em}{1.4ex}}
\def\qed{\hfill$\Box$}

% space at the beginning of an environment:
\def\@envspa{\hspace{0.3em}}
\def\@sa{\hspace{-0.2em}}
\def\@sb{\hspace{0.5em}}
\def\@sc{\hspace{-0.1em}}

\def\sk{\smallskip}		% space before and after theorems

\newtheorem{notation}{Notation}{\itshape}{}
\newtheorem{invariant}{Invariant}
\newtheorem*{hypothesis}{Hypothesis}
\newtheorem{technical}{}
\usepackage{thmtools}

\declaretheoremstyle[%
  spaceabove=-6pt,%
  spacebelow=6pt,%
  headfont=\normalfont\itshape,%
  postheadspace=1em,%
  qed=\qedsymbol,%
  headpunct={}
]{mystyle} 
\declaretheorem[name={Proof Sketch:},style=mystyle,unnumbered,
]{proofsketch}


%\newcommand{\proofsketch}{\noindent {\bf Proof Sketch: }}
%\newcommand{\qed}{\hfill\rule{2mm}{2mm}}
\newcommand\proofsketchend{\hfill\ensuremath{\qedhere}}
\newcommand\liquidtech{
	\href{http://goto.ucsd.edu/~rjhala/liquid/liquid_types_techrep.pdf}
		 {liquid types technical report}
}

\usepackage{listings}

% uncomment next line to restore colors
% \def\withcolor{}

\ifdefined\withcolor
	\definecolor{haskellblue}{rgb}{0.0, 0.0, 1.0}
	\definecolor{haskellblue}{rgb}{1.0, 0.0, 0.0}
	\definecolor{gray_ulisses}{gray}{0.55}
	\definecolor{castanho_ulisses}{rgb}{0.71,0.33,0.14}
	\definecolor{preto_ulisses}{rgb}{0.41,0.20,0.04}
	\definecolor{green_ulisses}{rgb}{0.0,0.4,0.0}
\else
	\definecolor{haskellblue}{gray}{0.1}
	\definecolor{haskellred}{gray}{0.1}
	\definecolor{gray_ulisses}{gray}{0.1}
	\definecolor{castanho_ulisses}{gray}{0.1}
	\definecolor{preto_ulisses}{gray}{0.1}
	\definecolor{green_ulisses}{gray}{0.1}
\fi


\def\codesize{\normalsize}

\lstdefinelanguage{HaskellUlisses} {
	basicstyle=\ttfamily\footnotesize,
	sensitive=true,
	morecomment=[l][\color{gray_ulisses}\ttfamily\codesize]{--},
	%% morecomment=[s][\color{gray_ulisses}\ttfamily\codesize]{\{-}{-\}},
	morestring=[b]",
	stringstyle=\color{haskellred},
	showstringspaces=false,
	numberstyle=\codesize,
	numberblanklines=true,
	showspaces=false,
	breaklines=true,
	showtabs=false,
    literate={
           {`}{{{$^{\backprime}{}$}}}1
           {'}{{{$^{\prime}{}$}}}1
           % {QED}{{{\color{lcolor}QED}}}3
           % {***}{{{\color{lcolor}***}}}3
           {?}{{{$\therefore$}}}1
           {<=}{{$\leq$}}1
           {theta}{{$\theta$}}1
           {env}{{$\Gamma$}}1
           {|-}{{$\vdash$}}1
           {<=!}{{{\color{lcolor}<=!}}}3
           {!=}{{$\neq$}}2
           {forall}{{$\forall$}}1
           {->}{{$\rightarrow$}}2
           {=*}{{$\eqfun$}}2
           {<=>}{{$\Leftrightarrow$}}3
           {=>}{{$\Rightarrow$}}2
           {<:}{{$\preceq$}}1
           {mempty}{{$\mempty$}}1
           {mappend}{{$\mappend$}}1
           {<>}{{$\mappend$}}1
           {stringMempty}{{$\stringMempty$}}1
           {<+>}{{$\stringMappend$}}1
           {stringMappend}{{$\stringMappend$}}1
           {listMempty}{{[]}}1
           {listMappend}{{++}}2
           {epsilon}{{$\epsilon$}}1
           {eta}{{$\eta$}}1
           {&&&}{&&&}3
           {&&}{{$\land$}}1
           {_m}{{${}_m$}}1
           {_n}{{${}_n$}}1
           {m^+}{{m${}^{+}$}}2
           },
	emph=
	{[1]
		FilePath,IOError,abs,acos,acosh,all,and,any,appendFile,approxRational,asTypeOf,asin,
		asinh,atan,atan2,atanh,basicIORun,break,catch,ceiling,chr,compare,concat,concatMap,
		const,cos,cosh,curry,cycle,decodeFloat,denominator,digitToInt,div,divMod,drop,
		dropWhile,either,elem,encodeFloat,enumFrom,enumFromThen,enumFromThenTo,enumFromTo,
		error,even,exp,exponent,fail,filter,flip,floatDigits,floatRadix,floatRange,floor,
		fmap,foldl,foldl1,foldr,foldr1,fromDouble,fromEnum,fromInt,fromInteger,
		fromRational,fst,gcd,getChar,getContents,getLine,head,id,inRange,index,init,intToDigit,
		interact,ioError,isAlpha,isAlphaNum,isAscii,isControl,isDenormalized,isDigit,isHexDigit,
		isIEEE,isInfinite,isLower,isNaN,isNegativeZero,isOctDigit,isPrint,isSpace,isUpper,iterate,
		last,lcm,length,lex,lexDigits,lexLitChar,lines,log,logBase,lookup,map,mapM,mapM_,max,
		maxBound,maximum,maybe,min,minBound,minimum,mod,negate,not,notElem,numerator,odd,
		or,pi,pred,primExitWith,print,product,properFraction,putChar,putStr,putStrLn,quot,
		quotRem,range,rangeSize,read,readDec,readFile,readFloat,readHex,readIO,readInt,readList,readLitChar,
		readLn,readOct,readParen,readSigned,reads,readsPrec,realToFrac,recip,rem,repeat,replicate,
		reverse,round,scaleFloat,scanl,scanl1,scanr,scanr1,seq,sequence,sequence_,show,showChar,showInt,
		showList,showLitChar,showParen,showSigned,showString,shows,showsPrec,significand,signum,sin,
		sinh,snd,span,splitAt,sqrt,subtract,succ,sum,tail,take,takeWhile,tan,tanh,threadToIOResult,toEnum,
		toInt,toInteger,toLower,toRational,toUpper,truncate,uncurry,undefined,unlines,until,unwords,unzip,
		unzip3,userError,words,writeFile,zip,zip3,zipWith,zipWith3,listArray,doParse,for,initTo,
        maxEvens,create,get,set,initialize,idVec,fastFib,fibMemo,
        insert,union,split,size,fromList,initUpto,trim,quickSort,insertSort,append,upperCase,
        copy, group, doDownLoop, mapAccumR, peekByteOff,
        pokeByteOff,spanByte, 
        good, bad, foo, explode, 
        fib, ack, 
        tLen,
        memcpy,writeChar,unsafeWrite,unsafeFreeze,
        singleton
	},
	emphstyle={[1]\color{haskellblue}},
	emph=
	{[2]
		Bool,Char,Double,Either,Float,IO,Integer,Int,Maybe,Ordering,Rational,Ratio,ReadS,ShowS,String,
		Word8,Nat,NonZero,Nat64,Text,ByteString,ByteStringSZ,ByteStringN,
        Ptr,ForeignPtr,CSize
        InPacket,Tree,Prop,TreeEq,TreeLt,Vec,
        NullTerm,IncrList,DecrList,UniqList,BST,MinHeap,MaxHeap,
        PtrN,ByteStringN,ByteStringEq,VO,ByteStringsEq,ByteStringNE
	},
	emphstyle={[2]\color{castanho_ulisses}},
	emph=
	{[3]
		case,class,data,deriving,do,else,if,return,def,import,in,infixl,infixr,instance,let,
		module,measure,predicate,of,primitive,then,refinement,type,where,lazy
	},
	emphstyle={[3]\color{preto_ulisses}\textbf},
	emph=
	{[4]
		quot,rem,div,mod,elem,notElem,seq
	},
	emphstyle={[4]\color{castanho_ulisses}\textbf},
	emph=
	{[5]
		PS,Tip,Node,EQ,False,GT,Just,LT,Left,Nothing,Right,True,Show,Eq,Ord,Num
	},
	emphstyle={[5]\color{green_ulisses}}
}

%%%ORIG
%%%\lstnewenvironment{code}
%%%{\textbf{Haskell Code} \hspace{1cm} \hrulefill \lstset{language=HaskellUlisses}}
%%%{\hrule\smallskip}

%V1
%\lstnewenvironment{code}
%{\smallskip \lstset{language=HaskellUlisses}}
%{\smallskip}

\lstnewenvironment{code}
{\lstset{language=HaskellUlisses}}
{}

\lstnewenvironment{mcode}
{\lstset{language=HaskellUlisses,mathescape}}
{}

\lstMakeShortInline[language=HaskellUlisses]@




%\lstMakeShortInline[language=HaskellUlisses,basicstyle=\ttfamily\normalsize,breakatwhitespace]@

\begin{document}

\title{
	\ifthenelse{\equal{\isTechReport}{true}}
    {{Technical Report:}}
    {}
    Refinement Types For Haskell
    \thanks{This work was supported by NSF grants 
      CNS-0964702, CNS-1223850, CCF-1218344, CCF-1018672,
      and a generous gift from Microsoft Research.
    }
  } 
\newcommand\showproof[1]{\texttt{proved}}
\newcommand\showproofsketch[1]{#1}%{\texttt{proof sketch}}
\newcommand\showprooftodo[1]{\texttt{TODO}}



\authorinfo{Niki Vazou \and Eric L. Seidel \and Ranjit Jhala}{UC San Diego}{}
\authorinfo{Dimitrios Vytiniotis \and Simon Peyton-Jones}{Microsoft Research}{}

\maketitle

%%\NV{Fonts:
%%SPJ:I would prefer to use math-italic for all non-terminals
%%Particular terminal symbols like Int, Bool, and keywords like 'measure' can be typewriter font
%%Types can be \tau instead of t}


\conferenceinfo{ICFP~'14}{September 1--6, 2014, Gothenburg, Sweden} 
\copyrightyear{2014} 
\copyrightdata{978-1-4503-2873-9/14/09} 
%% \doi{nnnnnnn.nnnnnnn} 


\begin{abstract}
Refinement Reflection turns your favorite programming 
language into a proof assistant by reflecting
the ​code implementing a​ user-defined function 
into the function's (output) refinement type. 
%
As a consequence, at \emph{uses} of the function, 
the function definition is unfolded into the refinement logic 
in a precise, predictable and most
importantly, programmer controllable way.
%
In the logic, we encode functions and lambdas using uninterpreted symbols
preserving SMT-based decidable verification. 
In the language, we provide a library of combinators that lets programmers 
compose proofs from basic refinements and function definitions.
%
We have implemented our approach in the Liquid Haskell system, 
thereby converting Haskell into an interactive proof assistant, 
that we used to verify a variety of properties ranging 
from arithmetic properties of higher order, recursive functions
to the Monoid, Applicative, Functor and Monad type class laws 
for a variety of instances.
\end{abstract}
\begin{comment}
Must program verifiers always choose between expressiveness
and automation?
%
On the one hand, tools based on higher order logics
and full dependent types impose no limits on expressiveness,
but require user-provided (perhaps, tactic-based) proofs.
%
On the other hand, tools based on Refinement Types~\cite{Rushby98,pfenningxi98}
trade expressiveness for automation. For example, the refinement types
%
\begin{code}
  type Pos     = {v:Int | 0 < v}
  type IntGE x = {v:Int | x <= v}
\end{code}
%
specify subsets of @Int@ corresponding to values
that are positive or larger than some other value @x@
respectively. By limiting the refinement predicates to
SMT-decidable logics~\cite{NelsonOppen}, refinement type
based verifiers eliminate the need for explicit proof terms,
and thus automate verification.

% We can specify contracts like pre- and post-conditions by
% suitably refining the input and output types of functions.

This high degree of automation has enabled the
use of refinement types for a variety of verification
tasks, ranging from array bounds checking~\cite{LiquidPLDI08},
termination and totality checking~\cite{LiquidICFP14},
protocol validation~\cite{GordonTOPLAS2011,FournetCCS11},
and securing web applications~\cite{SwamyOAKLAND11}.
%
Unfortunately, this automation comes at a price.
To ensure predictable and decidable type checking, we must
limit the logical formulas appearing in specification types
to decidable (typically quantifier free) first order theories,
thereby precluding \emph{higher-order} specifications that
are essential for \emph{modular} verification.
\end{comment}

In this chapter we introduce \emph{Bounded Refinement Types} which enable 
\emph{bounded quantification} over refinements. 
%
Previously (Chapter~\ref{chapter:abstractrefinements}),
we developed Abstract Refinement Types, a mechanism
for quantifying type signatures over abstract refinement parameters.
%
We preserved decidability of checking and inference
by encoding abstractly refined types with uninterpreted functions
obeying the decidable axioms of congruence~\cite{NelsonOppen}. 
%
While useful,
refinement quantification was not enough to enable higher order abstractions
requiring fine grained \emph{dependencies between} abstract refinements.
%
In this chapter, we solve this problem by enriching signatures
with bounded quantification. 
%
The \emph{bounds} correspond to Horn
implications between abstract refinements, which, as in the classical
setting, correspond to subtyping constraints that must be satisfied 
by the concrete refinements used at any call-site. This
addition proves to be remarkably effective.

\begin{itemize}
\item
First, we demonstrate via a series of short examples how bounded refinements
enable the specification and verification of diverse textbook higher order
abstractions that were hitherto beyond the scope of decidable refinement
typing~(\S~\ref{sec:boundedrefinementtypes:overview}).

\item
Second, we formalize bounded types and show how bounds are translated
into ``ghost'' functions, reducing type checking and inference to the
``unbounded'' setting of chapter~\ref{chapter:abstractrefinements}, 
thereby ensuring that checking
remains decidable. Furthermore, as the bounds are Horn constraints, we
can directly reuse the abstract interpretation of Liquid Typing~\citep{LiquidPLDI08}
to automatically infer concrete refinements at instantiation
sites~(\S~\ref{sec:check}).

\item
Third, to demonstrate the expressiveness of bounded refinements, we
use them to build a typed library for extensible dictionaries, to
then implement a relational algebra library on top of those
dictionaries, and to finally build a library for type-safe
database access~(\S~\ref{sec:database}).

\item
Finally, we use bounded refinements to develop a \emph{Refined State Transformer}
monad for stateful functional programming, based upon Filli\^atre's method
for indexing the monad with pre- and post-conditions~\citep{Filliatre98}.
%
We use bounds to develop branching and looping combinators whose types
signatures capture the derivation rules of Floyd-Hoare logic, thereby
obtaining a library for writing verified stateful computations~(\S~\ref{sec:state}).
%
We use this library to develop a refined IO monad that tracks capabilities
at a fine-granularity, ensuring that functions only access specified
resources~(\S~\ref{sec:files}).
\end{itemize}

We have implemented Bounded Refinement Types in \toolname.
The source code of the examples (with slightly more verbose concrete syntax)
is at \cite{liquidhaskellgithub}.
%
While the construction of these verified abstractions is possible with full
dependent types, bounded refinements
%
keep checking automatic and decidable,
%
use abstract interpretation to automatically synthesize
refinements (\ie pre- and post-conditions and loop invariants),
and most importantly
%
enable retroactive or \emph{gradual} verification as when
erase the refinements, we get valid programs in the
host language.
%
Thus, bounded refinements point a way towards 
both automated and expressive verification. 
%
%%% Local Variables:
%%% mode: latex
%%% TeX-master: "main"
%%% End:

\section{Overview}
\label{sec:refinementreflection:overview}
\label{sec:examples}

We begin with an overview of refinement reflection and
how it allows us to write proofs \emph{of} and \emph{by}
Haskell functions.

\subsection{Refinement Types}

First, we recall some preliminaries about refinement types
and how they enable shallow specification and verification.

\mypara{Refinement types} are the source program's (here
Haskell's) types decorated with logical predicates drawn
from a(n SMT decidable) logic~\citep{ConstableS87,Rushby98}.
%
For example, we can define the @Nat@ type by refining
Haskell's @Int@ type with a predicate @0 <= v@:
%
\begin{code}
  type Nat = { v:Int | 0 <= v }
\end{code}
%
Here, @v@ names the value described by the type:
the above can be read as the
``set of @Int@ values @v@ that are not less than 0".
The refinement is drawn from the logic of quantifier
free linear arithmetic and uninterpreted functions
(QF-UFLIA~\cite{SMTLIB2}).

\mypara{Specification \& Verification}
%
We can use refinements to define and type the
textbook Fibonacci function as:
%
\begin{code}
  fib :: Nat -> Nat
  fib 0 = 0
  fib 1 = 1
  fib n = fib (n-1) + fib (n-2)
\end{code}
%
Here, the input type's refinement specifies a
\emph{pre-condition} that the parameters must
be @Nat@, which is needed to ensure termination,
and the output types's refinement specifies a
\emph{post-condition} that the result is also a @Nat@.
%
Refinement type checking lets us specify
and (automatically) verify the shallow property
that if @fib@ is invoked with a non-negative
@Int@, then it terminates and yields
a non-negative @Int@.

\mypara{Propositions}
%
We can use refinements to define a data type
representing propositions simply as an alias
for unit, a data type that carries no useful
runtime information:
%
\begin{mcode}
  type $\typp$ = ()
\end{mcode}
%
which can be \emph{refined} with
propositions about the code.
%
For example, the following states the proposition
$2 + 2$ equals $4$.
%
\begin{mcode}
  type Plus_2_2_eq_4 = { v: $\typp$ | 2 + 2 = 4 }
\end{mcode}
%
For clarity, we abbreviate the above type by omitting
the irrelevant basic type $\typp$ and variable @v@:
%
\begin{mcode}
  type Plus_2_2_eq_4 = { 2 + 2 = 4 }
\end{mcode}
%
Function types encode universally quantified propositions:
%
\begin{mcode}
  type Plus_com = x:Int -> y:Int -> { x + y = y + x }
\end{mcode}
%
The parameters @x@ and @y@ refer to input
values. Any inhabitant of the above type is a
proof that @Int@ addition is commutative.

\mypara{Proofs}
%
We \emph{prove} the above theorems by providing inhabitants to type specifications
in forms of Haskell programs. To ease this task \toolname
provides primitives to construct proof terms by
``casting'' expressions to \typp.
%
\begin{mcode}
  data QED = QED

  (**) :: a -> QED -> $\typp$
  _ ** _  = ()
\end{mcode}
%
To resemble mathematical proofs, we make this casting post-fix.
Thus, we write @e ** QED@ to cast @e@ to a value of \typp.
%
For example, we can prove the above propositions by writing
%
\begin{code}
  pf_plus_2_2 :: Plus_2_2_eq_4
  pf_plus_2_2 = trivial ** QED

  pf_plus_comm :: Plus_comm
  pf_plus_comm = \x y -> trivial ** QED

  trivial = ()
\end{code}
%
Via standard refinement type checking, the above code yields
the respective verification conditions (VCs),
%
\begin{align*}
                      2 + 2 & = 4 \\
  \forall \ x,\ y\ .\ x + y & = y + x
\end{align*}
%
which are easily proved valid by the SMT solver, allowing us
to prove the respective propositions.

\mypara{A Note on Bottom:} Readers familiar with Haskell's
semantics may be feeling anxious about whether the
dreaded ``bottom", which inhabits all types, makes our
proofs suspect.
%
Fortunately, as described in \cite{Vazou14}, \toolname
ensures that all terms with non-trivial refinements
provably evaluate to (non-bottom) values, thereby making
our proofs sound.

\subsection{Refinement Reflection}

Suppose we wish to prove properties about the @fib@
function, \eg @fib 2@ equals @1@.
%
\begin{code}
  type fib2_eq_1 = { fib 2 = 1 }
\end{code}
%
%% \NV{By Standard refinement type checking, you mean liquid types, not FStar}
Standard refinement type checking runs into two problems.
%
First, for decidability and soundness, arbitrary user-defined
functions do not belong the refinement logic, \ie we cannot
\emph{refer} to @fib@ in a refinement.
%
Second, the only information that a refinement type checker
has about the behavior of @fib@ is its shallow type
specification @Nat -> Nat@ which is far too weak to verify
@fib2_eq_1@.
%
To address both problems, we use the following annotation,
which sets in motion the three steps of refinement reflection:
%
\begin{code}
  reflect fib
\end{code}

\mypara{Step 1: Definition}
%
The annotation tells \toolname to declare an
\emph{uninterpreted function} @fib :: Int -> Int@
in the refinement logic.
%
By uninterpreted, we mean that the logical @fib@
is \emph{not} connected to the program function
@fib@; in the logic, @fib@
only satisfies the \emph{congruence axiom}
%
$$\forall n, m.\ n = m\ \Rightarrow\ \fib{n} = \fib{m}$$
%
On its own, the uninterpreted function is not
terribly useful, as it does not let us prove
% It lets us prove theorems like
% $$\forall m,\ n.\ m = n \Rightarrow \fib{m} = \fib{n}$$
%
%% \begin{code}
  %% fib_cong :: n:Nat -> m:Nat -> {m=n => fib m = fib n}
  %% fib_cong = trivial ** QED
%% \end{code}
%% %
%but not
@fib2_eq_1@ which requires reasoning about the
\emph{definition} of @fib@.

\mypara{Step 2: Reflection}
%
In the next key step, \toolname reflects the
definition into the refinement type of @fib@
by automatically strengthening the user defined
type for @fib@ to:
%
\begin{code}
  fib :: n:Nat -> { v:Nat | fibP v n }
\end{code}
%
where @fibP@ is an alias for a refinement
\emph{automatically derived} from the
function's definition:
%
\begin{mcode}
  fibP v n = v = if n = 0 then 0 else
                 if n = 1 then 1 else
                 fib(n-1) + fib(n-2)
\end{mcode}

\mypara{Step 3: Application}
%
With the reflected refinement type,
each application of @fib@ in the code
automatically unfolds the @fib@ definition
\textit{once} in the logic.
%
We prove @fib2_eq_1@ by:
%
\begin{code}
  pf_fib2 :: { fib 2 = 1 }
  pf_fib2 = let t0 = #fib# 0 
                t1 = #fib# 1
                t2 = #fib# 2 
            in  ()
\end{code}
%
We write @#f#@ to denote places where the
unfolding of @f@'s definition is important.
%
Via refinement typing, the above proof yields the
following verification condition that is
discharged by the SMT solver, even though @fib@
is uninterpreted:
%
\begin{align*}
   (\fibdef\ (\fib\ 0)\ 0) \ \wedge\ (\fibdef\ (\fib\ 1)\ 1) \ \wedge\ 
   (\fibdef\ (\fib\ 2)\ 2) \  \Rightarrow\ (\fib{2} = 1)
\end{align*}
%
Note that the verification of @pf_fib2@ relies
merely on the fact that @fib@ was applied
to (\ie unfolded at) @0@, @1@ and @2@.
%
The SMT solver automatically \emph{combines}
the facts, once they are in the antecedent.
The following is also verified:
%
\begin{code}
  pf_fib2' :: { fib 2 = 1 }
  pf_fib2' = [ #fib# 0, #fib# 1, #fib# 2 ] ** QED
\end{code}
%
%
Thus, unlike classical dependent typing, refinement
reflection \emph{does not} perform any type-level
computation.

\mypara{Reflection vs. Axiomatization}
%
An alternative \emph{axiomatic} approach,
used by Dafny~\citep{dafny} and
\fstar~\citep{fstar},
is to encode @fib@ using a universally
quantified SMT formula (or axiom):
$$\forall n.\ \fibdef\ (\fib\ n)\ n$$
%
Axiomatization offers greater automation than
reflection. Unlike \toolname, Dafny
%and \fstar
will verify the following by
\emph{automatically instantiating} the above
axiom at @2@, @1@ and @0@:
%
\begin{code}
  axPf_fib2 :: { fib 2 = 1 }
  axPf_fib2 = trivial ** QED
\end{code}

The automation offered by axioms is a bit of a
devil's bargain, as axioms render checking of
the VCs \emph{undecidable}.
%
In practice, automatic axiom instantation can
easily lead to infinite ``matching loops''.
%
For example, the existence of a term \fib{n} in a VC
can trigger the above axiom, which may then produce
the terms \fib{(n-1)} and \fib{(n-2)}, which may then
recursively give rise to further instantiations
\emph{ad infinitum}.
%
To prevent matching loops an expert must carefully
craft ``triggers'' and provide a ``fuel''
parameter~\citep{Amin2014ComputingWA} that can be
used to restrict the numbers of the SMT unfoldings,
which ensure termination, but can cause the axiom
to not be instantiated at the right places.
%
In short, per the authors of Dafny, the
undecidability of the VC checking and its
attendant heuristics makes verification
unpredictable~\citep{Leino16}.

\subsection{Structuring Proofs}

In contrast to the axiomatic approach,
with refinement reflection, the VCs are
deliberately designed to always fall in
an SMT-decidable logic, as function symbols
are uninterpreted.
%
It is up to the programmer to unfold the
definitions at the appropriate places,
which we have found, with careful design
of proof combinators, to be quite
a natural and pleasant experience.
%
To this end, we have developed a library
of proof combinators that permits reasoning
about equalities and linear arithmetic,
inspired by Agda~\citep{agdaequational}.

\mypara{``Equation'' Combinators}
%
We equip \toolname with a family of
equation combinators @op.@ for each
logical operator @op@ in
$\{=, \not =, \leq, <, \geq, > \}$,
the operators in the theory QF-UFLIA.
%
The refinement type of @op.@  \emph{requires}
that $x \odot y$ holds and then \emph{ensures}
that the returned value is equal to @x@.
%
For example, we define @=.@ as:
%
\begin{code}
  (=.) :: x:a -> y:{a| x=y} -> {v:a| v=x}
  x =. _ = x
\end{code}
%
and use it to write the following ``equational" proof:
%
\begin{code}
  eqPf_fib2 :: { fib 2 = 1 }
  eqPf_fib2 =  #fib# 2
            =. #fib# 1 + #fib# 0
            =. 1
            ** QED
\end{code} %$

\mypara{``Because'' Combinators}
%
Often, we need to compose ``lemmata'' into larger
theorems. For example, to prove @fib 3 = 2@ we
may wish to reuse @eqPf_fib2@ as a lemma.
%
To this end, \toolname has a ``because'' combinator:
%
\begin{mcode}
  ($\because$) :: ($\typp$ -> a) -> $\typp$ -> a
  f $\because$ y = f y
\end{mcode}
%
The operator is simply an alias for function
application that lets us write
%
@ x op. y $\because$ p@ (instead of @(op.) x y p@)
where @(op.)@ is extended to accept an \textit{optional} third proof
argument via Haskell's typeclass mechanisms.
%
We use the because combinator to
prove that @fib 3 = 2@ with a Haskell function:
%
\begin{mcode}
  eqPf_fib3 :: { fib 3 = 2 }
  eqPf_fib3 =  #fib# 3
            =. fib 2 + #fib# 1
            =. 2              $\because$ eqPf_fib2
            ** QED
\end{mcode}

\mypara{Arithmetic and Ordering}
%
SMT based refinements let us go well beyond just equational
reasoning. Next, lets see how we can use arithmetic and
ordering to prove that @fib@ is (locally) increasing,
%
\ie for all $n$, $\fib{n} \leq \fib{(n+1)}$
%
\begin{mcode}
  fibUp :: n:Nat -> { fib n <= fib (n+1) }
  fibUp n
    | n == 0
    =  #fib# 0 <. #fib# 1
    ** QED

    | n == 1
    =  fib 1 <=. fib 1 + fib 0 <=. #fib# 2
    ** QED

    | otherwise
    =  #fib# n
    =. fib (n-1) + fib (n-2)
    <=. fib n     + fib (n-2) $\because$ fibUp (n-1)
    <=. fib n     + fib (n-1) $\because$ fibUp (n-2)
    <=. #fib# (n+1)
    ** QED
\end{mcode} %$

\mypara{Case Splitting and Induction}
%
The proof @fibUp@ works by induction on @n@.
%
In the \emph{base} cases @0@ and @1@, we simply assert
the relevant inequalities. These are verified as the
reflected refinement unfolds the definition of
@fib@ at those inputs.
%
The derived VCs are (automatically) proved
as the SMT solver concludes $0 < 1$ and $1 + 0 \leq 1$
respectively.
%
In the \emph{inductive} case, @fib n@ is unfolded
to  @fib (n-1) + fib (n-2)@, which, because of the
induction hypothesis (applied by invoking @fibUp@
at @n-1@ and @n-2@) and the SMT solver's arithmetic
reasoning, completes the proof.

\mypara{Higher Order Theorems}
%
Refinements smoothly accomodate higher-order reasoning.
%
For example, lets prove that every locally increasing
function is monotonic, \ie
if @f z <= f (z+1)@ for all @z@,
then @f x <= f y@ for all @x < y@.
%
\begin{mcode}
  fMono :: f:(Nat -> Int)
        -> fUp:(z:Nat -> {f z <= f (z+1)})
        -> x:Nat
        -> y:{x < y}
        -> {f x <= f y} / [y]
  fMono f inc x y
    | x + 1 == y
    =  f x <=. f (x+1) $\because$ fUp x
           <=. f y
           ** QED

    | x + 1 < y
    =  f x <=. f (y-1) $\because$ fMono f fUp x (y-1)
           <=. f y     $\because$ fUp (y-1)
           ** QED
\end{mcode}
%
We prove the theorem by induction
on @y@, which is specified by the
annotation @/ [y]@ which states
that @y@ is a well-founded
termination metric that decreases
at each recursive call~\citep{Vazou14}.
%
% All reflected functions are proved terminating.
% When the annotation metric is not explicit Liquid Haskell
% successfully uses heuristics to automatically prove termination. 
%
If @x+1 == y@, then we use @fUp x@.
%
Otherwise, @x+1 < y@, and we use
the induction hypothesis \ie apply
@fMono@ at @y-1@, after which
transitivity of the less-than
ordering finishes the proof.
%
We can use the general @fMono@
theorem to prove that @fib@
increases monotonically:
%
\begin{code}
  fibMono :: n:Nat -> m:{n<m} -> {fib n <= fib m}
  fibMono = fMono fib fibUp
\end{code}


\subsection{Case Study: Deterministic Parallelism}
\label{sec:detpar}

%% The natural integration of deep verification with a language like Haskell makes
%% it possible to engage in lightweight, incremental verification of program
%% properties.

One benefit of an in-language prover is that it lowers the barrier to {\em
  small} verification efforts that touch only a fraction of the program, and yet
ensure critical invariants that Haskell's type system cannot.  Here we consider
parallel programming, which is commonly considered error prone and entails
proof obligations on the user that typically go unchecked.

The situation is especially precarious with parallel programming frameworks that
claim to be {\em determinstic} and thus usable within purely functional
programs.  These include Deterministic Parallel Java (DPJ \cite{DPJ}), Concurrent
Revisions for .NET~\cite{concurrent-revisions-oopsla}, and Haskell's
LVish~\cite{kuper2014freeze}, Accelerate~\cite{accelerate-icfp13}, and
REPA~\cite{repa-icfp10}.
%
Accelerate's parallel fold function, for instance, claims to be
deterministic---and its purely functional type means the Haskell optimizer will
{\em assume} its referential transparency---but its determinism depends on an
associativity guarantee which must be assured {\em by the programmer} rather than the
type system.
%
Thus simply folding the minus function, @fold (-) 0 arr@, is sufficient to
violate determinism and Haskell's pure semantics.


Likewise, DPJ goes to pains to develop a new type system for parallel
programming, but then provides a ``commutes'' annotation for methods updating
shared state, compromising the {\em guarantee} and going back to trusting the
user. LVish has the same Achilles heel. Consider set insertion:

\begin{code}
  insert :: Ord a => a -> Set s a -> Par s ()
\end{code}

Here @insert@ returns an (effectful) @Par@ computation, which can be run within a
pure function to produce a pure result.  At first glance it would seem that
trusting the implementation of the concurrent set is sufficient to assure a
deterministic outcome.  Yet the interface has an @Ord@ constraint. This
 polymorphic function works with user-defined data types, and thus
user-defined orderings.  What if the user fails to implement a total order?
Then, even a correct implementation of, e.g. a concurrent
skiplist~\cite{concurrent-skiplist}, can reveal
different insertion orders due to concurrency.

%% verifiedInsert :: HasPut e => VerifiedOrd a
%%                => a -> ISet s a -> Par e s ()

% \mypara{LVish}
%% We demonstrate the use of \toolname{} to ensure guarantees of
%% deterministic parallel programming. We choose this case study, because, to the
%% best of our knowledge, there exists no practical deterministic parallel
%% programming system, including user-defined parallel folds, which does not have
%% {\em soundness holes}---due to trusted assumptions of user code.

%% {\em LVish}\cite{kuper2014freeze} is a programming library for Haskell, which
%% exposes effectful parallel programming against lattice-variables (LVars) whose
%% states change monotonically during parallel regions of program execution. LVish
%% programs operate on Haskell data types, and LVish requires the operations on
%% these datatypes to satisfy some first order laws, which cannot be expressed in
%% Haskell. However, we can leverage \toolname to verify these properties for
%% arbitrary user-defined datatypes.

%% LVish provides two implementations of concurrent sets, @PureSet@ and @SLSet@,
%% where the underlying data structure is a size-balanced binary tree and
%% concurrent skiplist respectively. The @insert@ operation on a set requires a
%% total ordering on the elements, we can express that in the type signature by
%% \new{The implementation doesn't change, in fact, the}
%% @VerifiedOrd@ \new{methods do not even need to exist at runtime. A sufficiently
%%   smart compiler could optimize away these proof obligations during code
%%   generation.}
% \RN{Let's save the issue of runtime impact for the eval.}

In summary, parallel programs naturally need to communicate, but the mechanisms
of that communication---such as folds or inserts into a shared
structure---typically carry additional proof obligations.  This in turn makes
parallelism a liability.  But we can remove the risk with verification.

% But what if we could use verification to remove the risk?

% through contributions to shared structures (otherwise they are really separate
% programs)


\mypara{Verified typeclasses}
%
Our solution is simply to change the @Ord@ constraint above to
@VerifiedOrd@.
\begin{mcode}
  insert :: VerifiedOrd a => a -> Set s a -> Par s ()
\end{mcode}
%
This constraint changes the interface but not the implementation of @insert@.
%
% \NV{Why does insert now requires Verified Ord? Is it using the extra methods
% in the implementation?}
%% \note{VerifiedSemigroup story + lifting + isomorphism ("bootstrapping
%%   instances") + detpar propaganda}
%
%% It is an informal requirement when using
%% typeclasses in GHC that some typeclass laws be satisfied. For example, the @Ord@
%% typeclass in GHC requires that the $\leq$ operation be a total order. Using
%% \toolname, we can extend it to include the required properties of a total order,
%% which we call a @VerifiedOrd@.
The additional methods of the verified type class don't add operational
capabilities, but rather impose additional proof obligations:

\begin{code}
  class Ord a => VerifiedOrd a where
   antisym :: x:a -> y:a -> { x <= y && y <= x => x = y }
   trans   :: x:a -> y:a -> z:a -> { x <= y && y <= z => x <= z }
   total   :: x:a -> y:a -> { x <= y || y <= x }
\end{code}

% ---------------------------------------------------------------
% \mypara{Verified Monoids}

Similarly, we can extend
the @Monoid@ typeclass to a @VerifiedMonoid@, with refinements
expressing @Monoid@ laws.
%
\begin{code}
  class Monoid a => VerifiedMonoid a where
   lident :: x:a -> { mempty <> x = x }
   rident :: x:a -> { x <> mempty = x }
   assoc  :: x:a -> y:a -> z:a -> { x <> (y <> z) = (x <> y) <> z }
\end{code}
The @VerifiedMonoid@ typeclass constraint requires the binary operation
to be associative, thus can be safely used to fold on
an unknown number of processors.
%% A parallel fold requires the underlying binary operation to be associative and
%% have a well-behaved identity element, or a @Monoid@.


%% We can then extend the @ParFoldable@ typeclass to a @VerifiedParFoldable@ which
%% enforces a @VerifiedMonoid@ constraint.

%% \NV{Not sure if the below code adds any information: too difficult to follow,
%%   especially for non Haskell people} \NV{I suggest to say similarly to Verified
%%   Ord and add a link to appendix}
%%\RN{I concur with Niki -- we often whitewash away details of the library for
%%  presentation purposes.  E.g. we are not going to explain effect signatures in
%%  this paper.}

%% \begin{code}
%% class ParFoldable c
%%    => VerifiedParFoldable c where
%%   verifiedPmapFold :: forall m e s a .
%%   ( ParFuture m, HasGet e
%%   , HasPut e, FutContents m a,
%%   , VerifiedMonoid a )
%%   => (ElemOf c -> m e s a) -- compute one
%%                            -- result
%%   -> c                     -- element generator
%%                            -- to consume
%%   -> m e s a
%% \end{code}


%%  -------------------------------------------------------------------------

\mypara{Verified instances for primitive types}
@VerifiedOrd@ instances for primitive types like @Int@, @Double@ are trivial to
write; they just appeal to the SMT solver's built-in theories.
%
For example, the following is a valid totality proof on @Int@.
\begin{code}
  totInt :: x:Int -> y:Int -> {x <= y || y <= x}
  totInt _ _ = trivial ** QED
\end{code}

\mypara{Verified instances for algebraic datatypes}
%
To prove the class laws for user defined algebraic datatypes,
refinement reflection allows for structurally inductive proof terms.
%
For example, we can inductively define Peano numerals
%
\begin{code}
  data Peano = Z | S Peano
\end{code}
%
We can compare two @Peano@ numbers via
\begin{code}
  reflect leq :: Peano -> Peano -> Bool
  leq Z _         = True
  leq (S n) Z     = False
  leq (S n) (S m) = leq n m
\end{code}
%
In \S~\ref{sec:refinementreflection:theory} we will describe
exactly how the reflection mechanism (illustrated
via @fibP@) is extended to account for ADTs like @Peano@.
%
\toolname automatically checks
that @leq@ is total~\citep{Vazou14}, which
lets us safely @reflect@ it into the logic.

Next, we prove that @leq@ is total on @Peano@ numbers
%
\begin{mcode}
  totalPeano :: n:Peano -> m:Peano -> {leq n m || leq m n} / [toInt n + toInt m]
  totalPeano Z m = leq Z m ** QED
  totalPeano n Z = leq Z n ** QED
  totalPeano (S n) (S m)
   =  leq (S n) (S m) || leq (S m) (S n)
   =. leq n m || leq m n
   =. True $\because$ totalPeano m n
   ** QED
\end{mcode}
The proof goes by induction, splitting cases on
whether the number is zero or non-zero. Consequently,
we pattern match on the parameters @n@ and @m@, and furnish
separate proofs for each case.
%
In the ``zero" cases, we simply unfold the definition
of @leq@.
%
In the ``successor" case, after unfolding we (literally)
apply the induction hypothesis by using the because operator.
%
The termination hint @[toInt n + toInt m]@,
where @toInt@ maps @Peano@ numbers to integers,
is used to verify well-formedness of the @totalPeano@
proof term.
%
\toolname's termination and totality checker
use the hint to
verify that we are in fact doing induction
properly~(\S~\ref{sec:types-reflection}).

Similarly to @totalPeano@, we can define the rest of the @VerifiedOrd@
proof methods and use them to create the verified instance.
%
\begin{code}
  instance Ord Peano where
    (<=) = leq

  instance VerifiedOrd Peano where
    total = totalPeano
\end{code}
%
Proving all the four @VerifiedOrd@ laws
is a burden on the programmer.
%becomes a burden as the datatype grows more complicated.
%
Since @Peano@ is isomorphic to @Nat@s,
next we present how
to reduce the @Peano@ proofs into the
SMT automated integer proofs.

\mypara{Isomorphisms}
%
In order to reuse proofs for a custom datatype,
we provide a way to translate verified instances between isomorphic types~\cite{barthe2001type}.
%% If our datatype is isomorphic to a nesting of binary sums and products, we
%% should be able to reusing existing proofs.
%% To verify operations on custom data types efficiently, we
%% need to be able lift verified instances on one type to another.
%
We design a typeclass @Iso@ which witnesses the fact that
two types are isomorphic.
%, with respect to the built-in equality in \toolname{}
%which is a congruence.

\begin{mcode}
  class Iso a b where
    to      :: a -> b
    from    :: b -> a
    to$\circ$from :: x:a -> {to (from x) = x}
    from$\circ$to :: x:a -> {from (to x) = x}
\end{mcode}
%
For two isomorphic types @a@ and @b@
we compare instances of @b@ using @a@'s
comparison method.
%
\begin{mcode}
  instance (Ord a, Iso a b) => Ord b where
    x <= y = from x <= from y
\end{mcode}
%
Then, we prove that @VerifiedOrd@ laws are closed under isomorphisms.
%
For example, we prove totality of comparison on @b@s
using the @VerifiedOrd@ totality on @a@s

\begin{mcode}
  isoTotal :: (VerifiedOrd a, Iso a b) => x:b -> y:b -> {x <= y || y <= x}
  isoTotal x y
   =  x <= y || y <= x
   =. (from x) <= (from y) || (from y) <= (from x)
      $\because$ total (from x) (from y)
   ** QED
\end{mcode}
%
We use @isoTotal@ to create a verified instance on @b@s.
\begin{mcode}
  instance (VerifiedOrd a, Iso a b) => VerifiedOrd b where
    total   = isoTotal
\end{mcode}
%
With the above technique,
and using Haskell's instances,
getting a @VerifiedOrd@ instance for @Peano@
reduces to definition of an @Iso Nat Peano@.
%\VC{Iso (Either () Peano) Peano}

\mypara{Proof Composition via Products}
Finally, we present a mechanism to automatically
reduce proofs on product types to proofs of the product components.
%
For example, lexicographic ordering preserves the ordering laws.
%
First, we use class instances to define lexicographic ordering.
%
\begin{mcode}
  instance (VerifiedOrd a, VerifiedOrd b) => Ord (a, b) where
    (x1, y1) <= (x2, y2) = if x1 == x2 then y1 <= y2 else x1 <= x2
\end{mcode}
%
Then, we prove that lexicographic ordering
preserves the ordering laws.
%
For example, it preserves totality.
%
\begin{mcode}
  prodTotal :: (VerifiedOrd a, VerifiedOrd b)
            => p:(a, b) -> q:(a, b) -> {p <= q || q <= p}
  prodTotal p@(x1, y1) q@(x2, y2)
   =  p <= q || q <= p
   =. if x1 == x2 then (y1 <= y2 || y2 <= y1) else True 
      $\because$ total x1 x2
   =. if x1 == x2 then True                   else True 
      $\because$ total y1 y2
   ** QED
\end{mcode}
%
Finally, using the @prodTotal@ proof method,
we conclude that each instance defined via the lexicographic
ordering is indeed a verified instance.
%
\begin{mcode}
  instance (VerifiedOrd a, VerifiedOrd b) => VerifiedOrd (a, b) where
    total   = prodTotal
\end{mcode}
%
For example the type @(Peano, Peano)@ is derived to be a @VerifiedOrd@ instance.

In short, we can decompose an algebraic datatype into an isomorphic type using sums and
products to generate verified instances for arbitrary Haskell
datatypes. This could be combined with the Glasgow Haskell Compiler's (GHC) support
for generics~\cite{ghc-generics} to automate the derivation of verified instances
for user datatypes.
In \S\ref{sec:eval-parallelism}, we use these ideas to develop fully safe
interfaces to LVish modules, as well as verifying programming patterns from DPJ.

\newcommand\hnull{\ensuremath{\text{[]}}\xspace}

\subsection{Measures: From Integers to Data Types}\label{sec:measures}

\begin{figure}
\centering
\captionsetup{justification=centering}
$$
\begin{array}{lrcl}
{\emphbf{Definition}} &
  \mathit{def} & ::=  &  \mathtt{measure} \ f :: \tau \\
              & &      &  \quad eq_1 \ldots eq_n       \\[0.05in]

{\emphbf{Equation}}   & 
  \mathit{eq}  & ::=  &   f\ (D\ \overline{x}) = r    \\[0.15in] 

{\emphbf{Equation to Type}} &
\quad \embed{f\ (D\ \overline{x}) = r} & \defeq & D :: \overline{\tbind{x}{\tau}} \rightarrow \tref{\mathtt{v}}{\tau}{}{f\ \mathtt{v} = r}
\end{array}
$$
\caption{Syntax of Measures.}
\label{fig:measures}
\end{figure}



So far, all our examples have used only integer and boolean expressions in refinements.
To describe properties of algebraic data types, we use \emph{measures},
introduced in prior work on Liquid Types~\cite{LiquidPLDI09}.
%
Measures are inductively defined functions that can be used in refinements and
provide an efficient way to axiomatize properties of data types.
%
For example, @emp@ determines whether a list is empty:
%
\begin{code}
  measure emp  :: [Int] -> Bool
    emp []     = true
    emp (x:xs) = false
\end{code}
The syntax for measures deliberately looks like Haskell, but it is \emph{far} more
restricted, and should really be considered as a separate language.
A measure has exactly one argument and is defined by a list of equations,
each of which has a simple pattern on the left hand side (Figure~\ref{fig:measures}).
The right-hand side of the equation is a refinement expression $r$.
Measure definitions are typechecked in the usual way; we omit the typing rules which are standard.
(Our metatheory does not support type polymorphism,
so here we simply reason about lists of integers;
however, our implementation supports polymorphism).

\paragraph{Denotational semantics}
The denotational semantics of types in \hlang in \Sref{sec:den-sem} is readily extended to
support measures.  In \hlang a refinement $r$ is an arbitrary expression and
calls to a measure are evaluated in the usual way by pattern matching.
For example, with the above definition of @emp@ it is straightforward to show that
\begin{align}
  \mathtt{[1, 2, 3]} \dcolon \tref{\mathtt{v}}{[\tint]}{}{\mathtt{not}\ (\mathtt{emp}\ \mathtt{v})} \label{type:len}
\end{align}
as the refinement @not (emp ([1, 2, 3]))@ evaluates to $\tttrue$.

\mypara{Measures as Axioms}
How can we reason about invocations of measures in the decidable logic of VCs?
A natural approach is to treat a measure like @emp@ as an uninterpreted function
and add logical axioms that capture its behaviour. This looks easy: each equation 
of the measure definition corresponds to an axiom, thus:
%
\begin{align*}
\ttemp\ \hnull &= \tttrue\\
\forall \ttx, \ttxs.\, \ttemp\ (\ttx:\ttxs) &= \ttfalse
\end{align*}
%
Under these axioms the judgement~\ref{type:len} is indeed valid. 
% % Measures as data constructor refinements

\mypara{Measures as Refinements in Types of Data Constructors}
Axiomatizing measures is \emph{precise}; that is, 
the axioms exactly capture the meaning of measures.
Alas, axioms render SMT solvers \emph{inefficient}, and render the VC mechanism \emph{unpredictable}, 
as one must rely on various brittle syntactic matching and instantiation heuristics~\cite{simplifyj}.

Instead, we use a different approach that is \emph{both} precise \emph{and} efficient.
The key idea is this: \emph{instead of translating each measure equation into an axiom, 
we translate each equation into a refined type for the corresponding data constructor}~\citep{LiquidPLDI09}.
This translation is given in Figure~\ref{fig:measures}.
For example, the definition of the measure @emp@ yields the following refined types for the list data constructors:
$$
\begin{array}{lcl}
\hnull  & :: & \ttreft{v}{[\tint]}{emp\ v = true}\\
{:}  & :: & \tfun{\ttx}{\tint}{\tfun{\ttxs}{[\tint]}{\ttreft{v}{[\tint]}{emp\ v = false}}}
\end{array}
$$
These types ensure that:
%
~(1) each time a list value is \emph{constructed}, 
its type carries the appropriate emptiness information. 
Thus our system is able to statically decide that 
(\ref{type:len}) is valid and
~(2) each time a list value is \emph{matched}, 
the appropriate emptiness information is used to 
improve precision of pattern matching, as we see next.

\mypara{Using Measures}
\label{sec:pattern-match}
As an example, we use the measure @emp@ to 
provide an appropriate type for the @head@ function:
%
\begin{code}
  head    :: {v:[Int] | not (emp v)} -> Int 
  head xs = case xs of
              (x:_) -> x
              []    -> error "yikes"  

  error   :: {v:String | false} -> a
  error   = undefined
\end{code}
%
@head@ is safe as its input type stipulates that it will only 
be called with lists that are \emph{not} @[]@, and so
@error "..."@ is dead code.
%
The call to @error@ generates the subtyping query
%
\begin{align*}
   \tbind{\ttxs}{\tref{\ttxs}{[\tint]}{\trivial}{\lnot (\ttemp\ \ttxs)}}, \
   \tbind{\ttb}{\tttref{\ttb}{[\tint]}{\trivial}{(\ttemp\ \ttxs)= true}} 	
	 & \vdash \subtref{\tttrue}{\ttfalse} 
\end{align*}
%
The match-binder $\ttb$ holds the result of the 
match~\cite{SulzmannCJD07}. In the \texttt{[]} case,
we assign it the refinement of the type of \texttt{[]} 
which is $(\ttemp\ \ttxs) = \tttrue$. %~\cite{LiquidPLDI09}.
%
Since the call is done inside a @case-of@ expressions 
both @xs@ and @b@ are in WHNF,
thus they have \Wnf types. 
  
The verifier \emph{accepts} the program as the above subtyping reduces to the valid VC:
\begin{align*}
\lnot (\ttemp\ \ttxs) \wedge ((\ttemp\ \ttxs)= \tttrue) \Rightarrow\ & \tttrue \Rightarrow\ \ttfalse
\end{align*}
%
Thus, our system supports idiomatic 
Haskell, \eg taking the @head@ of an infinite list:
%
\begin{code}
  ex x     = head (repeat x)
  
  repeat   :: Int -> {v:[Int] | not (emp v)}
  repeat y = y : repeat y
\end{code}
%

\mypara{Multiple Measures}
If a type has multiple measures, we simply refine each data constructor's type
with the \emph{conjunction} of the refinements from each measure.
%
For example, consider a measure that computes the length of a list:
\begin{code}
  measure len  :: [Int] -> Int
    len ([])   = 0
    len (x:xs) = 1 + len xs
\end{code}
%
Using the translation of Figure~\ref{fig:measures},
we get the following types for list's data constructors.
%
\begin{align*}
\text{[]}  & ::  \ttreft{v}{[\tint]}{len\ v = 0}\\
{:}  & ::  \tfun{\ttx}{\tint}{\tfun{\ttxs}{[\tint]}{\ttreft{v}{[\tint]}{len\ v = 1 + (len\ xs)}}}\\
\intertext{The final types for list data are the 
conjunction of the refinements from $\mathtt{len}$ and $\mathtt{emp}$:}\\
\text{[]}  & ::  \ttreft{v}{[\tint]}{emp\ v = true \land len\ v = 0}\\
{:}  & ::  \tfun{\ttx}{\tint}{\tfun{\ttxs}{[\tint]}
           {\ttreft{v}{[\tint]}{emp\ v = false \land len\ v = 1 + (len\ xs)}}}
\end{align*}



\section{Declarative Typing: \texorpdfstring{\undeclang}{LamU}}\label{sec:language}\label{sec:undec}

Next, we formalize our stratified refinement type system, in two steps.
%
First,  we present a core calculus \undeclang, 
with a general $\beta$-reduction semantics. We describe the syntax,
operational semantics, and sound but undecidable declarative typing 
rules for \undeclang. 
%
Second, in \Sref{sec:typing}, we describe \logiclang, a subset 
of \undeclang that forms a decidable logic of refinements and 
use it to obtain \declang with decidable SMT-based algorithmic typing.

\subsection{Syntax}\label{sec:undec:syntax} 

\begin{figure}[!t]
\centering
\captionsetup{justification=centering}
$$
\begin{array}{rrcl}

\emphbf{Constants} \quad 
  & c & ::=    & 0,1,-1,\ldots \spmid \etrue, \efalse \\
  &   & \spmid & +,-,\ldots \spmid =, <, \ldots \spmid \ecrash 
  \\[0.05in]

\emphbf{Values} \quad 
  & \val & ::= &  c \spmid \efun{x}{\typ}{e} \spmid \edapp{D}{e}
  \\[0.05in] 

\emphbf{Expressions} \quad 
  & e & ::=    & \val \spmid x \spmid \eapp{e}{e} \spmid \elet{x}{e}{e} \\ 
  &   & \spmid & \ecase{e}{D}{\overline{x}}{e}{x} \\[0.05in] 

\emphbf{Refinements} \quad 
  & r & ::= &   e \\[0.05in] 

\emphbf{Basic Types} \quad 
  & \Base & ::= & \tint \spmid \tbool \spmid \ttct \\[0.05in] 

\emphbf{Types} \quad 
  & \typ & ::= & \tref{v}{\Base}{}{r} \spmid \tfunref{x}{\typ}{\typ}{v}{e} \\[0.1in]
% \hrule width 0.48\textwidth
\emphbf{Contexts} \quad 
  & C
  & ::= 
  &   	 \bullet 
  \spmid \eapp{C}{e} 
  \spmid \eapp{c}{C} 
  \spmid D\ \overline{e}\ C\ \overline{e}\\
  &&\spmid &
  \ecase{C}{D}{\overline{y}}{e}{x}
  \\[0.05in] 
\end{array}
$$

\judgementHead{Reduction}{\eval{e}{e}}

$$
\begin{array}{rcl}
\eval{C[e]&}{&C[e']} \qquad \text{if}\ \eval{e}{e'} \\
	\eval{\eapp{c}{v}&}{& \ceval{c}{v}}\\
\eval{\eapp{(\efun{x}{\tau_x}{e})}{e_x}&}{&e\sub{x}{e_x}}\\
	\eval{\elet{x}{e_x}{e}&}{&e\sub{x}{e_x}} \\
	\eval{\ecase{D_j\ \overline{e}}{D_i}{\overline{y_i}}{e_i}{x}&}
	{&e_j\sub{x}{D_j\ \overline{e}}\sub{\overline{y_j}}{\overline{e}}} \\
\end{array}
$$

\caption{Syntax and Operational Semantics of $\protect \undeclang$.}
\label{fig:undeclang}
\label{fig:operational}
\end{figure}


Figure~\ref{fig:undeclang} summarizes the syntax of \undeclang, 
which is essentially the calculus \hlang~\cite{Knowles10} 
\emph{without} the dynamic checking features (like casts), but 
\emph{with} the addition of data constructors.
%
In \undeclang, as in \hlang, refinement expressions $r$ are not drawn from a decidable 
logical sublanguage, but can be arbitrary expressions $e$
(hence $r ::= e$ in Figure~\ref{fig:undeclang}). 
This choice allows us to prove preservation and progress, 
but renders typechecking undecidable. 
 
%The syntactic elements of \undeclang are layered into 
%primitive constants, values, and expressions.

\spara{Constants}
The primitive constants of \undeclang include  
$\tttrue$, $\ttfalse$, $\mathtt{0}$, $\mathtt{1}$, $\mathtt{-1}$, \etc,
and arithmetic and logical operators like $\mathtt{+}$, $\mathtt{-}$, 
$\mathtt{\leq}$,$\mathtt{/}$, $\mathtt{\land}$, $\mathtt{\lnot}$.
%
In addition, we include a special \emph{untypable} constant $\ecrash$ 
that models ``going wrong''. Primitive operations return a $\ecrash$
when invoked with inputs outside their domain, \eg when $\mathtt{/}$ 
is invoked with $\mathtt{0}$ as the divisor or when $\mathtt{assert}$ is 
applied to $\mathtt{false}$.

\spara{Data Constructors}
We encode data constructors as special constants. 
Each data type has an arity $\arity{T}$ that represents
the exact number of data constructors that return a value of 
type $T$.
%
For example the data type \tintlist, which represents 
lists of integers, has two data constructors: $\dnull$ and $\dcons$,
\ie has arity $2$.
%%$D^\tintlist_1 \defeq \dnull$ and
%% $D^\tintlist_2 \defeq \dcons$.


\spara{Values \& Expressions}
The values of \undeclang include constants, 
$\lambda$-abstractions $\efun{x}{\typ}{e}$, and 
fully applied data constructors $D$ that wrap expressions.
%
The expressions of \undeclang include values, as well as
variables $x$, 
applications $\eapp{e}{e}$, 
and the $\mathtt{case}$ 
and $\mathtt{let}$ expressions.

\subsection{Operational Semantics}

Figure~\ref{fig:operational} summarizes the small 
step, contextual $\beta$-reduction semantics for 
\undeclang.
%
We allow for reductions under data constructors, 
and thus, values may be further reduced.
%
We write \evalj{e}{e'}{j} if there exist $e_1,\ldots,e_j$ such that
$e$ is $e_1$, $e'$ is $e_j$ and $\forall i,j, 1 \leq i < j$, we have
\eval{e_i}{e_{i+1}}.
%
We write \evals{e}{e'} if there exists a (finite) $j$ such that
$\evalj{e}{e'}{j}$.

\spara{Constants} Application of a constant requires the
argument to be reduced to a value; in a single step the 
expression is reduced to the output of the primitive 
constant operation. 
%
For example, consider $=$, the primitive equality operator 
on integers. We have $\ceval{=}{n} \defeq =_n$
where $\ceval{=_n}{m}$ equals \etrue iff $m$ is the same as $n$.

\subsection{Types}

\undeclang types include basic types, which are \emph{refined} with predicates, 
and dependent function types.
%
\emph{Basic types} $\Base$ comprise integers, booleans, and a family of data-types 
$T$ (representing lists, trees \etc).
%
For example, the data type \tintlist represents lists of integers.
%
We refine basic types with predicates (boolean valued expressions $e$) to obtain
\emph{basic refinement types} $\tref{v}{\Base}{}{e}$.
%
Finally, we have dependent \emph{function types} $\tfun{x}{\typ_x}{\typ}$ 
where the input $x$ has the type $\typ_x$ and the output $\typ$ may
refer to the input binder $x$.

\spara{Notation} We write $\Base$ to abbreviate $\tref{v}{\Base}{}{\etrue}$
and \tfunbasic{\typ_x}{\typ} to abbreviate \tfun{x}{\typ_x}{\typ} if 
$x$ does not appear in $\typ$. 
%
We use $\_$ for unused binders.
We write $\tref{v}{\tnat}{l}{r}$ to abbreviate $\tref{v}{\tint}{l}{0 \leq v \wedge r}$.


\spara{Denotations}
%
Each type $\typ$ \emph{denotes} a set of expressions $\interp{\typ}$,
that are defined via the dynamic semantics~\cite{Knowles10}.
%
Let \erase{\typ} be the type we get if we erase all refinements 
from $\typ$ and $\hastypebasesmall{\Env}{e}{\erase{\typ}}$ be the 
standard typing relation for the typed lambda calculus.
%
Then, we define the denotation of types as: 
\begin{align*}
\interp{\tref{x}{\ttbase}{}{r}} \defeq & 
    \{e \mid  \hastypebasesmall{\emptyset}{e}{\ttbase},
              \mbox{ if } \evals{e}{w} 
              \mbox{ then } \evals{\SUBST{r}{x}{w}}{\etrue} \}\\
\interp{\tfun{x}{\typ_x}{\typ}} \defeq & 
    \{e \mid  \hastypebasesmall{\emptyset}{e}{\erase{\tfunbasic{\typ_x}{\typ}}}, 
              \forall e_x \in \interp{\typ_x}.\ \eapp{e}{e_x} \in \interp{\typ\sub{x}{e_x}}
    \}
\end{align*}


\spara{Constants}
For each constant $c$ we define its type \constty{c}
such that $c \in \interp{\constty{c}}$, \eg
%
$$
\begin{array}{lcl}
\constty{3} &\doteq& \tttref{v}{\tint}{}{v = 3}\\
\constty{+} &\doteq& \tfun{\ttx}{\tint}{\tfun{\tty}{\tint}{\tttref{v}{\tint}{}{v = x + y}}}\\
\constty{/} &\doteq& \tfunbasic{\tint}{\tfunbasic{\tttref{v}{\tint}{}{v > 0}}{\tint}}\\
\constty{\eerror{\typ}} &\doteq& \tfunbasic{\tttref{v}{\tint}{}{\efalse}}{\typ}
\end{array}
$$
%
So, by definition we get the constant typing lemma.
%
\begin{lemma}{[Constant Typing]}\label{lemma:constants}
For every constant $c$, $c \in \interp{\constty{c}}$.
\end{lemma}
%
Thus, if $\constty{c} \defeq \tfun{x}{\typ_x}{\typ}$, then for every value 
$w \in \interp{\typ_x}$, we require that $\ceval{c}{w} \in \interp{\typ\sub{x}{w}}$.
%
For every value $w \not \in \interp{\typ_x}$, it suffices to define $\ceval{c}{w}$
as \ecrash, a special untyped value.

\spara{Data Constructors}
%As discussed in ~\Sref{sec:measures}, 
The types of data constructor constants are refined 
with predicates that track the semantics of the 
\emph{measures} associated with the data type.
%
For example, as discussed in \Sref{sec:measures} 
we use @emp@ to refine the list data constructors' types:
$$
\begin{array}{lcl}
\constty{\dnull}  & \defeq & \tttref{v}{\tintlist}{}{\eisNull{v}}\\
\constty{\dcons}  & \defeq & \tfunbasic{\tint}{\tfunbasic{\tintlist}{\tttref{v}{\tintlist}{}{\lnot (\eisNull{v})}}}
\end{array}
$$
%
By construction it is easy to prove that Lemma~\ref{lemma:constants}
holds for data constructors.
%
For example, $\ttemp\ \dnull$ goes to $\tttrue$.
%%We \emph{compose} multiple measures for a type by 
%%refining the constructors with the \emph{conjunction} 
%%of each measure's refinements.
%


\subsection{Type Checking}\label{subsec:typing}

\newcommand\restrictdecidable[2]{#2}
\begin{figure}[t!]
\centering
\captionsetup{justification=centering}

\judgementHead{Well-Formedness}{\undeciswellformed{\Gamma}{\tau}}
$$
\inference
   {\undechastype{\Gamma, \tbind{v}{\Base}}
                 {\restrictdecidable{p}{r}}{\tbool}
   }
   {\undeciswellformed{\Gamma}{\tref{v}{\Base}{}{\restrictdecidable{p}{r}}}}
   [\rwbase]
\qquad
\inference{
	\undeciswellformed{\Gamma}{\tau_x} &&
	\undeciswellformed{\Gamma, \tbind{x}{\tau_x}}{\tau}
}{
	\undeciswellformed{\Gamma}{\tfunref{x}{\tau_x}{\tau}{v}{e}}
}[\rwfun]
$$

\judgementHead{Subtyping}{\undecissubtype{\Gamma}{\tau_1}{\tau_2}}

$$
\inference{
  {\forall \sto\in \interp{\Env}. 
  		 \interp{\thetasub{\sto}{\tref{v}{\Base}{}{\restrictdecidable{p_1}{r_1}}}} 
  		\subseteq   \interp{\thetasub{\sto}{\tref{v}{\Base}{}{\restrictdecidable{p_2}{r_2}}}}}
}{
	\undecissubtype{\Gamma}
		{\tref{v}{\Base}{}{\restrictdecidable{p_1}{r_1}}}
		{\tref{v}{\Base}{}{\restrictdecidable{p_2}{r_2}}}
}[\rsubbase]
$$
$$
\inference{
	\undecissubtype{\Gamma}{\tau'_x}{\tau_x} &&
	\undecissubtype{\Gamma, \tbind{x}{\tau'_x}}{\tau}{\tau'}
}{
	\undecissubtype{\Gamma}{\tfunref{x}{\tau_x}{\tau}{v}{e_1}}{\tfunref{x}{\tau'_x}{\tau'}{v}{e_2}}
}[\rsubfun]
$$

\judgementHead{Typing}{\undechastype{\Gamma}{e}{\tau}}
$$
\inference{
	(x,\tau) \in \Gamma 
}{
	\undechastype{\Gamma}{x}{\tau}
}[\rtvar]
\qquad
\inference{
}{
	\undechastype{\Gamma}{c}{\constty{c}}
}[\rtconst]
$$
$$
\inference{
	\undechastype{\Gamma}{e}{\tau'} &&
	\undecissubtype{\Gamma}{\tau'}{\tau} &&
	\undeciswellformed{\Gamma}{\tau} &&
}{
	\undechastype{\Gamma}{e}{\tau}
}[\rtsub]
$$
$$
\inference{
	\undechastype{\Gamma, \tbind{x}{\tau_x}}{e}{\tau} &&
	\undeciswellformed{\Gamma}{\tau_x}
}{
	\undechastype{\Gamma}{\efun{x}{\tau_x}{e}}{(\tfun{x}{\tau_x}{\tau})}
}[\rtfun]
$$
$$
\inference{
	\undechastype{\Gamma}{e_1}{(\tfunref{x}{\tau_{x}}{\tau}{v}{e_v})} &&
	\undechastype{\Gamma}{\restrictdecidable{y}{e_2}}{\tau_{x}}
}{
	\undechastype{\Gamma}{\eapp{e_1}{\restrictdecidable{y}{e_2}}}{\tau\sub{x}{\restrictdecidable{y}{e_2}}}
}[\rtapp]
$$
$$
\inference{
	\undechastype{\Gamma}{e_x}{\tau_{x}} &&
	\undechastype{\Gamma,\tbind{x}{\tau_x}}{e}{\tau} &&
	\undeciswellformed{\Gamma}{\tau}
}{
	\undechastype{\Gamma}{\elet{x}{e_x}{e}}{\tau}
}[\rtlet]
$$

$$\inference{
	\undechastype{\Gamma}{e}{\tref{v}{T}{}{r}} &&
	 \undeciswellformed{\Gamma}{\tau}\\
     \forall i. \constty{D_i} = \overline{\tbind{y_j}{\tau_j}} \rightarrow \tref{v}{T}{}{r_i} &&
      \undechastype{\Gamma, \overline{\tbind{y_j}{\tau_j}}, \tbind{x}{\tref{v}{T}{}{r \land r_i}}}{e_i}{\tau}  \\
}{
	\undechastype{\Gamma}{\ecase{e}{D_i}{\overline{y_j}}{e_i}{x}}{\tau}
}[\rtcase]$$
\caption{Type-checking for \undeclang}
\label{fig:typing}
\end{figure}



Next, we present the type-checking judgments and rules of \undeclang. 

\spara{Environments and Closing Substitutions}
A \emph{type environment} $\Env$ is a sequence of type bindings 
$\tbind{x_1}{\typ_1},\ldots,\tbind{x_n}{\typ_n}$. An environment
denotes a set of \emph{closing substitutions} $\sto$ which are 
sequences of expression bindings: 
$\gbind{x_1}{e_1}, \ldots, \gbind{x_n}{e_n}$ such that:
$$
\interp{\Env} \defeq  \{\sto \mid \forall \tbind{x}{\typ} \in \Env. 
                                    \sto(x) \in \interp{\thetasub{\sto}{\typ}} \}
$$

\spara{Judgments}
We use environments to define three kinds of
rules: Well-formedness, Subtyping, 
and Typing~\cite{Knowles10,GordonTOPLAS2011}.
%
%\spara{Well-formedness}
A judgment \undeciswellformed{\Env}{\typ} states that 
the refinement type $\typ$ is well-formed in 
the environment $\Env$.
%
Intuitively, the type $\typ$ is well-formed if all
the refinements in $\typ$ are $\tbool$-typed in $\Env$.
%
%\spara{Subtyping} 
A judgment \undecissubtype{\Env}{\typ_1}{\typ_2} states 
that the type $\typ_1$ is a subtype of %the type 
$\typ_2$ in the environment $\Env$.
%
Informally, $\typ_1$ is a subtype of $\typ_2$ if, when 
the free variables of $\typ_1$ and $\typ_2$ 
are bound to expressions described by $\Env$,
the denotation of $\typ_1$ 
is \emph{contained in} the denotation of $\typ_2$. 
%
Subtyping of basic types reduces to denotational containment checking.
%
%\spara{Implication} 
%%A judgment \issubref{\Env}{p_1}{p_2} states 
%%that the predicate $p_1$ \emph{implies} 
%%the predicate $p_2$ in the environment $\Env$.
%
That is, for any closing substitution $\sto$
in the denotation of $\Env$, for every expression $e$, 
if $e \in \interp{\thetasub{\sto}{\typ_1}}$ then 
$ e \in \interp{\thetasub{\sto}{\typ_2}}$.
%
%\spara{Typing}
A judgment \undechastype{\Env}{e}{\typ} states that
the expression $e$ has the type $\typ$ in 
the environment $\Env$.
That is, when the free variables in $e$ are 
bound to expressions described by $\Env$, the 
expression $e$ will evaluate to a value 
described by $\typ$.

\mypara{Soundness}
In~\cite{vazou14techrep}, we use the (undecidable) \rsubbase to prove that each step 
of evaluation preserves typing and that if an expression
is not a value, then it can be further evaluated:
%
\begin{itemize}
\item\textbf{Preservation:} 
	If \undechastype{\emptyset}{e}{\typ} and \eval{e}{e'}, 
	then \undechastype{\emptyset}{e'}{\typ}. 
\item\textbf{Progress:}
	If \undechastype{\emptyset}{e}{\typ} and $e \not = w$,
	then \eval{e}{e'}. 
\end{itemize}
%
We combine the above to prove that evaluation preserves 
typing and that a well typed term will not \ecrash.
%
\begin{theorem}{[Soundness of \undeclang]}\label{thm:refinedhaskell:safety}
\begin{itemize}
\item\textbf{Type-Preservation:} If \undechastype{\emptyset}{e}{\typ}, %NV with the v -> w edit this didn't fit in 1 line
       $\evals{e}{w}$ then $\undechastype{\emptyset}{w}{\typ}$.
\item\textbf{Crash-Freedom:} If \undechastype{\emptyset}{e}{\typ} 
        then $\evals{e\not}{\ecrash}$.
\end{itemize}
\end{theorem}

We prove the above following the overall recipe of~\cite{Knowles10}. 
Crash-freedom follows from type-preservation, as \ecrash has no type.
%
The Substitution Lemma, in particular, follows from a connection between
the typing relation and type denotations:

\begin{lemma}{[Denotation Typing]}\label{lem:denotation}
If $\undechastype{\emptyset}{e}{\typ}$ then $e \in \interp{\typ}$.
\end{lemma} 

%%% Local Variables: 
%%% mode: latex
%%% TeX-master: "main"
%%% End: 

\begin{figure}
\emphbf{Well Formedness}\hfill{\fbox{\iswellformed{\env}{\typ}}}\\

$$
\inference{
	\hastype{\env,\tbind{v}{\btyp}}{\refa}{\tbool}
}{
	\iswellformed{\env}{\tref{v}{\btyp}{\refa}}
}[\rwbase]
\inference{
	\iswellformed{\env}{\typ_x} &
	\iswellformed{\env,\tbind{x}{\typ_x}}{\typ}
}{
	\iswellformed{\env}{\tfun{x}{\typ_x}{\typ}}
}[\rwfun]
$$

\emphbf{Subtyping}\hfill{\fbox{\issubtype{\env}{\typ_1}{\typ_2}}}\\

$$
\inference{
	\forall \sub\in\interp{\env}.
	\interp{\applysub{\sub}{\tref{v}{\btyp}{\refa_1}}}
	\subseteq
	\interp{\applysub{\sub}{\tref{v}{\btyp}{\refa_2}}}
}{
	\issubtype{\env}{\tref{v}{\btyp}{\refa_1}}{\tref{v}{\btyp}{\refa_2}}
}[\rsubbase]
$$

$$
\inference{
	\issubtype{\env}{\typ_x'}{\typ_x} &
	\issubtype{\env,\tbind{x}{\typ_x'}}{\typ}{\typ'}
}{
	\issubtype{\env}{\tfun{x}{\typ_x}{\typ}}{\tfun{x}{\typ_x'}{\typ'}}
}[\rsubfun]
$$

\emphbf{Typing}\hfill{\fbox{\hastype{\env}{\prog}{\typ}}}\\
$$
\inference{
	\tbind{x}{\gtyp}\in\env
}{
	\hastype{\env}{x}{\gtyp}
}[\rtvar]
\qquad
\inference{
}{
	\hastype{\env}{c}{\constty{c}}
}[\rtconst]
$$

$$
\inference{
	\hastype{\env}{\prog}{\typ'} &
	\issubtype{\env}{\typ'}{\typ}
}{
	\hastype{\env}{\prog}{\typ}
}[\rtsub]
$$
$$
\inference{
    % \haseq{\btyp} &
	\hastype{\env}{e}{\tref{v}{\btyp}{\reft_r}}
}{
	\hastype{\env}{e}{\tref{v}{\btyp}{\reft_r\land v = e}}
}[\rtexact]
$$
$$
\inference{
	\hastype{\env, \tbind{x}{\typ_x}}{e}{\typ}
}{
	\hastype{\env}{\efun{x}{\typ}{e}}{\tfun{x}{\typ_x}{\typ}}
}[\rtfun]
$$

$$
\inference{
	\hastype{\env}{e_1}{(\tfun{x}{\typ_x}{\typ})} &&
	\hastype{\env}{e_2}{\typ_x}
}{
	\hastype{\env}{e_1\ e_2}{\typ}
}[\rtapp]
$$

%%% $$
%%% \inference{
	%%% \hastype{\env}{e}{\gtyp_x} &
	%%% \hastype{\env, \tbind{x}{\gtyp_x}}{\prog}{\typ} &
	%%% \iswellformed{\env}{\typ}
%%% }{
	%%% \hastype{\env}{\elet{x}{e}{\gtyp_x}{}{\prog}}{\typ}
%%% }[\rtlet]
%%% $$
%%%
%%% $$
%%% \inference{
	%%% \hastype{\env, \tbind{x}{\gtyp_x}}{e}{\gtyp_x} &
	%%% \hastype{\env, \tbind{x}{\gtyp_x}}{\prog}{\typ} &
	%%% \iswellformed{\env}{\typ}
%%% }{
	%%% \hastype{\env}{\eletrec{x}{e}{\gtyp_x}{}{\prog}}{\typ}
%%% }[\rtletrec]
%%% $$
%% \NV{For soundness, it is important that f cannot appear in $\gtyp_2$}
$$
\inference
	{\hastype{\env, \tbind{x}{\gtyp_x}}{\bd_x}{\gtyp_x} &
	 \iswellformed{\env, \tbind{x}{\gtyp_x}}{\typ_x} \\
	 \hastype{\env, \tbind{x}{\gtyp_x}}{\bd}{\gtyp} &
	 \iswellformed{\env}{\typ}}
	{\hastype{\env}{\eletb{x}{\gtyp_x}{\bd_x}{\bd}}{\typ}}
	[\rtlet]
$$

$$
\inference
	{\hastype{\env}
	 				 {\eletb{f}{\exacttype{\gtyp_f}{e}}{e}{\prog}}
					 {\typ}
	}
	{\hastype{\env}
					 {\erefb{f}{\gtyp_f}{e}{\prog}}
					 {\typ}
	}[\rtreflect]
$$

$$
\inference{
	\hastype{\env}{e}{\tref{v}{T}{e_r}} & \iswellformed{\env}{\typ} \\
	& \forall i. \constty{\dc_i} = \tfunbasic{\overline{\tbind{y_j}{\typ_j}}}{\tref{v}{T}{e_{r_i}}} \\
	& \hastype{\env, \overline{\tbind{y_j}{\typ_j}}, \tbind{x}{\tref{v}{T}{e_r \land e_{r_i}}} }{e_i}{\typ}
}{
	\hastype{\env}{\ecase{x}{e}{\dc_i}{\overline{y_i}}{e_i}}{\typ}
}[\rtcase]
$$
$$
\inference{
	\iswellformed{\env}{\typ}
}{
	\hastype{\env}
	        {\efix{}}
	        {\tfun{x}{\typ}{\tfun{y}{\typ}{\typ}}}
}[\rtfix]
$$
\caption{Typing of \corelan}
\label{fig:typing}
\end{figure}

% \newcommand\rectyp{\FinTy{\tnat}}

\newcommand\factyp{\typ}

\section{Implementation: \toolname}\label{sec:haskell}

We have implemented \declang in \toolname (\Sref{sec:overview}). 
Next, we describe the key steps in the transition from \declang to \lhaskell.

\subsection{Termination}

%Let ${\factyp \defeq \tfunbasic{\FinTy{\tnat}}{\FinTy{\tnat}}}$.
Haskell's recursive functions of type ${\tfunbasic{\FinTy{\tnat}}{\typ}}$ 
are represented, in GHC's Core \cite{SulzmannCJD07} as
${\mathtt{let\ rec}\ f = \ \efun{n}{}{e}}$ which is operationally
equivalent to ${\mathtt{let}\ f = \ \etfix{\typ}\ (\efun{n}{}{\efun{f}{}{e}})}$.
Given the type of $\etfix{\typ}$, checking that $f$ has 
type $\tfunbasic{\FinTy{\tnat}}{\typ}$ reduces to checking
$e$ in a \emph{termination-weakened environment} where
$$f \ \colon \ \tfunbasic{\tref{v}{\tnat}{\finite}{v < n}}{\typ}$$
%
Thus, \toolname proves termination just as \declang 
does: by checking the body in the above environment, 
where the recursive binder is called with $\tnat$ 
inputs that are strictly smaller than $n$.

\mypara{Default Metric}
For example, \toolname proves that 
%
\begin{code}
  fac n = if n == 0 then 1 else n * fac (n-1)
\end{code}
%
has type $\tfunbasic{\FinTy{\tnat}}{\FinTy{\tnat}}$ 
by typechecking the body of @fac@ 
in a termination-weakened environment 
${\mathtt{fac}\ : \tfunbasic{\tref{\ttv}{\tnat}{\finite}{\ttv < \ttn}}{\FinTy{\tnat}}}$
The recursive call generates the subtyping query:
\begin{align*}
\tbind{\ttn}{\{0 \leq \ttn\}}, \lnot (\ttn = 0) \vdash_D &\  \subt{\ttref{v=n-1}}{\ttref{0 \leq v \wedge v < n}}\\ 
%\tlref{n}{\tint}{\finite}{0 \leq n}, \lnot (n = 0) \vdash_D &\  \subt{\ttref{v=n-1}}{\ttref{0 \leq v \wedge v < n}}\\ 
\intertext{Which reduces to the valid VC}
0 \leq \ttn \wedge \lnot (\ttn = 0) \Rightarrow &\   (\ttv = \ttn-1) \Rightarrow (0 \leq \ttv \wedge \ttv < \ttn)
\end{align*}
%
proving that $\mathtt{fac}$ terminates, in essence because the
\emph{first parameter} forms a \emph{well-founded decreasing metric}.

\mypara{Refinements Enable Termination} 
Consider Euclid's GCD:
%
\begin{code}
  gcd :: a:Nat -> {v:Nat | v < a} -> Nat 
  gcd a 0 = a
  gcd a b = gcd b (a `mod` b)
\end{code}
%
Here, the first parameter is decreasing, but this requires
the fact that the second parameter is smaller than the first 
and that @mod@ returns results smaller than its second 
parameter. Both facts are easily expressed as refinements, 
but elude non-extensible checkers~\cite{Giesl11}.

\mypara{Explicit Termination Metrics}
The indexed-fixpoint combinator technique is easily extended to 
cases where some parameter \emph{other} than the first is the 
well-founded metric. For example, consider: 
% As an example, consider the tail-recursive factorial:
%
\begin{code}
  tfac     :: Nat -> n:Nat -> Nat / [n] 
  tfac x n | n == 0    = x
           | otherwise = tfac (n*x) (n-1)
\end{code}
%
We specify that the \emph{last parameter} is decreasing by 
specifying an explicit termination metric @/ [n]@ in the 
type signature.
%
\toolname \emph{desugars} the 
termination metric into a new $\tnat$-valued \emph{ghost parameter} @d@ 
whose value is always equal to the termination metric @n@:
\begin{code}
  tfac :: d:Nat -> Nat -> {n:Nat | d = n} -> Nat 
  tfac d x n | n == 0    = x
             | otherwise = tfac (n-1) (n*x) (n-1)
\end{code}
%
Type checking, as before, checks the body in an environment where 
the first argument of @tfac@ is weakened, \ie, requires proving @d > n-1@.
%
So, the system needs to know that the ghost argument @d@ 
represents the decreasing metric.
%
We capture this information in the type signature of @tfac@ where the \emph{last} 
argument exactly specifies that @d@ is the termination metric @n@, \ie, @d = n@.
%
Note that since the termination metric can depend on any argument, 
it is important to refine the last argument,
so that all arguments are in scope, with the fact that @d@ is the termination metric.

To generalize, desugaring of termination metrics proceeds as follows.
Let $f$ be a recursive function with parameters $\overline{x}$, and 
termination metric $\mu(\overline{x})$. Then \toolname will
\begin{itemize}
\item add a $\tnat$-valued ghost first parameter $d$ in the definition of $f$, 
\item weaken the last argument of $f$ with the refinement $d = \mu(\overline{x})$, %and
\item at each recursive call of $f\ \overline{e}$, 
apply $\mu(\overline{e})$ as the first argument.
\end{itemize}  
%
%%As will shall see this technique can be used 
%%when the termination metric is any logical expression.

\mypara{Explicit Termination Expressions} 
Let us now apply the previous technique in a function where
none of the parameters themselves decrease across recursive calls,
but there is some \emph{expression} that forms the decreasing metric.
%
%Sometimes, none of the parameters themselves decrease across recursive calls,
%but there is some \emph{expression} that forms the decreasing metric.
%
Consider @range lo hi@, which returns the list of 
@Int@s from @lo@ to @hi@:
%
We generalize the explicit metric specification to 
\emph{expressions} like @hi-lo@. \toolname \emph{desugars} the 
expression into a new $\tnat$-valued \emph{ghost parameter} 
whose value is always equal to @hi-lo@, that is:
\begin{code}
  range :: lo:Nat -> {hi:Nat | hi >= lo} -> [Nat] 
         / [hi-lo]
  range lo hi 
    | lo < hi = lo : range (lo + 1) hi
    | _       = [] 
\end{code}
%
Here, neither parameter is decreasing (indeed, the first one
is \emph{increasing}) but @hi-lo@ decreases across each call. 
%
We generalize the explicit metric specification to 
\emph{expressions} like @hi-lo@. \toolname \emph{desugars} the 
expression into a new $\tnat$-valued \emph{ghost parameter} 
whose value is always equal to @hi-lo@, that is:
%
\begin{code}
  range lo hi = go (hi-lo) lo hi
    where 
      go :: d:Nat -> lo:Nat 
         -> {hi:Nat | d = hi - lo} -> [Nat]
      go d lo hi
       | lo < hi = l : go (hi-(lo+1)) (lo+1) hi 
       | _       = []
\end{code}
%
After which, it proves @go@ terminating, by showing 
that the first argument @d@ is a \tnat that decreases across each 
recursive call (as in @fac@ and @tfac@).

\mypara{Recursion over Data Types}
The above strategy generalizes easily to functions that recurse
over (finite) data structures like arrays, lists, and trees.
In these cases, we simply use \emph{measures} to project the 
structure onto \tnat, thereby reducing the verification to 
the previously seen cases. For each user defined type, \eg
%
\begin{code}
  data L [sz] a = N | C a (L a)
\end{code}
%
we can define a \emph{measure}
%
\begin{code}
  measure sz  :: L a -> Nat
    sz (C x xs) = 1 + (sz xs)
    sz N        = 0
\end{code}
%
We prove that @map@ terminates using the type:
%
\begin{code}
  map :: (a -> b) -> xs:L a -> L b / [sz xs]
  map f (C x xs) = C (f x) (map f xs)
  map f N        = N
\end{code}
%
That is, by simply using @(sz xs)@  as the 
decreasing metric.

\mypara{Generalized Metrics Over Datatypes}
Finally, in many functions there is no single argument 
whose (measure) provably decreases. For example, consider:
%
\begin{code}
  merge :: xs:_ -> ys:_ -> _ / [sz xs + sz ys]
  merge (C x xs) (C y ys)
    | x < y     = x `C` (merge xs  (y `C` ys))
    | otherwise = y `C` (merge (x `C` xs)  ys)
\end{code}
%
from the homonymous sorting routine. Here, neither parameter
decreases, but the \emph{sum} of their sizes does. 
%
As before \toolname desugars the decreasing expression into 
a ghost parameter and thereby proves termination (assuming, 
of course, that the inputs were finite lists, \ie 
$\FinTy{\mathtt{L}}\ a$.)

\mypara{Automation: Default Size Measures}
Structural recursion on the first argument is a common pattern 
in \lhaskell code.
%
\toolname automates termination proofs for this common case,
by allowing users to specify a \emph{size measure} 
for each data type, (\eg @sz@ for @L a@).
%
Now, if \emph{no} termination metric is given, by default 
\toolname assumes that the \emph{first} argument whose type
has an associated size measure decreases.
%
Thus, in the above, we need not specify metrics for @fac@ 
or @gcd@ or @map@ as the size measure is automatically 
used to prove termination. 
%
This simple heuristic allows us to {automatically}
prove 67\% of recursive functions terminating.

%%% \mypara{Summary}
%%% To sum up, 
%%% %
%%% \begin{itemize}
%%% \item No termination check for functions marked @lazy@, 
%%% \item If no explicit termination metrice, then first 
%%%       argument with size measure used by default,
%%% \item Otherwise, explicit termination metric desugared 
%%%       into ghost @nat@ parameter that is used to prove 
%%%       termination.
%%% \end{itemize}

\subsection{Non-termination}

By default, \toolname checks that every function is 
terminating. We show in \Sref{sec:evaluation} that 
this is in fact the overwhelmingly common case in practice.
%
However, annotating a function as @lazy@ deactivates 
\toolname's termination check (and marks the result as a 
\Div type).
%
This allows us to check functions that are 
non-terminating, and allows \toolname to prove safety 
properties of programs that manipulate \emph{infinite} 
data, such as streams, which arise idiomatically with 
Haskell's lazy semantics.
% 
For example, consider the classic @repeat@ function:
%
\begin{code}
  repeat x = x `C` repeat x
\end{code}
%
We cannot use the $\etfix{}$ combinators to 
represent this kind of recursion, and hence, 
use the non-terminating $\efix{}$ combinator 
instead. 

Let us see how we can use refinements to statically 
distinguish between finite and infinite streams. 
The direct, \emph{global} route of using a measure
%
\begin{code}
  measure inf    :: L a -> Prop 
    inf (C x xs) = inf xs
    inf N        = false 
\end{code}
%
to describe infinite lists is unavailable as such 
a measure, and hence, the corresponding refinement
would be non-terminating.
%
Instead, we describe infinite lists in \emph{local} 
fashion, by stating that each \emph{tail} is non-empty.

\mypara{Step 1: Abstract Refinements} 
We can parametrize a datatype with abstract 
refinements that relate sub-parts of the 
structure \cite{vazou13}. 
For example, we parameterize the list type as:
%
\begin{code}
  data L a <p :: L a -> Prop> 
    = N | C a {v: L<p> a | (p v)}
\end{code}
%
which parameterizes the list with a refinement 
@p@ which holds \emph{for each} tail of the list, 
\ie holds for each of the second arguments to 
the @C@ constructor in each sub-list.


\mypara{Step 2: Measuring Emptiness} Now, we can write a measure that 
states when a list is \emph{empty}
%
\begin{code}
  measure emp  :: L a -> Prop 
    emp N        = true
    emp (C x xs) = false
\end{code}
%
As described in \Sref{sec:typing}, \toolname translates the 
abstract refinements and measures into refined types for 
@N@ and @C@.

\mypara{Step 3: Specification \& Verification}
Finally, we can use the abstract refinements and measures to 
write a type alias describing a refined version of @L a@ 
representing infinite streams:
%
\begin{code}
  type Stream a = 
    {xs: L <{\v -> not(emp v)}> a | not(emp xs)}
\end{code}
%
We can now type @repeat@ as:
%
\begin{code}
  lazy repeat :: a -> Stream a
  repeat x    = x `C` repeat x 
\end{code}
%
The @lazy@ keyword \emph{deactivates} termination checking, and 
marks the output as a \Div type.
%
Even more interestingly, we can prove safety properties of 
infinite lists, for example:
%
\begin{code}
  take            :: Nat -> Stream a -> L a
  take 0 _        = N
  take n (C x xs) = x `C` take (n-1) xs
  take _ N        = error "never happens"
\end{code}
%
\toolname proves, similar to the @head@ example from
\Sref{sec:overview}, that we never match a @N@ when 
the input is a @Stream@.

\mypara{Finite vs. Infinite Lists}
%
Thus, the combination of refinements 
and labels allows our stratified type 
system to specify and verify whether 
a list is finite or infinite.
%
Note that:
%
$\FinTy{\mathtt{L}}\ a$ represents
\emph{finite} lists \ie those 
produced using the (inductive) 
terminating fixpoint combinators,
%
$\WnfTy{\mathtt{L}}\ a$ represents 
(potentially) infinite lists which 
are guaranteed to reduce to values, 
\ie non-diverging computations that
yield finite or infinite lists,
and
$\DivTy{\mathtt{L}}\ a$ represents 
computations that may diverge or 
produce a finite or infinite list.

\subsection{User Specifications and Type Inference}

In program verification it is common that the user provides functional
specification that the code should satisfy.
In \toolname these specifications can be provided as type signatures 
for @let@-bound variables.
%
Consider the typechecking rules of Figure~\ref{fig:typing}
that is used by \declang.
%
$$
\inference{
	\hastype{\Gamma}{e_x}{\tau_{x}} &&
	\hastype{\Gamma,x\colon\tau_x}{e}{\tau} &&
	\iswellformed{\Gamma}{\tau}
}{
	\hastype{\Gamma}{\elet{x}{e_x}{e}}{\tau}
}[\rtlet]
$$
%
Note that \rtlet \emph{guesses} an appropriate type $\tau_x$
for $e_x$ and binds this type to $x$ to typecheck $e$.

\toolname allows the user to specify the type $\tau_x$ for top level bindings.
%
For every binding \elet{x}{e_x}{\dots}, if the user provides a type specification $\tau_x$,
\toolname checks using the appropriate environment 
(1)~that the specified type is well-formed, and 
(2)~that the expression $e_x$ typechecks under the specification $\tau_x$.
%
For the other top level bindings, \ie those without user-provided specifications, 
as well as all local bindings, \toolname uses the Liquid Types~\citep{LiquidPLDI08} 
framework to infer refinement types, thus greatly reducing the number of annotations 
required from the user.

%%
%%\mypara{Liquid Types} 
%%The Liquid Types method infers refinements in three steps. 
%%%
%%First, we create refinement \emph{templates} for the unknown, 
%%to-be-inferred refinement types. The \emph{shape} of the template 
%%is determined by the Haskell's unrefined type it corresponds to. 
%%The template is just the shape refined with fresh refinement variables
%%$\kappa$ denoting the unknown refinements at each type position. 
%%For example, from a type ${\tfun{x}{t_x}{t}}$ we create 
%%the template ${\tfun{x}{\tref{v}{t_x}{}{\kappa_x}}{\tref{v}{t}{}{\kappa}}}$.
%%%
%%Second, we perform type checking using the templates (in place of the
%%unknown types.) Each wellformedness check becomes a wellformedness
%%constraint over the templates, and hence over the individual $\kappa$,
%%constraining which variables can appear in $\kappa$.
%%Each subsumption check becomes a subtyping constraint
%%between the templates, which can be further simplified, via syntactic
%%subtyping rules, to a logical implication query between the variables
%%$\kappa$.
%%%
%%Third, we solve the resulting system of logical implication constraints
%%(which can be cyclic) via abstract interpretation --- in particular,
%%monomial predicate abstraction over a set of logical qualifiers
%%\cite{Houdini,LiquidPLDI08}. The solution is a map from $\kappa$ to
%%conjunctions of qualifiers, which, when plugged back into the templates,
%%yields the inferred refinement types.

%%%%%%%%%%%%%%%%%%%%%%%%%%%%%%%%%%%%%%%%%%%%%%%%%%%%%%%%%%%%%%%%
%%%%%%%%%%%%%%%%%%%%%%%%%%%%%%%%%%%%%%%%%%%%%%%%%%%%%%%%%%%%%%%%
%%%%%%%%%%%%%%%%%%%%%%%%%%%%%%%%%%%%%%%%%%%%%%%%%%%%%%%%%%%%%%%%

%%%%%%%%   \section{Implementation: \toolname}\label{sec:haskell}
%%%%%%%%   We implemented \undeclang in \toolname (\S~\ref{sec:overview}), 
%%%%%%%%   a type based verifier for \lhaskell.
%%%%%%%%   %
%%%%%%%%   Before~\citep{vazou13} \toolname assumed 
%%%%%%%%   eager semantics. 
%%%%%%%%   %
%%%%%%%%   We modified the tool as suggested in 
%%%%%%%%   \S~\ref{sec:language}: 
%%%%%%%%   the environment of each VC 
%%%%%%%%   contains refinements of
%%%%%%%%   only provably terminating bindings.
%%%%%%%%   %
%%%%%%%%   This modification enables us to 
%%%%%%%%   reason about coinductive data, like @Streams@
%%%%%%%%   (\S~\ref{subsec:infinite})
%%%%%%%%   and forces us to prove termination (\S~\ref{subsec:termination})
%%%%%%%%   
%%%%%%%%   \subsection{Reasoning about Streams}\label{subsec:infinite}
%%%%%%%%   We use Abstract Refinements~\citep{vazou13} to 
%%%%%%%%   define a list data type parameterized with
%%%%%%%%    a type @a@ and a predicate @p@:
%%%%%%%%   \begin{code}
%%%%%%%%     data L [size] a <p :: L a -> Prop> 
%%%%%%%%     	= N
%%%%%%%%     	| C (x:a) ({xs: L <p> a | p xs})  
%%%%%%%%   \end{code}
%%%%%%%%   Ignoring the @[size]@ annotation,
%%%%%%%%   the above defines a list @L@
%%%%%%%%   such that for any type @a@
%%%%%%%%   and any predicate @p@,
%%%%%%%%   if a value has type @L <p> a@
%%%%%%%%   it is either @N@ or
%%%%%%%%   @C x xs@ where
%%%%%%%%   the tail @xs@ satisfies the predicate @p@.
%%%%%%%%   %
%%%%%%%%   The predicate should recursively hold, 
%%%%%%%%   thus, \emph{all} the tails of a @List <p> a@
%%%%%%%%   satisfy @p@.
%%%%%%%%   
%%%%%%%%   But what is a property that all tails satisfy?
%%%%%%%%   By definition, a Stream
%%%%%%%%   is a list in which @N@ does not appear, 
%%%%%%%%   \ie a list in which \emph{all} the tails 
%%%%%%%%   are constructed by @C@.
%%%%%%%%   %
%%%%%%%%   We use an @isCons@ measure that checks if a list is 
%%%%%%%%   constructed by @C@ to define 
%%%%%%%%   a @Stream@ type alias:
%%%%%%%%   \begin{code}
%%%%%%%%   type Stream a = {v: L <isCons> a | isCons v}
%%%%%%%%   
%%%%%%%%   measure isCons :: L a -> Prop
%%%%%%%%    isCons N        = false
%%%%%%%%    isCons (C x xs) = true
%%%%%%%%   \end{code}
%%%%%%%%   
%%%%%%%%   In \toolname, 
%%%%%%%%   the above definitions, 
%%%%%%%%   translate into the appropriate t
%%%%%%%%   types for lists data constructors:
%%%%%%%%   \begin{code}
%%%%%%%%   N :: forall a, <p :: L a -> Prop>. 
%%%%%%%%          {v: L <p> a | not (isCons v)}
%%%%%%%%   C :: forall a, <p :: L a -> Prop>.
%%%%%%%%           a -> {xs : L <p> a | p xs}
%%%%%%%%        -> {v: L <p> a | isCons v}
%%%%%%%%   \end{code}
%%%%%%%%   %
%%%%%%%%   Thus @N@ returns a list that falsifies @isCons@
%%%%%%%%   and has any abstract
%%%%%%%%   predicate @<p>@,
%%%%%%%%   while @C x xs@ returns a list that satisfies
%%%%%%%%   @isCons@ and
%%%%%%%%   has any abstract refinements
%%%%%%%%   that is satisfied by @xs@, \ie @p xs@ holds.
%%%%%%%%   %
%%%%%%%%   When in the type of @C@ the abstract refinement is instantiated 
%%%%%%%%   with @isCons@ we get that
%%%%%%%%   ``a list is a @Stream@ \emph{iff} so is its tail'':
%%%%%%%%   \vbox{@C :: a -> Stream a -> Stream a@}
%%%%%%%%   %
%%%%%%%%   Given so, we prove that @repeat x@ returns a @Stream@:
%%%%%%%%   \begin{code}
%%%%%%%%   repeat :: a -> Stream a
%%%%%%%%   repeat x = x `C` xs where xs = repeat x 
%%%%%%%%   \end{code}
%%%%%%%%   Assuming the type of @repeat@ we get that
%%%%%%%%   @xs :: Stream a@.
%%%%%%%%   %
%%%%%%%%   Thus @C x xs@ is also a @Stream@.
%%%%%%%%   
%%%%%%%%   Finally, we prove that one can safely @take@ @n@ elements from a @Stream@:
%%%%%%%%   \begin{code}
%%%%%%%%   take :: Nat -> Stream a -> a
%%%%%%%%   take 0 _           = N
%%%%%%%%   take n ys@(C x xs) = x `C` take (n-1) xs
%%%%%%%%   \end{code}
%%%%%%%%   To prove @take@ safe we need to prove that 
%%%%%%%%   at the recursive call @xs :: Stream a@; but 
%%%%%%%%   this is provable as @ys :: Stream@.
%%%%%%%%   %%Even though this information exists, it is generally guarded 
%%%%%%%%   %%by the requirement of @ys@ begin a value.
%%%%%%%%   %%But the recursive call happens after @ys@ is matched
%%%%%%%%   %%thus @ys@'s is indeed a value, which unguards its type
%%%%%%%%   %%and concludes our proof.
%%%%%%%%   
%%%%%%%%   With these types for @take@ and @repeat@, 
%%%%%%%%   \toolname soundly proves 
%%%%%%%%   @take n $ repeat x@
%%%%%%%%   safe for any @n@ and @x@.
%%%%%%%%   
%%%%%%%%   \subsection{Proving Termination}\label{subsec:termination} %$
%%%%%%%%   Typing @repeat@ did not require 
%%%%%%%%   information from the environment, but (\S~\ref{sec;overview})
%%%%%%%%   this is not the general case.
%%%%%%%%   In \undeclang, to use information from the environment, 
%%%%%%%%   you need to prove termination.
%%%%%%%%   This is not really a problem, as \toolname 
%%%%%%%%   implements a rather practical termination prover.
%%%%%%%%   
%%%%%%%%   Assume the definition of a recursive \lhaskell function 
%%%%%%%%   @f = \n -> e@ with type specification 
%%%%%%%%   @f :: n:Nat -> t[n/x]@.
%%%%%%%%   %
%%%%%%%%   To map @f@ in \declang we need to use the 
%%%%%%%%   $\mathtt{fix}$ constant.
%%%%%%%%   By the definition of \etfix{\tau}, 
%%%%%%%%   @f@ behaves exactly as its fixed point version, 
%%%%%%%%   or typing @f@ is equivalent to typing:
%%%%%%%%   $\etfix{\tau}\ (\efun{n}{}{\efun{f}{}{e}})
%%%%%%%%      \colon\colon
%%%%%%%%      \tfun{n}{\tnat^\lfinite}{\tau\sub{x}{n}}
%%%%%%%%   $
%%%%%%%%   %
%%%%%%%%   Which, by the \rtapp rule reduces to 
%%%%%%%%   \begin{align*}
%%%%%%%%   \efun{n}{}{\efun{f}{}{e}}
%%%%%%%%      &\colon\colon
%%%%%%%%   \tfun{n}{\tnat^\lfinite}{
%%%%%%%%   \tfunbasic{(
%%%%%%%%   	\tfunbasic{\tref{n'}{\tnat}{\lfinite}{n' < n}}{\tau\sub{x}{n'}}
%%%%%%%%   )}{\tau\sub{x}{n}}}
%%%%%%%%   \end{align*}
%%%%%%%%   %
%%%%%%%%   Interestingly, the above allows 
%%%%%%%%   the result type of @f@ to be finite, 
%%%%%%%%   thus its success implies that
%%%%%%%%   @f@ terminates!
%%%%%%%%   %
%%%%%%%%   So, \toolname proves @f@ terminates, 
%%%%%%%%   if its body typechecks 
%%%%%%%%   assuming a second argument
%%%%%%%%   \begin{code}
%%%%%%%%     f :: {n' : Nat | n' < n} -> t[n'/x]
%%%%%%%%   \end{code}
%%%%%%%%   
%%%%%%%%   As a concrete example, 
%%%%%%%%   \toolname proves that
%%%%%%%%   fibonacci function terminates exactly
%%%%%%%%   as \undeclang would prove it.
%%%%%%%%   %
%%%%%%%%   \begin{code}
%%%%%%%%     fib  :: Nat -> Int
%%%%%%%%     fib 0 = 1
%%%%%%%%     fib 1 = 1
%%%%%%%%     fib n = fib (n-1) + fib (n-2)
%%%%%%%%   \end{code}
%%%%%%%%   %
%%%%%%%%   To typecheck the body of @fib@,
%%%%%%%%   \toolname assumes a second argument
%%%%%%%%    @fib :: {n':Nat | n' < n} -> Int@
%%%%%%%%   and checks that each call to @fib@, 
%%%%%%%%   \ie each recursive call, is performed with 
%%%%%%%%   decreasing arguments, proving that @fib@
%%%%%%%%   \emph{must} terminate.
%%%%%%%%   
%%%%%%%%   \newcommand{\tmetric}{metric\xspace}
%%%%%%%%   \newcommand{\smetric}{\ensuremath{\mu}\xspace}
%%%%%%%%   Now consider a @range l h@ function
%%%%%%%%   that returns a list ranging from @l@ to @h@:
%%%%%%%%   
%%%%%%%%   \begin{code}
%%%%%%%%   range :: l:Nat -> h:Nat -> [Nat]
%%%%%%%%   range l h | l < h     = l : range (l+1) h
%%%%%%%%             | otherwise = [] 
%%%%%%%%   \end{code}
%%%%%%%%   We cannot apply the previous reasoning 
%%%%%%%%   to prove @range@ terminating.
%%%%%%%%   Yet, @range@ does terminate as at each recursive 
%%%%%%%%   call the \tmetric $\smetric(l, h) = h - l$ decreases.
%%%%%%%%   %
%%%%%%%%   One option would be to use a ghost first argument 
%%%%%%%%   that captures exactly this \tmetric.
%%%%%%%%   %
%%%%%%%%   Then, at a recursive call with actual parameters 
%%%%%%%%   $e_l$ and $e_h$ we should prove that the 
%%%%%%%%   termination condition 
%%%%%%%%   $TC \doteq 0 \leq \smetric(e_l, e_h) < \smetric(l, h)$ holds.
%%%%%%%%   
%%%%%%%%   \toolname captures exactly this behaviour.
%%%%%%%%   We annotate the type of @range@ with the
%%%%%%%%   \emph{termination \tmetric} $\smetric(l, h)$, 
%%%%%%%%   \ie an integer expression that refers to @range@'s arguments
%%%%%%%%   \begin{code}
%%%%%%%%   range :: l:Nat -> h:Nat -> [Nat] / [h - l]
%%%%%%%%   \end{code}
%%%%%%%%   at each function call \toolname checks whether
%%%%%%%%   $TC$ holds; 
%%%%%%%%   and either succeeds and proves termination
%%%%%%%%   or fails with a ``Termination Check Error''.
%%%%%%%%   
%%%%%%%%   \begin{comment}
%%%%%%%%   Next, consider the Ackermann function.
%%%%%%%%   %
%%%%%%%%   \begin{code}
%%%%%%%%     ack m n 
%%%%%%%%       | m == 0    = n + 1
%%%%%%%%       | n == 0    = ack (m-1) 1 
%%%%%%%%       | otherwise = ack (m-1) (ack m (n-1))
%%%%%%%%   \end{code}
%%%%%%%%   %
%%%%%%%%   There exists no integer termination metric that decreases at each recursive call.
%%%%%%%%   %
%%%%%%%%   However @ack@ terminates because at each call \emph{either}
%%%%%%%%   @m@ decreases \emph{or} @m@ remains the same and @n@ decreases. 
%%%%%%%%   %
%%%%%%%%   In other words, the pair @(m,n)@ strictly decreases according to
%%%%%%%%   \emph{lexicographic} ordering. 
%%%%%%%%   %
%%%%%%%%   To capture this requirement we extend termination metric
%%%%%%%%   from an integer to a list of integers
%%%%%%%%   and at each recursive call we check that this list is
%%%%%%%%   lexicographically decreasing.
%%%%%%%%   %
%%%%%%%%   In the case of
%%%%%%%%   @ack@ this list will simply be the parameters @m@
%%%%%%%%   and @n@:
%%%%%%%%   %
%%%%%%%%   \begin{code}
%%%%%%%%     ack :: m:Nat -> n:Nat -> Nat / [m,n]
%%%%%%%%   \end{code}
%%%%%%%%   %
%%%%%%%%   Thus, \toolname uses lexicographic ordering on 
%%%%%%%%   a list of natural numbers to prove termination.
%%%%%%%%   %
%%%%%%%%   Termination metrics could be generalized to 
%%%%%%%%   any \emph{well-found} metric.
%%%%%%%%   
%%%%%%%%   \spara{Mutual Recursion}
%%%%%%%%   %
%%%%%%%%   Equipped with termination metrics
%%%%%%%%   \toolname instantiates a powerful
%%%%%%%%   termination checker that like~\citep{XiTerminationLICS01}
%%%%%%%%   proves termination even for mutual recursive functions.
%%%%%%%%   %
%%%%%%%%   Consider the mutual recursive functions @isEven@ and @isOdd@
%%%%%%%%   \begin{code}
%%%%%%%%   {-@ isEven :: n:Nat -> Bool / [n, 0] @-}
%%%%%%%%   {-@ isOdd  :: n:Nat -> Bool / [n, 1] @-}
%%%%%%%%   
%%%%%%%%   isEven 0 = True
%%%%%%%%   isEven n = isOdd $ n-1
%%%%%%%%   
%%%%%%%%   isOdd n  = not $ isEven n 
%%%%%%%%   \end{code}
%%%%%%%%   Each call terminates as either @isEven@
%%%%%%%%   calls @isOdd@ with a decreasing argument, 
%%%%%%%%   or the argument remains the same, and @isOdd@
%%%%%%%%   calls @isEven@ that should then decrease the argument.
%%%%%%%%   % 
%%%%%%%%   We capture this reasoning using two lexicographic pairs:
%%%%%%%%   each function has its own metric, 
%%%%%%%%   and when @isEven@ calls @isOdd@
%%%%%%%%   the metric of the caller $(n, 0)$
%%%%%%%%   should be greater that callee's metric
%%%%%%%%   $(n-1, 1)$.
%%%%%%%%   %
%%%%%%%%   Similarly, at @isEven@'s call-site 
%%%%%%%%   \toolname verifies that	
%%%%%%%%   $(n, 1) > (n, 0)$.
%%%%%%%%   %
%%%%%%%%   For example, the call @isEven m@
%%%%%%%%   will fire the decreasing metric sequence
%%%%%%%%   $(m, 0) > (m-1, 1) > (m-1, 0) > (m-2, 1) > \dots$
%%%%%%%%   that ultimate terminates for \textit{any}
%%%%%%%%   natural number $m$.
%%%%%%%%   \end{comment}
%%%%%%%%   \NV{ack and isEven used to be here}
%%%%%%%%   
%%%%%%%%   \spara{Measuring the Size of Structures}
%%%%%%%%   Consider again the list \emph{user-defined} data type 
%%%%%%%%   and let us define define its @size@ by a measure:
%%%%%%%%   \begin{code}
%%%%%%%%     size (C x xs) = 1 + size xs
%%%%%%%%     size N        = 0
%%%%%%%%   \end{code}
%%%%%%%%   %
%%%%%%%%   From our discussion so far, 
%%%%%%%%   \toolname can prove that a @map@
%%%%%%%%   function terminates:
%%%%%%%%   \begin{code}
%%%%%%%%   map :: (a -> b) -> xs:L a -> L b / [size xs]
%%%%%%%%   map f (C x xs) = C (f x) (map f xs)
%%%%%%%%   map f N        = N
%%%%%%%%   \end{code}
%%%%%%%%   Turns out that structural recursion on the 
%%%%%%%%   first recursive argument is 
%%%%%%%%   a common pattern in \lhaskell code, so
%%%%%%%%   \toolname proves @map@ terminating 
%%%%%%%%   \emph{without} the need
%%%%%%%%   of the termination metric.
%%%%%%%%   %
%%%%%%%%   If no termination metric is provided, 
%%%%%%%%   \toolname assumes that the first
%%%%%%%%   argument with size information decreases.
%%%%%%%%   %
%%%%%%%%   Data type definitions%, like the one for list @L@
%%%%%%%%   can be annotated
%%%%%%%%   with a measure defining the size of the type;
%%%%%%%%   for @L a@ this measure is @size@.
%%%%%%%%   %
%%%%%%%%   \toolname comes with two build-in sizing information.
%%%%%%%%   For @Int@s it uses their value and for standard lists @[a]@
%%%%%%%%   uses @len@ as defined in \S~\ref{sec:overview}.
%%%%%%%%   %
%%%%%%%%   Even na\"{\i}ve, the above heuristic turns out to be 
%%%%%%%%   quite useful in practise\ref{sec:evaluation}. 
%%%%%%%%   
%%%%%%%%   \mypara{Using Measure Information}
%%%%%%%%   In many functions there is no argument that 
%%%%%%%%   provably decreases its size; 
%%%%%%%%   still the sizes of the arguments can be used to 
%%%%%%%%   express sophisticated termination metrics, as
%%%%%%%%   in the bellow @merge@ function that 
%%%%%%%%   merges two sorted lists.
%%%%%%%%   \begin{code}
%%%%%%%%   merge :: xs:L a -> ys:L a -> L a 
%%%%%%%%          / [size xs + size ys]
%%%%%%%%   merge (C x xs) (C y ys)
%%%%%%%%     | x < y     = x `C` merge xs     (y:ys)
%%%%%%%%     | otherwise = y `C` merge (x:xs) ys
%%%%%%%%   \end{code}
%%%%%%%%   Here the default metric, \ie @size xs@
%%%%%%%%   is not decreasing,
%%%%%%%%   instead we use @size@ to express the more complicated 
%%%%%%%%   termination metric @size xs + size ys@, which witch \toolname
%%%%%%%%   happily proves termination.
%%%%%%%%   
%%%%%%%%   Finally, annotating a function as @lazy@ will deactivate
%%%%%%%%   \toolname's termination check,
%%%%%%%%   %
%%%%%%%%   which concludes how \toolname proves termination:
%%%%%%%%   \begin{itemize}
%%%%%%%%   \item No termination check for @lazy@ functions
%%%%%%%%   \item Termination metrics are used to prove termination, if provided
%%%%%%%%   \item The \textit{first} argument 
%%%%%%%%   	with sizing information should decrease.
%%%%%%%%   \end{itemize}

\section{Evaluation: Strengths \& Limitations}\label{sec:stringmatcher:evaluation}

Verification of Parallel String Matching is the first realistic
proof that uses (Liquid) Haskell
to prove properties \textit{about} program functions.
%
In this section we use the String Matching proof
to quantitatively and qualitatively evaluate theorem proving in Haskell.

\paragraph{Quantitative Evaluation.}
The Correctness of Parallel String Matching proof
can be found online~\cite{implementation}.
%
Verification time, that is the time Liquid Haskell needs to check the proof,
is 75 sec on a dual-core Intel Core i5-4278U processor.
%
The proof consists of \textit{1839} lines of code.
%
Out of those
\begin{itemize}
\item \textit{226} are Haskell ``runtime'' code,
\item \textit{112} are liquid comments on the ``runtime'' Haskell code,
\item \textit{1307} are Haskell proof terms, that is functions with @Proof@ result type, and
\item \textit{194}  are liquid comments to specify theorems.
\end{itemize}
Counting both liquid comments and Haskell proof terms as verification code,
we conclude that the proof requires 7x the lines of ``runtime'' code.
%
This ratio is high and takes us to 2006 Coq,
when Leroy~\cite{Leroy06formalcertification} verified
the initial CompCert C compiler with
the ratio of verification to compiler lines being 6x.
% 4,400 lines of compiler code and 28,000 lines devoted to verification.

\paragraph{Strengths.}
Though currently verbose,
deep verification using Liquid Haskell has many benefits.
%
First and foremost,
\textit{the target code is written in the general purpose Haskell}
and thus can use advanced Haskell features, including
type literals, deriving instances, inline annotations
and optimized library functions like @ByteString@.
Even diverging functions can coexist with the target code, as long
as they are not reflected into logic~\cite{Vazou14}.

Moreover, \textit{SMTs are used to automate the proofs}
over key theories like linear arithmetic and equality.
%
As an example, associativity of @(+)@
is assumed throughout the proofs while shifting indices.
%
Our proof could be further automated 
by mapping refined strings to SMT strings and  
using the automated SMT string theory.
%
We did not follow this approach because we want to show that
our techinique can be used to prove any (and not only domain specific)
program properties.

Finally, we get further automation via
\textit{Liquid Type Inference}~\cite{LiquidPLDI08}.
%
Properties about program functions,
expressed as type specifications with unit result,
often depend on program invariants,
expressed as vanilla refinement types, and vice versa.
%
For example, we need the invariant that all indices of
a string matcher are good indices
to prove associativity of @(mappend)@.
%%NV simplify the above.
%%%For example, we need the invariant that all indices of
%%%a string matcher are good indices
%%%to prove that
%%%when the target is smaller than the input
%%%then the indices field is an empty list,
%%%which is required to prove associativity of @(mappend)@
%
Even though Liquid Haskell cannot currently synthesize proof terms,
it performs really well at inferring and propagating program invariants (like good indices)
via the abstract interpretation framework of Liquid Types.

\paragraph{Limitations.}
There are severe limitations that should be addressed
to make theorem proving in Haskell a pleasant and usable technique.
%
As mentioned earlier \textit{the proofs are verbose}.
%
There are a few cases where the proofs require domain specific knowledge.
%
For example, to prove associativity of string matching
@x mappend (y mappend z) = (x mappend y) mappend z@
we need a theorem that performs case analysis on the relative length of
the input field of @y@ and the target string.
%
Unlike this case split though, most proofs
do not require domain specific knowledge and merely proceed
by term rewriting and structural inductuction
that should be automated
via Coq-like~\cite{coq-book} tactics or/and Dafny-like~\cite{dafny} heuristics.
%
For example, synquid~\cite{polikarpova16} could be used to automatically
synthesize proof terms.

Currently, we suffer from two engineering limitations.
%
First, all reflected function should exist in the same module,
as reflection needs access to the function implementation
that is unknown for imported functions.
%
This is the reason why we need to use a user defined,
instead of Haskell's built-in, list.
%
In our implementation we used @CPP@ as a current workaround
of the one module restriction.
%
Second, class methods
cannot be currently reflected.
%
Our current workaround is to define Haskell functions instead
of class instances.
For example (@append@, @nil@) and (@concatStr@, @emptyStr@)
define the monoid methods of List and Refined String respectively.

Overall, we believe that the strengths outweigh the limitations which
will be addressed in the near future,
rendering Haskell a powerful theorem prover.

\chapter{Related Work}\label{chapter:related}

\toolname combines ideas from four main lines of research fields. 
%
It is a 
refinement type checker (\S~\ref{related:refinementtypes})
that enjoys SMT-based (\S~\ref{related:smtbased}) automated type checking. 
Via Refinement Reflection we touch the expressiveness 
of fully dependently types systems (\S~\ref{related:dependenttypes}), 
getting an automated and expressive verifier for Haskell programs (\S~\ref{related:haskell}).  
%

%%%%%%%%%%%%%%%%%%%%%%%%%%%%%%%%%%%%%%%%%%%%%%%%%%%%%%%%%%%%%%%%%%%%%%%%%%%%%%%
%%%%%%%%%%%%%%% Classic refinement types %%%%%%%%%%%%%%%%%%%%%%%%%%%%%%%%%%%%%%
%%%%%%%%%%%%%%%%%%%%%%%%%%%%%%%%%%%%%%%%%%%%%%%%%%%%%%%%%%%%%%%%%%%%%%%%%%%%%%%

\section{Refinement Types}\label{related:refinementtypes}

\mypara{Standard Refinement Types}
Refinement Types were introduced by Freeman and
Pfenning~\cite{FreemanPfenning91}, with refinements limited to
restrictions on the structure of algebraic datatypes. 
%
Freeman and Pfenning carefully designed the refinement logic 
to ensure \textit{decidable type inference}
via the notion of predicate subtyping (PVS~\cite{Rushby98}).
% 
The goal of refinement types is 
to refine the type system of an existing, general purpose,  
target language so that it
\textit{rejects more programs} as ill typed, 
unlike dependent type systems, 
that aim to increase the expressiveness 
and alter the semantics of the language.

\mypara{Applications of Refinement Types}
Xi and Pfenning implemented DML~\cite{pfenningxi98}
a refinement type checker for ML 
where arrays are indexed by terms 
from Presburger arithmetic to statically eliminate array bound checking. 
%
Since then, refinement types have been implemented for various general purpose languages, 
including 
ML~\cite{GordonTOPLAS2011,LiquidPLDI08},
C~\cite{deputy,LiquidPOPL10},
Racket~\cite{RefinedRacket}
and Scala~\cite{refinedscala}
to prove various correctness properties ranging from safe memory accessing 
to correctness of security protocols.
%
All the above systems operate under CBV semantics 
that implicitly assume that all free variables are bound to values. 
%
This assumption, that breaks under Haskell's lazy semantics,
turned out to be crucial for the soundness 
of refinement type checking.
To restore soundness in \toolname we 
use a refinement type based termination checker 
to distinguish between provably terminating and potential diverging free variables. 


%%%%%%%%%%%%%%%%%%%%%%%%%%%%%%%%%%%%%%%%%%%%%%%%%%%%%%%%%%%%%%%%%%%%%%%%%%%%%%%
%%%%%%%%%%%%%%% Extensions of refinement types %%%%%%%%%%%%%%%%%%%%%%%%%%%%%%%%
%%%%%%%%%%%%%%%%%%%%%%%%%%%%%%%%%%%%%%%%%%%%%%%%%%%%%%%%%%%%%%%%%%%%%%%%%%%%%%%

\mypara{Reconciliation between Expressiveness and Decidability}
Reluctant to give up decidable type checking, 
many systems have pushed the expressiveness of refinement types
within decidable logics. 
%
Kawaguchi \etal~\cite{LiquidPLDI09} 
introduce \emph{recursive} and \emph{polymorphic} refinements for data
structure properties increasing the expressiveness but also the complexity 
of the underlying refinement system. 
%
\catalyst~\citep{catalyst} permits a form of
higher order specifications where refinements
are relations which may themselves be parameterized
by other relations.
%
However, to ensure decidable checking, \catalyst
is limited to relations that can be specified as
catamorphisms over inductive types, precluding
for example, theories like arithmetic.
%
In the same direction, Abstract and Bounded refinement types 
encode modular, higher order specifications 
using the decidable theory of uninterpreted functions. 
% 
All the above systems
only allow for ``shallow'' specifications, 
where the underlying solver can only reason about 
(decidable) abstractions of user defined functions 
and not the exact description of the function  implementations of the functions. 
%
Refinement Reflection, on the other hand, 
reflects user defined function definitions 
into the logic, allowing for ``deep'' program specifications
but requiring the user to manually provide cumbersome proof terms. 



\section{SMT-based verification}\label{related:smtbased}

Knowles and Flanagan~\cite{Knowles10} allow refinement predicates to
be arbitrary terms of the language being typechecked and present a
technique for deciding some typing obligations statically and
deferring others to runtime.
%; Gronksi \etal~\cite{Gronski06} present animplementation of such a system.

The \fstar system enables full dependent typing via
SMT solvers via a higher-order universally quantified
logic that permit specifications similar to ours
(\eg @compose@, @filter@ and @foldr@).
%% https://github.com/FStarLang/FStar/
%
While this approach is at least as expressive
as bounded refinements it has two drawbacks.
%
First, due to the quantifiers, the generated VCs
fall outside the SMT decidable theories.
This renders the type system undecidable (in theory),
forcing a dependency on the solver's unpredictable
quantifier instantiation heuristics (in practice).
%
Second, more importantly, % perhaps more importantly,
the higher order
predicates must be \emph{explicitly} instantiated,
placing a heavy annotation burden on the programmer.
%
In contrast, bounds permit decidable
checking, and are automatically instantiated
via Liquid Types.


\paragraph{SMT-Based Verification}
%
SMT solvers have been extensively used to automate
reasoning on verification languages like
Dafny~\cite{dafny}, Fstar~\cite{fstar} and Why3~\cite{why3}.
%
These languages are designed for verification,
thus have limited support for commonly used language
features like parallelism and optimized libraries
that we use in our verified implementation.
%


% We compare refinement reflection to the most closely related
% lines of work in the vast literature on program verification.

\mypara{SMT-Based Verification}
%
SMT-solvers have been extensively used to automate
program verification via Floyd-Hoare logics~\cite{Nelson81}.
%
Our work is inspired by Dafny's Verified
Calculations~\citep{LeinoPolikarpova16},
a framework for proving theorems in
Dafny~\citep{dafny}, but differs in
%
(1)~our use of reflection instead of axiomatization and
(2)~our use of refinements to compose proofs.
%
Dafny, and the related \fstar~\citep{fstar}
which like \toolname, uses types to compose
proofs, offer more automation by translating
recursive functions to SMT axioms.
However, unlike reflection, this axiomatic
approach renders typechecking and verification
undecidable (in theory) and leads to
unpredictability and divergence
(in practice)~\citep{Leino16}.
%\NV{CHECL Relational-F*, Barthe et al, from POPL 2014, and EasyCrypt}

%% In a work more closely related to
%% ours, \fstar uses refinement types
%% for program verification supporting
%% expressiveness of fully dependent types.
%% %
%% As in Dafny, \fstar directly translates
%% recursive functions to axioms in the logic
%% thus suffers from the ``butterfly effect''
%% and allows the user to explicitly write SMT tactics to control it.

%% Leino \etal~\citep{Leino16}
%% name this problem as the ``butterfly
%% effect'', in which minor modifications
%% to the program source cause significant
%% instabilities in verification and propose
%% trigger selection strategies to address it.
%% %
%% We avoid the ``butterfly effect'' by not
%% directly axiomatizing functions into logic.
%% Instead the information about the function's
%% body is exactly captured in function's result
%% type and user needs to explicitly invoke the function to push
%% the function's definition information into the logic.

\begin{comment}

\section{Dependent Type Systems}\label{related:dependenttypes}

%%%%%%%%%%%%%%%%%%%%%%%%%%%%%%%%%%%%%%%%%%%%%%%%%%%%%%%%%%%%%%%%%%%%%%%%%%%%%%%
%%%%%%%%%%%%%%% Fully dependent typing %%%%%%%%%%%%%%%%%%%%%%%%%%%%%%%%%%%%%%%%
%%%%%%%%%%%%%%%%%%%%%%%%%%%%%%%%%%%%%%%%%%%%%%%%%%%%%%%%%%%%%%%%%%%%%%%%%%%%%%%

\mypara{Dependent types}
%
Our work is also inspired by dependently typed
systems like Coq~\citep{coq-book} and
Agda~\citep{agda}.
%
Reflection shows how deep specification
and verification in the style of Coq and Agda
can be \emph{retrofitted} into existing languages
via refinement typing.
%
Furthermore, we can use SMT to significantly
automate reasoning over important theories like
arithmetic, equality and functions.
%
It would be interesting to investigate how
the tactics and sophisticated proof search
of Coq \etc can be adapted to the refinement setting.

% which allow for arbitrary expressiveness of the type system
% in the cost of automatic verification.
%
%% The syntax of \libname's operators is inspired by
%% Equational Reasoning in Agda~\citep{agda}.
%% Here we extended these equational operators
%% to support linear arithmetic and, for example, prove
%% properties of Ackermann function.
%% %
%% Unlike Adga, proof term are explicit in \libname,
%% we do not use heuristics to infer proofs.
\mypara{Higher order Logics and Dependent Type Systems}
%
including
NuPRL~\citep{Constable86},
Coq~\citep{coq-book}, Agda~\citep{norell07},
and even to some extent, \haskell~\citep{JonesVWW06, McBride02},
occupy the maximal extreme of the expressiveness spectrum.
However, in these settings, checking requires explicit
proof terms which can add considerable programmer overhead.
%
Our goal is to eliminate the programmer overhead of
proof construction by restricting specifications to
decidable, first order logics and to see how far
we can go without giving up on expressiveness.

%  Higher-order logics: Coq/HTT/F*/Agda which have explicit predicates, quantification 
A number of higher-order logics and corresponding verification tools
have been developed for reasoning about programs.
%
Example of systems of this type include NuPRL \cite{Constable86},
%F$_{<:}$ \cite{Cardelli91},
Coq \cite{coq-book}, F$^\star$ \cite{SwamyCFSBY11} and Agda \cite{norell07}
which support the development and verification of higher-order, 
pure functional programs.
%
While these systems are highly expressive, their expressiveness comes at the
cost of making logical validity checking undecidable.
%
To help automate validity checking, both built-in and user-provided
tactics are used to attempt to discharge proof obligations; however,
the user is ultimately responsible for manually proving any
obligations which the tactics are unable to discharge.



\spara{Dependent Types} are the basis of many verifiers, 
or more generally, proof assistants.
%
Verification of haskell code is possible with
``full'' dependently typed systems like Coq~\cite{coq-book}, 
Agda~\cite{norell07}, Idris~\cite{Brady13}, Omega~\cite{Sheard06}, and
 {$\lambda_\rightarrow$}~\cite{LohMS10}.
 %
 While these systems are highly expressive,
their expressiveness comes at the cost of making logical validity checking undecidable
thus rendering verification cumbersome.	
 %
 
 \spara{Dependent Types} are the basis of many verifiers, 
or more generally, proof assistants.
%
In this setting arbitrary terms may appear inside types,
so to prevent logical inconsistencies, and enable
the checking of type equivalence, all terms must
terminate.
%
``Full'' dependently typed systems like Coq~\cite{coq-book}, 
Agda~\cite{norell07}, and Idris~\cite{Brady13} typically use 
\emph{structural} checks where recursion is allowed on 
sub-terms of ADTs to ensure that \emph{all} terms terminate.
%
We differ in that, since the refinement logic is
restricted, we do not require that all functions terminate,
and hence, we can prove properties of possibly diverging 
functions like @collatz@ as well as lazy functions like @repeat@.
%
Recent languages like Aura~\citep{AURA} and Zombie~\citep{Zombie}
allow general recursion, but constrain the logic to a terminating 
sublanguage, as we do, to avoid reasoning 
about divergence in the logic.
%
In contrast to us, the above systems crucially assume 
\emph{call-by-value} semantics to ensure that binders are bound
to values, \ie cannot diverge.




\paragraph{Dependent Types}
Unlike Refinement Types, dependent type systems,
like Coq~\cite{coq-book}, Adga~\cite{agda} and Isabelle/HOL~\cite{isabelle} allow for ``deep'' specifications
which talk about program functions,
such as the program equivalence reasoning we presented.
%
Compared to (Liquid) Haskell,
these systems allow for tactics and heuristics
that automate proof term generation
but lack SMT automations and
general-purpose language features,
like non-termination, exceptions and IO.
%
Zombie~\cite{Zombie,Sjoberg2015} and Fstar~\cite{fstar} allow dependent types to
coexist with divergent and effectful programs,
but still lack the optimized libraries,
like @ByteSting@, which come
with a general purpose language
with long history, like Haskell.



\paragraph{Parallel Code Verification}

One work  closely related to ours is
SyDPaCC~\cite{SyDPaCC}, a Coq library that
automatically parallelizes list homomorphisms
by extracting parallel Ocaml versions of user provided Coq functions.
%
Unlike SyDPaCC, we are not automatically generating the parallel
function version, because of engineering limitations
(\S~\ref{sec:evaluation}).  Once these are addressed, code extraction
can be naturally implemented by turning
Theorem~\ref{theorem:two-level} into a Haskell type class with a
default parallelization method.
%
SyDPaCC used maximum prefix sum as a case study,
whose morphism verification is
much simpler than our string matching case study.
%
Finally, our implementation consists of
2K lines of Liquid Haskell, which we consider verbose and aim to
use tactics to simplify.
On the contrary, the SyDPaCC implementation
requires three different languages:
2K lines of Coq with tactics, 600 lines of Ocaml and 120 lines of C,
and is considered ``very concise''.

\mypara{Dependent Types for Non-Terminating Programs}
%
Zombie~\citep{Zombie, Sjoberg2015} integrates
dependent types in non terminating programs
and supports automatic reasoning for equality.
%
Vazou \etal have previously~\citep{Vazou14} shown
how Liquid Types can be used to check
non-terminating programs.
%
Reflection makes \toolname at least as
expressive as Zombie, \emph{without}
having to axiomatize the theory of
equality within the type system.
%
Consequently, in contrast to Zombie,
SMT based reflection lets \toolname
verify higher-order specifications
like @foldr_fusion@.

%%%%%%%%%%%%%%%%%%%%%%%%%%%%%%%%%%%%%%%%%%%%%%%%%%%%%%%%%%%%%%%%%%%%%%%%%%%%%%%
%%%%%%%%%%%%%%% alternative verification in Haskell %%%%%%%%%%%%%%%%%%%%%%%%%%%
%%%%%%%%%%%%%%%%%%%%%%%%%%%%%%%%%%%%%%%%%%%%%%%%%%%%%%%%%%%%%%%%%%%%%%%%%%%%%%%

\section{Verification in Haskell}\label{related:haskell}


\mypara{Proving Equational Properties}
% of Haskell Programs}
%
Several authors have proposed tools for proving
(equational) properties of (functional) programs.
%
%
HERMIT~\citep{Farmer15} proves equalities by rewriting
the GHC core language, guided by user specified scripts.
%
In contrast, our proofs are simply Haskell programs,
we can use SMT solvers to automate reasoning, and,
most importantly, we can connect the validity of
proofs with the semantics of the programs.


\spara{Static Contract Checkers} 
like ESCJava~\cite{ESCJava} are a classical way of verifying 
correctness through assertions and pre- and post-conditions. 
%
\cite{XuPOPL09} describes a static contract checker for 
Haskell that uses symbolic execution to unroll procedures
upto some fixed depth, yielding weaker ``bounded'' soundness
guarantees.
% 

Similarly, Zeno~\cite{ZENO} is an automatic Haskell 
prover that combines unrolling with heuristics for rewriting
and proof-search. 
%%Based on rewriting, it is sound but 
%%``Zeno might loop forever'' when faced with 
%%non-termination.
%
Finally, the Halo~\cite{halo} contract checker encodes 
Haskell programs into first-order logic by directly 
modeling the code's denotational semantics,
again, requiring heuristics for instantiating axioms 
describing functions' behavior.
%
   Haskell itself can be used to \emph{fake} ``lightweight'' dependent 
   types~\citep{ChakravartyKJ05,JonesVWW06,Weirich12}.
   In this style, the invariants are expressed in 
   a restricted~\citep{Jia10} total 
   index language and relationships (\eg $x<y$ and $y<z$) 
   are combined (\eg $x<z$) by explicitly constructing
   a term denoting the consequent from terms 
   denoting the antecedents.
   %
   On the plus side this ``constructive'' approach
   ensures soundness. 
   It is impossible to witness inconsistencies, 
   as doing so triggers diverging computations.
   %
   However, it is not easy to use restricted indices
   with explicitly constructed relations to verify 
   complex properties~\citep{LindleyM13}.


\spara{Totality Checking}
is feasible by GHC itself, via an option flag that warns of any incomplete patterns.
%
Regrettably, GHC's warnings are local, \ie
GHC will raise a warning for @head@'s partial definition, 
but not for its caller, as the programmer would desire.
%%(2)~ and preservative:
%%a warning will be raised for any incomplete pattern
%%without an attempt to reason if it is reachable or not.
%
Catch~\cite{catch}, 
a fully automated tool that tracks incomplete patterns,
addresses the above issue
%
by computing functions' pre- and post-conditions.
Moreover, catch statically analyses the code 
to track reachable incomplete patterns.
%
\toolname allows more precise analysis than catch, 
thus, by assigning the appropriate
types to $\star$Error functions (\S~\ref{sec:totality}) 
it tracks reachable incomplete patters 
%we get catch analysis
as a side-effect of verification.
 
 
 \spara{Termination Analysis}
is crucial for \toolname's soundness 
and is implemented in a technique inspired by~\cite{XiTerminationLICS01}, 
%
Various other authors have proposed techniques to verify termination of
recursive functions, either using the ``size-change
principle''~\cite{JonesB04,Sereni05}, or by annotating types with size indices
and verifying that the arguments of recursive calls have smaller
indices~\cite{HughesParetoSabry96,BartheTermination}.
%
To our knowledge, none of the above analyses have been empirically
evaluated on large and complex real-world libraries.

AProVE~\cite{Giesl11} implements a powerful, fully-automatic
termination analysis for Haskell based on term-rewriting.
%
Compared to AProVE,
encoding the termination proof via 
refinements provides advantages that are crucial in 
large, real-world code bases. 
Specifically, refinements
let us
%
(1) prove termination over a subset 
    (not all) of inputs; many functions (\eg @fac@) 
    terminate only on @Nat@ inputs and not all @Int@s,
%
(2) encode pre-conditions, 
    post-conditions, and auxiliary invariants that 
    are essential for proving termination, (\eg @qsort@),
%
(3) easily specify non-standard 
    decreasing metrics and prove termination, (\eg @range@).
%
In each case, the code could be (significantly) 
\emph{rewritten} to be amenable to AProVE but this defeats
the purpose of an automatic checker.
 
%%%%%%%%%%%%%%%%%%%%%%%%%%%%%%%%%%%%%%%%%%%%%%%%%%%%%%%%%%%%%%%%%%%%%%%%%%%%%%%
%%%%%%%%%%%%%%% dependent types in Haskell %%%%%%%%%%%%%%%%%%%%%%%%%%%%%%%%%%%%
%%%%%%%%%%%%%%%%%%%%%%%%%%%%%%%%%%%%%%%%%%%%%%%%%%%%%%%%%%%%%%%%%%%%%%%%%%%%%%%
 
 
 \mypara{Dependent Types in Haskell}
%
Integration of dependent types into Haskell
has been a long standing goal that dates back
to Cayenne~\citep{cayenne}, a Haskell-like,
fully dependent type language with undecidable
type checking.
%
In a recent line of work~\citep{EisenbergS14}
Eisenberg \etal aim to allow fully dependent
programming within Haskell, by making
``type-level programming ... at least as
  expressive as term-level programming''.
%
Our approach differs in two significant ways.
%
First, reflection allows SMT-aided verification,
which drastically simplifies proofs over key theories
like linear arithmetic and equality.
%
Second, refinements are completely erased at run-time.
That is, while both systems automatically lift Haskell
code to either uninterpreted logical functions
or type families, with refinements, the logical
functions are not accessible at run-time and
promotion cannot affect the semantics of
the program.
%
As an advantage (resp. disadvantage), refinements
cannot degrade (resp. optimize)
the performance of programs.

 \mypara{Our Relational Algebra Library} builds on a long
line of work on type safe database access.
%
The HaskellDB~\citep{haskellDB}
showed how phantom types could be used to eliminate
certain classes of errors.
%
Haskell's HList library~\citep{heterogeneous}
extends this work with type-level computation
features to encode heterogeneous lists, which
can be used to encode database schema, and
(unlike HaskellDB) statically reject accesses
of ``missing'' fields.
%
The HList implementation is non-trivial,
requiring new type-classes for new operations
(\eg @append@ing lists); \citep{thepipower}
shows how a dependently typed language greatly
simplifies the implementation.
%
Much of this simplicity can be recovered in
Haskell using the @singleton@ library~\citep{Weirich12}.
%
Our goal is to show that bounded refinements
are expressive enough to permit the construction
of rich abstractions like a relational algebra
and generic combinators for safe database access
while using SMT solvers to provide decidable
checking and inference. Further, unlike the
HList based approaches, refinements they can
be used to \emph{retroactively} or \emph{gradually}
verify safety; if we erase the types we still
get a valid Haskell program operating over
homogeneous lists.
We do not extende Haskell's expressivity 
 
 Haskell itself can be considered a dependently-typed language,
 as type level computation is allowed via 
 Type Families~\cite{McBride02},
 Singleton Types\cite{Weirich12}, 
 Generalized Algebraic  Datatypes (GADTs)~\cite{JonesVWW06, SchrijversJSV09}, 
 and type-level functions~\cite{ChakravartyKJ05}.
 %
Again, 
verification in haskell itself turns out to be quite painful~\cite{LindleyM13}.















\spara{Tracking Divergent Computations}
The notion of type stratification to track potentially 
diverging computations dates to at least~\citep{ConstableS87} 
which uses $\bar{\typ}$ to encode diverging terms, and types 
$\efix{}$ as $(\bar{\typ}\rightarrow\bar{\typ}) \rightarrow \bar{\typ}$).
%
More recently, \cite{Capretta05} tracks diverging 
computations within a \emph{partiality monad}.
%
Unlike the above, we use refinements to 
obtain terminating fixpoints (\etfix{}), which let us prove 
the vast majority (of sub-expressions) in real world libraries 
as non-diverging, avoiding the restructuring that would
be required by the partiality monad.

\spara{Termination Analyses}
Various authors have proposed techniques to verify termination 
of recursive functions, either using the ``size-change principle'' 
\cite{JonesB04,Sereni05}, or by annotating types with size indices 
and verifying that the arguments of recursive calls have smaller 
indices~\cite{HughesParetoSabry96,BartheTermination}.
%
Our use of refinements to encode terminating fixpoints is most 
closely related to~\cite{XiTerminationLICS01}, but this work 
also crucially assumes CBV semantics for soundness.

AProVE~\cite{Giesl11} implements a powerful, fully-automatic
termination analysis for Haskell based on term-rewriting.
%
While we could use an external analysis like AProVE,
we have found that encoding the termination proof via 
refinements provided advantages that are crucial in 
large, real-world code bases. Specifically, refinements
let us
%
(1) prove termination over a subset 
    (not all) of inputs; many functions (\eg @fac@) 
    terminate only on @Nat@ inputs and not all @Int@s,
%
(2) encode pre-conditions, 
    post-conditions, and auxiliary invariants that 
    are essential for proving termination, (\eg @gcd@),
%
(3) easily specify non-standard 
    decreasing metrics and prove termination, (\eg @range@).
%
In each case, the code could be (significantly) 
\emph{rewritten} to be amenable to AProVE but this defeats
the purpose of an automatic checker.
%
Finally, none of the above analyses have been empirically
evaluated on large and complex real-world libraries.


\spara{Static Contract Checkers} 
like ESCJava~\cite{ESCJava} are a classical way of verifying 
correctness through assertions and pre- and post-conditions. 
%
Side-effects like modifications of global variables are a 
well known issue for static checkers for imperative languages;
the standard approach is to use an effect analysis to determine
the ``modifies clause'' \ie the set of globals modified by a procedure.
%
Similarly, one can view our approach as implicitly computing 
the non-termination effects.
%
%
\cite{XuPOPL09} describes a static contract checker for 
Haskell that uses symbolic execution to unroll procedures
upto some fixed depth, yielding weaker ``bounded'' soundness
guarantees.
% 

%
Similarly, Zeno~\cite{ZENO} is an automatic Haskell 
prover that combines unrolling with heuristics for rewriting
and proof-search. 
Based on rewriting, it is sound but 
``Zeno might loop forever'' when faced with 
non-termination.
%
Finally, the Halo~\cite{halo} contract checker encodes 
Haskell programs into first-order logic by directly 
modeling the code's denotational semantics,
again, requiring heuristics for instantiating axioms 
describing functions' behavior. Halo's translation of Haskell
programs directly encodes constructors as uninterpreted functions,
axiomatized to be injective (as the denotational semantics requires).
This heavyweight encoding is more precise than predicate abstraction 
but leads to model-theoretic problems (outlined in the Halo paper) and 
affects the efficiency of the encoding when scaling to larger programs 
(see also \ref{sec:refinedhaskell:conclusion}, paragraph B) in the lack of specialized 
decisions procedures.
%
Unlike any of the above, our type-based approach does 
not rely on heuristics for unrolling recursive procedures, 
or instantiating axioms. 
%
Instead we are based on decidable SMT validity 
checking and abstract interpretation~\cite{LiquidPLDI08} 
which makes the tool predictable and the overall workflow
scale to the verification of large, real-world
code bases.

\end{comment}


\begin{comment}
\paragraph{Parallel Code Verification}
Dependent type theorem provers have been used before to
verify parallel code.
%
BSP-Why~\cite{bspwhy} is an extension to Why2 that is using both Coq and SMTs
to discharge user specified verification conditions.
%
Daum~\cite{daum07} used Isabelle to formalize the semantics
of a type-safe subset of C, 
by extending Schirmer's~\cite{schirmer06}
formalization of sequential imperative languages.
%
Finally, Swierstra~\cite{wouter10} formalized mutable arrays in Agda
and used them to reason about distributed maps and sums.

\mypara{Deterministic Parallelism}
%
Deterministic parallelism has plenty of theory but relatively few practical
implementations.  Early discoveries were based on limited producer-consumer
communication, such as single-assignment variables \cite{Tesler-1968,IStructures}, Kahn
process networks~\cite{kahn-1974}, and synchronous dataflow~\cite{lee-sdf}.
Other models use synchronous updates to shared state, as in
Esterel~\cite{synchronous-overview} or PRAM.  Finally, work on type systems for
permissions management \cite{permission-types,habanero-java-permissions},
supports the development of {\em non-interfering} parallel programs that access
disjoint subsets of the heap in parallel.  Parallel functional programming is
also non-interfering~\cite{manticore,multicore-ghc}.
%
Irrespective of which theory is used to support deterministic parallel
programming, practical implementations such as Cilk~\cite{cilk} or Intel
CnC~\cite{cnc} are limited by host languages with type systems insufficient to
limit side effects, much less prove associativity.  Conversely, dependently
typed languages like Agda and Idris do not have parallel programming APIs and
runtime systems.

% synchronous languages such as Esterel
\mypara{Our Approach for Verifying Stateful Computations} using monads
indexed by pre- and post-conditions is inspired by the method of
Filli\^atre~\citep{Filliatre98}, which was later enriched with
separation logic in Ynot~\citep{ynot}. In future work it would
be interesting to use separation logic based refinements to specify
and verify the complex sharing and aliasing patterns allowed by Ynot.
%
\fstar encodes stateful computations in a special Dijkstra
Monad~\citep{dijkstramonad} that replaces the two assertions with
a single (weakest-precondition) predicate transformer which
can be composed across sub-computations to yield a transformer
for the entire computation.
%
Our \RIO approach uses the idea of indexed monads but
has two concrete advantages.
%
First, we show how bounded refinements alone suffice to
let us fashion the \RIO abstraction from scratch.
%
Consequently, second, we automate inference of pre- and
post-conditions and loop invariants as refinement instantiation
via Liquid Typing~\citep{LiquidPLDI08}.


Systems~\citep{sousa16} and \citep{KobayashiRelational15}
extend classical safety verification algorithms,
respectively based on Floyd-Hoare logic and Refinement Types,
to the setting of relational or $k$-safety properties
that are assertions over $k$-traces of a program.
%
Thus, these methods can automatically prove that
certain functions are associative, commutative \etc.
but are restricted to first-order properties and
are not programmer-extensible.
%
Zeno~\citep{ZENO} generates proofs by term
rewriting and Halo~\citep{halo} uses an axiomatic
encoding to verify contracts.
%
Both the above are automatic, but unpredictable and not
programmer-extensible, hence, have been limited to
far simpler properties than the ones checked here.

\end{comment}
\section{Conclusions \& Future Work}\label{sec:conclusion}

Our goal is to use the recent advances in SMT solving to 
build automated refinement type-based verifiers for 
Haskell.
%
In this paper, we have made the following advances 
towards the goal. 
%
First, we demonstrated how the classical technique
for generating VCs from refinement subtyping queries
is unsound under lazy evaluation.
%
Second, we have presented a solution that addresses 
the unsoundness by stratifying types into those that 
are inhabited by terms that may diverge, those that must reduce 
to Haskell values, and those that must reduce to finite values, 
and have shown how refinement types may themselves 
be used to soundly verify the stratification. 
%
Third, we have developed an implementation of our 
technique in \toolname and have evaluated the tool 
on a large corpus comprising 10KLOC of widely used 
Haskell libraries. Our experiments empirically 
demonstrate the practical effectiveness of our
approach: using refinement types, we were able 
to prove 96\% of recursive functions as 
terminating, and to crucially use this information 
to prove a variety of functional correctness properties.

\mypara{Limitations}
While our approach is demonstrably effective 
\emph{in practice}, it relies critically on 
proving termination, which, while independently 
useful, is not wholly satisfying 
\emph{in theory}, as adding divergence shouldn't 
\emph{break} a safety proof.
%to quote~\cite{McMillanPersonal}: 
%\emph{``adding divergence shouldn't break your safety proofs.''}
%
%In our approach, we can prove a program safe, 
Our system can prove a program safe, 
but if the program is modified by making 
some functions non-deterministically diverge,
then, since we rely on termination, we
may no longer be able to prove safety.
%
Thus, in future work, it would be valuable to 
explore \emph{other} ways to reconcile laziness 
and refinement typing. We outline some routes 
and the challenging obstacles along them.


%% \mypara{1. Reject Inconsistent Environements}
%% We may be tempted to point the finger of blame at the
%% ``inconsistency'' itself. Unfortunately, this would be 
%% misguided -- inconsistencies are not a bug but a crucial 
%% feature of refinement type systems. 
%% %
%% They enable, among other things, \emph{path sensitivity} by
%% incorporating information from run-time tests (guards) and 
%% hence let us verify that expressions that throw catastrophic
%% exceptions (\eg @error e@) are indeed unreachable dead code 
%% and will not fail at run-time.
%% 
%% \mypara{2. CPS Transformation}
%% We might use a CPS transformation~\cite{PlotkinTCS75,WadlerICFP03} 
%% to convert the program into call-by-value.
%% We confess to be somewhat wary of the prospect of translating 
%% inferred types and errors \emph{back} to the source level after 
%% such a transformation.
%% Previous experience shows that the ability to map types and 
%% errors to source is critical for usability.

%% \mypara{3. Strictness Analysis}
%% We may want some form of \emph{strictness} or 
%% \emph{dependency analysis}~\cite{Mycroft80} to statically 
%% predict which binders must be evaluated, and only use
%% refinements for those binders in the environment. 
%% This route is problematic for many reasons. 
%% %
%% First, we aim to prove fine-grained functional correctness 
%% properties of programs. In addition to the well known limitations
%% of strictness analysis~\cite{HaskellWiki}, it 
%% is unclear how to develop an analysis that is sensitive
%% to the precise semantic (``path'') conditions under which 
%% the evaluation of different binders will be forced.
%% %
%% Second, and more importantly, it is often useful to add
%% \emph{ghost} values into the program for the sole purpose 
%% of making refinement types \emph{complete}~\cite{TerauchiPOPL13}. 
%% By construction, these values are not used by the program, 
%% and would be thrown away by a strictness analysis, thereby
%% precluding verification.

\mypara{A. Convert Lazy To Eager Evaluation}
%
One alternative might be to translate the program from lazy to eager evaluation,
for example, to replace every (thunk) $e$ with an abstraction $\efun{()}{}{e}$,
and every use of a lazy value $x$ with an application $x\ ()$. 
After this, we could simply assume eager evaluation, and so the usual refinement
type systems could be used to verify Haskell. Alas, no. 
While sound, this translation
doesn't solve the problem
of reasoning about divergence. 
%%While this translation
%%does soundly reject the @explode@ example, 
%%it doesn't solve the problem
%%of reasoning about divergence. 
A dependent function type
${\tfun{x}{\tint}{\tlref{\vv}{\tint}{}{\vv>\x}}}$
would be transformed to
${\tfun{x}{(\tfunbasic{()}{\tint})}
          {\tlref{\vv}{\tint}{}{\vv > \x\ ()}}}$
%
%%%\begin{code}
%%%  f :: x:Int -> {v:Int | v > x}
%%%\end{code}
%%%%
%%%would be transformed to
%%%%
%%%\begin{code}
%%%  f :: x:(() -> Int) -> {v:Int | v > x ()}
%%%\end{code}
%
The transformed type is problematic as it uses 
arbitrary function applications in the refinement logic!
%
The type is only sensible if $x\ ()$ provably reduces to a value, 
bringing us back to square one.

%%% This is highly problematic, because now we have function 
%%% applications in the logic! Now it seems that "x" is a 
%%% pretty harmless function but not really, because we're 
%%% essentially back in the same world where we have to be 
%%% *sure* that "x ()" actually reduces to a value! 
%%% 
%%% That is to say, essentially, this is the same as the 
%%% "direct" approach of, 
%%% 
%%%    x:Int -> {y | (isvalue x) => y > x}
%%% 
%%% 
%%% That is, the simple refinement in the original CBN is converted to 
%%% 
%%% 
%%% %
%%% Unfortunately, this doesn't 
%%% 
%%% that require CBV evaluation
%%% 
%%%   That use call by name and a strict language.  Now all the old techniques should work. 
%%% 
%%% convert the program from lazy to ea
%% We might use a CPS transformation~\cite{PlotkinTCS75,WadlerICFP03} 
%% to convert the program into call-by-value.
%% We confess to be somewhat wary of the prospect of translating 
%% inferred types and errors \emph{back} to the source level after 
%% such a transformation.
%% Previous experience shows that the ability to map types and 
%% errors to source is critical for usability.

\mypara{B. Explicit Reasoning about Divergence}
%
Another alternative is to enrich the refinement logic
% It is not really clear that it is THE only 
% Thus, the only other alternative is to enrich the refinement logic
with a \emph{value predicate} $\isvalue{x}$ that is true when 
``$x$ is a value'' and use the SMT solver to 
\emph{explicitly} reason about divergence.
%
(Note that $\isvalue{x}$ is equivalent to introducing a 
$\ebot$ constant denoting divergence, and 
writing $(x \not =\ \ebot)$.)
%
Unfortunately, this $\isvalue{x}$ predicate takes the VCs 
outside the scope of the standard efficiently decidable logics 
supported by SMT solvers.
%
%%To see why, 
%%the subtyping query (\ref{sub:good}) 
%%from @good@ in \S~\ref{sec:overview}
%%will reduce to the below VC
%%with explicit $\isvalue{x}$ predicates:
To see why, recall 
the subtyping query %(\ref{sub:good}) 
from @good@ in \Sref{sec:overview}. 
With explicit value predicates, 
this subtyping reduces to the VC:
%
\begin{align}
\begin{array}{l}
{(\isvalue{x} \Rightarrow x \geq 0)}\\ 
{(\isvalue{y} \Rightarrow y \geq 0)} 
\end{array} 
\Rightarrow
{(v = y+1)}   \Rightarrow {(v > 0)}\label{vc:good:explicit}
\end{align}
%
To prove the above valid, we require the knowledge 
that $(v = y+1)$ implies that $y$ is a value, \ie that 
$\isvalue{y}$ holds.
%
This fact, while obvious to a \emph{human} reader, is 
outside the decidable theories of linear arithmetic
of the existing SMT solvers.
%
Thus, existing solvers would be unable to prove (\ref{vc:good:explicit}) 
valid, causing us to reject @good@.

%%%%%%%%%%%%%%%%%%%%%%%%%%%%%%%%%%%%%%%%%%%%%%%%%%%%%%%%%%%%%%%%%%%%%%%%%%%%%%%
%%%%%%%%%%%%%%%%%%%%%%%%%%%%%%%%%%%%%%%%%%%%%%%%%%%%%%%%%%%%%%%%%%%%%%%%%%%%%%%
%%%%%%%%%%%%%%%%%%%%%%%%%%%%%%%%%%%%%%%%%%%%%%%%%%%%%%%%%%%%%%%%%%%%%%%%%%%%%%%

\mypara{Possible Fix: Explicit Reasoning With Axioms?}
%
One possible fix for the above would be to specify a collection of
\emph{axioms} that characterize how the value predicate behaves with 
respect to the other theory operators. 
%
For example, we might specify axioms like: 
%
\begin{align*}
\forall x,y,z. (x = y + z)\ &\Rightarrow\ (\isvalue{x} \wedge \isvalue{y} \wedge \isvalue{z})\\
% \forall x,y,z. (x = y - z)\ &\Rightarrow\ (\isvalue{x} \wedge \isvalue{y} \wedge \isvalue{z})\\\
\forall x,y. (x < y )\ &\Rightarrow\ (\isvalue{x} \wedge \isvalue{y})
% &\mathit{etc.}
\end{align*}
%
\etc. However, this is a non-solution for several reasons. 
First, it is not clear what a complete set of axioms is.
Second, there is the well known loss of predictable checking
that arises when using axioms, as one must rely on various 
brittle, syntactic matching and instantiation heuristics~\cite{simplifyj}. 
%
It is unclear how well these heuristics will work with the
sophisticated linear programming-based algorithms used to 
decide arithmetic theories. 
%
Thus, proper support for value predicates could require 
significant changes to existing decision procedures, 
making it impossible to use existing SMT solvers.


%%%%%%%%%%%%%%%%%%%%%%%%%%%%%%%%%%%%%%%%%%%%%%%%%%%%%%%%%%%%%%%%%%%%%%%%%%%%%%%
%%%%%%%%%%%%%%%%%%%%%%%%%%%%%%%%%%%%%%%%%%%%%%%%%%%%%%%%%%%%%%%%%%%%%%%%%%%%%%%
%%%%%%%%%%%%%%%%%%%%%%%%%%%%%%%%%%%%%%%%%%%%%%%%%%%%%%%%%%%%%%%%%%%%%%%%%%%%%%%

\mypara{Possible Fix: Explicit Reasoning With Types?}
%
Another possible fix would be to encode the behavior of the
value predicates within the refinement types for different 
operators, after which the predicate itself could be treated 
as an \emph{uninterpreted function} in the refinement 
logic~\cite{bradleybook}. For instance, we could type 
the primitives:
%
\begin{code}
 (+) :: x:Int -> y:Int
     -> {v | v  =  x + y && Val x && Val y}
 (<) :: x:Int -> y:Int 
     -> {v | v <=> x < y && Val x && Val y}
\end{code}
%
While this approach requires \emph{no} changes to the SMT 
machinery, it makes specifications complex and verbose. 
%
%% (and not unlike having to sprinkle explicit ``non-null'' 
%% checks all over pointer manipulating programs!)
%
We cannot just add the value predicates to the primitives' 
specifications. Consider 
%
\begin{code}
 choose b x y = if b then x+1 else y+2
\end{code}
%
To reason about the output of @choose@ we must type it as:
%
\begin{code}
 choose :: Bool -> x:Int -> y:Int
        -> {v|(v > x && Val x)||(v > y && Val y)}  
\end{code}
%
Thus, the value predicates will pervasively clutter all 
signatures with strictness information, making the system 
unpleasant to use.

%%%%%%%%%%%%%%%%%%%%%%%%%%%%%%%%%%%%%%%%%%%%%%%%%%%%%%%%%%%%%%%%%%%%%
%%%%%%%%%%%%%%%%%%%%%%%%%%%%%%%%%%%%%%%%%%%%%%%%%%%%%%%%%%%%%%%%%%%%%
%%%%%%%%%%%%%%%%%%%%%%%%%%%%%%%%%%%%%%%%%%%%%%%%%%%%%%%%%%%%%%%%%%%%%

\mypara{Divergence Requires 3-Valued Logic}
Finally, for either ``fix'', the value predicate poses a 
model-theoretic problem: 
what is the meaning of $\isvalue{x}$? 
%what meaning do we give $\isvalue{x}$? 
%
One sensible approach is to extend the universe with a family of 
\emph{distinct} $\bot$ constants, such that $\isvalue{\bot}$ is false.
%
These constants lead inevitably into a three-valued logic 
(in order to give meaning to formulas like $\bot = \bot$).
%
Thus, even if we were to find a way to reason with the value 
predicate via axioms or types, we would have to ensure that 
we properly handled the 3-valued logic within 
existing 2-valued SMT solvers.

\mypara{Future Work}
Thus, in future work it would be worthwhile to address the above 
technical and usability problems to enable explicit reasoning with 
the value predicate.
%
This explicit system would be \emph{more expressive} than our 
stratified approach, \eg would let us check 
%
%\begin{code}
  @let x = collatz 10 in 12 `div` x+1@
%\end{code}
%
by encoding strictness inside the logic. Nevertheless, we suspect
such a verifier would use stratification to eliminate the value
predicate in the common case.
%
At any rate, until these hurdles are crossed, we can take comfort in
stratified refinement types and can just \emph{eagerly}
use termination to prove safety for \emph{lazy} languages.

%% If we could address the above problems
%% Thus, at this point, even though the natural route is to reason explicitly
%% with value predicates, as they would address the theoretical
%% robustness-to-divergence limitation described above, 
%% it is unclear how the above problems can be solved, 
%% and we believe they may be promising directions for 
%% future work.
%
%
%
%% Thus, at this point, even though the natural route is to reason explicitly
%% with value predicates, as they would address the theoretical
%% robustness-to-divergence limitation described above, 
%% it is unclear how the above problems can be solved, 
%% and we believe they may be promising directions for 
%% future work.
%% 
%% %
%% Of course, that does not mean they are unsolvable, just 
%% that the presence of value predicates means we cannot 
%% use \emph{existing}, off-the-shelf SMT solvers to 
%% achieve our goal of a sound and predictable 
%% refinement type checker for Haskell.


%%% Local Variables: 
%%% mode: latex
%%% TeX-master: "main"
%%% End: 


\subsection*{Acknowledgements}
We thank 
Kenneth Knowles, 
Kenneth L. McMillan, 
Andrey Rybalchenko, 
Philip Wadler, 
and the reviewers for their 
excellent suggestions and feedback about 
earlier versions of this paper.

{
\bibliographystyle{plain}
\bibliography{sw}
}

\ifthenelse{\equal{\isTechReport}{true}}
{
\appendix
\section{Declarative Typing: \undeclang}

\subsection{Definitions}
To simplify the metatheory we extend \undeclang so that
\begin{itemize}
\item Supports stratified types, and
\item explicitly contains \ebot, a primitive that has any type, but does not evaluate. 
\end{itemize}


\begin{figure}
$$
\begin{array}{rrcl}

\emphbf{Constants} \quad 
  & c & ::=    & 0,1,-1,\ldots \spmid \etrue, \efalse \\
  &   & \spmid & +,-,\ldots \spmid =, <, \ldots \spmid \ecrash 
  \\[0.05in]

\emphbf{Values} \quad 
  & v & ::= &  c \spmid \efun{x}{\typ}{e} \spmid \edapp{D}{e}
  \\[0.05in] 

\emphbf{Expressions} \quad 
  & e & ::=    & \ebot \spmid v \spmid x \spmid \eapp{e}{e} \spmid \elet{x}{e}{e} \\ 
  &   & \spmid & \ecase{e}{D}{\overline{x}}{e}{x} \\[0.05in] 

\emphbf{Basic Types} \quad 
  & \tbase & ::= & \tint \spmid \tbool \spmid T \\[0.05in] 

\emphbf{Label} \quad 
  & l
  & ::= 
  & \trivial \spmid \finite 
  \\[0.05in] 
  
\emphbf{Types} \quad 
  & \typ & ::= & \tlref{v}{\tbase}{}{e} \spmid \tlref{v}{\tbase}{l}{e} \spmid
  				 \tfunref{x}{\typ}{\typ}{v}{e} \\ 
\end{array}
$$

\hrule width 0.48\textwidth

$$
\begin{array}{rrcl}
\emphbf{Contexts} \quad 
  & C
  & ::= 
  &   	 \bullet 
  \spmid \eapp{C}{e} 
  \spmid \eapp{c}{C} 
  \spmid D\ \overline{e}\ C\ \overline{e}\\
  &&\spmid &
  \ecase{C}{D}{\overline{y}}{e}{x}
  \\[0.05in] 
\end{array}
$$

\caption{\undeclang: Syntax}
\label{fig:undeclang}
\label{fig:operational}
\end{figure}

Then, we define the function \erase{\bullet} that erases the refinements in types and environments:
\begin{align*}
\erase{\tlref{v}{B}{l}{e}}&=B^{l} &
\erase{\emptyset}&=\emptyset\\
\erase{\tfunref{x}{\tau_x}{\tau}{v}{e}}&= \erase{\tau_x} \rightarrow \erase{\tau} &
\erase{x\colon\tau, \Gamma}&= x\colon\erase{\tau},\erase{\Gamma}
\end{align*}

and variable substitution on types:
\begin{align*}
(\tref{v}{B}{l}{e})\sub{y}{e_y} &=\tref{v}{B}{l}{e\sub{y}{e_y}}\\
(\tfunref{x}{\tau_x}{\tau}{v}{e})\sub{y}{e_y} &=
	\tfunref{x}{(\tau_x\sub{y}{e_y})}{(\tau\sub{y}{e_y})}{v}{e\sub{y}{e_y}}\\
\end{align*}


We extend the typing rules with another rule that types \ebot with
\textbf{any} type getting the rules as defined in Figure~\ref{fig:proofs:typing}.
%
\NV{Questions on the Abstract Refinement paper:}
\NV{1. WF-Abs-alpha: why we put alpha into the environment?}
\NV{2. We do we need the first two well-formedness rules?}
\newcommand\interp[1]{\ensuremath{[\! | #1 |\!]}}


\begin{figure}[!ht]
\judgementHead{Well-Formedness}{\isWellFormed{\Gamma}{\sigma}}

$$
\inference
  {}
  {\isWellFormed{\Gamma}{\true(\vref)}}
  [\wtTrue]
$$

$$
\inference
    {\isWellFormed{\Gamma}{\areft(\vref)} && 
     \hastype{\Gamma}{\rvapp{\rvar}{e} \ \vref}{\tbbool}
    }
    {\isWellFormed{\Gamma}{(\areft \wedge \rvapp{\rvar}{e})(\vref)}}
    [\wtRVApp]
$$

$$\inference
    {\hastype{\Gamma, \vref:b}{\reft}{\tbbool} \quad 
     \isWellFormed{\Gamma, \vref:b}{\areft(\vref)}
    }
    {\isWellFormed{\Gamma}{\tpref{b}{\areft}{\reft}}}
    [\wtBase]
$$

$$
\inference
    {
	\hastype{\Gamma}{\reft}{\tbbool} &&
    \isWellFormed{\Gamma}{\tau_x} &&
	\isWellFormed{\Gamma, x:\tau_x}{\tau}
    }
    {\isWellFormed{\Gamma}{\trfun{x}{\tau_x}{\tau}{\reft}}}
    [\wtFun]
$$


$$\begin{array}{ccc}
\inference
  {\isWellFormed{\Gamma, \rvar:\tau}{\sigma}}
  {\isWellFormed{\Gamma}{\tpabs{\rvar}{\tau}{\sigma}}}
  [\wtPred]
&
\quad
&
\inference
    {\isWellFormed{\Gamma}{\sigma}}
    {\isWellFormed{\Gamma}{\ttabs{\alpha}{\sigma}}}
    [\wtPoly]
\end{array}$$

\medskip \judgementHead{Subtyping}{\isSubType{\Gamma}{\sigma_1}{\sigma_2}}

$$
\inference
   {\text{SMT-Valid}(\inter{\Gamma} \land \inter{\areft_1\ \vref} \Rightarrow \inter{\reft_1} 
                 \Rightarrow \inter{\areft_2\ \vref} \land \inter{\reft_2})}
   {\isSubType{\Gamma}{\tpref{b}{\areft_1}{\reft_1}}{\tpref{b}{\areft_2}{\reft_2}}}
   [\tsubBase]
$$

$$
\inference
   {%\text{Valid}(\inter{\Gamma}\land \inter{e_1} \Rightarrow \inter{e_2}) \\
	\isSubType{\Gamma}{\tau_2}{\tau_1} &
	\isSubType{\Gamma, x_2:{\tau_2}}{\SUBST{\tau_1'}{x_1}{x_2}}{\tau_2'}	
   }
   {\isSubType{\Gamma}
	  {\trfun{x_1}{\tau_1}{\tau_1'}{\reft_1}}
	  {\trfun{x_2}{\tau_2}{\tau_2'}{\true}}
}[\tsubFun]
$$


$$
\inference
   {\isSubType{\Gamma, \rvar:\tau}{\sigma_1}{\sigma_2}}
   {\isSubType{\Gamma}{\tpabs{\rvar}{\tau}{\sigma_1}}{\tpabs{\rvar}{\tau}{\sigma_2}}}
   [\tsubPred]
$$
$$
\inference
   {\isSubType{\Gamma}{\sigma_1}{\sigma_2}}
   {\isSubType{\Gamma}{\ttabs{\alpha}{\sigma_1}}{\ttabs{\alpha}{\sigma_2}}}
   [\tsubPoly]
$$

\medskip \judgementHead{Type Checking}{$\hastype{\Gamma}{e}{\sigma}$}

$$
\inference
  {  \hastype{\Gamma}{e}{\sigma_2} && \isSubType{\Gamma}{\sigma_2}{\sigma_1} 
  && \isWellFormed{\Gamma}{\sigma_1}
  }
  {\hastype{\Gamma}{e}{\sigma_1}}
  [\tsub]
$$

$$
\inference
  {}
  {\hastype{\Gamma}{c}{\tc{c}}}
  [\tconst]
$$

$$
\inference
  {x: \tpref{b}{\areft}{\reft} \in \Gamma}
  {\hastype{\Gamma}{x}{\tpref{b}{\areft}{e \land \vref = x}}}
  [\tbase]
$$
$$
\inference
  {x:\tau \in \Gamma}
  {\hastype{\Gamma}{x}{\tau}} 
  [\tvariable]
$$

$$
\inference
   {\hastype{\Gamma, x:\tau_x}{e}{\tau} 
    && \isWellFormed{\Gamma}{\tau_x}
   }
   {\hastype{\Gamma}{\efunt{x}{\tau_x}{e}}{\tfun{x}{\tau_x}{\tau}}}
   [\tfunction]
$$
$$
\inference
   {\hastype{\Gamma}{e_1}{\tfun{x}{\tau_x}{\tau}} 
   &&  \hastype{\Gamma}{e_2}{\tau_x}
   }
   {\hastype{\Gamma}{\eapp{e_1}{e_2}}{\SUBST{\tau}{x}{e_2}}}
   [\tapp]
$$
$$
\inference
  {\hastype{\Gamma, \alpha}{e}{\sigma}}
  {\hastype{\Gamma}{\etabs{\alpha}{e}}{\ttabs{\alpha}{\sigma}}}
  [\tgen]
\qquad
\inference
  {\hastype{\Gamma}{e}{\ttabs{\alpha}{\sigma}} && 
   \isWellFormed{\Gamma}{\tau}
  }
  {\hastype{\Gamma}{\etapp{e}{\tau}}{\SUBST{\sigma}{\alpha}{\tau}}}
  [\tinst]
$$
$$
\inference
    {\hastype{\Gamma, \rvar:\tau}{e}{\sigma} &&
     \isWellFormed{\Gamma}{\tau} 
     % \tau \mbox{ is non-refined } 
     %\isWellFormed{\Gamma}{\tpabs{p}{\tau}{\pi}} && 
     %p \notin \fv{e}
    }
    {\hastype{\Gamma}{\epabs{\rvar}{\tau}{e}}{\tpabs{\rvar}{\tau}{\sigma}}}
    [\tpgen]
$$
$$
\inference
    {\hastype{\Gamma}{e}{\tpabs{\rvar}{\tau}{\sigma}} && 
     \hastype{\Gamma}{\efunbar{x:\tau_x}{\reft'}}{\tau}
    }
    {\hastype{\Gamma}
             {\epapp{e}{\efunbar{x:\tau_x}{\reft'}}}
             {\rpinst{\sigma}{\rvar}{\efunbar{x:\tau_x}{\reft'}}}
    }
    [\tpinst]
$$
\caption{\textbf{Static Semantics: Well-formedness, Subtyping and Type Checking}}
\label{fig:rules}
\end{figure}


\begin{figure}[!ht]
\judgementHead{Well-Formedness}{\isWellFormed{\Gamma}{\sigma}}

$$
\inference{
   \isWellFormed{\Gamma}{\Gamma_\constraint}             && 
	\isWellFormed{\Gamma, \Gamma_\constraint}{r_1} &&
	\isWellFormed{\Gamma, \Gamma_\constraint}{r_2} &&
	\isWellFormed{\Gamma}{\tau}
}{
	\isWellFormed{\Gamma}{\{\isSubType{\Gamma_\constraint}{r_1}{r_2}\} \carrow  \tau}
}[\wfBounded]
$$

$$
\inference{
}{
	\isWellFormed{\Gamma}{\emptyset}
}[\wfGammaEmpty]
$$
$$
\inference{
	\isWellFormed{\Gamma_1}{\sigma} && \isWellFormed{\Gamma_1, x\colon\sigma}{\Gamma_2}
}{
	\isWellFormed{\Gamma_1}{x\colon\sigma, \Gamma_2}
}[\wfGammaNonEmpty]
$$


\medskip \judgementHead{Subtyping}{\isSubType{\Gamma}{\sigma_1}{\sigma_2}}

$$
\inference{
    \isSubType{\Gamma_\constraint}{r_1}{r_2} && 
    \isSubType{\Gamma}{\tau_1}{\tau_2}
}{
   \isSubType{\Gamma}{\{\isSubType{\Gamma_\constraint}{r_1}{r_2}\} \carrow  \tau_1}{\tau_2}
}[\tsubPoly]
$$

\medskip \judgementHead{Type Checking}{$\hastype{\Gamma}{e}{\sigma}$}
$$
\inference{
	\text{fresh}\ x &&
	\hastype{\Gamma, x\colon\tref{\tbunit}{\interp{\isSubType{\Gamma_\constraint}{r_1}{r_2}}}}{e}{\tau}
}{
  \hastype{\Gamma}{e}{\{\isSubType{\Gamma_\constraint}{r_1}{r_2}\} \carrow  \tau}
}[\tsub]
$$
$$
\inference{
  \hastype{\Gamma}{e}{\{\isSubType{\Gamma_\constraint}{r_1}{r_2}\} \carrow  \tau}
  && \isSubType{\Gamma, \Gamma_\constraint}{r_1}{r_2}
}{
  \hastype{\Gamma}{e}{\tau}
}[\tsub]
$$
\caption{\textbf{Extensions In Reality}}
\label{fig:rules}
\end{figure}


\begin{figure}[!ht]
\judgementHead{Well-Formedness}{\isWellFormed{\Gamma}{\sigma}}

$$
\inference{
	\isWellFormed{\Gamma}{\constraint} &&
	\isWellFormed{\Gamma}{\tau}
}{
	\isWellFormed{\Gamma}{\tconstraint{\constraint}{\tau}}
}[\wfBounded]
$$

\medskip \judgementHead{Subtyping}{\isSubType{\Gamma}{\sigma_1}{\sigma_2}}

$$
\inference{
    \isSubType{\Gamma_\constraint}{r_1}{r_2} && 
    \isSubType{\Gamma}{\tau_1}{\tau_2}
}{
   \isSubType{\Gamma}
                    {\tconstraint{\isSubType{\Gamma_1}{r_{11}}{r_{12}}}{\tau_1}}
                    {\tconstraint{\isSubType{\Gamma_2}{r_{21}}{r_{22}}}{\tau_2}}
}[\tsubPoly]
$$

\medskip \judgementHead{Type Checking}{$\hastype{\Gamma}{e}{\sigma}$}
$$
\inference{
	\hastype{\Gamma}{e}{\tconstraint{\isSubType{\Gamma'}{r_1}{r_2}}{\tau}} &&
	\isSubType{\Gamma, \Gamma'}{r_1}{r_2}
}{
  \hastype{\Gamma}{e\ \econstantconstraint}{\tau}
}
$$
$$
\inference{
	\text{fresh}\ x &&
	\isWellFormed{\Gamma}{\phi} && \hastype{\Gamma, x\colon\tref{\tbunit}{\inter{\constraint}}}{e}{\tau}
}{
  \hastype{\Gamma}{(\econstraint{e})}{\tconstraint{\constraint}{\tau}}
}
$$
\caption{\textbf{Extensions In Theory}}
\label{fig:rules}
\end{figure}



\begin{figure}[!ht]
\judgementHead{Well-Formedness-Constraint}{\isWellFormed{\Gamma}{\constraint}}
$$
\inference{
	\isWellFormed{\Gamma}{\Gamma'} 
	&& \isWellFormed{\Gamma, \Gamma'}{r_1}
	&& \isWellFormed{\Gamma, \Gamma'}{r_2}
}{
	\isWellFormed{\Gamma}{(\isSubType{\Gamma'}{r_1}{r_2})}
}
$$

$$
\inference{
}{
	\isWellFormed{\Gamma}{\emptyset}
}% [\wfGammaEmpty]
\qquad
\inference{
	\isWellFormed{\Gamma_1}{\sigma} && \isWellFormed{\Gamma_1, x\colon\sigma}{\Gamma_2}
}{
	\isWellFormed{\Gamma_1}{x\colon\sigma, \Gamma_2}
}% [\wfGammaNonEmpty]
$$

\caption{\textbf{More Rules}}
\label{fig:rules}
\end{figure}

\begin{figure}
$$
\begin{array}{l}
\interp{\isSubType{\Gamma}{\tpref{b}{\areft_1}{\reft_1}}{\tpref{b}{\areft_2}{\reft_2}}} \\
  \qquad =\forall(dom(\Gamma), \vref) . \inter{\Gamma} \land \inter{\areft_1\ \vref} \Rightarrow \inter{\reft_1} 
                 \Rightarrow \inter{\areft_2\ \vref} \land \inter{\reft_2} 
\end{array}
$$
\label{fig:rules}
\end{figure}


\NV{make clear how forall is an expression}

\subsection{Translation}
Let $n$ be the the maximum number of constraints that appear in a type in the program. 
Then 





$$\begin{array}{rclcrcl}
\tx{\true\ \vref} & \defeq 
  & \true  
  & \quad \quad &

\tx{\tpabs{\rvar}{\tau}{\sigma}} & \defeq 
  & \tfun{\rvar}{\tx{\tau}}{\tx{\sigma}} \\

\tx{(\areft \land \rvapp{\rvar}{e})\ \vref} & \defeq 
  & \tx{\areft\ \vref} \land \eapp{\eapp{\rvar}{\overline{e}}}{\vref} 
  & \quad \quad &

\tx{\ttabs{\alpha}{\sigma}} & \defeq 
  & \ttabs{\alpha}{\tx{\sigma}} \\

\tx{\tpref{b}{\areft}{\reft}} & \defeq 
  & \tref{b}{\reft \land \tx{\areft\ \vref}} 
  & \quad \quad &

\tx{\tfun{x}{\tau_1}{\tau_2}} & \defeq 
  & \tfun{x}{\tx{\tau_1}}{\tx{\tau_2}} 
%\tx{\trfun{x}{\tau_1}{\tau_2}{\reft}} \defeq 
%  & \trfun{x}{\tx{\tau_1}}{\tx{\tau_2}}{\tx{\reft}} \\
\end{array}$$
Similarly, we translate \corelan terms $e$ to \conlan 
terms $\tx{e}$ by converting refinement abstraction and application 
to $\lambda$-abstraction and application
$$\begin{array}{rclcrcl}
\tx{x} & \defeq & x & \quad \quad \quad & \tx{c} & \defeq & c \\
\tx{\efunt{x}{\tau}{e}} & \defeq & \efunt{x}{\tx{\tau}}{\tx{e}} & \quad & \tx{\eapp{e_1}{e_2}} & \defeq & \eapp{\tx{e_1}}{\tx{e_2}} \\
\tx{\etabs{\alpha}{e}} & \defeq & \etabs{\alpha}{\tx{e}} & \quad & \tx{\etapp{e}{\tau}} & \defeq & \eapp{\tx{e}}{\tx{\tau}} \\
\tx{\epabs{\rvar}{\tau}{e}} &\defeq & \efunt{\rvar}{\tx{\tau}}{\tx{e}} & \quad & \tx{\epapp{e_1}{e_2}} &\defeq & \eapp{\tx{e_1}}{\tx{e_2}}
\end{array}$$








%

We define the denotations of types by combining the denotations 
of stratified types:
\begin{definition}{[Type Denotations]}
\begin{align*}
\interp{\tref{x}{\tbase}{}{p}} \defeq & 
    \{e \mid  \hastypebase{\emptyset}{e}{\tbase},
              \mbox{ if } \evals{e}{v} 
              \mbox{ then } \evals{\SUBST{p}{x}{v}}{\etrue} \}\\
\interp{\tlref{v}{\tbase}{\trivial}{p}} \defeq & 
    \interp{\tlref{v}{\tbase}{}{p}} \cap \{ e \mid \evals{e}{v} \}\\
\interp{\tlref{v}{\tbase}{\finite}{p}} \defeq & 
    \interp{\tlref{v}{\tbase}{\trivial}{p}} \cap \{ e \mid \evals{e}{d} \} \\
\interp{\tfun{x}{\typ_x}{\typ}} \defeq & 
    \{e \mid  \hastypebase{\emptyset}{e}{\erase{\tfunbasic{\typ_x}{\typ}}}, 
              \forall e_x \in \interp{\typ_x}.\ \eapp{e}{e_x} \in \interp{\typ\sub{x}{e_x}}
    \}
\end{align*}
\end{definition}

Finally, we define the constraints that should be satisfied by constants:
%
\begin{definition}{[Constants]}\label{def:constants}
For every basic type $T$ there are exactly  $n = \arity{T}$ 
data contractors with result type $T$, namely 
$\{D_T^i | 0 < i \leq n \}$.

\CRASH is an untyped constant.
%
For each constant $c \neq \CRASH$
\newcommand\pcond[1]{\ensuremath{}}
\newcommand\const{\ensuremath{c}}
\begin{enumerate}
\item \hastype{\emptyset}{c}{\constty{\const}} and \iswellformed{}{\constty{c}}
%
\item If $\constty{c} = \tfun{x}{\tau_x}{\tau}$, then for each $v$, 
	$\ceval{\const}{v}$ is defined and 
	if \hastype{\emptyset}{v}{\tau_x} then
	\shastype{}{\interp{c}(v)}{\tau\sub{x}{v}},
	otherwise  $\interp{c}(v) = \CRASH$.
%	
\item If $\constty{c} = \tref{v}{B}{l}{e}$, 
	then 
	$c \in \interp{\constty{c}}$ and 
	$\forall c', c' \neq c. c' \not \in \interp{\constty{c}}$ 
%
\item If $\constty{D_T^i} = \tfun{x_1}{\tau_1}{\dots\tfun{x_n}{\tau_n}{\tau}}$, 
then $\tau_i$ are unrefined and for every $e_i$ with $0 < i \leq n$,
such that \hastype{\emptyset}{e_i}{\tau_i}, 
$D_T^i\ \overline{e_i}\in \interp{\tau\sub{x_i}{e_i}}$.
\end{enumerate}
\end{definition}


\subsection{Denotational Typing}
We define denotational typing as follows:
\begin{align*}
\shastype{\Gamma}{e}{\tau} & \doteq
	\forall \theta . \theta\in\interp{\Gamma}\Rightarrow \theta\ e \in \interp{\theta \ \tau}\\
\sissubtype{\Gamma}{\tau_1}{\tau_2} & \doteq 
	\forall \theta . \theta\in\interp{\Gamma}\Rightarrow \interp{\theta\ \tau_1} \subseteq \interp{\theta\ \tau_2}
\end{align*}

And prove that syntactic typing implies denotational typing, 
\ie a general version of Lemma~\ref{lem:denotation} of the paper.



\begin{lemma}{[Denotation Typing]}\label{lem:proofs:denotation}
\begin{enumerate}
\item If \issubtype{\Gamma}{\tau_1}{\tau_2} then \sissubtype{\Gamma}{\tau_1}{\tau_2}. 
\item If \hastype{\Gamma}{e}{\tau} then \shastype{\Gamma}{e}{\tau}.
\end{enumerate}
\end{lemma} 
\begin{proof}
Helping Lemma:
\begin{lemma}\label{lemma:closesem}
If \evals{e}{e'} then $e' \in \interp{\tau}$ \textit{iff} $e \in \interp{\tau}$.
\end{lemma}
\begin{proofsketch}
Since the validity of $e \in \interp{\tau}$ depends on the evaluated $e$, 
the if direction is evident.
The only if direction follows from the deterministic operational semantics.
\end{proofsketch}

%
\begin{enumerate}
\item \label{proof:ssub} Assume \issubtype{\Gamma}{\tau_1}{\tau_2}. 
We will prove it by induction on the derivation tree:

\begin{itemize}
\item\rsubbase. We have
$$\issubtype{\Gamma}{\tref{v}{B}{l}{e_1}}{\tref{v}{B}{l}{e_2}}$$
By inversion we get 
$$\issubref{\Gamma, v\colon B}{e_1}{e_2}$$
By inversion of \rimpl we have
$$	\forall \theta. \theta\in \interp{\Gamma}\Rightarrow
	\generalconditionImpl{\thetasub{\theta}{e_1}}
						{\thetasub{\theta}{e_2}}
\ (1)$$

We want to prove 
$$\sissubtype{\Gamma}{\tref{v}{B}{l}{e_1}}{\tref{v}{B}{l}{e_2}}$$
Equivalently
$$	
	\forall \theta . \iswellformedtheta{\Gamma}{\theta} \Rightarrow 
	\interp{\theta\ \tref{v}{B}{l}{e_1}} \subseteq \interp{\theta\ \tref{v}{B}{l}{e_2}}
$$

Since the labels are the same it suffices to prove that
\begin{align*}
	\forall \theta . \iswellformedtheta{\Gamma}{\theta} & \Rightarrow 
		\{e \mid \hastype{}{e}{B} 
 			\land 
			\generalconditionInterp{e}{\thetasub{\theta}{e_1\sub{v}{e}}} 
		\}	
	\\& \subseteq 
		\{e \mid \hastype{}{e}{B} 
			\land 
			\generalconditionInterp{e}{\thetasub{\theta}{e_2\sub{v}{e}}}
		 \}	
\end{align*}
Since $e \in \interp{B}$, we have \iswellformed{\Gamma,v\colon B}{\theta,\sub{v}{e}}.
So, from $(1)$ for $\theta := \theta,\sub{v}{e}$
we have 
$$	
	\generalconditionImpl
		{\thetasub{\theta}{e_1\sub{v}{e}}}
		{\thetasub{\theta}{e_2\sub{v}{e}}}
$$
\item\rsubfun Assume
$$
	\issubtype{\Gamma}{\tfunref{x}{\tau_x}{\tau}{v}{e_1}}{\tfunref{x}{\tau'_x}{\tau'}{v}{e_2}}
$$
By inversion we have
$$	
	\issubtype{\Gamma}{\tau'_x}{\tau_x} \qquad
	\issubtype{\Gamma, x \colon \tau'_x}{\tau}{\tau'} 
$$
By IH
$$	
	\sissubtype{\Gamma}{\tau'_x}{\tau_x} \ (1) \qquad
	\sissubtype{\Gamma, x \colon \tau'_x}{\tau}{\tau'} \ (2)
$$
We want to show that 
$$
	\sissubtype{\Gamma}
		{\tfunref{x}{\tau_x}{\tau}{v}{e_1}}
		{\tfunref{x}{\tau'_x}{\tau'}{v}{e_2}}
$$
Equivalently
$$	
	\forall \theta . \iswellformedtheta{\Gamma}{\theta} \Rightarrow 
	\interp{\thetasub{\theta}{\tfunref{x}{\tau_x}{\tau}{v}{e_1}}} 
	\subseteq 
	\interp{\thetasub{\theta}{\tfunref{x}{\tau'_x}{\tau'}{v}{e_2}}}
$$
Equivalently
\begin{align*}
	&\forall \theta. \iswellformedtheta{\Gamma}{\theta} \\&\Rightarrow 
	\{e \mid \hastype{}{e}{\erase{\tau_x} \rightarrow \erase{\tau}} 
	\land 
	\forall e_x \in \interp{\thetasub{\theta}{\tau_x}}. \
	 \eapp{e}{e_x} \in \interp{\thetasub{\theta}{\tau\sub{x}{e_x}}} 
	 \}\\ &
	\subseteq 
	\{e \mid \hastype{}{e}{\erase{\tau'_x} \rightarrow \erase{\tau'}} 
	\land 
	\forall e_x \in \interp{\thetasub{\theta}{\tau'_x}}. \
	 \eapp{e}{e_x} \in \interp{\thetasub{\theta}{\tau'\sub{x}{e_x}}} 
	 \}
\end{align*}
The above holds, as for any $e, e_x$
if $e_x \in \interp{\thetasub{\theta}{\tau_x'}}$
then by $(1)$
$e_x \in \interp{\thetasub{\theta}{\tau_x}}$.
Also, by $(2)$
if $\eapp{e}{e_x} \in \interp{\thetasub{\theta}{\tau\sub{x}{e_x}}}$
then
$\eapp{e}{e_x} \in \interp{\thetasub{\theta}{\tau'\sub{x}{e_x}}}$.
\end{itemize}


\item Assume \hastype{\Gamma}{e}{\tau}. 
We will prove it by induction on the derivation tree.
\begin{itemize}
\item\rtvar Assume \hastype{\Gamma}{e}{\tau}
	where $e \equiv x$.
	By inversion we have
	$$(x,\tau) \in \Gamma$$
	We need to show that 
	$$	\forall \theta . \iswellformedtheta{\Gamma}{\theta} 
		\Rightarrow \thetasub{\theta}{x} \in \interp{\thetasub{\theta}{\tau}}$$
	Which holds by the definition of well-formed substitutions.

\item\rtconst. Assume \hastype{\Gamma}{e}{\tau}
	where $e \equiv c$  and $\tau\equiv\constty{c}$.
	Then \shastype{\Gamma}{e}{\tau} holds by Definition \ref{def:constants}.

\item\rtsub Assume \hastype{\Gamma}{e}{\tau}.
	By inversion
	$$
	\hastype{\Gamma}{e}{\tau'}\ (1) \qquad
	\issubtype{\Gamma}{\tau'}{\tau}\ (2) \qquad
	\iswellformed{\Gamma}{\tau}\ (3)
	$$
%
	By IH on $(1)$ we have
	$$\shastype{\Gamma}{e}{\tau'}\ (4)$$
%
	By \ref{proof:ssub} on $(2)$
	$$\sissubtype{\Gamma}{\tau'}{\tau}\ (5)$$
%
	By $(4)$ and $(5)$ we get
	$$\shastype{\Gamma}{e}{\tau}$$

\item\rtfun Assume \hastype{\Gamma}{e}{\tau},
	where $e \equiv \efun{x}{}{e'}$ and 
	$\tau \equiv\tfun{x}{\tau'_x}{\tau'}$.
	By inversion we get
	$$
	\hastype{\Gamma, x\colon\tau'_x}{e'}{\tau'}\ (1) \qquad
	\iswellformed{\Gamma}{\tau'_x}\ (2)
	$$
	By IH on $(1)$ we have
	$$
	\shastype{\Gamma, x\colon\tau'_x}{e'}{\tau'}\ (3)
	$$
	Equivalently
	$$	
	\forall \theta . \iswellformedtheta{(\Gamma,x\colon\tau'_x)}{(\theta\sub{x}{e_x})} 
		\Rightarrow \thetasub{(\theta\sub{x}{e_x})}{e'} \in 
		\interp{\thetasub{(\theta\sub{x}{e_x})}{\tau'}}\\
	$$
	Or
	$$	
	\forall \theta . \iswellformedtheta{\Gamma}{\theta} \Rightarrow
	\forall e_x . e_x \in \interp{\tau'_x} \Rightarrow
		\thetasub{\theta}{\eapp{e}{e_x}} \in \interp{\thetasub{\theta}{\tau'\sub{x}{e_x}}}\\
	$$
%
	Moreover, $\hastypebase{}{e}{\erase{\tau'_x}\rightarrow{\erase{\tau}}}$.
%
	So,
	$$	
	\forall \theta . \iswellformedtheta{\Gamma}{\theta}. \thetasub{\theta}{e}\in \interp{\thetasub{\theta}{\tau}}
	$$
	Or, $$\shastype{\Gamma}{e}{\tau}$$

\item\rtapp. Assume \hastype{\Gamma}{e}{\tau},
	where $e\equiv\eapp{e_1}{e_2}$ and $\tau\equiv\tau'\sub{x}{e_2}$.
	By inversion:
	$$
	\hastype{\Gamma}{e_1}{(\tfunref{x}{\tau'_{x}}{\tau'}{v}{e_r})}\ (1)\qquad
	\hastype{\Gamma}{e_2}{\tau'_{x}}\ (2)
	$$
	By IH we get
	$$
	\shastype{\Gamma}{e_1}{(\tfunref{x}{\tau'_{x}}{\tau'}{v}{e_r})}\ (3)\qquad
	\shastype{\Gamma}{e_2}{\tau'_{x}}\ (4)
	$$
	So 
	$$\forall \theta. \iswellformedtheta{\Gamma}{\theta}\Rightarrow
	\forall e_x \in \interp{\thetasub{\theta}{\tau'_x}} \Rightarrow
		\eapp{(\thetasub{\theta}{e_1})}{e_x} \in 
		\interp{\thetasub{\theta}{\tau'\sub{x}{e_x}}}
	\ (5)$$
	and
	$$\forall \theta. \iswellformedtheta{\Gamma}{\theta}\Rightarrow
		\thetasub{\theta}{e_2} \in 
		\interp{\thetasub{\theta}{\tau'_x}}
	\ (6)$$
%
	From $(5)$ and $(6)$, we get
	$$\forall \theta. \iswellformedtheta{\Gamma}{\theta}\Rightarrow
		\theta\ e \in \interp{\thetasub{\theta}{\tau}}
	$$
	Or $$\shastype{\Gamma}{e}{\tau}$$

\item\rtlet. Assume \hastype{\Gamma}{e}{\tau}, 
	where $e \equiv\elet{x}{e_x}{e'}$.
	By inversion:
	$$
	\hastype{\Gamma}{e_x}{\tau_{x}}\ (1) \qquad
	\hastype{\Gamma,x\colon\tau_x}{e'}{\tau}\ (2)\qquad
	\iswellformed{\Gamma}{\tau}\ (3)
	$$
	By IH we get
	$$
	\shastype{\Gamma}{e_x}{\tau_{x}}\ (4) \qquad
	\shastype{\Gamma,x\colon\tau_x}{e'}{\tau}\ (5)
	$$
	By $(5)$
	$$\forall \theta'. \iswellformedtheta{\Gamma, x:\tau_x}{\theta'}\Rightarrow
		\thetasub{\theta'}{e'} \in \interp{\thetasub{\theta'}{\tau}}
		\ (6)
	$$
	By $(4)$, 
	$$
	 	\iswellformedtheta{\Gamma}{\theta}
		\Rightarrow 
		\iswellformedtheta{\Gamma, x:\tau_x}{\theta\sub{x}{e_x}}
	\ (7)$$
	From $(6)$, $(7)$ and $(3)$, we get
	$$\forall \theta. \iswellformedtheta{\Gamma}{\theta}\Rightarrow
		\thetasub{\theta}{e'\sub{x}{e_x}} \in \interp{\thetasub{\theta}{\tau}}
	$$
	By Lemma \ref{lemma:closesem}, we get
	$$\forall \theta. \iswellformedtheta{\Gamma}{\theta}\Rightarrow
		\thetasub{\theta}{e} \in \interp{\thetasub{\theta}{\tau}}
	$$
	So, $$\shastype{\Gamma}{e}{\tau}$$
\item\rtbot Assume \hastype{\Gamma}{e}{\tau}, 
	where $e \equiv\ebot$ and $\tau \equiv \tref{v}{B}{}{p}$.
	Since \ebot does not evaluate, 
	$$\forall \theta. \iswellformedtheta{\Gamma}{\theta}\Rightarrow
		\thetasub{\theta}{e} \in \interp{\thetasub{\theta}{\tau}}
	$$
	So, $$\shastype{\Gamma}{e}{\tau}$$

\item\rtcase Assume \hastype{\Gamma}{e}{\tau}, 
	where $e' \equiv \ecase{e}{D^i_T}{\overline{y}}{e_i}{x}$.
	By inversion
$$
	l \not \in \{\finite, \trivial\} \Rightarrow \tau \ \text{is}\ \Div\ (1)\qquad
	\hastype{\Gamma}{e}{\tref{v}{T}{l}{e_T}} \ (2)\qquad
	 \iswellformed{\Gamma}{\tau}\ (3)
$$
$$
\forall i. 0 < i \leq \arity{T}\{
$$
$$
	\constty{D^i_T} = \tfun{y_1}{\tau_1}{\dots\rightarrow\tfun{y_n}{\tau_n}{\tref{v}{T}{}{e_{T_i}}}}\ (4)
$$
$$
		\hastype{\Gamma,  
				\overline{y_i\colon \tau_i},
				x\colon\tlref{v}{T}{\restrictdecidable{\trivial}
				{\addtechnical{}{\ltrivial}}
				}{e_T \land e_{T_i}}}{e_i}{\tau}\ (5) \}
$$

By IH on $(2)$ we get 
$$
	\shastype{\Gamma}{e'}{\tref{v}{T}{l}{e_T}}\ (6)
$$

We fix a $\theta$ such that $\iswellformedtheta{\Gamma}{\theta}$
We split cases on whether \thetasub{\theta}{e'} evaluates to a WNF or not:
\begin{itemize}
\item If \evals{\thetasub{\theta}{e'}}{v}.
By $(6)$, for some $i$ such that $0 < i \leq \arity{T}$, 
%
$\evals{\thetasub{\theta}{e'}}{D^i_T\ \overline{e_j}}$.

By IH on $(4)$ and the Definition~\ref{def:constants}
$$
		\shastype{\Gamma}{e_i\sub{y_i}{e_j}\sub{x}{e'}}{\tau}
$$
Finally, by Lemma~\ref{lemma:closesem}
$$
		\shastype{\Gamma}{e}{\tau}
$$
\item If $\thetasub{\theta}{e'}$, then by $(6)$
$l \not \in \{\finite, \trivial \}$.
Moreover, $e$ diverges so it trivially belongs to the 
interpretation of any \Div type, or by $(1)$
$$
		\shastype{\Gamma}{e}{\tau}
$$
%%$$
%%\interp{\tref{v}{T}{}{p}} \doteq
%%\{
%%e \mid \hastypebase{}{e}{T}, 
%%\evals{e}{D^i_T\overline{e_j}} \Rightarrow
%%\constty{D^i_T} = \tfun{y_1}{\tau_1}{\dots\rightarrow\tfun{y_n}{\tau_n}{\tref{v}{T}{}{q}}} \Rightarrow
%%e_i \in \interp{\tau_i\sub{y_j}{e_j}}, 
%%\evals{p \land q\sub{y_j}{e_j}}{\etrue}
%%\}
%%$$
\end{itemize}
\end{itemize}
\end{enumerate}

\end{proof}

We define \iswellformed{}{\Gamma}
as \iswellformed{}{\emptyset} and if \iswellformed{\Gamma}{\tau} then \iswellformed{}{\Gamma, x:\tau}.
Now, using Lemma~\ref{lem:proofs:denotation} we prove substitution Lemma:
\begin{lemma}{[Substitution]}\label{lemma:substitution}
If \hastype{\Gamma}{e_x}{\tau_x} and \iswellformed{}{\Gamma, x\colon\tau_x ,\Gamma'}, then 
\begin{enumerate}
\item If 
	\issubtype{\Gamma, x\colon\tau_x, \Gamma'}{\tau_1}{\tau_2}
	then
	\issubtype{\Gamma, \sub{x}{e_x}\Gamma'}{\sub{x}{e_x}\tau_1}{\sub{x}{e_x}\tau_2}.
\item If 
	\hastype{\Gamma, x\colon\tau_x, \Gamma'}{e}{\tau}
	then
	\hastype{\Gamma, \sub{x}{e_x}\Gamma'}{\sub{x}{e_x}e}{\sub{x}{e_x}\tau}.
\item If 
	\iswellformed{\Gamma, x\colon\tau_x, \Gamma'}{\tau}
	then
	\iswellformed{\Gamma, \sub{x}{e_x}\Gamma'}{\sub{x}{e_x}\tau}.
\end{enumerate}
\end{lemma}
\begin{proof}
\newcommand\generalconditionImpol[2]{\ensuremath{\evals{#1}{\etrue}\Rightarrow \evals{#2}{\etrue}}}
If \hastype{\Gamma}{e_x}{\tau_x} and \iswellformed{\Gamma, x\colon\tau_x ,\Gamma'}, then 
\begin{enumerate}
\item\label{proof:sub:sub} Assume
	$$\issubtype{\Gamma, x\colon\tau_x, \Gamma'}{\tau_1}{\tau_2}$$
We will prove the lemma by induction on the derivation tree.
\begin{itemize}
\item \rsubbase
Assume \issubtype{\Gamma, x\colon\tau_x, \Gamma'}{\tau_1}{\tau_2}
where $\tau_1 \equiv \tref{v}{B}{l}{e_1}$
and   $\tau_2 \equiv \tref{v}{B}{l}{e_2}$.
By inversion
	$$
	\issubref{\Gamma, x\colon\tau_x, \Gamma',v:B}{e_1}{e_2}
	$$
By inversion
	\begin{align*}
	\forall &\theta, e'_x, \theta',e .
	\iswellformedtheta{\Gamma, x\colon\tau_x, \Gamma',v:B}
		{\theta\sub{x}{e'_x}\theta'\sub{v}{e}}\\& \Rightarrow
	\generalconditionImpl{\thetasub{\theta\sub{x}{e'_x}\theta'\sub{v}{e}}{e_1}\\&}
						 {\thetasub{\theta\sub{x}{e'_x}\theta'\sub{v}{e}}{e_2}}
	\end{align*}

Since \hastype{\Gamma}{e_x}{\tau_x}, so
	\begin{align*}
	\forall &\theta, \theta',e .
	\iswellformedtheta{\Gamma,x\colon\tau_x, \Gamma',v:B}{\theta \sub{x}{e_x}\theta'\sub{v}{e}}\\& \Rightarrow
	\generalconditionImpl{\thetasub{\theta\sub{x}{e_x}\theta'\sub{v}{e}}{e_1}\\&}
						 {\thetasub{\theta\sub{x}{e_x}\theta'\sub{v}{e}}{e_2}}
	\end{align*}
Since \hastype{\Gamma}{e_x}{\tau_x}, so
	\begin{align*}
	\forall &\theta, \theta',e .
	\iswellformedtheta{\Gamma,\sub{x}{e_x}\Gamma',v:B}{\theta\theta'\sub{v}{e}}\\& \Rightarrow
	\generalconditionImpl{\thetasub{\theta\theta'\sub{v}{e}}{e_1\sub{x}{e_x}}\\&}
						 {\thetasub{\theta\theta'\sub{v}{e}}{e_2\sub{x}{e_x}}}
	\end{align*}
So,
	$$
	\issubref{\Gamma, \sub{x}{e_x}\Gamma',v:B}{e_1\sub{x}{e_x}}{e_2\sub{x}{e_x}}
	$$
Or
	$$
	\issubtype{\Gamma, \sub{x}{e_x}\Gamma',v:B}{t_1\sub{x}{e_x}}{t_2\sub{x}{e_x}}
	$$
\item \rsubfun
Assume \issubtype{\Gamma, x\colon\tau_x, \Gamma'}{\tau_1}{\tau_2},
where $\tau_1 \equiv \tfun{y}{\tau_y}{\tau}$
and   $\tau_2 \equiv \tfun{y}{\tau'_y}{\tau'}$.
By inversion
	$$
	\issubtype{\Gamma, x\colon\tau_x, \Gamma'}{\tau'_y}{\tau_y}\ (1) \qquad
	\issubtype{\Gamma, x\colon\tau_x, \Gamma',y\colon\tau'_y}{\tau}{\tau'}\ (2)
	$$
By IH	
	$$
	\issubtype{\Gamma, \sub{x}{e_x}\Gamma'}{\tau'_y\sub{x}{e_x}}{\tau_y\sub{x}{e_x}} 
	$$
	$$
	\issubtype{\Gamma, \sub{x}{e_x}\Gamma',y\colon\tau'_y\sub{x}{e_x}}{\tau\sub{x}{e_x}}{\tau'\sub{x}{e_x}}
	$$
By rule \rsubfun	
	$$
	\issubtype{\Gamma, \sub{x}{e_x}\Gamma'}{\tau_1\sub{x}{e_x}}{\tau_2\sub{x}{e_x}}
	$$
\end{itemize}


\item \label{proof:sub:type} 
Assume 
	\hastype{\Gamma, x\colon\tau_x, \Gamma'}{e}{\tau}.
We will prove the lemma by induction on the derivation tree.
\begin{itemize}
\item \rtvar Assume \hastype{\Gamma, x\colon\tau'_x, \Gamma'}{e}{\tau},
where $e \equiv y$.
By inversion 
$$(y,\tau )\in \Gamma, x\colon\tau'_x, \Gamma'$$
Assume
$$(y,\tau)\in \Gamma$$
By rule \rtvar
$$\hastype{\Gamma,\sub{x}{e_x}\Gamma'}{e}{\tau}$$
Since \iswellformed{}{\Gamma}, $x$ cannot appear in $\tau$
so $\tau\sub{x}{e_x}\equiv\tau$.
Also, $x\neq y$, so $e\sub{x}{e_x}\equiv e$.
So,
$$\hastype{\Gamma,\sub{x}{e_x}\Gamma'}{e\sub{x}{e_x}}{\tau\sub{x}{e_x}}$$
%
Assume
$$y \equiv x$$
By lemma's assumption 
$$\hastype{\Gamma}{e_x}{\tau_x}$$
so
$$\hastype{\Gamma,\sub{x}{e_x}\Gamma'}{e_x}{\tau'_x}$$
Since $x = y$, $e\sub{x}{e_x} \equiv e_x$.
Also, since $x \notin Dom(\Gamma)$ 
it cannot appear in $\tau'_x$,so
$\tau\sub{x}{e_x} \equiv \tau \equiv \tau'_x$.
So,
$$\hastype{\Gamma,\sub{x}{e_x}\Gamma'}{e\sub{x}{e_x}}{\tau\sub{x}{e_x}}$$
%
Otherwise, assume
$$(y,\tau)\in \Gamma'$$
So,
$$(y,\sub{x}{e_x}\tau)\in \sub{x}{e_x}\Gamma'$$
Also, $e\sub{x}{e_x}\equiv e \equiv y$.
By which and rule \rtvar, we get
$$\hastype{\Gamma,\sub{x}{e_x}\Gamma'}{e\sub{x}{e_x}}{\tau\sub{x}{e_x}}$$

\item Case \rtconst.
Assume \hastype{\Gamma, x\colon\tau_x, \Gamma'}{e}{\tau},
where $e \equiv c$ and $\tau\equiv\constty{c}$.
Since $e\sub{x}{e_x} \equiv e$ and $\tau\sub{x}{e_x}\equiv\tau$
$$\hastype{\Gamma,\sub{x}{e_x}\Gamma'}{e\sub{x}{e_x}}{\tau\sub{x}{e_x}}$$

\item\rtsub
Assume \hastype{\Gamma, x\colon\tau_x, \Gamma'}{e}{\tau}.
By inversion
$$
\hastype{\Gamma, x\colon\tau_x, \Gamma'}{e}{\tau'}\ (1)\qquad
\issubtype{\Gamma, x\colon\tau_x, \Gamma'}{\tau'}{\tau} \ (2)
$$
$$
\iswellformed{\Gamma, x\colon\tau_x, \Gamma'}{\tau} \ (3)
$$
By IH, \ref{proof:sub:sub} and \ref{proof:sub:wf}
$$
\hastype{\Gamma, \sub{x}{e_x}\Gamma'}{\sub{x}{e_x}e}{\sub{x}{e_x}\tau'}
$$
$$
\issubtype{\Gamma, \sub{x}{e_x}\Gamma'}{\sub{x}{e_x}\tau'}{\sub{x}{e_x}\tau}
$$
$$
\iswellformed{\Gamma, \sub{x}{e_x}\Gamma'}{\sub{x}{e_x}\tau}
$$
By rule \rtsub
$$\hastype{\Gamma,\sub{x}{e_x}\Gamma'}{e\sub{x}{e_x}}{\tau\sub{x}{e_x}}$$

\item\rtfun Assume \hastype{\Gamma, x\colon\tau_x, \Gamma'}{e}{\tau},
where $e\equiv\efun{y}{e'}$ and $\tau\equiv\tfun{y}{\tau'_y}{\tau'}$.
By inversion
	$$
	\hastype{\Gamma, x\colon\tau_x, \Gamma', y\colon\tau'_y}{e'}{\tau'}\ (1)\qquad
	\iswellformed{\Gamma, x\colon\tau_x, \Gamma'}{\tau'_y}\ (2)
	$$
By IH and \ref{proof:sub:wf}
	$$
	\hastype{\Gamma,\sub{x}{e_x} \Gamma', y\colon\sub{x}{e_x}\tau'_y}{\sub{x}{e_x}e'}{\sub{x}{e_x}\tau'} 	$$
	$$
	\iswellformed{\Gamma, \sub{x}{e_x}\Gamma'}{\sub{x}{e_x}\tau'_y}
	$$
	By rule \rtfun
	$$
	\hastype{\Gamma,\sub{x}{e_x} \Gamma'}{\sub{x}{e_x}e}{\sub{x}{e_x}\tau}
	$$
	
\item\rtapp Assume \hastype{\Gamma, x\colon\tau_x, \Gamma'}{e}{\tau},
where $e\equiv\eapp{e_1}{e_2}$ and $\tau\equiv\tau'\sub{y}{e_2}$.
By inversion
	$$
	\hastype{\Gamma, x\colon\tau_x, \Gamma'}{e_1}{\tfun{y}{\tau'_y}{\tau'}}\ (1)\qquad
	\hastype{\Gamma, x\colon\tau_x, \Gamma'}{e_2}{{\tau'_y}}\ (2)
	$$
By IH 
	$$
	\hastype{\Gamma,\sub{x}{e_x} \Gamma'}{\sub{x}{e_x}e_1}{\sub{x}{e_x}\tfun{y}{\tau'_y}{\tau'}} \qquad
	$$
	$$
	\hastype{\Gamma,\sub{x}{e_x} \Gamma'}{\sub{x}{e_x}e_2}{\sub{x}{e_x}{\tau'_y}}
	$$
	By rule \rtapp
	$$
	\hastype{\Gamma,\sub{x}{e_x} \Gamma'}{\sub{x}{e_x}e}{\sub{x}{e_x}\tau}
	$$

\item\rtlet Assume \hastype{\Gamma, x\colon\tau_x, \Gamma'}{e}{\tau},
where $e\equiv\elet{y}{e_y}{e'}$.
By inversion
	$$
	\hastype{\Gamma, x\colon\tau_x, \Gamma'}{e_y}{\tau_y}\ (1) \qquad
	\hastype{\Gamma, x\colon\tau_x, \Gamma', y\colon\tau_y}{e}'{\tau}\ (2)
	$$
	$$
	\iswellformed{\Gamma, x\colon\tau_x, \Gamma'}{\tau}\ (3)
	$$
By IH and \ref{proof:sub:wf}	
	$$
	\hastype{\Gamma, \sub{x}{e_x}\Gamma'}{e_y}{\tau_y}\ (4) \qquad
	\hastype{\Gamma, \sub{x}{e_x}\Gamma', y\colon\tau_y}{e}'{\tau}\ (5)
	$$
	$$
	\iswellformed{\Gamma, \sub{x}{e_x}\Gamma'}{\tau}\ (6)
	$$
So, 	$$
	\hastype{\Gamma,\sub{x}{e_x} \Gamma'}{\sub{x}{e_x}e}{\sub{x}{e_x}\tau}
	$$

\item\rtcase This case is similar to \rtlet.
\item\rtbot Assume \hastype{\Gamma, x\colon\tau_x, \Gamma'}{e}{\tau},
where $e\equiv\ebot$.
By inversion
$$	\iswellformed{\Gamma, x\colon\tau_x, \Gamma'}{\tau}$$
By \ref{proof:sub:wf}
$$	\iswellformed{\sub{x}{e_x}\Gamma'}{\sub{x}{e_x}\tau}$$
By rule \rtbot
	$$
	\hastype{\Gamma,\sub{x}{e_x} \Gamma'}{\sub{x}{e_x}e}{\sub{x}{e_x}\tau}
	$$
\end{itemize}




\item \label{proof:sub:wf}
Assume \iswellformed{\Gamma, x\colon\tau_x, \Gamma'}{\tau}.
We will prove it by induction on the derivation.
\begin{itemize}
\item \rwbase
Assume \iswellformed{\Gamma, x\colon\tau_x, \Gamma'}{\tau},
where $\tau\equiv\tref{v}{B}{l}{e}$.
By inversion
$$\hastypebase{\erase{\Gamma, x\colon\tau_x, \Gamma'},v\colon B}{e}{\tbool}$$
So,
$$\hastypebase{\erase{\Gamma, \sub{x}{e_x}\Gamma'},v\colon B}{e\sub{x}{e_x}}{\tbool}$$
By rule \rwbase
$$\iswellformed{\Gamma, \sub{x}{e_x}\Gamma'}{\tref{v}{B}{l}{e\sub{x}{e_x}}}$$
Or 
$$\iswellformed{\Gamma, \sub{x}{e_x}\Gamma'}{\tau\sub{x}{e_x}}$$
\item \rwfun
Assume \iswellformed{\Gamma, x\colon\tau_x, \Gamma'}{\tau},
where $\tau\equiv \tfun{y}{\tau'_y}{\tau'}$.
By inversion, we get
$$
	\iswellformed{\Gamma, x\colon\tau_x, \Gamma'}{\tau_x} \qquad
	\iswellformed{\Gamma, x\colon\tau_x, \Gamma', y \colon \tau'_y}{\tau'}
$$
By IH
$$
	\iswellformed{\Gamma, \sub{x}{e_x} \Gamma'}{\tau_x\sub{x}{e_x}}\qquad
	\iswellformed{\Gamma, \sub{x}{e_x}(\Gamma', y \colon \tau'_y)}{\tau'\sub{x}{e_x}}
$$
Due to $\alpha$-renaming, $x \neq y$, so
$$
	\iswellformed{\Gamma, \sub{x}{e_x} \Gamma'}{\tau'_y\sub{x}{e_x}}\qquad
	\iswellformed{\Gamma, \sub{x}{e_x}\Gamma', y \colon \sub{x}{e_x}\tau'_y}{\tau'\sub{x}{e_x}}
$$
By \rwfun
$$
	\iswellformed{\Gamma, \sub{x}{e_x} \Gamma'}{\tfun{y}{\tau'_y\sub{x}{e_x}}{\tau'\sub{x}{e_x}}}
$$
Or
$$
	\iswellformed{\Gamma, \sub{x}{e_x} \Gamma'}{\tau\sub{x}{e_x}}
$$
\end{itemize}
\end{enumerate}
\end{proof}


\subsection{Soundness}
Figure~\ref{fig:proofs:botomless} defines a \botomless{\bullet} predicate on expressions:

%\def\figone{%
\begin{figure*}[t!]
$$
\botomless{c} \qquad\botomless{x} \qquad \lnot \botomless{\ebot}
$$
$$
\botomless{D\ \overline{e_i}} \Leftrightarrow \bigwedge\botomless{e_i} \qquad
\botomless {\efun{x}{}{e}} \Leftrightarrow \botomless{e}
$$
$$
\botomless {e_1 \ e_2} \Leftrightarrow \botomless{e_1} \land \botomless{e_2} 
$$
$$
\botomless {\elet{x}{e_1}{e_2}} \Leftrightarrow \botomless{e_1} \land \botomless{e_2}
$$
$$
\botomless {\ecase{e}{D_i}{\overline{x}}{e_i}{x}} \Leftrightarrow \botomless{e} \land \bigwedge\botomless{e_i}
$$
\caption{\botomless{e}}
\label{fig:proofs:botomless}
\end{figure*}
%\global\let\figone\relax}
We prove Preservation and Progress only on expressions that do not contain \ebot:
%
\begin{lemma}[Preservation]\label{lemma:preservation}
If \hastype{\emptyset}{e}{\tau}, \botomless{e} and \eval{e}{e'} then \hastype{\emptyset}{e'}{\tau}.
\end{lemma}
\begin{proof}
Helping Lemmata:
\begin{lemma}\label{lemma:wftypes}
If \hastype{\Gamma}{e}{\tau} and \iswellformed{}{\Gamma} then \iswellformed{\Gamma}{\tau}.
\end{lemma}
\begin{proofsketch}
By case split on the derivation \iswellformed{\Gamma}{\tau}
\end{proofsketch}
\begin{lemma}\label{lemma:eval}
If \eval{e}{e'} then
	\issubtype{\emptyset}{\tau\sub{x}{e'}}{\tau\sub{x}{e}}
\end{lemma}
\begin{proofsketch}
By case split on the derivation \issubtype{\Gamma}{\tau\sub{x}{e'}}{\tau\sub{x}{e}}
\end{proofsketch}

Assume \botomless{e} and \hastype{\emptyset}{e}{\tau} and \eval{e}{e'}. 
We will prove the lemma by induction on the derivation tree. 
\begin{itemize}
\item Cases \rtvar, \rtconst, \rtfun and \rtbot trivially hold
       as there is no $e'$ for which \eval{e}{e'}. 

\item Case \rtsub. Assume \hastype{\emptyset}{e}{\tau}.
By inversion
$$	\hastype{\emptyset}{e}{\tau'} \ (1) \qquad
	\issubtype{\emptyset}{\tau'}{\tau}\ (2) \qquad
	\iswellformed{\emptyset}{\tau}\ (3)
$$

By IH on $(1)$
$$	\hastype{\emptyset}{e'}{\tau'} $$
By which, $(2), (3)$ and \rtsub
$$	\hastype{\emptyset}{e'}{\tau}$$

\item Case \rtlet. Assume \hastype{\emptyset}{e}{\tau}, 
where $e \equiv \elet{x}{e_x}{e_0}$. Then 
 $e' \equiv e_0\sub{x}{e_x}$.
By inversion
$$
	\hastype{\emptyset}{e_x}{\tau_{x}} \ (1) \qquad
	\hastype{x\colon\tau_x}{e_0}{\tau} \ (2) \qquad
	\iswellformed{\emptyset}{\tau} \ (3)
$$

By $(1)$, $(2)$ and Lemma \ref{lemma:substitution}, 
$$\hastype{\emptyset}{e'}{\tau\sub{x}{e_x}} \ (4)$$
By $(3)$ $x$ does not appear free in $\tau$, so, $\tau\sub{x}{e_x} \equiv \tau$ and
$$\hastype{\emptyset}{e'}{\tau}$$

\item Case \rtapp. Assume
$$	\hastype{\emptyset}{e}{\tau}\ (1)$$
where $e \equiv \eapp{e_1}{e_2}$, and
	  $\tau\equiv\tau'\sub{x}{e_2}$

By inversion
$$	
	\hastype{\emptyset}{e_1}{(\tfun{x}{\tau_{x}}{\tau'})}\ (2) \qquad
	\hastype{\emptyset}{e_2}{\tau_{x}}\ (3)
$$

We split cases on the structure of $e$.
\begin{itemize}
\item $e\equiv \eapp{c}{v_2}$.
Then, $e'\equiv\interp{c}(v_2)$.
By Definition \ref{def:constants},
$$\hastype{\emptyset}{e'}{\tau}$$

\item $e\equiv \eapp{c}{e_2}$ where $e_2$ is botomless and not a value, 
Then, by (3) and Lemma~\ref{lemma:progress},
\eval{e_2}{e_2'}, and $e' \equiv \eapp{e_1}{e_2'}$.
%
By IH on $(3)$
$$	\hastype{\emptyset}{e_2'}{\tau_{x}}$$
By which, $(2)$ and rule \rtapp we get
$$\hastype{\emptyset}{e'}{\tau'\sub{x}{e_2'}}\ (4)$$
By Lemma \ref{lemma:eval}
$$
	\issubtype{\emptyset}{\tau'\sub{x}{e_2'}}{\tau'\sub{x}{e_2}}\ (5)
$$
By $(1)$ and Lemma \ref{lemma:wftypes}, since \iswellformed{}{\emptyset}
$$
	\iswellformed{\emptyset}{\tau'\sub{x}{e_2}}\ (6)
$$
By $(4), (5), (6)$ and rule \rtsub
$$	\hastype{\emptyset}{e'}{\tau}$$

\item $e \equiv \eapp{\efun{x}{e_x}}{e_2}$.
Then, $e' \equiv e_x\sub{x}{e_2}$.

By inversion on $(2)$
$$
	\hastype{x\colon\tau_x}{e_x}{\tau'}
$$
By which, $(3)$ and Lemma \ref{lemma:substitution} (since \iswellformed{}{x\colon\tau_x})
$$\hastype{\emptyset}{e'}{\tau'}$$

\item $e \equiv \eapp{e_1}{e_2}$, where $e_1$ is botomless and not a value.
Then, by $(2)$ and Lemma \ref{lemma:progress}, \eval{e_1}{e_1'} and 
$e'\equiv\eapp{e_1'}{e_2}$
By IH on $(2)$
$$	\hastype{\emptyset}{e_1'}{(\tfun{x}{\tau_{x}}{\tau'})}
$$
By which, $(3)$ and rule \rtapp we get
$$	\hastype{\emptyset}{e'}{\tau}$$
\end{itemize}
\item Case \rtcase, assume $\hastype{\emptyset}{e}{\tau}$, 
where $e \equiv \ecase{e_0}{D_i}{\overline{y}}{e}{x}$.
By inversion
$$	\hastype{\emptyset}{e_T}{\tref{v}{T}{l}{e_T}}\ (1) $$
$$	 \iswellformed{\emptyset}{\tau}\ (2)$$
$$	\forall i, 0 < i \leq \arity{T}. (
		\constty{D^i_T} = \tfun{x_1}{\tau_1}{\dots\tfun{x_n}{\tau_n}{\tref{v}{T}{l}{e_{T_i}}}}\ (3)
$$
$$		\theta = \overline{\sub{x_i}{y_i}}\ (4) \qquad
		\hastype{x\colon\tlref{v}{T}{}{e_t \land e_{T_i}}, 
						\overline{y_i\colon \theta\ \tau_i}}{e_i}{\tau}\ (5)	
	)
$$
We split cases on the structure of $e_T$.
\begin{itemize}
\item Assume that $e_T \equiv D^i_T\ \overline{e}$,
then $e' \equiv e_i \sub{x}{e} \overline{\sub{y_i}{e_i}}$.

By $(5)$
$$
		\hastype{\overline{y_i\colon \theta\ \tau_i}, 
		x\colon\tlref{v}{T}{l}{e_t \land \theta\ e_{T_i}}}{e_i}{\tau}\	
$$

By inversion on $(1)$ 
\hastype{\emptyset}{e_j}{\tau_j\overline{\sub{x}{e}}} 
and
\hastype{\emptyset}{D^i_T\ \overline{e}}{\tref{v}{T}{}{e_{T_i}}\overline{\sub{x}{e}}}. 
So,
\hastype{\emptyset}{e_j}{\tau_j\overline{\sub{x}{y}\sub{y}{e}}} 
and
\hastype{\emptyset}{D^i_T\ \overline{e}}{\tref{v}{T}{}{e_{T_i}}\overline{\sub{x}{y}\sub{y}{e}}}. 
And,
\shastype{\emptyset}{e_j}{\tau_j\overline{\sub{x}{y}\sub{y}{e}}} 
and
\shastype{\emptyset}{D^i_T\ \overline{e}}{\tref{v}{T}{}{e_{T_i}}\overline{\sub{x}{y}\sub{y}{e}}}. 

Finally, by Definition~\ref{def:constants}
\shastype{\emptyset}{D^i_T\ \overline{e}}{\tref{v}{T}{}{e_{T_i} \land e_t}\overline{\sub{x}{y}\sub{y}{e}}}. 

Then, by Lemma \ref{lemma:substitution}

\hastype{\emptyset}{e'}{\tau \overline{\sub{y_i}{e_i}}\sub{x}{e}}.

Finally, by $(2)$, $\tau \overline{\sub{y_i}{e_i}}\sub{x}{e} \equiv \tau$, so
$$
\hastype{\emptyset}{e'}{\tau}.
$$

\item Otherwise, by $(1)$ and Lemma \ref{lemma:progress} \eval{e_0}{e'_0}.
So $e' \equiv \ecase{e'_0}{D_i}{\overline{y}}{e}{x}$.
By IH \hastype{\emptyset}{e'_0}{\tref{v}{T}{}{e_T}}, 
by which and $(1) - (6)$ $$\hastype{\emptyset}{e'}{\tau}$$
\end{itemize}

\end{itemize}
\end{proof}
\begin{lemma}[Progress]\label{lemma:progress}
If \hastype{\emptyset}{e}{\tau}, \botomless{e} and $e \neq v$ 
then there exists an $e'$ such that \botomless{e'} and \eval{e}{e'}.
\end{lemma}
\begin{proof}
Assume \hastype{\emptyset}{e}{\tau}.
We will prove the Lemma by induction on the derivation tree.
\begin{itemize}
\item Case \rtvar cannot occur, as $\Gamma = \emptyset$
\item Case \rtbot is trivial, 
		as $\lnot \botomless{e}$.
\item Cases \rtconst and \rtfun are trivial, 
		as $e = v$.
\item Case \rtsub. Assume \hastype{\emptyset}{e}{\tau}.
By inversion
$$	\hastype{\emptyset}{e}{\tau'}$$
By IH 
either $e \equiv v$ or there exists an botomless $e'$ such that \eval{e}{e'}.
\item Case \rtapp. Assume $$\hastype{\emptyset}{e}{\tau}\ (1)$$
where $e\equiv\eapp{e_1}{e_2}$ and $\tau\equiv\tau'\sub{x}{e_2}$.
By inversion
$$
	\hastype{\emptyset}{e_1}{(\tfun{x}{\tau_{x}}{\tau})}\ (2)\qquad
	\hastype{\emptyset}{e_2}{\tau_{x}}\ (3)
$$

We split cases on the structure of $e$.
\begin{itemize}
\item $e\equiv \eapp{c}{v_2}$.
Then, $e'\equiv\interp{c}(v_2)$ which is botomless by Definition of constants.

\item $e\equiv \eapp{c}{e_2}$ where $e_2$ is not a value, 
By IH on $(3)$ \eval{e_2}{e_2'} and  $e' \equiv \eapp{e_1}{e_2'}$

\item $e \equiv \eapp{\efun{x}{e_x}}{e_2}$.
Then, $e' \equiv e_x\sub{x}{e_2}$, which does not contain bottom.

\item $e \equiv \eapp{e_1}{e_2}$, where $e_1 \neq v$.
Then, by IH on $(2)$ \eval{e_1}{e_1'} and 
$e'\equiv\eapp{e_1'}{e_2}$.
\end{itemize}

\item Case \rtlet. Assume \hastype{\emptyset}{e}{\tau}, where 
$e \equiv \elet{x}{e_x}{e_0}$, then $e'\equiv e_0\sub{x}{e_x}$ which is botomless.

\item Case \rtcase. Assume \hastype{\emptyset}{e}{\tau}, where
$e \equiv \ecase{e_T}{D_{T_i}}{\overline{y}}{e_i}{x}$.
By inversion, 
$$
	\hastype{\Gamma}{e}{\tref{v}{T}{e_T}}\ (1)
$$
We split cases on the structure of $e_T$
\begin{itemize}
\item If $e_T$ is a value, then by $(1)$ it is of the form $e_T \equiv D_{T_i} \overline{e}$,
so $e' \equiv e_i \sub{x}{e_T}\overline{\sub{y}{e}} $
\item Otherwise, by IH there exists $e'_T$ such that \evals{e_T}{e'_T}, 
so $e' \equiv \ecase{e'_T}{D_{T_i}}{\overline{y}}{e_i}{x}$.
\end{itemize}
\end{itemize}
\end{proof}

We combine the above to prove 
\textit{Soundness of \undeclang}, \ie Theorem~\ref{thm:safety} in the paper:
%
\begin{theorem}{[Soundness of \undeclang]}\label{thm:proofs:safety}
If \hastype{\emptyset}{e}{\tau} and \botomless{e}, then 
\begin{itemize}
\item\textbf{Type-Preservation:} If 
       $\evals{e}{v}$ then $\hastypet{\emptyset}{v}{\typ}$.
\item\textbf{Crash-Freedom:} $\evals{e\not}{\ecrash}$.
\end{itemize}
\end{theorem}
\begin{proof}
1. 
Since \botomless{e} there exists by Lemma \ref{lemma:progress} a bottomless evaluation sequence 
$$
e \equiv e_0 \eval{}{} e_1 \eval{}{} \dots \eval{}{} \dots e_n \equiv v
$$
The Theorem is proven by $n$ applications of Preservation Lemma.

2. If $\evals{e}{\CRASH}$, then by Preservation \hastype{\emptyset}{\CRASH}{\tau}
which cannot happen, as \CRASH by definition is an untyped constant.
\end{proof}

\section{Tracking Substitutions}

Then we define the notion of tracking substitutions.
In Figure~\ref{fig:proofs:tracking} we extend the operational
semantics with a state $\sigma$, \ie a mapping from variables to 
expressions that tracks evaluation of its expressions

%\newcommand\teval[4]{\evalt{#1}{#2}{#3}{#4}}
\renewcommand\evalt[4]{{\teval{#1}{#3}{#2}{#4}}}

\begin{figure*}
\hfill\mbox{\evalt{\sigma}{\sigma}{e}{e}}
$$
\begin{array}{ll}
	%% CONTEXT
	\evalt{\sigma}{\sigma'}{\eapp{e_1}{e_2}}{\eapp{e_1'}{e_2}} 
	& \text{if}\ 
	\evalt{\sigma}{\sigma'}{e_1}{e_1'}\\
	%
	\evalt{\sigma}{\sigma'}{\eapp{c}{e}}{\eapp{c}{e'}} 
	& \text{if}\ 
	\evalt{\sigma}{\sigma'}{e}{e'}\\
	%
	\evalt{\sigma}{\sigma'}{\ecase{e}{D_i}{\overline{y_i}}{e_i}{x}}{\ecase{e'}{D_i}{\overline{y_i}}{e_i}{x}}
	&\text{if}\ \evalt{\sigma}{\sigma'}{e}{e'} \\
	%
	\evalt{\sigma}{\sigma'}
		{D\ \overline{e_i}\ e\ \overline{e_j}}
		{D\ \overline{e_i}\ e'\ \overline{e_j}}
	&\text{if}\ \evalt{\sigma}{\sigma'}{e}{e'} \\
	%
	%% EVALUATION
	\evalt{\sigma}{\sigma}{\eapp{c}{v}}{\ceval{c}{v}} &\\
	\evalt{\sigma}{\sigma}{\eapp{\efun{x}{}{e}}{e_x}}{e\sub{x}{e_x}} &\\
	\evalt{\sigma}{\sigma}{\elet{x}{e_x}{e}}{e\sub{x}{e_x}}& \\
	\evalt{\sigma}{\sigma}{\ecase{D_j\ \overline{e}}{D_i}{\overline{y_i}}{e_i}{x}}{e_j\sub{x}{D_j\ 			\overline{e}}\sub{\overline{y_j}}{\overline{e}}} \\
	%% TRACKING
\teval{(x,e_x)\sigma}{x}{(x,e'_x)\sigma'}{x} &\text{if}\ \teval{\sigma}{e_x}{\sigma'}{e'_x}\\
	\teval{(y,e_y)\sigma}{x}{(y,e_y)\sigma'}{e_x}
  &\text{if}\
	\teval{\sigma}{x}{\sigma'}{e_x}\\
	\teval{(x,v)\sigma}{x}{(x,v)\sigma}{v}
  &\text{if}\
  v \not = D\ \overline{e}\\
	\teval{(x,D\ \overline{y})\sigma}{x}{(x,D\ \overline{y})\sigma}{D\ \overline{y}}
  &\\
	\teval{(x,D\ \overline{e})\sigma}{x}{(x,D\ \overline{y})\overline{(y_i, e_i)}\sigma}{D\ \overline{y}}
  &\text{if}\
	\text{fresh}\ \overline{y_i}
  \\
\end{array}
$$
\caption{Tracking Substitutions}
\label{fig:proofs:tracking}
\end{figure*}

First we prove that evaluation to a constant exists \textit{iff} 
tracking evaluation to the same constant exists.
\begin{lemma}\label{lemma:teval}
$\forall\theta, e, c, \exists\theta'. \evals{\thetasub{\theta}{e}}{c} \Leftrightarrow 
	\tevals{\theta}{e}{\theta'}{c}$.
\end{lemma}
\showproofsketch{
\begin{proofsketch}
\begin{itemize} We prove each direction:
\item $\Rightarrow$.
Given the derivation $\evals{\thetasub{\theta}{e}}{v}$, we can track the appearances 
of each expressions $\theta(x_i)$ and its derivatives and replace them with $x_i$.
Thus, given the initial derivation we can transverse it
(left-to-right and post-order); 
for every tracked appearance we use the appropriate rules 
that update the stack every time a tracked expressions evaluates, ie., 
appears in the left hand side of a rule; 
and remove the multiple evaluations of expressions in the stack
and construct the evaluation 
\tevals{\theta}{e}{\theta'}{c}.

Note that if $\theta(x_i)$ goes to a value, then 
$\theta(x_i) \equiv e_0 \hookrightarrow \dots e_i \dots e_n \equiv v$.
By the way we transverse the tree, 
after the stack is updated to $e_k$ and before it is updated to $e_{k+1}$
all tracked computations for $x_i$ are $e_j, j \leq k$.

If $\theta(x_i)$ does not go to a value, it cannot appear in the left hand side of 
a rule, because evaluation would diverge, thus the stack is not updated for $x_i$.

When a tracked expression reaches a value, we use the appropriate value to 
substitute (and untrack) the value.
Since the result of the initial evaluation is a constant, 
then the result of the tracked computation is the same constant. 

\item $\Leftarrow$.
Given $\tevals{\theta}{e}{\theta'}{c}$
we can construct the derivation \evals{\thetasub{\theta}{e}}{c} replacing each query to the 
stack with the initial computation of the expression.
\end{itemize}
\end{proofsketch}
}

Then we define a \textit{bottomize} function \mkbot{\bullet}
that replaces non-evaluated expressions with \ebot:
\begin{definition}{[Bottomize]}
$$
\mkbot{\theta}(x) = 
\left\{
	\begin{array}{ll}
		D\ \overline{\mkbot{\theta}(y)}  & \mbox{if } \theta(x) = D\ \overline{y}\\
		v  & \mbox{if } \theta(x) = v \not = D\ \overline{y}\\
		\ebot & \mbox{otherwise}
	\end{array}
\right.
$$
\end{definition}

Using the bottomize function we show that evaluation does not depend 
on non-evaluated expressions:
\begin{lemma}\label{lemma:mkbot}
If \tevals{\theta}{e}{\theta'}{c}, 
then \evals{\mkbot{\theta'}\ e}{c}.
\end{lemma}
\showproofsketch{
\begin{proofsketch}
Since \tevals{\theta}{e}{\theta'}{c}$(1)$, 
then \tevals{\theta'}{e}{\theta'}{c}$(2)$:
From the evaluation tree $(1)$ we can construct the evaluation tree $(2)$.
The trees differ on store related rules.

Say that in $(1)$ the store in $x$ is updated, for an arbitrary $x$: 
\teval{(x,e_x)\theta_x}{x}{(x,e'_x)\theta_x}{x}
Since $(1)$ is finite, it should be that
$\tevals{\theta_x}{e_x}{\theta_x}{v} (3)$.
Call $v_x = D\ \overline{y}$ if $v = D\ \overline{e}$, $v$ otherwise.
Then in $(1)$ there should be a ``subtree'' with $(3)$
after which the value of $x$ cannot change in the store.
Or $\theta'(x) = v_x$.
We construct $(2)$ by removing the ``subtree'' with $(3)$.
After that all rules that relate store with $x$ will be the same
on $(1)$ and $(2)$.

If $x$ is not updated in $(1)$ then 
$x$ does not appear in the left hand side of a rule; 
thus $\theta'(x) = \theta(x)$.

We construct $\theta''(x)= \left\{
	\begin{array}{ll}
		v  & \mbox{if}\ \theta'(x)= v\\
		\ebot & \mbox{otherwise}
	\end{array}
\right.$

Then \tevals{\theta''}{e}{\theta''}{c}.
If $\theta'(x)$ is not a value, then it does not appear in the left hand side 
of any rule in $(2)$, thus evaluation of $e$ cannot depend on $x$.

Then by Lemma \ref{lemma:teval}, \evals{\thetasub{\theta''}{e}}{c}.
But $\mkbot{\theta'} e = \thetasub{\theta''}{e}$, so \evals{\thetasub{\mkbot{\theta'}}{e}}{c}.
\end{proofsketch}
}

Also, replacing \ebot with any expression yields the same evaluation:
\begin{lemma}\label{lemma:rmbot}
If \evals{\thetasub{\mkbot{\theta}}{e}}{c}, 
then \evals{\thetasub{\theta}{e}}{c}.
\end{lemma}
\showproofsketch{
\begin{proof}
Since \evals{\thetasub{\mkbot{\theta}}{e}}{c}$(1)$, then
\tevals{\mkbot{\theta}}{e}{\theta'}{c}$(2)$.
\ebot expressions in \mkbot{\theta} are not evaluated, 
otherwise $(2)$ would get stuck.
Thus they can be instantiated with any expression.
$\theta$ provides such an instantiation, thus
\tevals{\theta}{e}{\theta''}{c}$(3)$.
By Lemma \ref{lemma:teval}, \evals{\thetasub{\theta}{e}}{c}.
\end{proof}
}

Finally, we define lifting substitutions
%
\begin{definition}{[Lifting Substitutions]}
$
\trackevals{\sto}{\botsto} \doteq 
\exists e, e', \theta' \tevals{\theta}{e}{\theta'}{e'} \land \botsto = \mkbot{\theta'}
$
\end{definition}

and prove the Lifting Lemma
\begin{lemma}{[Lifting]}\label{lemma:proofs:lifting}
$\evals{\thetasub{\sto}{e}}{c}$ iff $\exists \trackevals{\sto}{\botsto}$ s.t. 
$\evals{\thetasub{\botsto}{e}}{c}$.
\end{lemma}
\showproofsketch{
\begin{proofsketch}
The $\Rightarrow$ direction follows immediately from Lemmata~\ref{lemma:teval} and~\ref{lemma:mkbot}.
The $\Leftarrow$ direction follows immediately from Lemmata~\ref{lemma:teval} and~\ref{lemma:rmbot}.
\end{proofsketch}
}



\section{Constants}

We can prove that all the above constants belong to the 
interpretations of their types.
%
\begin{theorem}{[Constants]}\label{thm:constant}
$c \in \interp{\constty{c}}$.
\end{theorem} 
%
The Theorem trivially holds for more of the constants.
For example, 
\newcommand\eqtype{\ensuremath{
	\tfun{x}{b^\lfinite}{
	\tfun{y}{b^\lfinite}{
	\tlref{v}{\tbool}{\lfinite}{v \Leftrightarrow x = y}
	}}
}}
$$= \in \interp{\eqtype}$$
as $\forall e_1, e_2, \evals{e_1}{d_1}, \evals{e_2}{d_2}\Rightarrow 
			\evals{(e_1=e_2 \Leftrightarrow e_1= e_2)}{\etrue}
			\land \exists d. \evals{(e_1 = e_2)}{d}$

Here we prove that for any type $\tau$, 
\efix{\tau} and \etfix{\tau} satisfy Theorem~\ref{thm:constant}.

Given the families of constants:
\begin{align*}
\ceval{\etfix{\tau}}{f} & \doteq \efun{n}{}{\efun{f}{}{\etfixn{\tau}{}{n}}}\\ 
%\ceval{\etfixf{\tau}{f}{??}}{n} &\doteq
%f\ n\ (\etfixn{\tau}{f}{n}\ f) \\
\ceval{\etfixn{\tau}{}{n}}{m} &\doteq
\efun{f}{}{
f\ m\ (\etfixn{\tau}{f}{m}\ f)} \\
\end{align*}
%
and their types
%
\begin{align*}
%\decr{\tau}{n} & \doteq \decrtypefull{\tau}{n}\\
\constty{\etfix{\tau}} &\doteq 
	(\decrty{\tau})
	 \rightarrow
	\tfun{m}{\tnat^\lfinite}{\tau\sub{x}{m}}\\ 
%\constty{\etfixf{\tau}{f}{n}} &\doteq 
%	 \etfixfty{\tau}{f}{n}\\ 
\constty{\etfixn{\tau}{f}{n}} &\doteq  
	(\decrty{\tau})
	\rightarrow\adecrty{\tau}{n}\\ 
%\constty{\etfixfn{\tau}{f}{n}} &\doteq 
%	 \etfixfty{\tau}{f}{n}\\ 
\end{align*}
we prove that the constants belong to the 
meanings of their types:
%
\begin{theorem}{[Terminating Fixpoint]}\label{thm:fixpoint}
\begin{enumerate}
\item\label{nfix}$\forall n. \etfixn{\tau}{f}{n} \in \constty{\etfixn{\tau}{f}{n}}$
\item\label{tfix}$\etfix{\tau} \in \constty{\etfix{\tau}}$
\item\label{fix}$\efix{\tau} \in \constty{\efix{\tau}}$, if the result of $\tau$ is a \Div type.
\end{enumerate}
\end{theorem}
\begin{proof}
\begin{itemize}
\item \ref{nfix}.
We prove that for all
%$f \in \interp{\decrty{\tau}}$
appropriate $f$
and $ m \in \interp{\tref{v}{\tnat}{\lfinite}{v < n}}$,
$e \equiv \etfixn{\tau}{f}{n}\ f \ m \in \interp{\tau\sub{x}{m}}$
% 
by induction on $n$.

For $n=0$, 
it is trivial, as 
there is no $m$ such that
$m \in \interp{\tref{v}{\tnat}{\lfinite}{v < 0}}$.

For the inductive step, $e$ reduces to 
$$
\etfixn{\tau}{f}{n}\ f\ m 
\hookrightarrow
\etfixfn{\tau}{f}{n}\ m 
\hookrightarrow
f\ m\ (\etfixn{\tau}{f}{m}\ f)\\
$$
By IH, since $m < n$,
$\etfixn{\tau}{f}{m} \in \constty{\etfixn{\tau}{f}{m}}$, 
so $f$ receives the appropriate arguments, 
and returns the appropriate result that proves the theorem.
%
\item \ref{tfix}.
We prove that 
for all appropriate $f$
% \\$f \in \interp{\decrty{\tau}}$
and     $ m \in \interp{\tnat^\lfinite}$,
$\etfix{\tau}\ f \ m \in \interp{\tau\sub{x}{m}}$.
%
Since $m \in \interp{\tref{v}{\tnat}{\lfinite}{v < m+1}}$
$$\etfixn{\tau}{f}{m+1}\ f \ m \in \interp{\tau\sub{x}{m}}$$
%
But operationally, 
$\etfixn{\tau}{f}{m+1}\ f \ m$
and
$\etfix{\tau}\ f \ m$
behave equivalently, which proves the theorem.
\item \ref{fix}. The prove for 
$\efix{\tau} \in \constty{\efix{\tau}}$.
is similar.
%
The only difference is that for the base case
\efixn{\tau}{0} should be defined to belong 
into the interpretation of any type.
%
Thus, it is defined as a diverging expression
and the type of \efix{\tau} is constrainted
to $\tau$'s with potentially diverging result. 
%
With refinement types we prove that the basic
\etfixn{\tau}{f}{0} operator
cannot be called, so we omit 
the definition of this basic case.
\end{itemize}
\end{proof}
%

\section{Algorithmic Typing: \declang}

Soundness of \declang trivially reduces to soundness of implication checking.
Here we give the detailed proof of the Approximation Theorem:
\begin{theorem}{[Approximation]}\label{thm:approximation} 
  If \decissubref{\Env}{p_1}{p_2} then \isimplied{\Env}{p_1}{p_2}.
\end{theorem}
\begin{proof}
To prove the above, let ${\VC \defeq \VCOND{\Env}{p_1}{p_2}}$. First, note that
if $\VC$ is u-valid then it is valid as the addition of axioms preserves
validity. Next, we prove that if the \VC is valid then \isimplied{\Env}{p_1}{p_2}.
%
We fix a $\theta$ for which 
$ \theta \in \interp{\Gamma}$ and 
$\evals{\thetasub{\theta}{q_1}}{\etrue}$
%
It suffices to prove that 
$\evals{\thetasub{\theta}{q_2}}{\etrue}$.

For all $(x_i, \tlref{x_i}{B}{\trivial}{p_i}) \in \Gamma$
there exists $(x_i, e_i) \in \theta$ and
\begin{align*}
	e_i \in \interp{\tref{x_i}{B}{\trivial}{p_i}}
&\Leftrightarrow
	e_i \in \interp{\tref{v}{B}{}{p_i}}
	\land
	\exists v. \evals{e_i}{v_i}  \\
&\Leftrightarrow
	\exists v. \evals{e_i}{v} \Rightarrow 
	\evals{\thetasub{\theta}{p_i\sub{x_i}{v_i}}}{\etrue}
	\land
	\exists v. \evals{e_i}{v_i}  \\
&\Leftrightarrow
	\evals{\thetasub{\theta}{p_i}}{\etrue}\\
\end{align*}

Thus we have that 
$\evals{\theta\ (\bigwedge p_i \land q_1)}{\etrue}$.
%
By Lemma~\ref{lemma:teval}
$\tevals{\theta}{(\bigwedge p_i \land q_1)}{\theta'}{\etrue}$.
%
Let $\rho = \mkbot{\theta'}$;
thus, $\trackevals{\theta}{\rho}$.
By Lemma~\ref{lemma:proofs:lifting}
$\evals{\thetasub{\rho}{\bigwedge p_i \land q_1}}{\etrue}$.
%
Moreover, by the construction of $\rho$, 
\hastype{\emptyset}{\thetasub{\rho}{\bigwedge p_i \land q_1}}{\tbool}.
Thus, by Equivalence Theorem~\ref{thm:equiv}
$\forall \sigma \in \embed{\rho}. 
\lmodels{\sigma}{\bigwedge p_i \land q_1}$.
%
By which and validity of $VC$
$\forall \sigma \in \embed{\rho}. 
\lmodels{\sigma}{q_2}$.
%
Using the other direction of Equivalence Theorem~\ref{thm:equiv}
$\evals{\rho\ q_2}{\etrue}$.
%
Finally, using the other direction of Lemma~\ref{lemma:proofs:lifting}
$\eval{\thetasub{\theta}{q_2}}{\etrue}$.
\end{proof}

\renewcommand\botsto{\ensuremath{\theta^\ebot}}
To conclude the proof we prove Equivalence Theorem.
Let \botsto be a substitution from variables to \textit{lifted values}.
We define the embedding of the substitution \embed{\botsto}
that maps \ebot to arbitrary elements of the logical domain:


\begin{definition}
$$
\instance{\rho}=
	\{(x_1, \botv_1), \dots, (x_n, \botv_n) \mid \botv_i \in \instance{\rho(x_i)}  \}
$$
\begin{align*}
\instance{\ebot}& = \dom &
\instance{D\ \overline{v}} &= 
	\{ D\ \overline{v} \mid v_i \in \instance{v} \} \\ 
\instance{n} &= \{n\} &
\instance{v} &= \{c_v\}, \text{otherwise}
\end{align*}
\end{definition}

Then we prove that
given a lifted substitution a predicate goes to \etrue
if and only if for any embedding the
predicate holds.

\begin{theorem}{[Equivalence]}\label{thm:equiv}
If \hastype{\emptyset}{\botsto(p)}{\tbool}, then
\begin{itemize}
\item $\evals{\botsto(p)}{\etrue}\ \mbox{iff}\ 
	\forall \sigma \in \embed{\botsto}. \lmodels{\sigma}{p}$.
\item $\evals{\botsto(p)}{\efalse}\ 
	\mbox{iff}\ 
	\forall \sigma \in \embed{\botsto}. \lmodels{\sigma\not}{p}$.
\end{itemize}
\end{theorem}
\begin{proof}
\newcommand\sigmamodel{\ensuremath{\sigma}}
\newcommand\interpI{\ensuremath{\mathcal{I}}}
\newcommand\interpIEq[3]{\ensuremath{\interpI_{\sigmamodel}(#2) = #3}}
\newcommand\interpIGEq[3]{\ensuremath{\interpI_{\sigmamodel}(#2) \sqsupseteq #3}}
\newcommand\interpINGEq[3]{\ensuremath{\interpI_{\sigmamodel}(#2) \not\sqsupseteq #3}}
\newcommand\interpINEq[3]{\ensuremath{\interpI_{\sigmamodel}(#2) \not = #3}}
\newcommand\interpIENGEq[4]{\ensuremath{\interpI_{\sigmamodel}(#2) = {#3}\not \sqsupseteq #4}}

To begin with we define a comparison between lifted values and 
elements of the logical domain:
\begin{definition}
\begin{align*}
\ebot &\sqsubset d & d &\sqsubseteq d & v &\sqsubseteq c_v
\end{align*}
\end{definition}

and a function $\interpI_\sigmamodel (t)$
that given a model $\sigmamodel$ and an (open) logical term $t$ returns an element in the logic:

\begin{definition}{[Interpretation]}
$$\interpI_{\sigmamodel} :: t \rightarrow d $$
%
\begin{align*}
\interpIEq{\sigmamodel}{n&}{n} & 
\interpIEq{\sigmamodel}{f\ \overline{t}&}{f_D\ (\overline{\interpI_{\sigmamodel}{t}})}\\
& & \interpIEq{\sigmamodel}{D\ \overline{t}&}{D\ \overline{\interpI_{\sigmamodel}{t}}}\\
\interpIEq{\sigmamodel}{x&}{\sigmamodel(x)}& \interpIEq{\sigmamodel}{t_1 \oplus t_2 &}{\interpI_{\sigmamodel}{t_2}\oplus_D \interpI_{\sigmamodel}{t_2}}\\
\end{align*}
\end{definition}

We relate the evaluation of logical terms with their interpretation
into the logic:
%
\begin{lemma}
If \hastype{\Gamma}{\thetasub{\botsto}{t}}{\tau}, then
$	\evals{\thetasub{\botsto}{t}}{\botv} 
	\Leftrightarrow 
	\forall\sigma\in \instance{\rho}.\interpIGEq{\sigma}{t}{\botv}$  
\end{lemma}
\begin{proof}
By induction on the structure of $t$.
\begin{itemize}
\item $t \equiv n$: \evals{\rho\ n}{n} and 
	  $\forall\sigmamodel \in \instance{\rho}\interpIEq{\sigmamodel}{n}{n}$
\item $t \equiv x$:
		\evals{\thetasub{\rho}{x}}{\rho(x)} and 
		$\forall\sigmamodel \in \instance{\rho}. 
		\interpIEq{\sigmamodel}{x}{\sigmamodel(x)} \sqsupseteq \rho(x)$
\item $t \equiv f\ \overline{t}$:

	$$
		\evals{\rho\ (f\ \overline{t})}{\botv} \Leftrightarrow
		\evals{f\ \overline{\rho\ t}}{\botv} \Leftrightarrow	
	$$
	$$	
		\exists \botv_i. \evals{\thetasub{\rho}{t_i}}{\botv_i} \text{ and } \evals{f ({\overline{\botv}})}{\botv} \Leftrightarrow	
	$$
	$$	
		\exists\botv_i\forall\sigmamodel \in \instance{\rho}.\interpIGEq{\instance{\rho}}{t_i}{\botv_i} \text{ and } 		
		\forall d_i\sqsupseteq\botv_i. f_D(\overline{d}) \sqsupseteq \botv \xLeftrightarrow{(*)}	
	$$
	$$	
		\forall\sigmamodel \in \instance{\rho}.\exists d_i\interpIEq{\sigmamodel}{t_i}{d_i} \text{ and } 		
		f_D(\overline{d}) \sqsupseteq \botv \Leftrightarrow	
	$$
	$$	
		\forall\sigmamodel \in \instance{\rho}.
		f_D(\overline{\interpI_{\sigmamodel}(t_i)}) \sqsupseteq \botv \Leftrightarrow	
	$$
	$$	
		\forall\sigmamodel \in \instance{\rho}.\interpIGEq{\sigmamodel}{f\ \overline{t}}{\botv} 
	$$

$(*)$ We can show that for each $f_D$ and $\botv$ 
	$$	
		\exists\botv_i\forall d_i. d_i\sqsupseteq\botv_i \Leftrightarrow f_D(\overline{d}) \sqsupseteq \botv 
	$$
ie, $\botv_i$ contains the least information required by $f_D$
to produce a result less than \botv.
Now, say $$ \exists\sigmamodel \in \instance{\rho}\forall d_i.\interpIENGEq{\sigmamodel}{t_i}{d_i}{\botv_i} $$
Then, by definition of $\botv_i$,
$f_D(\overline{d}) \not\sqsupseteq \botv $, which is a contradiction.
\item $t \equiv D\ \overline{t}$: 
	$$
		\evals{\rho\ (D\ \overline{t})}{D\ \overline{\botv}} \Leftrightarrow
		\evals{\rho\ t_i}{\botv_i} \Leftrightarrow
	$$
	$$
		\forall\sigmamodel \in \instance{\rho}.\interpIGEq{\sigmamodel}{t_i}{\botv_i} \Leftrightarrow
		\forall\sigmamodel \in \instance{\rho}.\interpIGEq{\sigmamodel}{D\ \overline{t}}{D\ \overline{\botv}} 
	$$
\item $t \equiv t_1 \oplus t_2 $
	$$
		\evals{\rho\ (t_1 \oplus t_2)}{d} \Leftrightarrow
		\evals{(\rho\ t_1) \oplus (\rho\ t_2)}{d} \Leftrightarrow	
	$$
	$$
		\exists d_1.\evals{\rho\ t_1}{d_1} \text{ and }  \evals{\oplus_{d_1} (\rho\ t_2)}{d} \Leftrightarrow
	$$
	$$	
		\exists d_1, d_2.\evals{\rho\ t_1}{d_1} \text{ and }  \evals{\rho\ t_2}{d_2} \text{ and } d_1 \oplus_D d_2 = d \Leftrightarrow	
	$$
	$$	
		\exists d_1, d_2.\forall\sigmamodel \in \instance{\rho}.\interpIEq{\sigmamodel}{t_1}{d_1} \text{ and }  
		\forall\sigmamodel \in \instance{\rho}.\interpIEq{\sigmamodel}{t_2}{d_2} \text{ and } d_1 \oplus_D d_2 = d \xLeftrightarrow{(*)}	
	$$
	$$	
		\forall\sigmamodel \in \instance{\rho}.
		\exists d_1, d_2.\interpIEq{\sigmamodel}{t_1}{d_1} \text{ and }  
		\interpIEq{\sigmamodel}{t_2}{d_2} \text{ and } d_1 \oplus_D d_2 = d \Leftrightarrow	
	$$
	$$	
		\forall\sigmamodel \in \instance{\rho}.\interpIEq{\sigmamodel}{t_1 \oplus t_2}{d} 
	$$
$(*)$ For $i = 1 , 2$, fix two instantiations
$\sigmamodel_1, \sigmamodel_2 \in \instance{\rho}$. 
Assume that $d_{i_{\sigmamodel_1}} \not = d_{i_{\sigmamodel_2}}$.
Then $\lnot \forall \sigmamodel \in \instance{\rho} \interpIEq{\sigmamodel}{t_i}{d} \Rightarrow \evals{\rho\ t_i \not}{d} \Rightarrow \lnot \hastype{\Gamma}{t_i}{b^\finite} \Rightarrow \lnot \hastype{\Gamma}{p}{\tbool}$.
\end{itemize}
\end{proof}

We use the above Lemma to prove the Theorem by induction on the structure of $p$.
\begin{itemize}
\item $ p \equiv \etrue$:
\begin{itemize}
\item \eval{\rho\ \etrue}{\etrue} and $\forall\forall\sigmamodel \in  \instance{\rho}. \sigmamodel \models \etrue$
\item \eval{\rho\ \etrue\not}{\efalse} and $\exists \forall\sigmamodel \in  \instance{\rho}. \sigmamodel \models \etrue$
\end{itemize}

\item $ p \equiv \efalse$:
\begin{itemize}
\item $\eval{\rho\ \efalse \not}{\etrue}$ and $\exists \sigmamodel \in \instance{\rho}. \sigmamodel \not \models \efalse$
\item $\eval{\rho\ \efalse}{\efalse}$ and $\forall \sigmamodel \in \instance{\rho}.\sigmamodel \not \models \efalse$
\end{itemize}
\item $ p \equiv \lnot q$:
\begin{itemize}
\item
$
	\evals{\rho\ (\lnot q)}{\etrue} \Leftrightarrow
	\evals{\lnot (\rho\ q)}{\etrue} \Leftrightarrow
	\evals{\rho\ q}{\efalse} 		\Leftrightarrow
	\forall \sigmamodel \in \instance{\rho}. \sigmamodel \not \models q	 \Leftrightarrow
	\forall \sigmamodel \in \instance{\rho}. \sigmamodel \models \lnot q \Leftrightarrow
	\forall \sigmamodel \in \instance{\rho}. \sigmamodel \models p
$
\item
$
	\evals{\rho\ (\lnot q)}{\efalse} \Leftrightarrow
	\evals{\lnot (\rho\ q)}{\efalse} \Leftrightarrow
	\evals{\rho\ q}{\etrue} 		\Leftrightarrow
	\forall \sigmamodel\in\instance{\rho}. \sigmamodel \models q	 \Leftrightarrow
	\forall \sigmamodel\in\instance{\rho}. \sigmamodel \not \models \lnot q \Leftrightarrow
	\forall \sigmamodel\in\instance{\rho}. \sigmamodel \not \models p
$
\end{itemize}

\item $ p \equiv p_1 \land p_2$:
\begin{itemize}
\item
$
	\evals{\rho\ (p_1 \land p_2)}{\etrue} \Leftrightarrow
$
$	
	\evals{(\rho\ p_1) \land (\rho\ p_2) }{\etrue} \Leftrightarrow
$
$	\evals{\rho\ p_1}{\etrue} \text{ and } \evals{\rho\ p_2}{\etrue} \Leftrightarrow
$\\
$
	{\forall\sigmamodel\in\instance{\rho}. \sigmamodel\models p_1} \text{ and } 
	{\forall\sigmamodel\in\instance{\rho}. \sigmamodel\models p_2} \Leftrightarrow
$\\
$	
	{\forall\sigmamodel\in\instance{\rho}. \instance{\rho}\models p_1 \land p_2}  \Leftrightarrow
	\forall\sigmamodel\in\instance{\rho}. \instance{\rho}  \models p		 		
$
\item
$
	\evals{\rho\ (p_1 \land p_2)}{\efalse} \Leftrightarrow
	\evals{(\rho\ p_1) \land (\rho\ p_2) }{\efalse} \Leftrightarrow
	\left\{
	\begin{array}{c}
		\evals{\rho\ p_1}{\efalse} \\
		 \text{OR}\\
		\evals{\rho\ p_2}{\efalse} \\
	\end{array}
	\right.
	\Leftrightarrow
	\left.
	\begin{array}{c}
		{\forall\sigmamodel\in\instance{\rho}. \sigmamodel\not\models p_1} \\
		 \text{OR}\\
		{\forall\sigmamodel\in\instance{\rho}. \sigmamodel\not\models p_2} \\
	\end{array}
	\right\}
$\\
$	
	  \Leftrightarrow
	{\forall\sigmamodel\in\instance{\rho}. \sigmamodel\not\models p_1 \land p_2}  \Leftrightarrow
	\forall\sigmamodel\in\instance{\rho}. \sigmamodel \not\models p		 		
$
\end{itemize}

\item $p \equiv t_1 = t_2$:
\begin{itemize}
\item
$
\begin{array}{lclclcl}
	&&\evals{\rho\ (t_1 = t_2)}{\etrue} 
	\\&\Leftrightarrow&
	\evals{(\rho\ t_1) = (\rho\ t_2)}{\etrue} &&\\
	&\Leftrightarrow&
	\exists d_1, d_2.\evals{\rho\ t_1}{d_1} &\text{and}& \evals{=_{d_1} (\rho\ t_2)}{\etrue} &\\
	&\Leftrightarrow&
	\exists d_1, d_2.\evals{\rho\ t_1}{d_1} &\text{and}& \evals{\rho\ t_2}{d_2} \\
	&&&\text{and}& d_1 =_D d_2\\
	&\Leftrightarrow&
	\exists d_1, d_2\forall\sigmamodel\in\instance{\rho}.\interpIEq{\sigmamodel}{t_1}{d_1} &\text{and}& 
	\forall\sigmamodel\in\instance{\rho}.\interpIEq{\sigmamodel}{t_2}{d_2} \\&&&\text{and}& d_1 =_D d_2\\
	&\xLeftrightarrow{(*)}&
	\forall\sigmamodel\in\instance{\rho}\exists d_1, d_2.\interpIEq{\sigmamodel}{t_1}{d_1} &\text{and}& 
	\interpIEq{\sigmamodel}{t_2}{d_2} \\&&&\text{and}& d_1 =_D d_2\\
	&\Leftrightarrow&
	\forall\sigmamodel\in \instance{\rho}. \sigmamodel \models t_1 = t_2  &&\\
\end{array}
$

\item
$
\begin{array}{lclclcl}
	&&\evals{\rho\ (t_1 = t_2)}{\efalse} \\&\Leftrightarrow&
	\evals{(\rho\ t_1) = (\rho\ t_2)}{\efalse} &&\\
	&\Leftrightarrow&
	\exists d_1, d_2.\evals{\rho\ t_1}{d_1} &\text{and}& \evals{=_{d_1} (\rho\ t_2)}{\efalse} &\\
	&\Leftrightarrow&
	\exists d_1, d_2.\evals{\rho\ t_1}{d_1} &\text{and}& \evals{\rho\ t_2}{d_2} \\&&&\text{and}& d_1 \not=_D d_2\\
	&\Leftrightarrow&
	\exists d_1, d_2\forall\sigmamodel\in\instance{\rho}.\interpIEq{\sigmamodel}{t_1}{d_1} &\text{and}& 
	\forall\sigmamodel\in\instance{\rho}.\interpIEq{\sigmamodel}{t_2}{d_2} \\&&&\text{and}& d_1 \not =_D d_2\\
	&\xLeftrightarrow{(*)}&
	\forall\sigmamodel\in\instance{\rho}\exists d_1, d_2.\interpIEq{\sigmamodel}{t_1}{d_1} &\text{and}& 
	\interpIEq{\sigmamodel}{t_2}{d_2} \\&&&\text{and}& d_1 \not =_D d_2\\
	&\Leftrightarrow&
	\forall\sigmamodel\in \instance{\rho}. \sigmamodel \not \models t_1 = t_2  &&\\
\end{array}
$
\end{itemize}
$(*)$ For $i = 1 , 2$, fix two instantiations
$\sigmamodel_1, \sigmamodel_2\in\instance{\rho}$. 
Assume that $d_{i_{\sigmamodel_1}} \not = d_{i_{\sigmamodel_2}}$.
Then $\lnot \forall \sigmamodel\in\instance{\rho} \interpIEq{\sigmamodel}{t_i}{d} \Rightarrow \evals{\rho\ t_i \not}{d} \Rightarrow \lnot \hastype{\Gamma}{t_i}{b^\finite} \Rightarrow \lnot \hastype{\Gamma}{p}{\tbool}$

\item $p \equiv t_1 < t_2$:
\begin{itemize}
\item
$
\begin{array}{lclclcl}
	&&\evals{\rho\ (t_1 < t_2)}{\etrue} \\&\Leftrightarrow&
	\evals{(\rho\ t_1) < (\rho\ t_2)}{\etrue} &&\\
	&\Leftrightarrow&
	\exists d_1.\evals{\rho\ t_1}{d_1} &\text{and}& \evals{<_{d_1} (\rho\ t_2)}{\etrue} &\\
	&\Leftrightarrow&
	\exists d_1, d_2.\evals{\rho\ t_1}{d_1} &\text{and}& \evals{\rho\ t_2}{d_2} \\&&&\text{and}& d_1 <_D d_2\\
	&\Leftrightarrow&
	\exists d_1, d_2.\forall\instance{\rho}.\interpIEq{\instance{\rho}}{t_1}{d_1} &\text{and}& 
	\forall\sigmamodel\in\instance{\rho}.\interpIEq{\sigmamodel}{t_2}{d_2} 
	\\&&&\text{and}& d_1 <_D d_2\\
	&\Leftrightarrow&
	\exists d_1, d_2.\forall\sigmamodel\in\instance{\rho}.
	\interpIEq{\sigmamodel}{t_1}{d_1} &\text{and}& 
	\interpIEq{\sigmamodel}{t_2}{d_2} \\&&&\text{and}& d_1 <_D d_2\\
	&\xLeftrightarrow{(*)}&
	\forall\sigmamodel\in\instance{\rho}\exists d_1, d_2.
	\interpIEq{\sigmamodel}{t_1}{d_1}&\text{and}&\interpIEq{\sigmamodel}{t_2}{d_2} \\&&&\text{and}& d_1 <_D d_2\\
	&\Leftrightarrow&
	\forall \sigmamodel\in\instance{\rho}. \sigmamodel \models t_1 < t_2  &&\\
\end{array}
$
\item
$
\begin{array}{lclclcl}
	&&\evals{\rho\ (t_1 < t_2)}{\efalse} \\&\Leftrightarrow&
	\evals{(\rho\ t_1) < (\rho\ t_2)}{\efalse} &&\\
	&\Leftrightarrow&
	\exists d_1.\evals{\rho\ t_1}{d_1} &\text{and}& \evals{<_{d_1} (\rho\ t_2)}{\efalse} &\\
	&\Leftrightarrow&
	\exists d_1, d_2.\evals{\rho\ t_1}{d_1} &\text{and}& \evals{\rho\ t_2}{d_2} \\&&&\text{and}& d_1 \not <_D d_2\\
	&\Leftrightarrow&
	\exists d_1, d_2\forall\sigmamodel\in\instance{\rho}.\interpIEq{\sigmamodel}{t_1}{d_1} &\text{and}& 
	\forall\sigmamodel\in\instance{\rho}.\interpIEq{\sigmamodel}{t_2}{d_2} \\&&&\text{and}& d_1 \not <_D d_2\\
	&\Leftrightarrow&
	\exists d_1, d_2\forall\sigmamodel\in\instance{\rho}.\interpIEq{\sigmamodel}{t_1}{d_1} &\text{and}& 
	\interpIEq{\sigmamodel}{t_2}{d_2} \\&&&\text{and}& d_1 \not <_D d_2\\
	&\xLeftrightarrow{(*)}&
	\forall\sigmamodel\in\instance{\rho}\exists d_1, d_2.
	\interpIEq{\sigmamodel}{t_1}{d_1} &\text{and}& 
	\interpIEq{\sigmamodel}{t_2}{d_2} \\&&&\text{and}& d_1 \not <_D d_2\\
	&\Leftrightarrow&
	\forall\sigmamodel\in \instance{\rho}. \sigmamodel\not\models t_1 < t_2  &&\\
\end{array}
$
\end{itemize}
$(*)$ For $i = 1 , 2$, fix two instantiations
$\sigmamodel_1, \sigmamodel_2 \in \instance{\rho}$. 
Assume that $d_{i_{\sigmamodel_1}} \not = d_{i_{\sigmamodel_2}}$.
Then $\lnot \forall\sigmamodel\in \instance{\rho} \interpIEq{\sigmamodel}{t_i}{d} \Rightarrow \evals{\rho\ t_i \not}{d} \Rightarrow \lnot \hastype{\Gamma}{t_i}{b^\finite} \Rightarrow \lnot \hastype{\Gamma}{p}{\tbool}$


\item $p \equiv t$:
\begin{itemize}
\item
$
\begin{array}{lclclcl}
	\evals{\rho\ t}{\etrue} &\Leftrightarrow&
	\forall\sigmamodel\in\instance{\rho}.\interpIEq{\sigmamodel}{t}{\etrue}
	\\&\Leftrightarrow&
	\forall\sigmamodel\in \instance{\rho}. \sigmamodel \models t\\
\end{array}
$
\item
$
\begin{array}{lclclcl}
	\evals{\rho\ t}{\efalse} &\Leftrightarrow&
	\forall\sigmamodel\in\instance{\rho}.\interpIEq{\sigmamodel}{t}{\efalse}
	\\&\Leftrightarrow&
	\forall\sigmamodel\in\instance{\rho}. \sigmamodel \not \models t\\
\end{array}
$
\end{itemize}
\end{itemize}
\end{proof}

\section{Implementation: \toolname}
Here we give some more examples on how we can use \toolname.
We start by proving termination on mutual recursive functions, 
using lexicographical ordering. 
%
Then we describe how we proved functional correctness on 
two commonly used functions, namely \texttt{ByteString}
and \texttt{Text}.

\subsection{Proving Termination}

   Next, consider the Ackermann function.
   %
   \begin{code}
     ack m n 
       | m == 0    = n + 1
       | n == 0    = ack (m-1) 1 
       | otherwise = ack (m-1) (ack m (n-1))
   \end{code}
   %
   There exists no integer termination metric that decreases at each recursive call.
   %
   However @ack@ terminates because at each call \emph{either}
   @m@ decreases \emph{or} @m@ remains the same and @n@ decreases. 
   %
   In other words, the pair @(m,n)@ strictly decreases according to
   \emph{lexicographic} ordering. 
   %
   To capture this requirement we extend termination metric
   from an integer to a list of integers
   and at each recursive call we check that this list is
   lexicographically decreasing.
   %
   In the case of
   @ack@ this list will simply be the parameters @m@
   and @n@:
   %
   \begin{code}
     ack :: m:Nat -> n:Nat -> Nat / [m,n]
   \end{code}
   %
   Thus, \toolname uses lexicographic ordering on 
   a list of natural numbers to prove termination.
   %
   Termination metrics could be generalized to 
   any \emph{well-found} metric.
   
   \spara{Mutual Recursion}
   %
   Equipped with termination metrics
   \toolname instantiates a powerful
   termination checker that like~\citep{XiTerminationLICS01}
   proves termination even for mutual recursive functions.
   %
   Consider the mutual recursive functions @isEven@ and @isOdd@
   \begin{code}
   {-@ isEven :: n:Nat -> Bool / [n, 0] @-}
   {-@ isOdd  :: n:Nat -> Bool / [n, 1] @-}
   
   isEven 0 = True
   isEven n = isOdd $ n-1
   
   isOdd n  = not $ isEven n 
   \end{code}
   Each call terminates as either @isEven@
   calls @isOdd@ with a decreasing argument, 
   or the argument remains the same, and @isOdd@
   calls @isEven@ that should then decrease the argument.
   % 
   We capture this reasoning using two lexicographic pairs:
   each function has its own metric, 
   and when @isEven@ calls @isOdd@
   the metric of the caller $(n, 0)$
   should be greater that callee's metric
   $(n-1, 1)$.
   %
   Similarly, at @isEven@'s call-site 
   \toolname verifies that	
   $(n, 1) > (n, 0)$.
   %
   For example, the call @isEven m@
   will fire the decreasing metric sequence
   $(m, 0) > (m-1, 1) > (m-1, 0) > (m-2, 1) > \dots$
   that ultimate terminates for \textit{any}
   natural number $m$.


\subsection{Bytestring}\label{sec:bytestring}
% The terms ``Haskell'' and ``pointer arithmetic'' 
% rarely occur in the same sentence. 
% Thus, from a verification point of view, the 
The single most important aspect of the \bytestring 
library,%~\cite{bytestring}, 
our first case study, is its pervasive intermingling of
high level abstractions like higher-order loops,
folds, and fusion, with low-level pointer 
manipulations in order to achieve high-performance. 
%
%% From the package description, \bytestring is, 
%% ``A time and space-efficient implementation of byte vectors using packed
%% Word8 arrays, suitable for high performance use, both in terms of large
%% data quantities, or high speed requirements. Byte vectors are encoded as
%% strict Word8 arrays of bytes, held in a ForeignPtr, and can be passed
%% between C and Haskell with little effort."
%
\bytestring is an appealing target for evaluating
\toolname, as refinement types are an ideal way to 
statically ensure the correctness of the delicate 
pointer manipulations, errors in which lie below 
the scope of dynamic protection.

The library spans $8$ files (modules) totaling about 3,500 lines.
We used \toolname to verify the library by giving precise 
types describing the sizes of internal pointers and bytestrings. 
These types are used in a modular fashion to verify the 
implementation of functional correctness properties of 
higher-level API functions which are built using 
lower-level internal operations. 
Next, we show the key invariants and how
\toolname reasons precisely about pointer
arithmetic and higher-order codes.

\spara{Key Invariants}
A (strict) @ByteString@ is a triple of a @pay@load pointer, 
an @off@set into the memory buffer referred to by the pointer 
(at which the string actually ``begins") and a @len@gth 
corresponding to the number of bytes in the string, which is 
the size of the buffer \emph{after} the @off@set, that
corresponds to the string.
%
We define a measure for the \emph{size} of 
a @ForeignPtr@'s buffer, and use it to define 
the key invariants as a refined datatype 
%
\begin{code}
  measure fplen  :: ForeignPtr a -> Int
  data ByteString = PS
   { pay :: ForeignPtr Word8
   , off :: {v:Nat | v       <= (fplen pay)}
   , len :: {v:Nat | off + v <= (fplen pay)} }
\end{code}
%
The definition states that 
the offset is a @Nat@ no bigger than the size of 
the @payload@'s buffer, and that
the sum of the @off@set and non-negative @len@gth
is no more than the size of the payload buffer.
Finally, we encode a @ByteString@'s size as a measure.
%
\begin{code}
  measure bLen   :: ByteString -> Int
  bLen (PS p o l) = l
\end{code}

\spara{Specifications}
We define a type alias for a @ByteString@ whose length is the same
as that of another, and use the alias to type the API 
function @copy@, which clones @ByteString@s.

\begin{code}
  type ByteStringEq B 
    = {v:ByteString | (bLen v) = (bLen B)}
  copy :: b:ByteString -> ByteStringEq b 
  copy (PS fp off len) 
    = unsafeCreate len $ \p -> 
        withForeignPtr fp $ \f ->
          memcpy len p (f `plusPtr` off) 
\end{code}

\spara{Pointer Arithmetic}
The simple body of @copy@ abstracts a fair bit of internal work. 
@memcpy sz dst src@, implemented in \C and accessed via the FFI is a potentially
dangerous, low-level operation, that copies @sz@ bytes starting
\emph{from} an address @src@ \emph{into} an address @dst@. 
Crucially, for safety, the regions referred to be @src@ and @dst@ 
must be larger than @sz@. We capture this requirement by defining
a type alias @PtrN a N@ denoting GHC pointers that refer to a region
bigger than @N@ bytes, and then specifying that the destination
and source buffers for @memcpy@ are large enough. 

\begin{code}
  type PtrN a N = {v:Ptr a | N <= (plen v)}
  memcpy :: sz:CSize -> dst:PtrN a siz 
                     -> src:PtrN a siz 
                     -> IO () 
\end{code}


The actual output for @copy@ is created and filled in using the 
internal function @unsafeCreate@ which is a wrapper around. 
% -- | Create ByteString of size @l@ and use
% --   action @f@ to fill it's contents.
\begin{code}
  create :: l:Nat -> f:(PtrN Word8 l -> IO ())
         -> IO (ByteStringN l)
  create l f = do
      fp <- mallocByteString l
      withForeignPtr fp $ \p -> f p
      return $! PS fp 0 l
\end{code}

% We include the comment to illustrate how the 
% refinement type captures the natural language 
% requirement in a machine checkable manner.
%
The type of @f@ specifies that the action
will only be invoked on a pointer of length at least 
@l@, which is verified by propagating the types of
@mallocByteString@ and @withForeignPtr@. 
%
The fact that the action is only invoked on such pointers 
is used to ensure that the value @p@ in the body of @copy@ 
is of size @l@. This, and the @ByteString@ 
invariant that the size of the payload @fp@ 
exceeds the sum of @off@ and @len@, ensures 
that the call to @memcpy@ is safe.

\spara{Interfacing with the Real World}
The above illustrates how \toolname analyzes code that interfaces 
with the ``real world" via the \C FFI. We specify the behavior 
of the world via a refinement typed interface. These types are then assumed
to hold for the corresponding functions, \ie generate pre-condition checks
and post-condition guarantees at usage sites within the Haskell code.


\spara{Higher Order Loops} 
@mapAccumR@ combines a @map@ and a @foldr@ over a @ByteString@. 
The function uses non-trivial recursion, and demonstrates 
the utility of abstract-interpretation based inference. 
%
\begin{code}
  mapAccumR f z b 
    = unSP $ loopDown (mapAccumEFL f) z b
\end{code}
%$
To enable fusion \cite{streamfusion} 
@loopDown@ uses a higher order @loopWrapper@ 
to iterate over the buffer with a @doDownLoop@ action:
%
%% DONE \ES{should we use a termination expression for ``loop'' even though it won't actually work atm in LH?}
\begin{code}
  doDownLoop f acc0 src dest len 
    = loop (len-1) (len-1) acc0
    where
     loop :: s:_ -> _ -> _ -> _ / [s+1]
     loop s d acc
       | s < 0 
       = return (acc :*: d+1 :*: len - (d+1))
       | otherwise       
       = do x <- peekByteOff src s
            case f acc x of
              (acc' :*: NothingS) -> 
                   loop (s-1) d acc'
              (acc' :*: JustS x') -> 
                   pokeByteOff dest d x'
                >> loop (s-1) (d-1) acc'
\end{code}

The above function iterates across the @src@ and @dst@ 
pointers from the right (by repeatedly decrementing the 
offsets @s@ and @d@ starting at the high @len@ down to @-1@). 
Low-level reads and writes are carried out using the 
potentially dangerous @peekByteOff@ and @pokeByteOff@ 
respectively. To ensure safety, we type these low level 
operations with refinements stating that they are only 
invoked with valid offsets @VO@ into the input buffer @p@.

\begin{code}
  type VO P    = {v:Nat | v < plen P}
  peekByteOff :: p:Ptr b -> VO p -> IO a
  pokeByteOff :: p:Ptr b -> VO p -> a -> IO ()
\end{code}

The function @doDownLoop@ is an internal function.
Via abstract interpretation~\cite{LiquidPLDI08}, 
\toolname infers that
%
(1)~@len@ is less than the sizes of @src@ and @dest@,
(2)~@f@ (here, @mapAccumEFL@) always returns a @JustS@, so
(3)~source and destination offsets satisfy $\mathtt{0 \leq s, d < {len}}$,
(4)~the generated @IO@ action returns a triple @(acc :*: 0 :*: len)@,
%
thereby proving the safety of the accesses in @loop@ \emph{and}
verifying that @loopDown@ and the API function @mapAccumR@ 
return a \bytestring whose size equals its input's.
 
To prove \emph{termination}, we add a \emph{termination expression} 
@s+1@ which is always non-negative and decreases at each call.

\spara{Nested Data}
@group@ splits a string like @"aart"@ into the list
@["aa","r","t"]@, \ie a list of
(a)~non-empty @ByteString@s whose 
(b)~total length equals that of the input. 
To specify these requirements, we define a measure for 
the total length of strings in a list and use it to
write an alias for a list of \emph{non-empty} strings
whose total length equals that of another string:

\begin{code}
  measure bLens :: [ByteString] -> Int 
  bLens ([])     = 0
  bLens (x:xs)   = bLen x + bLens xs
  
  type ByteStringNE 
    = {v:ByteString | bLen v > 0}
  type ByteStringsEq B
    = {v:[ByteStringNE] | bLens v = bLen b}
\end{code}
%
\toolname uses the above to verify that
%
\begin{code}
  group :: b:ByteString -> ByteStringsEq b
  group xs
   | null xs   = []
   | otherwise = let x        = unsafeHead xs
                     xs'      = unsafeTail xs
                     (ys, zs) = spanByte x xs' 
                 in (y `cons` ys) : group zs
\end{code}
%
The example illustrates why refinements are critical for
proving termination. \toolname determines that @unsafeTail@ 
returns a \emph{smaller} @ByteString@ than its input, and that
each element returned by @spanByte@ is no bigger than the 
input, concluding that @zs@ is smaller than @xs@, and hence
checking the body under the termination-weakened environment.

To see why the output type holds, let's look at @spanByte@,
which splits strings into a pair:
%
\begin{code}
  spanByte c ps@(PS x s l) 
    = inlinePerformIO $ withForeignPtr x $
          \p -> go (p `plusPtr` s) 0
    where
      go :: _ -> i:_ -> _ / [l-i]
      go p i 
        | i >= l    = return (ps, empty)
        | otherwise = do
            c' <- peekByteOff p i
            if c /= c'
              then let b1 = unsafeTake i ps
                       b2 = unsafeDrop i ps
                   in  return (b1, b2)
              else go p (i+1)
\end{code}
%
Via inference, \toolname verifies the safety of 
the pointer accesses, and determines that the 
sum of the lengths of the output pair of 
@ByteString@s equals that of the input @ps@.
@go@ terminates as @l-i@ is a well-founded 
decreasing metric.

%%% Local Variables: 
%%% mode: latex
%%% TeX-master: "main"
%%% End: 


\subsection{Text}\label{sec:text}
Next %, to give a qualitative sense of the kinds of properties analyzed 
% during the course of our evaluation, 
we present a brief overview of the verification of \libtext, which 
is the standard library used for serious unicode text processing. 
\libtext uses byte arrays and stream fusion to guarantee 
performance while providing a high-level API.
In our evaluation of \toolname on \libtext,%~\cite{text},
we focused on two types of properties: 
(1) the safety of array index and write operations, and 
(2) the functional correctness of the top-level API.
%
These are both made more interesting by the fact that 
\libtext internally encodes characters using UTF-16, 
in which characters are stored in either two or four bytes.
%
\libtext is a vast library spanning 39 modules and 5,700 lines of
code, however we focus on the 17 modules that are relevant
to the above properties.
%
While we have verified exact functional correctness size properties
for the top-level API, we focus here on the low-level functions 
and interaction with unicode.

\spara{Arrays and Texts}
A @Text@ consists of an (immutable) @Array@ of 16-bit words,
an offset into the @Array@, and a length describing the
number of @Word16@s in the @Text@.  
The @Array@ is created and filled using a
\emph{mutable} @MArray@. 
All write operations in \libtext are performed on @MArray@s 
in the @ST@ monad, but they are \emph{frozen} into @Array@s
before being used by the @Text@ constructor.
%
We write a measure for the size of an @MArray@ and use
it to type the write and freeze operations.
%
\begin{code}
  measure malen       :: MArray s -> Int
  predicate EqLen A MA = alen A = malen MA
  predicate Ok I A     = 0 <= I < malen A
  type VO A            = {v:Int| Ok v A} 
  
  unsafeWrite  :: m:MArray s
               -> VO m -> Word16 -> ST s ()
  unsafeFreeze :: m:MArray s
               -> ST s {v:Array | EqLen v m}
\end{code}

\spara{Reasoning about Unicode}
The function @writeChar@ (abbreviating the function \texttt{unsafeWrite} from \texttt{UnsafeChar})
writes a @Char@ into an @MArray@.
\libtext uses UTF-16 to represent characters internally,
meaning that every @Char@ will be encoded using two or 
four bytes (one or two @Word16@s).
%
\begin{code}
  writeChar marr i c
      | n < 0x10000 = do
          unsafeWrite marr i (fromIntegral n)
          return 1
      | otherwise = do
          unsafeWrite marr i lo
          unsafeWrite marr (i+1) hi
          return 2
      where n = ord c
            m = n - 0x10000
            lo = fromIntegral
               $ (m `shiftR` 10) + 0xD800
            hi = fromIntegral
               $ (m .&. 0x3FF) + 0xDC00
\end{code}
%
The UTF-16 encoding complicates the specification of the function
as we cannot simply require @i@ to be less than the length of 
@marr@; if @i@ were @malen marr - 1@ and @c@ required two 
@Word16@s, we would perform an out-of-bounds write. 
%
We account for this subtlety with a predicate that states 
there is enough @Room@ to encode @c@.
%
% measure ord         :: Char -> Int
\begin{code}
  predicate OkN I A N  = Ok (I+N-1) A
  predicate Room I A C = if ord C < 0x10000
                         then OkN I A 1
                         else OkN I A 2
  
  type OkSiz I A = {v:Nat  | OkN  I A v}
  type OkChr I A = {v:Char | Room I A v}
\end{code}
%
@Room i marr c@ says 
``if @c@ is encoded using one @Word16@, 
  then @i@ must be less than @malen marr@,
  otherwise @i@ must be less than @malen marr - 1@.''
%
@OkSiz I A@ is an alias for a valid number of @Word16@s 
remaining after the index @I@ of array @A@. 
@OkChr@ specifies the @Char@s for which there is room (to write)
at index @I@ in array @A@.
%
The specification for @writeChar@ states that given an array \hbox{@marr@,}
an index @i@, and a valid @Char@ for which there is room at index \hbox{@i@,}
the output is a monadic action returning the number of @Word16@ occupied
by the @char@.
%
\begin{code}
  writeChar :: marr:MArray s
            -> i:Nat
            -> OkChr i marr
            -> ST s (OkSiz i marr)
\end{code}
%
\spara{Bug}
Thus, clients of @writeChar@ should only call it with suitable indices
and characters.
%
Using \toolname we found an error in one client, @mapAccumL@, 
which combines a map and a fold over a @Stream@, and stores 
the result of the map in a @Text@. Consider the inner loop of @mapAccumL@.
%
% \begin{code}
% mapAccumL f z0 (Stream next0 s0 len) =
%   (nz, Text na 0 nl)
%  where
%   mlen = upperBound 4 len
%   (na,(nz,nl)) = runST $ do
%        (marr,x) <- (new mlen >>= \arr ->
%                     outer arr mlen z0 s0 0)
%        arr      <- unsafeFreeze marr
%        return (arr,x)
%   outer arr top = loop
%    where
%     loop !z !s !i =
%       case next0 s of
%         Done          -> return (arr, (z,i))
%         Skip s'       -> loop z s' i
%         Yield x s'
%           | j >= top  -> do
%             let top' = (top + 1) `shiftL` 1
%             arr' <- new top'
%             copyM arr' 0 arr 0 top
%             outer arr' top' z s i
%           | otherwise -> do
%             let (z',c) = f z x
%             d <- writeChar arr i c
%             loop z' s' (i+d)
%           where j | ord x < 0x10000 = i
%                   | otherwise       = i + 1
% \end{code}
\begin{code}
  outer arr top = loop
   where
    loop !z !s !i =
      case next0 s of
        Done          -> return (arr, (z,i))
        Skip s'       -> loop z s' i
        Yield x s'
          | j >= top  -> do
            let top' = (top + 1) `shiftL` 1
            arr' <- new top'
            copyM arr' 0 arr 0 top
            outer arr' top' z s i
          | otherwise -> do
            let (z',c) = f z x
            d <- writeChar arr i c
            loop z' s' (i+d)
          where j | ord x < 0x10000 = i
                  | otherwise       = i + 1
\end{code}
%
Let's focus on the @Yield x s'@ case.
%
We first compute the maximum index @j@ to 
which we will write and determine the safety of a write. 
%
If it is safe to write to @j@ we call the provided 
function @f@ on the accumulator @z@ and the character 
@x@, and write the \emph{resulting} character @c@ into the array. 
%
However, we know nothing about @c@, in particular, 
whether @c@ will be stored as one or two @Word16@s! 
Thus, \toolname flags the call to @writeChar@ as \emph{unsafe}.
The error can be fixed by lifting @f z x@ into the @where@ clause and defining the
write index @j@ by comparing @ord c@ (not @ord x@). \toolname (and the authors)
readily accepted our fix.

%% INCLUDEPROOF To illustrate why the call is in fact buggy, 
%% INCLUDEPROOF consider a sample iteration of @loop@ 
%% INCLUDEPROOF where @i = malen arr - 1@ and
%% INCLUDEPROOF @ord x < 0x10000@. 
%% INCLUDEPROOF %
%% INCLUDEPROOF In this case @j@ will equal @i@ and we will enter
%% INCLUDEPROOF the @otherwise@ branch. 
%% INCLUDEPROOF %
%% INCLUDEPROOF Next, suppose @f z x@ returns a
%% INCLUDEPROOF @c@ such that  @ord c >= 0x10000@. 
%% INCLUDEPROOF %
%% INCLUDEPROOF The action @writeChar arr i c@ will write to
%% INCLUDEPROOF indices @i@ \emph{and} @i+1@ of @arr@, but 
%% INCLUDEPROOF @i+1 = malen arr@ and is not a valid index 
%% INCLUDEPROOF for writing! 
%% INCLUDEPROOF %
%% INCLUDEPROOF The error lies dormant till the next loop 
%% INCLUDEPROOF iteration, when @i = malen arr + 1@ and we 
%% INCLUDEPROOF trigger the @j >= top@ branch. 
%% INCLUDEPROOF %
%% INCLUDEPROOF Here, we allocate a larger array and copy 
%% INCLUDEPROOF the contents of the previous array into the 
%% INCLUDEPROOF new array. 
%% INCLUDEPROOF %
%% INCLUDEPROOF The @copyM arr' 0 arr 0 top@ call
%% INCLUDEPROOF only copies @top@ elements, \ie it 
%% INCLUDEPROOF \emph{does not}
%% INCLUDEPROOF copy the element \emph{at} \texttt{top},
%% INCLUDEPROOF \emph{losing} a @Word16@ and so 
%% INCLUDEPROOF yielding the wrong  output.
%% INCLUDEPROOF The fix is to replace...
%% INCLUDEPROOF \begin{code}
%% INCLUDEPROOF    | j >= top  -> do ...
%% INCLUDEPROOF    | otherwise -> do
%% INCLUDEPROOF      d <- writeChar arr i c
%% INCLUDEPROOF      loop z' s' (i+d)
%% INCLUDEPROOF    where (z',c) = f z x
%% INCLUDEPROOF          j | ord c < 0x10000 = i
%% INCLUDEPROOF            | otherwise       = i + 1
%% INCLUDEPROOF \end{code}

%%% Local Variables: 
%%% mode: latex
%%% TeX-master: "main"
%%% End: 


}
{}

\end{document}

%%% Local Variables: 
%%% mode: latex
%%% TeX-master: "main"
%%% End: 
