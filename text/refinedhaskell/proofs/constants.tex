\section{Constants}

We can prove that all the above constants belong to the 
interpretations of their types.
%
\begin{theorem}{[Constants]}\label{thm:constant}
$c \in \interp{\constty{c}}$.
\end{theorem} 
%
The Theorem trivially holds for more of the constants.
For example, 
\newcommand\eqtype{\ensuremath{
	\tfun{x}{b^\lfinite}{
	\tfun{y}{b^\lfinite}{
	\tlref{v}{\tbool}{\lfinite}{v \Leftrightarrow x = y}
	}}
}}
$$= \in \interp{\eqtype}$$
as $\forall e_1, e_2, \evals{e_1}{d_1}, \evals{e_2}{d_2}\Rightarrow 
			\evals{(e_1=e_2 \Leftrightarrow e_1= e_2)}{\etrue}
			\land \exists d. \evals{(e_1 = e_2)}{d}$

Here we prove that for any type $\tau$, 
\efix{\tau} and \etfix{\tau} satisfy Theorem~\ref{thm:constant}.

Given the families of constants:
\begin{align*}
\ceval{\etfix{\tau}}{f} & \doteq \efun{n}{}{\efun{f}{}{\etfixn{\tau}{}{n}}}\\ 
%\ceval{\etfixf{\tau}{f}{??}}{n} &\doteq
%f\ n\ (\etfixn{\tau}{f}{n}\ f) \\
\ceval{\etfixn{\tau}{}{n}}{m} &\doteq
\efun{f}{}{
f\ m\ (\etfixn{\tau}{f}{m}\ f)} \\
\end{align*}
%
and their types
%
\begin{align*}
%\decr{\tau}{n} & \doteq \decrtypefull{\tau}{n}\\
\constty{\etfix{\tau}} &\doteq 
	(\decrty{\tau})
	 \rightarrow
	\tfun{m}{\tnat^\lfinite}{\tau\sub{x}{m}}\\ 
%\constty{\etfixf{\tau}{f}{n}} &\doteq 
%	 \etfixfty{\tau}{f}{n}\\ 
\constty{\etfixn{\tau}{f}{n}} &\doteq  
	(\decrty{\tau})
	\rightarrow\adecrty{\tau}{n}\\ 
%\constty{\etfixfn{\tau}{f}{n}} &\doteq 
%	 \etfixfty{\tau}{f}{n}\\ 
\end{align*}
we prove that the constants belong to the 
meanings of their types:
%
\begin{theorem}{[Terminating Fixpoint]}\label{thm:fixpoint}
\begin{enumerate}
\item\label{nfix}$\forall n. \etfixn{\tau}{f}{n} \in \constty{\etfixn{\tau}{f}{n}}$
\item\label{tfix}$\etfix{\tau} \in \constty{\etfix{\tau}}$
\item\label{fix}$\efix{\tau} \in \constty{\efix{\tau}}$, if the result of $\tau$ is a \Div type.
\end{enumerate}
\end{theorem}
\begin{proof}
\begin{itemize}
\item \ref{nfix}.
We prove that for all
%$f \in \interp{\decrty{\tau}}$
appropriate $f$
and $ m \in \interp{\tref{v}{\tnat}{\lfinite}{v < n}}$,
$e \equiv \etfixn{\tau}{f}{n}\ f \ m \in \interp{\tau\sub{x}{m}}$
% 
by induction on $n$.

For $n=0$, 
it is trivial, as 
there is no $m$ such that
$m \in \interp{\tref{v}{\tnat}{\lfinite}{v < 0}}$.

For the inductive step, $e$ reduces to 
$$
\etfixn{\tau}{f}{n}\ f\ m 
\hookrightarrow
\etfixfn{\tau}{f}{n}\ m 
\hookrightarrow
f\ m\ (\etfixn{\tau}{f}{m}\ f)\\
$$
By IH, since $m < n$,
$\etfixn{\tau}{f}{m} \in \constty{\etfixn{\tau}{f}{m}}$, 
so $f$ receives the appropriate arguments, 
and returns the appropriate result that proves the theorem.
%
\item \ref{tfix}.
We prove that 
for all appropriate $f$
% \\$f \in \interp{\decrty{\tau}}$
and     $ m \in \interp{\tnat^\lfinite}$,
$\etfix{\tau}\ f \ m \in \interp{\tau\sub{x}{m}}$.
%
Since $m \in \interp{\tref{v}{\tnat}{\lfinite}{v < m+1}}$
$$\etfixn{\tau}{f}{m+1}\ f \ m \in \interp{\tau\sub{x}{m}}$$
%
But operationally, 
$\etfixn{\tau}{f}{m+1}\ f \ m$
and
$\etfix{\tau}\ f \ m$
behave equivalently, which proves the theorem.
\item \ref{fix}. The prove for 
$\efix{\tau} \in \constty{\efix{\tau}}$.
is similar.
%
The only difference is that for the base case
\efixn{\tau}{0} should be defined to belong 
into the interpretation of any type.
%
Thus, it is defined as a diverging expression
and the type of \efix{\tau} is constrainted
to $\tau$'s with potentially diverging result. 
%
With refinement types we prove that the basic
\etfixn{\tau}{f}{0} operator
cannot be called, so we omit 
the definition of this basic case.
\end{itemize}
\end{proof}
%
