\section{Implementation: \toolname}
Here we give some more examples on how we can use \toolname.
We start by proving termination on mutual recursive functions, 
using lexicographical ordering. 
%
Then we describe how we proved functional correctness on 
two commonly used functions, namely \texttt{ByteString}
and \texttt{Text}.

\subsection{Proving Termination}

   Next, consider the Ackermann function.
   %
   \begin{code}
     ack m n 
       | m == 0    = n + 1
       | n == 0    = ack (m-1) 1 
       | otherwise = ack (m-1) (ack m (n-1))
   \end{code}
   %
   There exists no integer termination metric that decreases at each recursive call.
   %
   However @ack@ terminates because at each call \emph{either}
   @m@ decreases \emph{or} @m@ remains the same and @n@ decreases. 
   %
   In other words, the pair @(m,n)@ strictly decreases according to
   \emph{lexicographic} ordering. 
   %
   To capture this requirement we extend termination metric
   from an integer to a list of integers
   and at each recursive call we check that this list is
   lexicographically decreasing.
   %
   In the case of
   @ack@ this list will simply be the parameters @m@
   and @n@:
   %
   \begin{code}
     ack :: m:Nat -> n:Nat -> Nat / [m,n]
   \end{code}
   %
   Thus, \toolname uses lexicographic ordering on 
   a list of natural numbers to prove termination.
   %
   Termination metrics could be generalized to 
   any \emph{well-found} metric.
   
   \spara{Mutual Recursion}
   %
   Equipped with termination metrics
   \toolname instantiates a powerful
   termination checker that like~\citep{XiTerminationLICS01}
   proves termination even for mutual recursive functions.
   %
   Consider the mutual recursive functions @isEven@ and @isOdd@
   \begin{code}
   {-@ isEven :: n:Nat -> Bool / [n, 0] @-}
   {-@ isOdd  :: n:Nat -> Bool / [n, 1] @-}
   
   isEven 0 = True
   isEven n = isOdd $ n-1
   
   isOdd n  = not $ isEven n 
   \end{code}
   Each call terminates as either @isEven@
   calls @isOdd@ with a decreasing argument, 
   or the argument remains the same, and @isOdd@
   calls @isEven@ that should then decrease the argument.
   % 
   We capture this reasoning using two lexicographic pairs:
   each function has its own metric, 
   and when @isEven@ calls @isOdd@
   the metric of the caller $(n, 0)$
   should be greater that callee's metric
   $(n-1, 1)$.
   %
   Similarly, at @isEven@'s call-site 
   \toolname verifies that	
   $(n, 1) > (n, 0)$.
   %
   For example, the call @isEven m@
   will fire the decreasing metric sequence
   $(m, 0) > (m-1, 1) > (m-1, 0) > (m-2, 1) > \dots$
   that ultimate terminates for \textit{any}
   natural number $m$.


\subsection{Bytestring}\label{sec:bytestring}
% The terms ``Haskell'' and ``pointer arithmetic'' 
% rarely occur in the same sentence. 
% Thus, from a verification point of view, the 
The single most important aspect of the \bytestring 
library,%~\cite{bytestring}, 
our first case study, is its pervasive intermingling of
high level abstractions like higher-order loops,
folds, and fusion, with low-level pointer 
manipulations in order to achieve high-performance. 
%
%% From the package description, \bytestring is, 
%% ``A time and space-efficient implementation of byte vectors using packed
%% Word8 arrays, suitable for high performance use, both in terms of large
%% data quantities, or high speed requirements. Byte vectors are encoded as
%% strict Word8 arrays of bytes, held in a ForeignPtr, and can be passed
%% between C and Haskell with little effort."
%
\bytestring is an appealing target for evaluating
\toolname, as refinement types are an ideal way to 
statically ensure the correctness of the delicate 
pointer manipulations, errors in which lie below 
the scope of dynamic protection.

The library spans $8$ files (modules) totaling about 3,500 lines.
We used \toolname to verify the library by giving precise 
types describing the sizes of internal pointers and bytestrings. 
These types are used in a modular fashion to verify the 
implementation of functional correctness properties of 
higher-level API functions which are built using 
lower-level internal operations. 
Next, we show the key invariants and how
\toolname reasons precisely about pointer
arithmetic and higher-order codes.

\spara{Key Invariants}
A (strict) @ByteString@ is a triple of a @pay@load pointer, 
an @off@set into the memory buffer referred to by the pointer 
(at which the string actually ``begins") and a @len@gth 
corresponding to the number of bytes in the string, which is 
the size of the buffer \emph{after} the @off@set, that
corresponds to the string.
%
We define a measure for the \emph{size} of 
a @ForeignPtr@'s buffer, and use it to define 
the key invariants as a refined datatype 
%
\begin{code}
  measure fplen  :: ForeignPtr a -> Int
  data ByteString = PS
   { pay :: ForeignPtr Word8
   , off :: {v:Nat | v       <= (fplen pay)}
   , len :: {v:Nat | off + v <= (fplen pay)} }
\end{code}
%
The definition states that 
the offset is a @Nat@ no bigger than the size of 
the @payload@'s buffer, and that
the sum of the @off@set and non-negative @len@gth
is no more than the size of the payload buffer.
Finally, we encode a @ByteString@'s size as a measure.
%
\begin{code}
  measure bLen   :: ByteString -> Int
  bLen (PS p o l) = l
\end{code}

\spara{Specifications}
We define a type alias for a @ByteString@ whose length is the same
as that of another, and use the alias to type the API 
function @copy@, which clones @ByteString@s.

\begin{code}
  type ByteStringEq B 
    = {v:ByteString | (bLen v) = (bLen B)}
  copy :: b:ByteString -> ByteStringEq b 
  copy (PS fp off len) 
    = unsafeCreate len $ \p -> 
        withForeignPtr fp $ \f ->
          memcpy len p (f `plusPtr` off) 
\end{code}

\spara{Pointer Arithmetic}
The simple body of @copy@ abstracts a fair bit of internal work. 
@memcpy sz dst src@, implemented in \C and accessed via the FFI is a potentially
dangerous, low-level operation, that copies @sz@ bytes starting
\emph{from} an address @src@ \emph{into} an address @dst@. 
Crucially, for safety, the regions referred to be @src@ and @dst@ 
must be larger than @sz@. We capture this requirement by defining
a type alias @PtrN a N@ denoting GHC pointers that refer to a region
bigger than @N@ bytes, and then specifying that the destination
and source buffers for @memcpy@ are large enough. 

\begin{code}
  type PtrN a N = {v:Ptr a | N <= (plen v)}
  memcpy :: sz:CSize -> dst:PtrN a siz 
                     -> src:PtrN a siz 
                     -> IO () 
\end{code}


The actual output for @copy@ is created and filled in using the 
internal function @unsafeCreate@ which is a wrapper around. 
% -- | Create ByteString of size @l@ and use
% --   action @f@ to fill it's contents.
\begin{code}
  create :: l:Nat -> f:(PtrN Word8 l -> IO ())
         -> IO (ByteStringN l)
  create l f = do
      fp <- mallocByteString l
      withForeignPtr fp $ \p -> f p
      return $! PS fp 0 l
\end{code}

% We include the comment to illustrate how the 
% refinement type captures the natural language 
% requirement in a machine checkable manner.
%
The type of @f@ specifies that the action
will only be invoked on a pointer of length at least 
@l@, which is verified by propagating the types of
@mallocByteString@ and @withForeignPtr@. 
%
The fact that the action is only invoked on such pointers 
is used to ensure that the value @p@ in the body of @copy@ 
is of size @l@. This, and the @ByteString@ 
invariant that the size of the payload @fp@ 
exceeds the sum of @off@ and @len@, ensures 
that the call to @memcpy@ is safe.

\spara{Interfacing with the Real World}
The above illustrates how \toolname analyzes code that interfaces 
with the ``real world" via the \C FFI. We specify the behavior 
of the world via a refinement typed interface. These types are then assumed
to hold for the corresponding functions, \ie generate pre-condition checks
and post-condition guarantees at usage sites within the Haskell code.


\spara{Higher Order Loops} 
@mapAccumR@ combines a @map@ and a @foldr@ over a @ByteString@. 
The function uses non-trivial recursion, and demonstrates 
the utility of abstract-interpretation based inference. 
%
\begin{code}
  mapAccumR f z b 
    = unSP $ loopDown (mapAccumEFL f) z b
\end{code}
%$
To enable fusion \cite{streamfusion} 
@loopDown@ uses a higher order @loopWrapper@ 
to iterate over the buffer with a @doDownLoop@ action:
%
%% DONE \ES{should we use a termination expression for ``loop'' even though it won't actually work atm in LH?}
\begin{code}
  doDownLoop f acc0 src dest len 
    = loop (len-1) (len-1) acc0
    where
     loop :: s:_ -> _ -> _ -> _ / [s+1]
     loop s d acc
       | s < 0 
       = return (acc :*: d+1 :*: len - (d+1))
       | otherwise       
       = do x <- peekByteOff src s
            case f acc x of
              (acc' :*: NothingS) -> 
                   loop (s-1) d acc'
              (acc' :*: JustS x') -> 
                   pokeByteOff dest d x'
                >> loop (s-1) (d-1) acc'
\end{code}

The above function iterates across the @src@ and @dst@ 
pointers from the right (by repeatedly decrementing the 
offsets @s@ and @d@ starting at the high @len@ down to @-1@). 
Low-level reads and writes are carried out using the 
potentially dangerous @peekByteOff@ and @pokeByteOff@ 
respectively. To ensure safety, we type these low level 
operations with refinements stating that they are only 
invoked with valid offsets @VO@ into the input buffer @p@.

\begin{code}
  type VO P    = {v:Nat | v < plen P}
  peekByteOff :: p:Ptr b -> VO p -> IO a
  pokeByteOff :: p:Ptr b -> VO p -> a -> IO ()
\end{code}

The function @doDownLoop@ is an internal function.
Via abstract interpretation~\cite{LiquidPLDI08}, 
\toolname infers that
%
(1)~@len@ is less than the sizes of @src@ and @dest@,
(2)~@f@ (here, @mapAccumEFL@) always returns a @JustS@, so
(3)~source and destination offsets satisfy $\mathtt{0 \leq s, d < {len}}$,
(4)~the generated @IO@ action returns a triple @(acc :*: 0 :*: len)@,
%
thereby proving the safety of the accesses in @loop@ \emph{and}
verifying that @loopDown@ and the API function @mapAccumR@ 
return a \bytestring whose size equals its input's.
 
To prove \emph{termination}, we add a \emph{termination expression} 
@s+1@ which is always non-negative and decreases at each call.

\spara{Nested Data}
@group@ splits a string like @"aart"@ into the list
@["aa","r","t"]@, \ie a list of
(a)~non-empty @ByteString@s whose 
(b)~total length equals that of the input. 
To specify these requirements, we define a measure for 
the total length of strings in a list and use it to
write an alias for a list of \emph{non-empty} strings
whose total length equals that of another string:

\begin{code}
  measure bLens :: [ByteString] -> Int 
  bLens ([])     = 0
  bLens (x:xs)   = bLen x + bLens xs
  
  type ByteStringNE 
    = {v:ByteString | bLen v > 0}
  type ByteStringsEq B
    = {v:[ByteStringNE] | bLens v = bLen b}
\end{code}
%
\toolname uses the above to verify that
%
\begin{code}
  group :: b:ByteString -> ByteStringsEq b
  group xs
   | null xs   = []
   | otherwise = let x        = unsafeHead xs
                     xs'      = unsafeTail xs
                     (ys, zs) = spanByte x xs' 
                 in (y `cons` ys) : group zs
\end{code}
%
The example illustrates why refinements are critical for
proving termination. \toolname determines that @unsafeTail@ 
returns a \emph{smaller} @ByteString@ than its input, and that
each element returned by @spanByte@ is no bigger than the 
input, concluding that @zs@ is smaller than @xs@, and hence
checking the body under the termination-weakened environment.

To see why the output type holds, let's look at @spanByte@,
which splits strings into a pair:
%
\begin{code}
  spanByte c ps@(PS x s l) 
    = inlinePerformIO $ withForeignPtr x $
          \p -> go (p `plusPtr` s) 0
    where
      go :: _ -> i:_ -> _ / [l-i]
      go p i 
        | i >= l    = return (ps, empty)
        | otherwise = do
            c' <- peekByteOff p i
            if c /= c'
              then let b1 = unsafeTake i ps
                       b2 = unsafeDrop i ps
                   in  return (b1, b2)
              else go p (i+1)
\end{code}
%
Via inference, \toolname verifies the safety of 
the pointer accesses, and determines that the 
sum of the lengths of the output pair of 
@ByteString@s equals that of the input @ps@.
@go@ terminates as @l-i@ is a well-founded 
decreasing metric.

%%% Local Variables: 
%%% mode: latex
%%% TeX-master: "main"
%%% End: 


\subsection{Text}\label{sec:text}
Next %, to give a qualitative sense of the kinds of properties analyzed 
% during the course of our evaluation, 
we present a brief overview of the verification of \libtext, which 
is the standard library used for serious unicode text processing. 
\libtext uses byte arrays and stream fusion to guarantee 
performance while providing a high-level API.
In our evaluation of \toolname on \libtext,%~\cite{text},
we focused on two types of properties: 
(1) the safety of array index and write operations, and 
(2) the functional correctness of the top-level API.
%
These are both made more interesting by the fact that 
\libtext internally encodes characters using UTF-16, 
in which characters are stored in either two or four bytes.
%
\libtext is a vast library spanning 39 modules and 5,700 lines of
code, however we focus on the 17 modules that are relevant
to the above properties.
%
While we have verified exact functional correctness size properties
for the top-level API, we focus here on the low-level functions 
and interaction with unicode.

\spara{Arrays and Texts}
A @Text@ consists of an (immutable) @Array@ of 16-bit words,
an offset into the @Array@, and a length describing the
number of @Word16@s in the @Text@.  
The @Array@ is created and filled using a
\emph{mutable} @MArray@. 
All write operations in \libtext are performed on @MArray@s 
in the @ST@ monad, but they are \emph{frozen} into @Array@s
before being used by the @Text@ constructor.
%
We write a measure for the size of an @MArray@ and use
it to type the write and freeze operations.
%
\begin{code}
  measure malen       :: MArray s -> Int
  predicate EqLen A MA = alen A = malen MA
  predicate Ok I A     = 0 <= I < malen A
  type VO A            = {v:Int| Ok v A} 
  
  unsafeWrite  :: m:MArray s
               -> VO m -> Word16 -> ST s ()
  unsafeFreeze :: m:MArray s
               -> ST s {v:Array | EqLen v m}
\end{code}

\spara{Reasoning about Unicode}
The function @writeChar@ (abbreviating the function \texttt{unsafeWrite} from \texttt{UnsafeChar})
writes a @Char@ into an @MArray@.
\libtext uses UTF-16 to represent characters internally,
meaning that every @Char@ will be encoded using two or 
four bytes (one or two @Word16@s).
%
\begin{code}
  writeChar marr i c
      | n < 0x10000 = do
          unsafeWrite marr i (fromIntegral n)
          return 1
      | otherwise = do
          unsafeWrite marr i lo
          unsafeWrite marr (i+1) hi
          return 2
      where n = ord c
            m = n - 0x10000
            lo = fromIntegral
               $ (m `shiftR` 10) + 0xD800
            hi = fromIntegral
               $ (m .&. 0x3FF) + 0xDC00
\end{code}
%
The UTF-16 encoding complicates the specification of the function
as we cannot simply require @i@ to be less than the length of 
@marr@; if @i@ were @malen marr - 1@ and @c@ required two 
@Word16@s, we would perform an out-of-bounds write. 
%
We account for this subtlety with a predicate that states 
there is enough @Room@ to encode @c@.
%
% measure ord         :: Char -> Int
\begin{code}
  predicate OkN I A N  = Ok (I+N-1) A
  predicate Room I A C = if ord C < 0x10000
                         then OkN I A 1
                         else OkN I A 2
  
  type OkSiz I A = {v:Nat  | OkN  I A v}
  type OkChr I A = {v:Char | Room I A v}
\end{code}
%
@Room i marr c@ says 
``if @c@ is encoded using one @Word16@, 
  then @i@ must be less than @malen marr@,
  otherwise @i@ must be less than @malen marr - 1@.''
%
@OkSiz I A@ is an alias for a valid number of @Word16@s 
remaining after the index @I@ of array @A@. 
@OkChr@ specifies the @Char@s for which there is room (to write)
at index @I@ in array @A@.
%
The specification for @writeChar@ states that given an array \hbox{@marr@,}
an index @i@, and a valid @Char@ for which there is room at index \hbox{@i@,}
the output is a monadic action returning the number of @Word16@ occupied
by the @char@.
%
\begin{code}
  writeChar :: marr:MArray s
            -> i:Nat
            -> OkChr i marr
            -> ST s (OkSiz i marr)
\end{code}
%
\spara{Bug}
Thus, clients of @writeChar@ should only call it with suitable indices
and characters.
%
Using \toolname we found an error in one client, @mapAccumL@, 
which combines a map and a fold over a @Stream@, and stores 
the result of the map in a @Text@. Consider the inner loop of @mapAccumL@.
%
% \begin{code}
% mapAccumL f z0 (Stream next0 s0 len) =
%   (nz, Text na 0 nl)
%  where
%   mlen = upperBound 4 len
%   (na,(nz,nl)) = runST $ do
%        (marr,x) <- (new mlen >>= \arr ->
%                     outer arr mlen z0 s0 0)
%        arr      <- unsafeFreeze marr
%        return (arr,x)
%   outer arr top = loop
%    where
%     loop !z !s !i =
%       case next0 s of
%         Done          -> return (arr, (z,i))
%         Skip s'       -> loop z s' i
%         Yield x s'
%           | j >= top  -> do
%             let top' = (top + 1) `shiftL` 1
%             arr' <- new top'
%             copyM arr' 0 arr 0 top
%             outer arr' top' z s i
%           | otherwise -> do
%             let (z',c) = f z x
%             d <- writeChar arr i c
%             loop z' s' (i+d)
%           where j | ord x < 0x10000 = i
%                   | otherwise       = i + 1
% \end{code}
\begin{code}
  outer arr top = loop
   where
    loop !z !s !i =
      case next0 s of
        Done          -> return (arr, (z,i))
        Skip s'       -> loop z s' i
        Yield x s'
          | j >= top  -> do
            let top' = (top + 1) `shiftL` 1
            arr' <- new top'
            copyM arr' 0 arr 0 top
            outer arr' top' z s i
          | otherwise -> do
            let (z',c) = f z x
            d <- writeChar arr i c
            loop z' s' (i+d)
          where j | ord x < 0x10000 = i
                  | otherwise       = i + 1
\end{code}
%
Let's focus on the @Yield x s'@ case.
%
We first compute the maximum index @j@ to 
which we will write and determine the safety of a write. 
%
If it is safe to write to @j@ we call the provided 
function @f@ on the accumulator @z@ and the character 
@x@, and write the \emph{resulting} character @c@ into the array. 
%
However, we know nothing about @c@, in particular, 
whether @c@ will be stored as one or two @Word16@s! 
Thus, \toolname flags the call to @writeChar@ as \emph{unsafe}.
The error can be fixed by lifting @f z x@ into the @where@ clause and defining the
write index @j@ by comparing @ord c@ (not @ord x@). \toolname (and the authors)
readily accepted our fix.

%% INCLUDEPROOF To illustrate why the call is in fact buggy, 
%% INCLUDEPROOF consider a sample iteration of @loop@ 
%% INCLUDEPROOF where @i = malen arr - 1@ and
%% INCLUDEPROOF @ord x < 0x10000@. 
%% INCLUDEPROOF %
%% INCLUDEPROOF In this case @j@ will equal @i@ and we will enter
%% INCLUDEPROOF the @otherwise@ branch. 
%% INCLUDEPROOF %
%% INCLUDEPROOF Next, suppose @f z x@ returns a
%% INCLUDEPROOF @c@ such that  @ord c >= 0x10000@. 
%% INCLUDEPROOF %
%% INCLUDEPROOF The action @writeChar arr i c@ will write to
%% INCLUDEPROOF indices @i@ \emph{and} @i+1@ of @arr@, but 
%% INCLUDEPROOF @i+1 = malen arr@ and is not a valid index 
%% INCLUDEPROOF for writing! 
%% INCLUDEPROOF %
%% INCLUDEPROOF The error lies dormant till the next loop 
%% INCLUDEPROOF iteration, when @i = malen arr + 1@ and we 
%% INCLUDEPROOF trigger the @j >= top@ branch. 
%% INCLUDEPROOF %
%% INCLUDEPROOF Here, we allocate a larger array and copy 
%% INCLUDEPROOF the contents of the previous array into the 
%% INCLUDEPROOF new array. 
%% INCLUDEPROOF %
%% INCLUDEPROOF The @copyM arr' 0 arr 0 top@ call
%% INCLUDEPROOF only copies @top@ elements, \ie it 
%% INCLUDEPROOF \emph{does not}
%% INCLUDEPROOF copy the element \emph{at} \texttt{top},
%% INCLUDEPROOF \emph{losing} a @Word16@ and so 
%% INCLUDEPROOF yielding the wrong  output.
%% INCLUDEPROOF The fix is to replace...
%% INCLUDEPROOF \begin{code}
%% INCLUDEPROOF    | j >= top  -> do ...
%% INCLUDEPROOF    | otherwise -> do
%% INCLUDEPROOF      d <- writeChar arr i c
%% INCLUDEPROOF      loop z' s' (i+d)
%% INCLUDEPROOF    where (z',c) = f z x
%% INCLUDEPROOF          j | ord c < 0x10000 = i
%% INCLUDEPROOF            | otherwise       = i + 1
%% INCLUDEPROOF \end{code}

%%% Local Variables: 
%%% mode: latex
%%% TeX-master: "main"
%%% End: 

